\documentclass[12pt,a4paper]{article}
\usepackage[UTF8]{ctex}
\usepackage{amsmath}
\usepackage{amssymb}
\usepackage{amsthm}
\usepackage{geometry}
\geometry{left=2.5cm,right=2.5cm,top=2.5cm,bottom=2.5cm}

\title{子水平集(Sublevel Set)详解}
\author{}
\date{\today}

\begin{document}

\maketitle

\section{引言}

子水平集(Sublevel Set)是凸分析和优化理论中的重要概念,与凸函数和拟凸函数密切相关。理解子水平集对于理解凸优化的性质非常重要。

\section{子水平集的定义}

\subsection{基本定义}

\textbf{子水平集}:对于函数 $f: \mathbb{R}^n \to \mathbb{R}$ 和实数 $\alpha \in \mathbb{R}$,$f$ 的 $\alpha$-子水平集定义为:

\begin{equation}
S_\alpha = \{\mathbf{x} \in \text{dom } f \mid f(\mathbf{x}) \leq \alpha\}
\end{equation}

\textbf{含义}:
\begin{itemize}
\item 所有函数值不超过 $\alpha$ 的点的集合
\item 是函数定义域的一个子集
\item $\alpha$ 是"水平"或"阈值"
\end{itemize}

\subsection{上水平集}

\textbf{上水平集}(Superlevel Set):对于函数 $f$ 和实数 $\alpha$,$f$ 的 $\alpha$-上水平集定义为:

\begin{equation}
S'_\alpha = \{\mathbf{x} \in \text{dom } f \mid f(\mathbf{x}) \geq \alpha\}
\end{equation}

\textbf{关系}:上水平集是子水平集的"对偶"概念。

\section{几何直观}

\subsection{一维情况}

\textbf{例子}:$f(x) = x^2$

\textbf{子水平集}:$S_\alpha = \{x \mid x^2 \leq \alpha\}$

\begin{itemize}
\item 如果 $\alpha < 0$:$S_\alpha = \emptyset$(空集)
\item 如果 $\alpha = 0$:$S_0 = \{0\}$(单点)
\item 如果 $\alpha > 0$:$S_\alpha = [-\sqrt{\alpha}, \sqrt{\alpha}]$(区间)
\end{itemize}

\textbf{几何意义}:
\begin{itemize}
\item 在函数图像上,画一条水平线 $y = \alpha$
\item 子水平集是函数图像在这条水平线"下方"(包括线上)的所有点的横坐标
\item 对于 $f(x) = x^2$,子水平集是区间 $[-\sqrt{\alpha}, \sqrt{\alpha}]$
\end{itemize}

\subsection{二维情况}

\textbf{例子}:$f(x, y) = x^2 + y^2$

\textbf{子水平集}:$S_\alpha = \{(x, y) \mid x^2 + y^2 \leq \alpha\}$

\begin{itemize}
\item 如果 $\alpha < 0$:$S_\alpha = \emptyset$(空集)
\item 如果 $\alpha = 0$:$S_0 = \{(0, 0)\}$(单点)
\item 如果 $\alpha > 0$:$S_\alpha$ 是半径为 $\sqrt{\alpha}$ 的圆盘(包括边界)
\end{itemize}

\textbf{几何意义}:
\begin{itemize}
\item 在三维空间中,函数图像是抛物面 $z = x^2 + y^2$
\item 子水平集是水平面 $z = \alpha$ "下方"的所有点的投影
\item 对于 $f(x, y) = x^2 + y^2$,子水平集是圆盘
\end{itemize}

\subsection{等高线(等值线)}

\textbf{等高线}:函数值等于 $\alpha$ 的所有点的集合:

\begin{equation}
C_\alpha = \{\mathbf{x} \in \text{dom } f \mid f(\mathbf{x}) = \alpha\}
\end{equation}

\textbf{关系}:
\begin{itemize}
\item 等高线是子水平集的"边界"
\item 子水平集是等高线及其"内部"的区域
\item 上水平集是等高线及其"外部"的区域
\end{itemize}

\section{子水平集与凸性}

\subsection{凸函数的子水平集}

\textbf{重要定理}:如果 $f$ 是凸函数,则其所有子水平集都是凸集。

\textbf{证明}:

设 $f$ 是凸函数,$\mathbf{x}_1, \mathbf{x}_2 \in S_\alpha$,即 $f(\mathbf{x}_1) \leq \alpha$,$f(\mathbf{x}_2) \leq \alpha$。

对于 $\theta \in [0, 1]$,考虑 $\mathbf{x} = \theta \mathbf{x}_1 + (1-\theta) \mathbf{x}_2$:

\begin{align}
f(\mathbf{x}) &= f(\theta \mathbf{x}_1 + (1-\theta) \mathbf{x}_2) \\
&\leq \theta f(\mathbf{x}_1) + (1-\theta) f(\mathbf{x}_2) \quad \text{(凸性)} \\
&\leq \theta \alpha + (1-\theta) \alpha = \alpha
\end{align}

因此 $f(\mathbf{x}) \leq \alpha$,即 $\mathbf{x} \in S_\alpha$。

所以 $S_\alpha$ 是凸集。$\square$

\subsection{拟凸函数的子水平集}

\textbf{定义}:函数 $f$ 是拟凸的,当且仅当其所有子水平集都是凸集。

\textbf{等价性}:
\begin{itemize}
\item $f$ 是拟凸函数 $\Leftrightarrow$ 所有子水平集 $S_\alpha$ 都是凸集
\item 这是拟凸函数的等价定义
\end{itemize}

\textbf{几何意义}:
\begin{itemize}
\item 拟凸函数的"等值线"围成的区域是凸的
\item 函数值"不会在中间突然升高"
\end{itemize}

\subsection{非凸函数的子水平集}

\textbf{例子}:$f(x) = \sin x$

\textbf{子水平集}:$S_\alpha = \{x \mid \sin x \leq \alpha\}$

\begin{itemize}
\item 如果 $\alpha \geq 1$:$S_\alpha = \mathbb{R}$(整个实数轴)
\item 如果 $\alpha < -1$:$S_\alpha = \emptyset$(空集)
\item 如果 $-1 \leq \alpha < 1$:$S_\alpha$ 是多个区间的并集(非凸)
\end{itemize}

\textbf{结论}:$\sin x$ 不是拟凸函数(某些子水平集不是凸集)。

\section{具体例子}

\subsection{例子1:线性函数}

\textbf{函数}:$f(\mathbf{x}) = \mathbf{a}^T \mathbf{x} + b$

\textbf{子水平集}:$S_\alpha = \{\mathbf{x} \mid \mathbf{a}^T \mathbf{x} + b \leq \alpha\} = \{\mathbf{x} \mid \mathbf{a}^T \mathbf{x} \leq \alpha - b\}$

\textbf{几何意义}:
\begin{itemize}
\item 子水平集是半空间(half-space)
\item 半空间是凸集
\item 因此线性函数是凸函数(也是拟凸函数)
\end{itemize}

\subsection{例子2:二次函数}

\textbf{函数}:$f(\mathbf{x}) = \mathbf{x}^T \mathbf{P} \mathbf{x} + \mathbf{q}^T \mathbf{x} + r$,其中 $\mathbf{P} \succeq 0$

\textbf{子水平集}:$S_\alpha = \{\mathbf{x} \mid \mathbf{x}^T \mathbf{P} \mathbf{x} + \mathbf{q}^T \mathbf{x} + r \leq \alpha\}$

\textbf{几何意义}:
\begin{itemize}
\item 如果 $\mathbf{P} \succ 0$:子水平集是椭球(ellipsoid)
\item 如果 $\mathbf{P} \succeq 0$:子水平集是椭球或椭圆柱
\item 椭球是凸集
\item 因此二次函数($\mathbf{P} \succeq 0$)是凸函数
\end{itemize}

\subsection{例子3:范数函数}

\textbf{函数}:$f(\mathbf{x}) = \|\mathbf{x}\|_p$($p \geq 1$)

\textbf{子水平集}:$S_\alpha = \{\mathbf{x} \mid \|\mathbf{x}\|_p \leq \alpha\}$

\textbf{几何意义}:
\begin{itemize}
\item $p = 1$:$L_1$ 范数,子水平集是"菱形"(diamond)
\item $p = 2$:$L_2$ 范数,子水平集是球(ball)
\item $p = \infty$:$L_\infty$ 范数,子水平集是"方盒"(box)
\item 所有这些形状都是凸集
\item 因此范数函数是凸函数
\end{itemize}

\subsection{例子4:最大值函数}

\textbf{函数}:$f(\mathbf{x}) = \max\{x_1, x_2, \ldots, x_n\}$

\textbf{子水平集}:$S_\alpha = \{\mathbf{x} \mid \max\{x_1, \ldots, x_n\} \leq \alpha\} = \{\mathbf{x} \mid x_i \leq \alpha, \forall i\}$

\textbf{几何意义}:
\begin{itemize}
\item 子水平集是"方盒" $\{\mathbf{x} \mid x_i \leq \alpha, \forall i\}$
\item 方盒是凸集
\item 因此最大值函数是凸函数
\end{itemize}

\subsection{例子5:单峰函数}

\textbf{函数}:$f(x) = 
\begin{cases}
x^2 & \text{if } x \leq 0 \\
0 & \text{if } x > 0
\end{cases}$

\textbf{子水平集}:
\begin{itemize}
\item 如果 $\alpha < 0$:$S_\alpha = \emptyset$
\item 如果 $\alpha = 0$:$S_0 = [0, +\infty)$
\item 如果 $\alpha > 0$:$S_\alpha = [-\sqrt{\alpha}, +\infty)$
\end{itemize}

\textbf{分析}:
\begin{itemize}
\item 所有子水平集都是凸集(区间)
\item 因此函数是拟凸的
\item 但函数不是凸的(在 $x=0$ 处不满足凸性)
</{itemize>

\section{子水平集的性质}

\subsection{基本性质}

\begin{enumerate}
\item \textbf{单调性}:如果 $\alpha_1 \leq \alpha_2$,则 $S_{\alpha_1} \subseteq S_{\alpha_2}$
   \begin{itemize}
   \item 较小的 $\alpha$ 对应较小的子水平集
   \item 子水平集随 $\alpha$ 增大而增大
   \end{itemize}

\item \textbf{交集}:$S_\alpha = \bigcap_{\beta > \alpha} S_\beta$
   \begin{itemize}
   \item 子水平集是所有更大子水平集的交集
   \end{itemize}

\item \textbf{闭性}:如果 $f$ 是下半连续的,则 $S_\alpha$ 是闭集
   \begin{itemize}
   \item 对于凸函数,子水平集通常是闭的
   \end{itemize}
\end{enumerate}

\subsection{凸性保持}

\begin{enumerate}
\item \textbf{凸函数的子水平集}:如果 $f$ 是凸函数,则所有 $S_\alpha$ 都是凸集
\item \textbf{拟凸函数的子水平集}:如果 $f$ 是拟凸函数,则所有 $S_\alpha$ 都是凸集
\item \textbf{逆命题}:如果所有子水平集都是凸集,则函数是拟凸的
\end{enumerate}

\section{在优化中的应用}

\subsection{可行域}

\textbf{约束优化问题}:
\begin{align}
\begin{array}{ll}
\text{minimize} & f_0(\mathbf{x}) \\
\text{subject to} & f_i(\mathbf{x}) \leq 0, \quad i = 1, \ldots, m
\end{array}
\end{align}

\textbf{可行域}:
\begin{equation}
\mathcal{F} = \bigcap_{i=1}^m \{\mathbf{x} \mid f_i(\mathbf{x}) \leq 0\} = \bigcap_{i=1}^m S_{0}^{(i)}
\end{equation}

其中 $S_{0}^{(i)}$ 是 $f_i$ 的 $0$-子水平集。

\textbf{关键观察}:
\begin{itemize}
\item 如果所有 $f_i$ 都是凸函数,则可行域是凸集
\item 可行域是多个凸子水平集的交集
\item 凸集的交集是凸集
\end{itemize}

\subsection{最优值}

\textbf{最优值}:$p^* = \inf\{f_0(\mathbf{x}) \mid \mathbf{x} \in \mathcal{F}\}$

\textbf{子水平集视角}:
\begin{itemize}
\item 最优值 $p^*$ 是最小的 $\alpha$,使得 $S_\alpha \cap \mathcal{F} \neq \emptyset$
\item 即:最小的 $\alpha$,使得子水平集与可行域有交集
\end{itemize}

\subsection{$\epsilon$-次优集}

\textbf{定义}:$\epsilon$-次优集是函数值不超过 $p^* + \epsilon$ 的所有可行点:

\begin{equation}
\mathcal{S}_\epsilon = \{\mathbf{x} \in \mathcal{F} \mid f_0(\mathbf{x}) \leq p^* + \epsilon\} = S_{p^* + \epsilon} \cap \mathcal{F}
\end{equation}

\textbf{性质}:
\begin{itemize}
\item 对于凸优化问题,$\epsilon$-次优集是凸集
\item 对于拟凸优化问题,$\epsilon$-次优集也是凸集
\end{itemize}

\section{子水平集的构造}

\subsection{通过函数值判断}

\textbf{方法}:
\begin{enumerate}
\item 给定函数 $f$ 和水平 $\alpha$
\item 对于每个点 $\mathbf{x}$,计算 $f(\mathbf{x})$
\item 如果 $f(\mathbf{x}) \leq \alpha$,则 $\mathbf{x} \in S_\alpha$
\item 如果 $f(\mathbf{x}) > \alpha$,则 $\mathbf{x} \notin S_\alpha$
\end{enumerate}

\subsection{通过等高线}

\textbf{方法}:
\begin{enumerate}
\item 找到等高线 $C_\alpha = \{\mathbf{x} \mid f(\mathbf{x}) = \alpha\}$
\item 确定等高线的"内部"和"外部"
\item 子水平集是等高线及其"内部"的区域
\end{enumerate}

\section{总结}

\subsection{关键概念}

\begin{enumerate}
\item \textbf{定义}:
   \begin{itemize}
   \item 子水平集:$S_\alpha = \{\mathbf{x} \mid f(\mathbf{x}) \leq \alpha\}$
   \item 所有函数值不超过 $\alpha$ 的点的集合
   \end{itemize}

\item \textbf{几何意义}:
   \begin{itemize}
   \item 在函数图像上,水平线 $y = \alpha$ "下方"的所有点
   \item 等高线及其"内部"的区域
   \end{itemize}

\item \textbf{与凸性的关系}:
   \begin{itemize}
   \item 凸函数的子水平集是凸集
   \item 拟凸函数 $\Leftrightarrow$ 所有子水平集都是凸集
   \end{itemize}

\item \textbf{在优化中的应用}:
   \begin{itemize}
   \item 可行域是约束函数的子水平集的交集
   \item 最优值与子水平集和可行域的交集相关
   \end{itemize}
\end{enumerate}

\subsection{记忆技巧}

\begin{enumerate}
\item \textbf{子水平集}:想象函数图像,画一条水平线,取"下方"的所有点
\item \textbf{与凸性}:凸函数的子水平集是凸的,这是凸性的重要特征
\item \textbf{在优化中}:约束 $f(\mathbf{x}) \leq 0$ 定义了子水平集 $S_0$
\end{enumerate}

理解子水平集,是理解凸函数和拟凸函数的关键!

\end{document}

