\documentclass[12pt,a4paper]{article}
\usepackage[UTF8]{ctex}
\usepackage{amsmath}
\usepackage{amssymb}
\usepackage{amsthm}
\usepackage{geometry}
\geometry{left=2.5cm,right=2.5cm,top=2.5cm,bottom=2.5cm}

\title{为什么设 $\alpha = \max\{f(\mathbf{x}_1), f(\mathbf{x}_2)\}$?}
\subtitle{理解证明策略中的关键选择}
\author{}
\date{\today}

\begin{document}

\maketitle

\section{问题提出}

在证明"所有子水平集是凸集 $\Rightarrow$ 拟凸函数"时,我们设:

\begin{equation}
\alpha = \max\{f(\mathbf{x}_1), f(\mathbf{x}_2)\}
\end{equation}

\textbf{问题}:为什么选择这个特定的 $\alpha$ 值?这个选择的巧妙之处在哪里?

\section{证明目标回顾}

\subsection{需要证明什么?}

\textbf{目标}:证明如果所有子水平集都是凸集,则函数是拟凸的。

\textbf{具体需要证明}:对于任意 $\mathbf{x}_1, \mathbf{x}_2 \in \text{dom } f$ 和 $\theta \in [0, 1]$,有:

\begin{equation}
f(\theta \mathbf{x}_1 + (1-\theta) \mathbf{x}_2) \leq \max\{f(\mathbf{x}_1), f(\mathbf{x}_2)\}
\end{equation}

\subsection{已知条件}

\textbf{已知}:所有子水平集 $S_\alpha = \{\mathbf{x} \mid f(\mathbf{x}) \leq \alpha\}$ 都是凸集。

\textbf{这意味着}:对于任意 $\alpha$,如果 $\mathbf{x}_1, \mathbf{x}_2 \in S_\alpha$,则 $\theta \mathbf{x}_1 + (1-\theta) \mathbf{x}_2 \in S_\alpha$。

\section{为什么选择 $\alpha = \max\{f(\mathbf{x}_1), f(\mathbf{x}_2)\}$?}

\subsection{关键观察}

\textbf{目标}:证明 $f(\theta \mathbf{x}_1 + (1-\theta) \mathbf{x}_2) \leq \max\{f(\mathbf{x}_1), f(\mathbf{x}_2)\}$。

\textbf{策略}:找到一个合适的 $\alpha$,使得:
\begin{enumerate}
\item $\mathbf{x}_1, \mathbf{x}_2 \in S_\alpha$(这样可以利用 $S_\alpha$ 的凸性)
\item $\alpha = \max\{f(\mathbf{x}_1), f(\mathbf{x}_2)\}$(这样结论正好是我们需要的)
\end{enumerate}

\subsection{为什么这个选择有效?}

\textbf{选择}:$\alpha = \max\{f(\mathbf{x}_1), f(\mathbf{x}_2)\}$

\textbf{验证条件1}:$\mathbf{x}_1, \mathbf{x}_2 \in S_\alpha$?

\begin{itemize}
\item $f(\mathbf{x}_1) \leq \max\{f(\mathbf{x}_1), f(\mathbf{x}_2)\} = \alpha$,因此 $\mathbf{x}_1 \in S_\alpha$ ✓
\item $f(\mathbf{x}_2) \leq \max\{f(\mathbf{x}_1), f(\mathbf{x}_2)\} = \alpha$,因此 $\mathbf{x}_2 \in S_\alpha$ ✓
\end{itemize}

\textbf{验证条件2}:$\alpha$ 的值正好是我们需要的?

\begin{itemize}
\item $\alpha = \max\{f(\mathbf{x}_1), f(\mathbf{x}_2)\}$ 正是我们要证明的不等式右边
\item 如果能够证明 $f(\theta \mathbf{x}_1 + (1-\theta) \mathbf{x}_2) \leq \alpha$,就完成了证明
\end{itemize}

\textbf{完美匹配}:这个选择同时满足两个条件!

\section{完整的证明逻辑}

\subsection{证明步骤}

\textbf{步骤1}:取任意 $\mathbf{x}_1, \mathbf{x}_2 \in \text{dom } f$ 和 $\theta \in [0, 1]$。

\textbf{步骤2}:设 $\alpha = \max\{f(\mathbf{x}_1), f(\mathbf{x}_2)\}$。

\textbf{为什么这样设?}
\begin{itemize}
\item 这是我们要证明的不等式右边的值
\item 这样如果证明了 $f(\theta \mathbf{x}_1 + (1-\theta) \mathbf{x}_2) \leq \alpha$,就完成了证明
\end{itemize}

\textbf{步骤3}:验证 $\mathbf{x}_1, \mathbf{x}_2 \in S_\alpha$。

\begin{itemize}
\item $f(\mathbf{x}_1) \leq \max\{f(\mathbf{x}_1), f(\mathbf{x}_2)\} = \alpha$,所以 $\mathbf{x}_1 \in S_\alpha$ ✓
\item $f(\mathbf{x}_2) \leq \max\{f(\mathbf{x}_1), f(\mathbf{x}_2)\} = \alpha$,所以 $\mathbf{x}_2 \in S_\alpha$ ✓
\end{itemize}

\textbf{步骤4}:由于 $S_\alpha$ 是凸集,且 $\mathbf{x}_1, \mathbf{x}_2 \in S_\alpha$,有:

\begin{equation}
\theta \mathbf{x}_1 + (1-\theta) \mathbf{x}_2 \in S_\alpha
\end{equation}

\textbf{步骤5}:根据子水平集的定义,$\theta \mathbf{x}_1 + (1-\theta) \mathbf{x}_2 \in S_\alpha$ 意味着:

\begin{equation}
f(\theta \mathbf{x}_1 + (1-\theta) \mathbf{x}_2) \leq \alpha = \max\{f(\mathbf{x}_1), f(\mathbf{x}_2)\}
\end{equation}

\textbf{步骤6}:这正是拟凸函数的定义!$\square$

\section{为什么不能选择其他 $\alpha$?}

\subsection{如果 $\alpha < \max\{f(\mathbf{x}_1), f(\mathbf{x}_2)\}$}

\textbf{问题}:假设 $\alpha < \max\{f(\mathbf{x}_1), f(\mathbf{x}_2)\}$。

\textbf{后果}:
\begin{itemize}
\item 设 $\max\{f(\mathbf{x}_1), f(\mathbf{x}_2)\} = f(\mathbf{x}_1)$(不失一般性)
\item 则 $f(\mathbf{x}_1) > \alpha$,因此 $\mathbf{x}_1 \notin S_\alpha$
\item 无法利用 $S_\alpha$ 的凸性(因为 $\mathbf{x}_1$ 不在 $S_\alpha$ 中)
\end{itemize}

\textbf{结论}:这个选择无效。

\subsection{如果 $\alpha > \max\{f(\mathbf{x}_1), f(\mathbf{x}_2)\}$}

\textbf{问题}:假设 $\alpha > \max\{f(\mathbf{x}_1), f(\mathbf{x}_2)\}$。

\textbf{分析}:
\begin{itemize}
\item $\mathbf{x}_1, \mathbf{x}_2 \in S_\alpha$ ✓(可以)
\item 由于 $S_\alpha$ 是凸集,$\theta \mathbf{x}_1 + (1-\theta) \mathbf{x}_2 \in S_\alpha$ ✓
\item 因此 $f(\theta \mathbf{x}_1 + (1-\theta) \mathbf{x}_2) \leq \alpha$ ✓
\end{itemize}

\textbf{问题}:
\begin{itemize}
\item 我们只能得到 $f(\theta \mathbf{x}_1 + (1-\theta) \mathbf{x}_2) \leq \alpha$
\item 但 $\alpha > \max\{f(\mathbf{x}_1), f(\mathbf{x}_2)\}$
\item 这比我们需要的结果弱(不够紧)
\item 无法直接得到 $f(\theta \mathbf{x}_1 + (1-\theta) \mathbf{x}_2) \leq \max\{f(\mathbf{x}_1), f(\mathbf{x}_2)\}$
\end{itemize}

\textbf{结论}:这个选择虽然可以,但不是最优的(不够紧)。

\subsection{最优选择:$\alpha = \max\{f(\mathbf{x}_1), f(\mathbf{x}_2)\}$}

\textbf{优势}:
\begin{enumerate}
\item \textbf{确保两点在 $S_\alpha$ 中}:$f(\mathbf{x}_1) \leq \alpha$ 且 $f(\mathbf{x}_2) \leq \alpha$
\item \textbf{直接得到目标}:$f(\theta \mathbf{x}_1 + (1-\theta) \mathbf{x}_2) \leq \alpha = \max\{f(\mathbf{x}_1), f(\mathbf{x}_2)\}$
\item \textbf{最紧的界}:$\alpha$ 正好是我们需要证明的上界
\end{enumerate}

\section{几何直观}

\subsection{子水平集的包含关系}

\textbf{关键观察}:子水平集有包含关系。

如果 $\alpha_1 < \alpha_2$,则 $S_{\alpha_1} \subseteq S_{\alpha_2}$。

\textbf{原因}:如果 $f(\mathbf{x}) \leq \alpha_1$,则 $f(\mathbf{x}) \leq \alpha_2$。

\subsection{选择 $\alpha$ 的策略}

\textbf{目标}:找到一个 $\alpha$,使得:
\begin{itemize}
\item $\mathbf{x}_1, \mathbf{x}_2 \in S_\alpha$(可以利用凸性)
\item $\alpha$ 尽可能小(得到最紧的界)
\end{itemize}

\textbf{最优选择}:$\alpha = \max\{f(\mathbf{x}_1), f(\mathbf{x}_2)\}$

\textbf{为什么最优?}
\begin{itemize}
\item 这是使得 $\mathbf{x}_1, \mathbf{x}_2 \in S_\alpha$ 的最小 $\alpha$ 值
\item 如果 $\alpha$ 更小,至少有一个点不在 $S_\alpha$ 中
\item 如果 $\alpha$ 更大,虽然可以,但得到的界不够紧
\end{itemize}

\section{具体例子}

\subsection{例子:$f(x) = x^2$,$x_1 = -2$,$x_2 = 3$}

\textbf{函数值}:
\begin{itemize}
\item $f(x_1) = f(-2) = 4$
\item $f(x_2) = f(3) = 9$
\item $\max\{f(x_1), f(x_2)\} = 9$
\end{itemize}

\textbf{选择 $\alpha = 9$}:

\textbf{子水平集}:$S_9 = \{x \mid x^2 \leq 9\} = [-3, 3]$

\textbf{验证}:
\begin{itemize}
\item $x_1 = -2 \in [-3, 3] = S_9$ ✓
\item $x_2 = 3 \in [-3, 3] = S_9$ ✓
\end{itemize}

\textbf{凸组合}:对于 $\theta = 0.5$,$x = 0.5 \times (-2) + 0.5 \times 3 = 0.5$

\textbf{验证}:
\begin{itemize}
\item $x = 0.5 \in [-3, 3] = S_9$ ✓
\item $f(0.5) = 0.25 \leq 9 = \max\{4, 9\}$ ✓
\end{itemize}

\textbf{如果选择 $\alpha = 10$}:

\textbf{子水平集}:$S_{10} = \{x \mid x^2 \leq 10\} = [-\sqrt{10}, \sqrt{10}]$

\textbf{验证}:
\begin{itemize}
\item $x_1, x_2 \in S_{10}$ ✓
\item $x = 0.5 \in S_{10}$ ✓
\item $f(0.5) = 0.25 \leq 10$
\end{itemize}

\textbf{问题}:虽然成立,但 $10 > 9$,得到的界不够紧。

\textbf{如果选择 $\alpha = 5$}:

\textbf{子水平集}:$S_5 = \{x \mid x^2 \leq 5\} = [-\sqrt{5}, \sqrt{5}]$

\textbf{验证}:
\begin{itemize}
\item $x_1 = -2 \in [-\sqrt{5}, \sqrt{5}]$?$-\sqrt{5} \approx -2.24$,$-2 > -2.24$,所以 $-2 \in S_5$ ✓
\item $x_2 = 3 \in [-\sqrt{5}, \sqrt{5}]$?$\sqrt{5} \approx 2.24$,$3 > 2.24$,所以 $3 \notin S_5$ ✗
\end{itemize}

\textbf{问题}:$x_2$ 不在 $S_5$ 中,无法利用 $S_5$ 的凸性。

\section{证明策略的总结}

\subsection{为什么这个选择是巧妙的?}

\begin{enumerate}
\item \textbf{最小充分条件}:
   \begin{itemize}
   \item $\alpha = \max\{f(\mathbf{x}_1), f(\mathbf{x}_2)\}$ 是使得 $\mathbf{x}_1, \mathbf{x}_2 \in S_\alpha$ 的最小 $\alpha$ 值
   \item 如果 $\alpha$ 更小,至少有一个点不在 $S_\alpha$ 中
   \end{itemize}

\item \textbf{直接得到目标}:
   \begin{itemize}
   \item $\alpha$ 正好是我们要证明的不等式右边
   \item 一旦证明了 $f(\theta \mathbf{x}_1 + (1-\theta) \mathbf{x}_2) \leq \alpha$,就完成了证明
   \end{itemize}

\item \textbf{完美匹配}:
   \begin{itemize}
   \item 既满足"两点在 $S_\alpha$ 中"的条件(可以利用凸性)
   \item 又满足"$\alpha$ 是目标上界"的条件(直接得到结论)
   \end{itemize}
\end{enumerate}

\subsection{证明技巧}

\textbf{一般策略}:
\begin{enumerate}
\item 确定目标:要证明什么?
\item 找到合适的参数:使得已知条件可以应用
\item 选择最优参数:既满足条件,又直接得到目标
\end{enumerate}

\textbf{在这个证明中}:
\begin{itemize}
\item 目标:$f(\theta \mathbf{x}_1 + (1-\theta) \mathbf{x}_2) \leq \max\{f(\mathbf{x}_1), f(\mathbf{x}_2)\}$
\item 已知条件:所有子水平集都是凸集
\item 策略:设 $\alpha = \max\{f(\mathbf{x}_1), f(\mathbf{x}_2)\}$,利用 $S_\alpha$ 的凸性
\end{itemize}

\section{总结}

\subsection{关键理解}

\begin{enumerate}
\item \textbf{为什么选择 $\alpha = \max\{f(\mathbf{x}_1), f(\mathbf{x}_2)\}$?}
   \begin{itemize}
   \item 这是使得 $\mathbf{x}_1, \mathbf{x}_2 \in S_\alpha$ 的最小 $\alpha$ 值
   \item 这样可以直接利用 $S_\alpha$ 的凸性
   \item 同时 $\alpha$ 正好是我们要证明的上界
   \end{itemize}

\item \textbf{证明逻辑}:
   \begin{itemize}
   \item 设 $\alpha = \max\{f(\mathbf{x}_1), f(\mathbf{x}_2)\}$
   \item 验证 $\mathbf{x}_1, \mathbf{x}_2 \in S_\alpha$
   \item 利用 $S_\alpha$ 的凸性,得到凸组合在 $S_\alpha$ 中
   \item 根据子水平集定义,得到 $f(\theta \mathbf{x}_1 + (1-\theta) \mathbf{x}_2) \leq \alpha$
   \item 完成证明
   \end{itemize}

\item \textbf{为什么这个选择是巧妙的?}
   \begin{itemize}
   \item 最小充分条件:最小的 $\alpha$ 使得两点都在 $S_\alpha$ 中
   \item 直接得到目标:$\alpha$ 正好是目标上界
   \item 完美匹配:同时满足两个条件
   \end{itemize}
\end{enumerate}

理解这个证明策略,有助于掌握类似的证明技巧!

\end{document}

