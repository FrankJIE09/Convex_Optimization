\documentclass[12pt,a4paper]{article}
\usepackage[UTF8]{ctex}
\usepackage{amsmath}
\usepackage{amssymb}
\usepackage{amsthm}
\usepackage{geometry}
\geometry{left=2.5cm,right=2.5cm,top=2.5cm,bottom=2.5cm}

\title{可微函数的凸性条件详解}
\author{}
\date{\today}

\begin{document}

\maketitle

\section{引言}

对于可微函数,凸性有更简洁的判定条件。一阶条件和二阶条件提供了判断函数凸性的实用方法,这些条件不仅有严格的数学证明,还有直观的几何意义。

\section{凸函数的基本定义回顾}

\subsection{定义}

\textbf{凸函数}:函数 $f: \mathbb{R}^n \to \mathbb{R}$ 是凸函数,如果:

\begin{enumerate}
\item 定义域 $\text{dom } f$ 是凸集

\item 对于任意 $\mathbf{x}_1, \mathbf{x}_2 \in \text{dom } f$ 和 $\theta \in [0, 1]$,有:
   \begin{equation}
   f(\theta \mathbf{x}_1 + (1-\theta) \mathbf{x}_2) \leq \theta f(\mathbf{x}_1) + (1-\theta) f(\mathbf{x}_2)
   \end{equation}
\end{enumerate}

\subsection{一维情况的简化}

对于一元函数 $f: \mathbb{R} \to \mathbb{R}$,凸性条件为:

对于任意 $x_1, x_2 \in \text{dom } f$ 和 $\theta \in [0, 1]$,有:

\begin{equation}
f(\theta x_1 + (1-\theta) x_2) \leq \theta f(x_1) + (1-\theta) f(x_2)
\end{equation}

\section{一阶条件(First-Order Condition)}

\subsection{定理陈述}

\textbf{一阶条件}:如果函数 $f: \mathbb{R}^n \to \mathbb{R}$ 是可微的,则 $f$ 是凸函数,当且仅当对于任意 $\mathbf{x}, \mathbf{y} \in \text{dom } f$,有:

\begin{equation}
f(\mathbf{y}) \geq f(\mathbf{x}) + \nabla f(\mathbf{x})^T (\mathbf{y} - \mathbf{x})
\end{equation}

\textbf{一维情况}:对于一元函数 $f: \mathbb{R} \to \mathbb{R}$,条件简化为:

\begin{equation}
f(y) \geq f(x) + f'(x)(y - x)
\end{equation}

\subsection{几何意义}

\textbf{核心思想}:函数图像在任意点处的切线都在函数图像下方(或重合)。

\textbf{详细解释}:
\begin{itemize}
\item 在函数图像上任意一点 $(x, f(x))$ 处作切线
\item 切线的方程是:$y = f(x) + f'(x)(t - x)$
\item 一阶条件要求:对于所有 $y$,有 $f(y) \geq f(x) + f'(x)(y - x)$
\item 这意味着函数图像在切线的上方(或重合)
\item 换句话说:切线是函数图像的"支撑线"(supporting line)
\end{itemize}

\textbf{直观理解}:
\begin{itemize}
\item 凸函数的图像是"向下凸"的(concave upward)
\item 任意点的切线都在函数图像下方
\item 函数图像"向上弯曲"
\end{itemize}

\subsection{证明思路}

\textbf{必要性}(凸函数 $\Rightarrow$ 一阶条件):

如果 $f$ 是凸函数,则对于任意 $x, y$ 和 $\theta \in (0, 1]$,有:

\begin{equation}
f(x + \theta(y - x)) \leq (1-\theta) f(x) + \theta f(y)
\end{equation}

整理得:

\begin{equation}
\frac{f(x + \theta(y - x)) - f(x)}{\theta} \leq f(y) - f(x)
\end{equation}

令 $\theta \to 0$,左边趋于 $f'(x)(y - x)$,因此:

\begin{equation}
f'(x)(y - x) \leq f(y) - f(x)
\end{equation}

即:$f(y) \geq f(x) + f'(x)(y - x)$。

\textbf{充分性}(一阶条件 $\Rightarrow$ 凸函数):

如果一阶条件成立,可以证明对于任意 $x_1, x_2$ 和 $\theta \in [0, 1]$,有:

\begin{equation}
f(\theta x_1 + (1-\theta) x_2) \leq \theta f(x_1) + (1-\theta) f(x_2)
\end{equation}

证明需要使用中值定理和条件。

\subsection{具体例子}

\textbf{例子1:$f(x) = x^2$}

\textbf{验证一阶条件}:

$f'(x) = 2x$,一阶条件要求:

\begin{equation}
f(y) \geq f(x) + f'(x)(y - x)
\end{equation}

即:$y^2 \geq x^2 + 2x(y - x) = 2xy - x^2$

整理得:$y^2 - 2xy + x^2 = (y - x)^2 \geq 0$ ✓

\textbf{几何验证}:
\begin{itemize}
\item 在点 $(x, x^2)$ 处的切线:$y = x^2 + 2x(t - x) = 2xt - x^2$
\item 函数图像 $y = t^2$ 在切线上方
\item 例如:在 $x = 1$ 处,切线是 $y = 2t - 1$,函数 $y = t^2$ 在 $t = 2$ 处,$4 \geq 3$ ✓
\end{itemize}

\textbf{例子2:$f(x) = e^x$}

\textbf{验证一阶条件}:

$f'(x) = e^x$,一阶条件要求:

\begin{equation}
e^y \geq e^x + e^x(y - x) = e^x(1 + y - x)
\end{equation}

这等价于:$e^{y-x} \geq 1 + (y - x)$

设 $t = y - x$,需要证明:$e^t \geq 1 + t$ 对所有 $t$ 成立。

这可以通过 $e^t$ 的泰勒展开或函数 $g(t) = e^t - 1 - t$ 的最小值来证明。

\textbf{例子3:$f(x) = -x^2$(凹函数)}

\textbf{验证一阶条件}:

$f'(x) = -2x$,一阶条件要求:

\begin{equation}
-y^2 \geq -x^2 + (-2x)(y - x) = -x^2 - 2x(y - x) = -2xy + x^2
\end{equation}

整理得:$-y^2 + 2xy - x^2 = -(y - x)^2 \geq 0$

但 $(y - x)^2 \geq 0$,所以 $-(y - x)^2 \leq 0$,等号仅在 $y = x$ 时成立。

\textbf{结论}:一阶条件不满足,因此 $f(x) = -x^2$ 不是凸函数(是凹函数)。

\section{二阶条件(Second-Order Condition)}

\subsection{定理陈述}

\textbf{二阶条件}:如果函数 $f: \mathbb{R}^n \to \mathbb{R}$ 是二阶可微的,则 $f$ 是凸函数,当且仅当:

\begin{equation}
\nabla^2 f(\mathbf{x}) \succeq 0, \quad \forall \mathbf{x} \in \text{dom } f
\end{equation}

即:Hessian矩阵在所有点处都是半正定的。

\textbf{一维情况}:对于一元函数 $f: \mathbb{R} \to \mathbb{R}$,条件简化为:

\begin{equation}
f''(x) \geq 0, \quad \forall x \in \text{dom } f
\end{equation}

\subsection{几何意义}

\textbf{核心思想}:函数的二阶导数非负,即函数"向上弯曲"。

\textbf{详细解释}:
\begin{itemize}
\item 二阶导数 $f''(x)$ 表示函数的"曲率"
\item $f''(x) > 0$:函数在该点"向上弯曲"(concave upward)
\item $f''(x) < 0$:函数在该点"向下弯曲"(concave downward)
\item $f''(x) = 0$:函数在该点可能是拐点
\item 凸函数要求 $f''(x) \geq 0$ 对所有 $x$ 成立
\end{itemize}

\textbf{直观理解}:
\begin{itemize}
\item 凸函数的图像是"向上弯曲"的
\item 任意点的曲率非负
\item 函数"加速向上"或"匀速"
\end{itemize}

\subsection{证明思路}

\textbf{必要性}(凸函数 $\Rightarrow$ 二阶条件):

如果 $f$ 是凸函数,则一阶条件成立。对一阶条件关于 $y$ 求二阶导数,可以得到 $f''(x) \geq 0$。

\textbf{充分性}(二阶条件 $\Rightarrow$ 凸函数):

如果 $f''(x) \geq 0$ 对所有 $x$ 成立,可以使用泰勒展开证明一阶条件,从而证明凸性。

\textbf{泰勒展开方法}:

对于任意 $x, y$,存在 $\xi$ 在 $x$ 和 $y$ 之间,使得:

\begin{equation}
f(y) = f(x) + f'(x)(y - x) + \frac{1}{2} f''(\xi)(y - x)^2
\end{equation}

如果 $f''(\xi) \geq 0$,则:

\begin{equation}
f(y) \geq f(x) + f'(x)(y - x)
\end{equation}

这是一阶条件,因此 $f$ 是凸函数。

\subsection{具体例子}

\textbf{例子1:$f(x) = x^2$}

\textbf{计算二阶导数}:

$f'(x) = 2x$,$f''(x) = 2 > 0$ 对所有 $x$ 成立。

\textbf{结论}:$f(x) = x^2$ 是凸函数。✓

\textbf{例子2:$f(x) = e^x$}

\textbf{计算二阶导数}:

$f'(x) = e^x$,$f''(x) = e^x > 0$ 对所有 $x$ 成立。

\textbf{结论}:$f(x) = e^x$ 是凸函数。✓

\textbf{例子3:$f(x) = \log x$($x > 0$)}

\textbf{计算二阶导数}:

$f'(x) = \frac{1}{x}$,$f''(x) = -\frac{1}{x^2} < 0$ 对所有 $x > 0$ 成立。

\textbf{结论}:$f(x) = \log x$ 不是凸函数(是凹函数)。✗

\textbf{例子4:$f(x) = x^3$}

\textbf{计算二阶导数}:

$f'(x) = 3x^2$,$f''(x) = 6x$。

\textbf{分析}:
\begin{itemize}
\item 当 $x > 0$ 时,$f''(x) > 0$(凸的)
\item 当 $x < 0$ 时,$f''(x) < 0$(凹的)
\item 当 $x = 0$ 时,$f''(x) = 0$(拐点)
\end{itemize}

\textbf{结论}:$f(x) = x^3$ 不是凸函数(在 $x < 0$ 时是凹的)。

\textbf{例子5:$f(x) = |x|$}

\textbf{问题}:$f(x) = |x|$ 在 $x = 0$ 处不可微,因此不能直接使用二阶条件。

\textbf{但可以验证}:
\begin{itemize}
\item 当 $x > 0$ 时,$f(x) = x$,$f''(x) = 0 \geq 0$
\item 当 $x < 0$ 时,$f(x) = -x$,$f''(x) = 0 \geq 0$
\item 在 $x = 0$ 处,虽然不可微,但可以通过定义证明是凸函数
\end{itemize}

\section{多元函数的情况}

\subsection{一阶条件(多元)}

\textbf{定理}:如果函数 $f: \mathbb{R}^n \to \mathbb{R}$ 是可微的,则 $f$ 是凸函数,当且仅当对于任意 $\mathbf{x}, \mathbf{y} \in \text{dom } f$,有:

\begin{equation}
f(\mathbf{y}) \geq f(\mathbf{x}) + \nabla f(\mathbf{x})^T (\mathbf{y} - \mathbf{x})
\end{equation}

\textbf{几何意义}:
\begin{itemize}
\item 在点 $\mathbf{x}$ 处的切超平面(tangent hyperplane)
\item 切超平面的方程:$z = f(\mathbf{x}) + \nabla f(\mathbf{x})^T (\mathbf{y} - \mathbf{x})$
\item 一阶条件要求:函数图像在切超平面的上方(或重合)
\end{itemize}

\subsection{二阶条件(多元)}

\textbf{定理}:如果函数 $f: \mathbb{R}^n \to \mathbb{R}$ 是二阶可微的,则 $f$ 是凸函数,当且仅当:

\begin{equation}
\nabla^2 f(\mathbf{x}) \succeq 0, \quad \forall \mathbf{x} \in \text{dom } f
\end{equation}

即:Hessian矩阵 $\nabla^2 f(\mathbf{x})$ 在所有点处都是半正定的。

\textbf{Hessian矩阵}:

\begin{equation}
\nabla^2 f(\mathbf{x}) = \begin{pmatrix}
\frac{\partial^2 f}{\partial x_1^2} & \frac{\partial^2 f}{\partial x_1 \partial x_2} & \cdots \\
\frac{\partial^2 f}{\partial x_2 \partial x_1} & \frac{\partial^2 f}{\partial x_2^2} & \cdots \\
\vdots & \vdots & \ddots
\end{pmatrix}
\end{equation}

\textbf{半正定矩阵}:矩阵 $\mathbf{A}$ 是半正定的,如果对于任意 $\mathbf{v} \neq \mathbf{0}$,有 $\mathbf{v}^T \mathbf{A} \mathbf{v} \geq 0$。

\subsection{多元函数的例子}

\textbf{例子1:$f(\mathbf{x}) = \|\mathbf{x}\|_2^2 = \mathbf{x}^T \mathbf{x}$}

\textbf{计算Hessian}:

\begin{align}
\nabla f(\mathbf{x}) &= 2\mathbf{x} \\
\nabla^2 f(\mathbf{x}) &= 2\mathbf{I} \succ 0
\end{align}

\textbf{结论}:$f(\mathbf{x}) = \|\mathbf{x}\|_2^2$ 是凸函数(严格凸)。✓

\textbf{例子2:$f(\mathbf{x}) = \mathbf{x}^T \mathbf{P} \mathbf{x}$,其中 $\mathbf{P} \succeq 0$}

\textbf{计算Hessian}:

\begin{align}
\nabla f(\mathbf{x}) &= 2\mathbf{P}\mathbf{x} \\
\nabla^2 f(\mathbf{x}) &= 2\mathbf{P} \succeq 0
\end{align}

\textbf{结论}:如果 $\mathbf{P} \succeq 0$,则 $f$ 是凸函数。✓

\textbf{例子3:$f(\mathbf{x}) = \log(\sum_{i=1}^n e^{x_i})$}

\textbf{计算Hessian}:

可以证明 $\nabla^2 f(\mathbf{x}) \succeq 0$。

\textbf{结论}:$f$ 是凸函数。✓

\section{一阶条件与二阶条件的关系}

\subsection{等价性}

\textbf{定理}:对于可微函数,以下条件等价:

\begin{enumerate}
\item $f$ 是凸函数(基本定义)
\item 一阶条件成立:$f(\mathbf{y}) \geq f(\mathbf{x}) + \nabla f(\mathbf{x})^T (\mathbf{y} - \mathbf{x})$
\item (如果二阶可微)二阶条件成立:$\nabla^2 f(\mathbf{x}) \succeq 0$
\end{enumerate}

\subsection{证明思路}

\begin{itemize}
\item 基本定义 $\Leftrightarrow$ 一阶条件:通过极限和定义证明
\item 一阶条件 $\Leftrightarrow$ 二阶条件:通过泰勒展开证明
\end{itemize}

\subsection{使用建议}

\begin{enumerate}
\item \textbf{一阶条件}:
   \begin{itemize}
   \item 适用于可微函数
   \item 几何直观强
   \item 适合理论分析
   \end{itemize}

\item \textbf{二阶条件}:
   \begin{itemize}
   \item 适用于二阶可微函数
   \item 计算方便(只需计算Hessian)
   \item 适合实际判断
   \end{itemize}
</enumerate>

\section{严格凸函数}

\subsection{定义}

\textbf{严格凸函数}:函数 $f$ 是严格凸的,如果对于任意 $\mathbf{x}_1 \neq \mathbf{x}_2$ 和 $\theta \in (0, 1)$,有:

\begin{equation}
f(\theta \mathbf{x}_1 + (1-\theta) \mathbf{x}_2) < \theta f(\mathbf{x}_1) + (1-\theta) f(\mathbf{x}_2)
\end{equation}

\subsection{可微函数的条件}

\textbf{一阶条件}:如果 $f$ 是可微的严格凸函数,则:

\begin{equation}
f(\mathbf{y}) > f(\mathbf{x}) + \nabla f(\mathbf{x})^T (\mathbf{y} - \mathbf{x}), \quad \forall \mathbf{x} \neq \mathbf{y}
\end{equation}

\textbf{二阶条件}:如果 $f$ 是二阶可微的严格凸函数,则:

\begin{equation}
\nabla^2 f(\mathbf{x}) \succ 0, \quad \forall \mathbf{x} \in \text{dom } f
\end{equation}

即:Hessian矩阵是正定的(不仅仅是半正定)。

\section{总结}

\subsection{一阶条件}

\begin{enumerate}
\item \textbf{条件}:$f(\mathbf{y}) \geq f(\mathbf{x}) + \nabla f(\mathbf{x})^T (\mathbf{y} - \mathbf{x})$
\item \textbf{几何意义}:函数图像在任意点处的切线(或切超平面)都在函数图像下方
\item \textbf{适用}:可微函数
\item \textbf{优势}:几何直观强,适合理论分析
</enumerate>

\subsection{二阶条件}

\begin{enumerate}
\item \textbf{条件}:$\nabla^2 f(\mathbf{x}) \succeq 0$(或 $f''(x) \geq 0$ 在一维情况)
\item \textbf{几何意义}:函数的二阶导数非负,函数"向上弯曲"
\item \textbf{适用}:二阶可微函数
\item \textbf{优势}:计算方便,适合实际判断
\end{enumerate}

\subsection{关系}

\begin{enumerate}
\item 基本定义 $\Leftrightarrow$ 一阶条件 $\Leftrightarrow$ 二阶条件(对于可微函数)
\item 一阶条件更直观,二阶条件更实用
\item 两者都提供了判断凸性的有效方法
</enumerate>

理解这些条件,是判断函数凸性的关键工具!

\end{document}

