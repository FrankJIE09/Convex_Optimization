\documentclass[12pt,a4paper]{article}
\usepackage[UTF8]{ctex}
\usepackage{amsmath}
\usepackage{amssymb}
\usepackage{amsthm}
\usepackage{geometry}
\geometry{left=2.5cm,right=2.5cm,top=2.5cm,bottom=2.5cm}

\title{为什么二次函数的子水平集是椭球?}
\author{}
\date{\today}

\begin{document}

\maketitle

\section{问题提出}

\textbf{问题}:对于二次函数 $f(\mathbf{x}) = \mathbf{x}^T \mathbf{P} \mathbf{x} + \mathbf{q}^T \mathbf{x} + r$,其子水平集:

\begin{equation}
S_\alpha = \{\mathbf{x} \mid \mathbf{x}^T \mathbf{P} \mathbf{x} + \mathbf{q}^T \mathbf{x} + r \leq \alpha\}
</equation}

\textbf{为什么}:
\begin{itemize}
\item 如果 $\mathbf{P} \succ 0$:子水平集是椭球(ellipsoid)
\item 如果 $\mathbf{P} \succeq 0$:子水平集是椭球或椭圆柱(ellipsoid or elliptic cylinder)
\end{itemize}

\section{椭球的定义}

\subsection{标准椭球}

\textbf{椭球}:在 $\mathbb{R}^n$ 中,以原点为中心的标准椭球定义为:

\begin{equation}
\mathcal{E} = \left\{\mathbf{x} \in \mathbb{R}^n \mid \sum_{i=1}^n \frac{x_i^2}{a_i^2} \leq 1\right\}
</equation>

其中 $a_i > 0$ 是椭球在各坐标轴方向的半轴长度。

\subsection{一般椭球}

\textbf{一般椭球}:通过仿射变换,一般椭球可以表示为:

\begin{equation}
\mathcal{E} = \{\mathbf{x} \in \mathbb{R}^n \mid (\mathbf{x} - \mathbf{x}_c)^T \mathbf{P}^{-1} (\mathbf{x} - \mathbf{x}_c) \leq 1\}
</equation>

其中:
\begin{itemize}
\item $\mathbf{x}_c$ 是椭球的中心
\item $\mathbf{P} \succ 0$ 是正定矩阵,决定了椭球的形状和方向
\end{itemize}

\subsection{等价形式}

\textbf{等价形式}:椭球也可以表示为:

\begin{equation}
\mathcal{E} = \{\mathbf{x} \in \mathbb{R}^n \mid \mathbf{x}^T \mathbf{A} \mathbf{x} + \mathbf{b}^T \mathbf{x} + c \leq 0\}
</equation>

其中 $\mathbf{A} \succ 0$。

\section{二次函数的子水平集}

\subsection{一般形式}

\textbf{二次函数}:$f(\mathbf{x}) = \mathbf{x}^T \mathbf{P} \mathbf{x} + \mathbf{q}^T \mathbf{x} + r$

\textbf{子水平集}:$S_\alpha = \{\mathbf{x} \mid \mathbf{x}^T \mathbf{P} \mathbf{x} + \mathbf{q}^T \mathbf{x} + r \leq \alpha\}$

\textbf{等价形式}:

\begin{equation}
S_\alpha = \{\mathbf{x} \mid \mathbf{x}^T \mathbf{P} \mathbf{x} + \mathbf{q}^T \mathbf{x} + (r - \alpha) \leq 0\}
</equation>

\section{情况1:$\mathbf{P} \succ 0$(正定)}

\subsection{配方法}

\textbf{目标}:将二次型配成完全平方形式。

\textbf{步骤1}:完成平方

\begin{align}
\mathbf{x}^T \mathbf{P} \mathbf{x} + \mathbf{q}^T \mathbf{x} + (r - \alpha) &= \mathbf{x}^T \mathbf{P} \mathbf{x} + \mathbf{q}^T \mathbf{x} + (r - \alpha) \\
&= (\mathbf{x} + \frac{1}{2}\mathbf{P}^{-1}\mathbf{q})^T \mathbf{P} (\mathbf{x} + \frac{1}{2}\mathbf{P}^{-1}\mathbf{q}) - \frac{1}{4}\mathbf{q}^T \mathbf{P}^{-1}\mathbf{q} + (r - \alpha)
\end{align}

\textbf{详细推导}:

\begin{align}
&(\mathbf{x} + \mathbf{P}^{-1}\mathbf{q}/2)^T \mathbf{P} (\mathbf{x} + \mathbf{P}^{-1}\mathbf{q}/2) \\
&= \mathbf{x}^T \mathbf{P} \mathbf{x} + \mathbf{x}^T \mathbf{P} \mathbf{P}^{-1}\mathbf{q} + (\mathbf{P}^{-1}\mathbf{q}/2)^T \mathbf{P} \mathbf{x} + (\mathbf{P}^{-1}\mathbf{q}/2)^T \mathbf{P} \mathbf{P}^{-1}\mathbf{q}/2 \\
&= \mathbf{x}^T \mathbf{P} \mathbf{x} + \mathbf{x}^T \mathbf{q} + \mathbf{q}^T \mathbf{x}/2 + \mathbf{q}^T \mathbf{P}^{-1}\mathbf{q}/4 \\
&= \mathbf{x}^T \mathbf{P} \mathbf{x} + \mathbf{q}^T \mathbf{x} + \mathbf{q}^T \mathbf{P}^{-1}\mathbf{q}/4
\end{align}

因此:

\begin{align}
\mathbf{x}^T \mathbf{P} \mathbf{x} + \mathbf{q}^T \mathbf{x} &= (\mathbf{x} + \mathbf{P}^{-1}\mathbf{q}/2)^T \mathbf{P} (\mathbf{x} + \mathbf{P}^{-1}\mathbf{q}/2) - \mathbf{q}^T \mathbf{P}^{-1}\mathbf{q}/4
\end{align}

\textbf{步骤2}:代入子水平集

\begin{align}
\mathbf{x}^T \mathbf{P} \mathbf{x} + \mathbf{q}^T \mathbf{x} + (r - \alpha) &\leq 0 \\
(\mathbf{x} + \mathbf{P}^{-1}\mathbf{q}/2)^T \mathbf{P} (\mathbf{x} + \mathbf{P}^{-1}\mathbf{q}/2) - \mathbf{q}^T \mathbf{P}^{-1}\mathbf{q}/4 + (r - \alpha) &\leq 0 \\
(\mathbf{x} + \mathbf{P}^{-1}\mathbf{q}/2)^T \mathbf{P} (\mathbf{x} + \mathbf{P}^{-1}\mathbf{q}/2) &\leq \mathbf{q}^T \mathbf{P}^{-1}\mathbf{q}/4 - (r - \alpha)
\end{align}

\textbf{步骤3}:标准化

设 $\mathbf{x}_c = -\mathbf{P}^{-1}\mathbf{q}/2$(椭球中心),$\beta = \mathbf{q}^T \mathbf{P}^{-1}\mathbf{q}/4 - (r - \alpha)$。

如果 $\beta > 0$,则:

\begin{equation}
(\mathbf{x} - \mathbf{x}_c)^T \mathbf{P} (\mathbf{x} - \mathbf{x}_c) \leq \beta
</equation}

进一步标准化:

\begin{equation}
(\mathbf{x} - \mathbf{x}_c)^T \frac{\mathbf{P}}{\beta} (\mathbf{x} - \mathbf{x}_c) \leq 1
</equation}

\textbf{步骤4}:椭球的标准形式

由于 $\mathbf{P} \succ 0$,$\mathbf{P}/\beta \succ 0$,可以写成:

\begin{equation}
(\mathbf{x} - \mathbf{x}_c)^T \mathbf{P}^{-1} (\mathbf{x} - \mathbf{x}_c) \leq 1
</equation}

其中 $\mathbf{P}$ 被重新定义为 $\beta \mathbf{P}$(仍然是正定的)。

\textbf{结论}:这是椭球的标准形式!$\square$

\subsection{几何意义}

\textbf{椭球的特征}:
\begin{itemize}
\item \textbf{中心}:$\mathbf{x}_c = -\mathbf{P}^{-1}\mathbf{q}/2$
\item \textbf{形状}:由矩阵 $\mathbf{P}$ 的特征值和特征向量决定
\item \textbf{大小}:由 $\beta$ 决定
\item \textbf{方向}:由 $\mathbf{P}$ 的特征向量决定
\end{itemize}

\textbf{为什么是椭球?}
\begin{itemize}
\item 二次型 $\mathbf{x}^T \mathbf{P} \mathbf{x}$ 在 $\mathbf{P} \succ 0$ 时定义了一个椭球
\item 通过平移($\mathbf{x} - \mathbf{x}_c$)和缩放,得到标准椭球
\item 椭球是"拉伸的球"
\end{itemize}

\section{情况2:$\mathbf{P} \succeq 0$(半正定)}

\subsection{特征值分解}

\textbf{设}:$\mathbf{P}$ 的特征值分解为 $\mathbf{P} = \mathbf{Q} \boldsymbol{\Lambda} \mathbf{Q}^T$,其中:
\begin{itemize}
\item $\mathbf{Q}$ 是正交矩阵
\item $\boldsymbol{\Lambda} = \text{diag}(\lambda_1, \ldots, \lambda_n)$,$\lambda_i \geq 0$
\end{itemize}

\textbf{情况2a}:如果所有特征值 $\lambda_i > 0$(即 $\mathbf{P} \succ 0$),则子水平集是椭球(见情况1)。

\textbf{情况2b}:如果某些特征值为 0。

\textbf{设}:$\lambda_1, \ldots, \lambda_k > 0$,$\lambda_{k+1} = \ldots = \lambda_n = 0$($k < n$)。

\subsection{坐标变换}

\textbf{设}:$\mathbf{y} = \mathbf{Q}^T \mathbf{x}$,则:

\begin{align}
\mathbf{x}^T \mathbf{P} \mathbf{x} &= \mathbf{x}^T \mathbf{Q} \boldsymbol{\Lambda} \mathbf{Q}^T \mathbf{x} \\
&= \mathbf{y}^T \boldsymbol{\Lambda} \mathbf{y} \\
&= \sum_{i=1}^k \lambda_i y_i^2
\end{align}

\textbf{子水平集}:

\begin{equation}
S_\alpha = \left\{\mathbf{y} \mid \sum_{i=1}^k \lambda_i y_i^2 + \tilde{\mathbf{q}}^T \mathbf{y} + (r - \alpha) \leq 0\right\}
</equation>

其中 $\tilde{\mathbf{q}} = \mathbf{Q}^T \mathbf{q}$。

\subsection{椭圆柱的情况}

\textbf{关键观察}:在 $y_{k+1}, \ldots, y_n$ 方向上没有约束(因为 $\lambda_{k+1} = \ldots = \lambda_n = 0$)。

\textbf{结果}:
\begin{itemize}
\item 在 $y_1, \ldots, y_k$ 方向上:形成椭球(或椭圆,如果 $k = 2$)
\item 在 $y_{k+1}, \ldots, y_n$ 方向上:无约束(可以延伸到无穷)
\item 整体形状:椭圆柱(elliptic cylinder)
\end{itemize}

\textbf{具体例子}:

在 $\mathbb{R}^3$ 中,如果 $\lambda_1, \lambda_2 > 0$,$\lambda_3 = 0$:

\begin{equation}
S_\alpha = \left\{(y_1, y_2, y_3) \mid \lambda_1 y_1^2 + \lambda_2 y_2^2 + \text{线性项} \leq 0\right\}
</equation}

在 $y_1, y_2$ 平面上是椭圆,在 $y_3$ 方向上无约束,形成椭圆柱。

\section{具体例子}

\subsection{例子1:$\mathbf{P} \succ 0$,椭球}

\textbf{函数}:$f(x, y) = x^2 + 2y^2$

\textbf{矩阵}:$\mathbf{P} = \begin{pmatrix} 1 & 0 \\ 0 & 2 \end{pmatrix} \succ 0$

\textbf{子水平集}:$S_\alpha = \{(x, y) \mid x^2 + 2y^2 \leq \alpha\}$

\textbf{分析}:
\begin{itemize}
\item 如果 $\alpha < 0$:$S_\alpha = \emptyset$
\item 如果 $\alpha = 0$:$S_0 = \{(0, 0)\}$(单点)
\item 如果 $\alpha > 0$:$S_\alpha = \left\{(x, y) \mid \frac{x^2}{\alpha} + \frac{y^2}{\alpha/2} \leq 1\right\}$(椭圆)
\end{itemize}

\textbf{几何意义}:
\begin{itemize}
\item 这是以原点为中心的椭圆
\item 长半轴:$\sqrt{\alpha}$($x$ 方向)
\item 短半轴:$\sqrt{\alpha/2}$($y$ 方向)
\end{itemize}

\subsection{例子2:$\mathbf{P} \succeq 0$,椭圆柱}

\textbf{函数}:$f(x, y, z) = x^2 + 2y^2$(不依赖于 $z$)

\textbf{矩阵}:$\mathbf{P} = \begin{pmatrix} 1 & 0 & 0 \\ 0 & 2 & 0 \\ 0 & 0 & 0 \end{pmatrix} \succeq 0$

\textbf{子水平集}:$S_\alpha = \{(x, y, z) \mid x^2 + 2y^2 \leq \alpha\}$

\textbf{分析}:
\begin{itemize}
\item 在 $x, y$ 平面上:椭圆 $\{(x, y) \mid x^2 + 2y^2 \leq \alpha\}$
\item 在 $z$ 方向上:无约束($z$ 可以取任意值)
\item 整体形状:椭圆柱(椭圆沿 $z$ 轴延伸)
\end{itemize}

\textbf{几何意义}:
\begin{itemize}
\item 这是以 $z$ 轴为轴的椭圆柱
\item 横截面是椭圆
\item 在 $z$ 方向上无限延伸
\end{itemize}

\subsection{例子3:一般二次函数}

\textbf{函数}:$f(x, y) = x^2 + 2xy + 2y^2 + x + y + 1$

\textbf{矩阵}:$\mathbf{P} = \begin{pmatrix} 1 & 1 \\ 1 & 2 \end{pmatrix}$

\textbf{验证}:$\det(\mathbf{P}) = 2 - 1 = 1 > 0$,且 $\text{tr}(\mathbf{P}) = 3 > 0$,因此 $\mathbf{P} \succ 0$。

\textbf{配方法}:

\begin{align}
f(x, y) &= x^2 + 2xy + 2y^2 + x + y + 1 \\
&= (x + y)^2 + y^2 + x + y + 1
\end{align}

通过配方法,可以写成:

\begin{equation}
f(x, y) = (\mathbf{x} - \mathbf{x}_c)^T \mathbf{P} (\mathbf{x} - \mathbf{x}_c) + c
</equation}

其中 $\mathbf{x}_c$ 是中心,$c$ 是常数。

\textbf{子水平集}:椭球(椭圆,因为 $n = 2$)。

\section{为什么是椭球?几何直观}

\subsection{二次型的几何意义}

\textbf{二次型}:$\mathbf{x}^T \mathbf{P} \mathbf{x}$(当 $\mathbf{P} \succ 0$ 时)

\textbf{几何意义}:
\begin{itemize}
\item 这是"拉伸的球"
\item 通过特征值分解,可以找到主轴方向
\item 在不同方向上有不同的"拉伸"程度
\end{itemize}

\textbf{标准形式}:

通过坐标变换,可以写成:

\begin{equation}
\sum_{i=1}^n \frac{y_i^2}{a_i^2} \leq 1
</equation}

这是椭球的标准形式。

\subsection{从球到椭球}

\textbf{单位球}:$\{\mathbf{x} \mid \|\mathbf{x}\|_2^2 \leq 1\} = \{\mathbf{x} \mid \mathbf{x}^T \mathbf{I} \mathbf{x} \leq 1\}$

\textbf{椭球}:$\{\mathbf{x} \mid \mathbf{x}^T \mathbf{P} \mathbf{x} \leq 1\}$,其中 $\mathbf{P} \succ 0$

\textbf{关系}:
\begin{itemize}
\item 椭球是单位球经过线性变换得到的
\item 变换矩阵是 $\mathbf{P}^{1/2}$($\mathbf{P}$ 的平方根)
\item 不同方向有不同的拉伸
\end{itemize}

\section{总结}

\subsection{关键结论}

\begin{enumerate}
\item \textbf{$\mathbf{P} \succ 0$}:
   \begin{itemize}
   \item 子水平集是椭球
   \item 通过配方法可以化成标准椭球形式
   \item 中心:$\mathbf{x}_c = -\mathbf{P}^{-1}\mathbf{q}/2$
   \end{itemize}

\item \textbf{$\mathbf{P} \succeq 0$}:
   \begin{itemize}
   \item 如果所有特征值 $> 0$:椭球
   \item 如果某些特征值 $= 0$:椭圆柱
   \item 在零特征值对应的方向上无约束
   \end{itemize}
</enumerate>

\subsection{证明思路}

\begin{enumerate}
\item \textbf{配方法}:将二次型配成完全平方形式

\item \textbf{坐标变换}:通过特征值分解找到主轴方向

\item \textbf{标准化}:化成椭球的标准形式

\item \textbf{几何理解}:椭球是"拉伸的球"
</enumerate>

\subsection{记忆技巧}

\begin{enumerate}
\item \textbf{正定矩阵}:所有方向都有约束 → 椭球

\item \textbf{半正定矩阵}:某些方向无约束 → 椭圆柱

\item \textbf{配方法}:将一般二次型化成标准形式

\item \textbf{几何直观}:椭球 = 拉伸的球
</enumerate>

理解为什么二次函数的子水平集是椭球,有助于理解二次规划和椭球优化问题!

\end{document}

