\documentclass[12pt,a4paper]{article}
\usepackage[UTF8]{ctex}
\usepackage{amsmath}
\usepackage{amssymb}
\usepackage{amsthm}
\usepackage{geometry}
\geometry{left=2.5cm,right=2.5cm,top=2.5cm,bottom=2.5cm}

\title{拟凸函数(Quasiconvex Function)详解}
\author{}
\date{\today}

\begin{document}

\maketitle

\section{引言}

拟凸函数(Quasiconvex Function)是凸函数概念的推广,比凸函数更一般,但仍然保持一些良好的性质。理解拟凸函数对于理解更广泛的优化问题非常重要。

\section{拟凸函数的定义}

\subsection{标准定义}

\textbf{拟凸函数}:函数 $f: \mathbb{R}^n \to \mathbb{R}$ 是拟凸的,如果其定义域是凸集,且对于任意 $\mathbf{x}_1, \mathbf{x}_2 \in \text{dom } f$ 和任意 $\theta \in [0, 1]$,有:

\begin{equation}
f(\theta \mathbf{x}_1 + (1-\theta) \mathbf{x}_2) \leq \max\{f(\mathbf{x}_1), f(\mathbf{x}_2)\}
\end{equation}

\textbf{等价定义}:函数 $f$ 是拟凸的,如果其所有子水平集(sublevel set)都是凸集。

\textbf{子水平集}:
\begin{equation}
S_\alpha = \{\mathbf{x} \in \text{dom } f \mid f(\mathbf{x}) \leq \alpha\}
\end{equation}

\subsection{拟凹函数}

\textbf{拟凹函数}:函数 $f$ 是拟凹的,如果 $-f$ 是拟凸的。

等价地,$f$ 是拟凹的,如果:
\begin{equation}
f(\theta \mathbf{x}_1 + (1-\theta) \mathbf{x}_2) \geq \min\{f(\mathbf{x}_1), f(\mathbf{x}_2)\}
\end{equation}

\textbf{拟线性函数}:既是拟凸又是拟凹的函数。

\section{与凸函数的关系}

\subsection{凸函数 $\Rightarrow$ 拟凸函数}

\textbf{定理}:如果 $f$ 是凸函数,则 $f$ 是拟凸函数。

\textbf{证明}:

如果 $f$ 是凸函数,则:
\begin{equation}
f(\theta \mathbf{x}_1 + (1-\theta) \mathbf{x}_2) \leq \theta f(\mathbf{x}_1) + (1-\theta) f(\mathbf{x}_2)
\end{equation}

由于 $\theta \in [0, 1]$,有:
\begin{equation}
\theta f(\mathbf{x}_1) + (1-\theta) f(\mathbf{x}_2) \leq \max\{f(\mathbf{x}_1), f(\mathbf{x}_2)\}
\end{equation}

因此:
\begin{equation}
f(\theta \mathbf{x}_1 + (1-\theta) \mathbf{x}_2) \leq \max\{f(\mathbf{x}_1), f(\mathbf{x}_2)\}
\end{equation}

所以 $f$ 是拟凸函数。$\square$

\subsection{拟凸函数 $\not\Rightarrow$ 凸函数}

\textbf{反例}:$f(x) = \sqrt{|x|}$ 是拟凸的,但不是凸的。

\textbf{验证}:
\begin{itemize}
\item 子水平集 $S_\alpha = \{x \mid \sqrt{|x|} \leq \alpha\} = \{x \mid |x| \leq \alpha^2\}$ 是凸集
\item 但 $f$ 不是凸函数(二阶导数不满足)
\end{itemize}

\subsection{关系总结}

\begin{itemize}
\item \textbf{凸函数} $\subset$ \textbf{拟凸函数}
\item 所有凸函数都是拟凸的
\item 但拟凸函数不一定是凸的
\item 拟凸函数是凸函数的推广
\end{itemize}

\section{几何意义}

\subsection{拟凸函数的几何特征}

\textbf{核心思想}:拟凸函数的函数值在任意两点连线上不超过两点的最大值。

\textbf{几何描述}:
\begin{itemize}
\item 取函数图像上任意两点
\item 连接这两点的线段
\item 函数在这条线段上的值不超过两点的最大值
\item 换句话说:函数值"不会在中间突然升高"
\end{itemize}

\subsection{子水平集的几何意义}

\textbf{关键性质}:拟凸函数的所有子水平集都是凸集。

\textbf{几何直观}:
\begin{itemize}
\item 子水平集 $S_\alpha = \{\mathbf{x} \mid f(\mathbf{x}) \leq \alpha\}$ 是"函数值不超过 $\alpha$ 的所有点"
\item 如果所有子水平集都是凸集,则函数是拟凸的
\item 这意味着"等值线"围成的区域是凸的
\end{itemize}

\textbf{例子}:
\begin{itemize}
\item \textbf{凸函数}:所有子水平集都是凸集(更严格)
\item \textbf{拟凸函数}:所有子水平集都是凸集(更一般)
\item \textbf{非拟凸函数}:某些子水平集不是凸集
\end{itemize}

\section{具体例子}

\subsection{例子1:凸函数是拟凸的}

\textbf{函数}:$f(x) = x^2$

\textbf{验证}:
\begin{itemize}
\item $f$ 是凸函数
\item 因此 $f$ 是拟凸函数
\item 子水平集:$S_\alpha = \{x \mid x^2 \leq \alpha\} = [-\sqrt{\alpha}, \sqrt{\alpha}]$ 是凸集
\end{itemize}

\subsection{例子2:拟凸但不是凸的函数}

\textbf{函数}:$f(x) = \sqrt{|x|}$

\textbf{验证}:
\begin{itemize}
\item 子水平集:$S_\alpha = \{x \mid \sqrt{|x|} \leq \alpha\} = \{x \mid |x| \leq \alpha^2\} = [-\alpha^2, \alpha^2]$ 是凸集
\item 因此 $f$ 是拟凸函数
\item 但 $f$ 不是凸函数(二阶导数条件不满足)
\end{itemize}

\subsection{例子3:单峰函数}

\textbf{函数}:$f(x) = 
\begin{cases}
x^2 & \text{if } x \leq 0 \\
0 & \text{if } x > 0
\end{cases}$

\textbf{验证}:
\begin{itemize}
\item 这是一个"单峰"函数
\item 子水平集都是凸集
\item 因此是拟凸函数
\item 但不是凸函数(在 $x=0$ 处不满足凸性)
\end{itemize}

\subsection{例子4:Ceiling函数}

\textbf{函数}:$f(x) = \lceil x \rceil$(向上取整)

\textbf{验证}:
\begin{itemize}
\item 子水平集:$S_\alpha = \{x \mid \lceil x \rceil \leq \alpha\} = (-\infty, \alpha]$ 是凸集
\item 因此是拟凸函数
\item 但不是凸函数(不连续,更不用说凸性)
\end{itemize}

\subsection{例子5:拟凹函数}

\textbf{函数}:$f(x) = \log x$($x > 0$)

\textbf{验证}:
\begin{itemize}
\item $f$ 是凹函数(因此是拟凹函数)
\item 子水平集:$S_\alpha = \{x > 0 \mid \log x \geq \alpha\} = [e^\alpha, +\infty)$ 是凸集
\item 因此是拟凹函数
\end{itemize}

\section{拟凸函数的性质}

\subsection{基本性质}

\begin{enumerate}
\item \textbf{子水平集是凸集}:所有子水平集 $S_\alpha$ 都是凸集
\item \textbf{局部最小值}:拟凸函数的局部最小值是全局最小值
\item \textbf{单调函数}:单调函数是拟凸的(也是拟凹的,即拟线性的)
\end{enumerate}

\subsection{保持拟凸性的运算}

\begin{enumerate}
\item \textbf{非负加权最大值}:如果 $f_i$ 是拟凸的,则 $\max\{f_1, \ldots, f_k\}$ 是拟凸的
\item \textbf{复合}:如果 $g$ 是拟凸的,$h$ 是单调递增的,则 $h \circ g$ 是拟凸的
\item \textbf{仿射变换}:如果 $f$ 是拟凸的,则 $f(\mathbf{A}\mathbf{x} + \mathbf{b})$ 是拟凸的
\end{enumerate}

\subsection{与凸函数的区别}

\begin{table}[h]
\centering
\begin{tabular}{|l|l|l|}
\hline
\textbf{性质} & \textbf{凸函数} & \textbf{拟凸函数} \\
\hline
定义 & $f(\theta \mathbf{x}_1 + (1-\theta) \mathbf{x}_2) \leq \theta f(\mathbf{x}_1) + (1-\theta) f(\mathbf{x}_2)$ & $f(\theta \mathbf{x}_1 + (1-\theta) \mathbf{x}_2) \leq \max\{f(\mathbf{x}_1), f(\mathbf{x}_2)\}$ \\
\hline
子水平集 & 凸集 & 凸集 \\
\hline
局部最优 & 全局最优 & 全局最优 \\
\hline
可微性 & 不一定可微 & 不一定可微 \\
\hline
二阶条件 & $\nabla^2 f \succeq 0$ & 无简单的二阶条件 \\
\hline
\end{tabular}
\caption{凸函数与拟凸函数的对比}
\end{table}

\section{拟凸优化问题}

\subsection{定义}

\textbf{拟凸优化问题}:
\begin{align}
\begin{array}{ll}
\text{minimize} & f_0(\mathbf{x}) \\
\text{subject to} & f_i(\mathbf{x}) \leq 0, \quad i = 1, \ldots, m \\
& \mathbf{A}\mathbf{x} = \mathbf{b}
\end{array}
\end{align}

其中 $f_0$ 是拟凸函数,$f_i$ 是凸函数。

\subsection{性质}

\begin{enumerate}
\item \textbf{可行集是凸集}:由于约束函数是凸的,可行集是凸集
\item \textbf{局部最优 = 全局最优}:拟凸函数的局部最小值是全局最小值
\item \textbf{$\epsilon$-次优集是凸集}:对于拟凸优化问题,$\epsilon$-次优集是凸集
\end{enumerate}

\subsection{求解方法}

\textbf{二分法}:
\begin{enumerate}
\item 对于拟凸优化问题,可以使用二分法
\item 通过求解一系列凸优化问题来找到最优解
\item 利用子水平集是凸集的性质
\end{enumerate}

\textbf{基本思想}:
\begin{itemize}
\item 对于给定的 $\alpha$,判断是否存在可行解使得 $f_0(\mathbf{x}) \leq \alpha$
\item 这是一个凸可行性问题(因为子水平集是凸集)
\item 通过二分搜索找到最小的 $\alpha$
\end{itemize}

\section{判断拟凸性的方法}

\subsection{方法1:通过定义}

\textbf{步骤}:
\begin{enumerate}
\item 取定义域中任意两点 $\mathbf{x}_1, \mathbf{x}_2$
\item 对于任意 $\theta \in [0, 1]$,检查:
   \begin{equation}
   f(\theta \mathbf{x}_1 + (1-\theta) \mathbf{x}_2) \leq \max\{f(\mathbf{x}_1), f(\mathbf{x}_2)\}
   \end{equation}
\item 如果对所有点都成立,则是拟凸的
\end{enumerate}

\subsection{方法2:通过子水平集}

\textbf{步骤}:
\begin{enumerate}
\item 对于任意 $\alpha$,构造子水平集 $S_\alpha = \{\mathbf{x} \mid f(\mathbf{x}) \leq \alpha\}$
\item 检查 $S_\alpha$ 是否为凸集
\item 如果所有子水平集都是凸集,则函数是拟凸的
\end{enumerate}

\subsection{方法3:通过一阶条件(可微函数)}

\textbf{一阶条件}:如果 $f$ 是可微的拟凸函数,则对于任意 $\mathbf{x}, \mathbf{y} \in \text{dom } f$,如果 $f(\mathbf{y}) \leq f(\mathbf{x})$,则:

\begin{equation}
\nabla f(\mathbf{x})^T (\mathbf{y} - \mathbf{x}) \leq 0
</equation>

\textbf{注意}:这个条件比凸函数的一阶条件弱。

\section{常见拟凸函数}

\subsection{单调函数}

\textbf{定理}:单调函数是拟线性的(既是拟凸又是拟凹)。

\textbf{例子}:
\begin{itemize}
\item $f(x) = x$:单调递增,拟线性
\item $f(x) = e^x$:单调递增,拟线性
\item $f(x) = \log x$:单调递增,拟线性
\end{itemize}

\subsection{单峰函数}

\textbf{定义}:函数只有一个"峰"(最大值点)。

\textbf{例子}:
\begin{itemize}
\item $f(x) = -x^2$:单峰函数,拟凸
\item $f(x) = e^{-x^2}$:单峰函数,拟凸
\end{itemize}

\subsection{分片线性函数}

\textbf{例子}:
\begin{itemize}
\item $f(x) = |x|$:拟凸(也是凸的)
\item $f(x) = \max\{0, x\}$:拟凸(也是凸的)
\end{itemize}

\section{拟凸函数与优化}

\subsection{优势}

\begin{enumerate}
\item \textbf{更一般}:比凸函数更一般,可以描述更多问题
\item \textbf{保持良好性质}:局部最优 = 全局最优
\item \textbf{子水平集凸性}:便于求解
\end{enumerate}

\subsection{局限性}

\begin{enumerate}
\item \textbf{无简单的二阶条件}:不像凸函数有 $\nabla^2 f \succeq 0$
\item \textbf{求解更复杂}:通常需要二分法等特殊方法
\item \textbf{理论分析更困难}:比凸函数更复杂
\end{enumerate}

\section{总结}

\subsection{关键概念}

\begin{enumerate}
\item \textbf{拟凸函数定义}:
   \begin{itemize}
   \item $f(\theta \mathbf{x}_1 + (1-\theta) \mathbf{x}_2) \leq \max\{f(\mathbf{x}_1), f(\mathbf{x}_2)\}$
   \item 等价地:所有子水平集都是凸集
   \end{itemize}

\item \textbf{与凸函数的关系}:
   \begin{itemize}
   \item 凸函数 $\Rightarrow$ 拟凸函数
   \item 拟凸函数 $\not\Rightarrow$ 凸函数
   \item 拟凸函数是凸函数的推广
   \end{itemize}

\item \textbf{几何意义}:
   \begin{itemize}
   \item 函数值在任意两点连线上不超过两点的最大值
   \item 所有子水平集都是凸集
   \end{itemize}

\item \textbf{优化性质}:
   \begin{itemize}
   \item 局部最优 = 全局最优
   \item $\epsilon$-次优集是凸集
   \item 可以使用二分法求解
   \end{itemize}
</enumerate>

\subsection{记忆技巧}

\begin{enumerate}
\item \textbf{拟凸}:想象函数值"不会在中间突然升高"
\item \textbf{子水平集}:所有"等值线"围成的区域是凸的
\item \textbf{与凸函数}:拟凸比凸更一般,但保持局部最优 = 全局最优
\end{enumerate}

理解拟凸函数,有助于理解更广泛的优化问题,特别是那些目标函数不是凸的,但仍有良好性质的问题!

\end{document}

