\documentclass[12pt,a4paper]{article}
\usepackage[UTF8]{ctex}
\usepackage{amsmath}
\usepackage{amssymb}
\usepackage{amsthm}
\usepackage{geometry}
\geometry{left=2.5cm,right=2.5cm,top=2.5cm,bottom=2.5cm}

\title{拟凸函数与子水平集的等价性证明}
\subtitle{为什么"所有子水平集都是凸集"等价于"拟凸函数"?}
\author{}
\date{\today}

\begin{document}

\maketitle

\section{定理陈述}

\textbf{定理}:函数 $f: \mathbb{R}^n \to \mathbb{R}$ 是拟凸函数,当且仅当其所有子水平集都是凸集。

\textbf{数学表述}:

$f$ 是拟凸函数 $\Leftrightarrow$ 对于任意 $\alpha \in \mathbb{R}$,子水平集 $S_\alpha = \{\mathbf{x} \in \text{dom } f \mid f(\mathbf{x}) \leq \alpha\}$ 都是凸集。

\section{子水平集的定义回顾}

\subsection{定义}

\textbf{子水平集}:对于函数 $f: \mathbb{R}^n \to \mathbb{R}$ 和实数 $\alpha \in \mathbb{R}$,$f$ 的 $\alpha$-子水平集定义为:

\begin{equation}
S_\alpha = \{\mathbf{x} \in \text{dom } f \mid f(\mathbf{x}) \leq \alpha\}
</equation>

\textbf{含义}:所有函数值不超过 $\alpha$ 的点的集合。

\section{拟凸函数的定义回顾}

\subsection{标准定义}

\textbf{拟凸函数}:函数 $f: \mathbb{R}^n \to \mathbb{R}$ 是拟凸的,如果:

\begin{enumerate}
\item 定义域 $\text{dom } f$ 是凸集

\item 对于任意 $\mathbf{x}_1, \mathbf{x}_2 \in \text{dom } f$ 和 $\theta \in [0, 1]$,有:
   \begin{equation}
   f(\theta \mathbf{x}_1 + (1-\theta) \mathbf{x}_2) \leq \max\{f(\mathbf{x}_1), f(\mathbf{x}_2)\}
   \end{equation}
</enumerate}

\section{证明:拟凸函数 $\Rightarrow$ 所有子水平集是凸集}

\subsection{目标}

\textbf{需要证明}:如果 $f$ 是拟凸函数,则对于任意 $\alpha \in \mathbb{R}$,子水平集 $S_\alpha = \{\mathbf{x} \mid f(\mathbf{x}) \leq \alpha\}$ 是凸集。

\subsection{证明步骤}

\textbf{步骤1}:取任意 $\alpha \in \mathbb{R}$,考虑子水平集 $S_\alpha = \{\mathbf{x} \in \text{dom } f \mid f(\mathbf{x}) \leq \alpha\}$。

\textbf{步骤2}:取 $S_\alpha$ 中任意两点 $\mathbf{x}_1, \mathbf{x}_2 \in S_\alpha$。

根据子水平集的定义:
\begin{align}
f(\mathbf{x}_1) &\leq \alpha \\
f(\mathbf{x}_2) &\leq \alpha
\end{align}

\textbf{步骤3}:对于任意 $\theta \in [0, 1]$,考虑凸组合 $\mathbf{x} = \theta \mathbf{x}_1 + (1-\theta) \mathbf{x}_2$。

\textbf{步骤4}:由于 $f$ 是拟凸函数,有:

\begin{equation}
f(\mathbf{x}) = f(\theta \mathbf{x}_1 + (1-\theta) \mathbf{x}_2) \leq \max\{f(\mathbf{x}_1), f(\mathbf{x}_2)\}
</equation>

\textbf{步骤5}:由于 $f(\mathbf{x}_1) \leq \alpha$ 且 $f(\mathbf{x}_2) \leq \alpha$,有:

\begin{equation}
\max\{f(\mathbf{x}_1), f(\mathbf{x}_2)\} \leq \alpha
</equation}

\textbf{步骤6}:因此:

\begin{equation}
f(\mathbf{x}) \leq \max\{f(\mathbf{x}_1), f(\mathbf{x}_2)\} \leq \alpha
</equation}

\textbf{步骤7}:所以 $f(\mathbf{x}) \leq \alpha$,即 $\mathbf{x} \in S_\alpha$。

\textbf{步骤8}:由于 $\mathbf{x}_1, \mathbf{x}_2$ 和 $\theta$ 是任意的,$S_\alpha$ 中任意两点的凸组合都在 $S_\alpha$ 中,因此 $S_\alpha$ 是凸集。$\square$

\subsection{证明总结}

\begin{enumerate}
\item 取子水平集 $S_\alpha$ 中任意两点
\item 应用拟凸函数的定义
\item 利用 $\max\{f(\mathbf{x}_1), f(\mathbf{x}_2)\} \leq \alpha$
\item 得到凸组合也在 $S_\alpha$ 中
\item 因此 $S_\alpha$ 是凸集
</enumerate}

\section{证明:所有子水平集是凸集 $\Rightarrow$ 拟凸函数}

\subsection{目标}

\textbf{需要证明}:如果所有子水平集 $S_\alpha$ 都是凸集,则 $f$ 是拟凸函数。

即:对于任意 $\mathbf{x}_1, \mathbf{x}_2 \in \text{dom } f$ 和 $\theta \in [0, 1]$,有:

\begin{equation}
f(\theta \mathbf{x}_1 + (1-\theta) \mathbf{x}_2) \leq \max\{f(\mathbf{x}_1), f(\mathbf{x}_2)\}
</equation>

\subsection{证明步骤}

\textbf{步骤1}:取任意 $\mathbf{x}_1, \mathbf{x}_2 \in \text{dom } f$ 和 $\theta \in [0, 1]$。

\textbf{步骤2}:设 $\alpha = \max\{f(\mathbf{x}_1), f(\mathbf{x}_2)\}$。

\textbf{关键观察}:
\begin{itemize}
\item $f(\mathbf{x}_1) \leq \alpha$,因此 $\mathbf{x}_1 \in S_\alpha$
\item $f(\mathbf{x}_2) \leq \alpha$,因此 $\mathbf{x}_2 \in S_\alpha$
\item 由于 $S_\alpha$ 是凸集,对于任意 $\theta \in [0, 1]$,有:
   \begin{equation}
   \theta \mathbf{x}_1 + (1-\theta) \mathbf{x}_2 \in S_\alpha
   \end{equation}
\end{itemize}

\textbf{步骤3}:由于 $\theta \mathbf{x}_1 + (1-\theta) \mathbf{x}_2 \in S_\alpha$,根据子水平集的定义:

\begin{equation}
f(\theta \mathbf{x}_1 + (1-\theta) \mathbf{x}_2) \leq \alpha = \max\{f(\mathbf{x}_1), f(\mathbf{x}_2)\}
</equation>

\textbf{步骤4}:这正是拟凸函数的定义!

\textbf{步骤5}:由于 $\mathbf{x}_1, \mathbf{x}_2$ 和 $\theta$ 是任意的,$f$ 是拟凸函数。$\square$

\subsection{证明总结}

\begin{enumerate}
\item 取任意两点 $\mathbf{x}_1, \mathbf{x}_2$
\item 设 $\alpha = \max\{f(\mathbf{x}_1), f(\mathbf{x}_2)\}$
\item 由于 $S_\alpha$ 是凸集,凸组合在 $S_\alpha$ 中
\item 根据子水平集定义,得到拟凸性
\end{enumerate}

\section{完整等价性证明}

\subsection{方向1:拟凸函数 $\Rightarrow$ 所有子水平集是凸集}

\textbf{已证明}:见第4节。

\subsection{方向2:所有子水平集是凸集 $\Rightarrow$ 拟凸函数}

\textbf{已证明}:见第5节。

\subsection{结论}

由于两个方向都已证明,等价性成立:

\begin{equation}
f \text{ 是拟凸函数 } \Leftrightarrow \text{ 所有子水平集 } S_\alpha \text{ 都是凸集}
\end{equation}

\section{几何直观}

\subsection{拟凸函数的几何特征}

\textbf{核心思想}:函数值在任意两点连线上不超过两点的最大值。

\textbf{几何描述}:
\begin{itemize}
\item 取函数图像上任意两点
\item 连接这两点的线段
\item 函数在这条线段上的值不超过两点的最大值
\end{itemize}

\subsection{子水平集的几何特征}

\textbf{核心思想}:所有"等值线"围成的区域是凸的。

\textbf{几何描述}:
\begin{itemize}
\item 子水平集 $S_\alpha$ 是函数值不超过 $\alpha$ 的所有点
\item 如果所有子水平集都是凸集,则"等值线"围成的区域都是凸的
\item 这意味着函数值"不会在中间突然升高"
\end{itemize}

\subsection{等价性的几何理解}

\textbf{为什么等价?}

\begin{itemize}
\item \textbf{如果函数是拟凸的}:
   \begin{itemize}
   \item 函数值在任意两点连线上不超过最大值
   \item 因此,如果两点都在 $S_\alpha$ 中(函数值 $\leq \alpha$)
   \item 则连接两点的线段上的点函数值也 $\leq \alpha$
   \item 所以线段在 $S_\alpha$ 中,$S_\alpha$ 是凸集
   \end{itemize}

\item \textbf{如果所有子水平集都是凸集}:
   \begin{itemize}
   \item 对于任意两点,设 $\alpha = \max\{f(\mathbf{x}_1), f(\mathbf{x}_2)\}$
   \item 两点都在 $S_\alpha$ 中
   \item 由于 $S_\alpha$ 是凸集,连接两点的线段在 $S_\alpha$ 中
   \item 因此线段上的点函数值 $\leq \alpha = \max\{f(\mathbf{x}_1), f(\mathbf{x}_2)\}$
   \item 所以函数是拟凸的
   \end{itemize}
\end{itemize}

\section{具体例子}

\subsection{例子1:凸函数是拟凸的}

\textbf{函数}:$f(x) = x^2$

\textbf{验证方法1(通过定义)}:

对于任意 $x_1, x_2$ 和 $\theta \in [0, 1]$:

\begin{align}
f(\theta x_1 + (1-\theta) x_2) &= (\theta x_1 + (1-\theta) x_2)^2 \\
&\leq \theta x_1^2 + (1-\theta) x_2^2 \quad \text{(凸性)} \\
&\leq \max\{x_1^2, x_2^2\} = \max\{f(x_1), f(x_2)\}
\end{align}

因此是拟凸的。✓

\textbf{验证方法2(通过子水平集)}:

子水平集:$S_\alpha = \{x \mid x^2 \leq \alpha\}$

\begin{itemize}
\item 如果 $\alpha < 0$:$S_\alpha = \emptyset$(空集是凸集)
\item 如果 $\alpha \geq 0$:$S_\alpha = [-\sqrt{\alpha}, \sqrt{\alpha}]$(区间是凸集)
\end{itemize}

所有子水平集都是凸集,因此是拟凸的。✓

\subsection{例子2:拟凸但不是凸的函数}

\textbf{函数}:$f(x) = \sqrt{|x|}$

\textbf{验证方法1(通过定义)}:

可以验证拟凸性条件成立。

\textbf{验证方法2(通过子水平集)}:

子水平集:$S_\alpha = \{x \mid \sqrt{|x|} \leq \alpha\} = \{x \mid |x| \leq \alpha^2\} = [-\alpha^2, \alpha^2]$

所有子水平集都是区间(凸集),因此是拟凸的。✓

但 $f$ 不是凸函数(二阶导数条件不满足)。

\subsection{例子3:非拟凸函数}

\textbf{函数}:$f(x) = \sin x$

\textbf{子水平集}:$S_\alpha = \{x \mid \sin x \leq \alpha\}$

对于 $\alpha = 0$,$S_0 = \{x \mid \sin x \leq 0\}$ 包含多个不相交的区间,不是凸集。

因此 $f$ 不是拟凸函数。

\section{应用:判断拟凸性}

\subsection{方法1:通过定义}

\textbf{步骤}:
\begin{enumerate}
\item 取定义域中任意两点 $\mathbf{x}_1, \mathbf{x}_2$
\item 对于任意 $\theta \in [0, 1]$,检查:
   \begin{equation}
   f(\theta \mathbf{x}_1 + (1-\theta) \mathbf{x}_2) \leq \max\{f(\mathbf{x}_1), f(\mathbf{x}_2)\}
   \end{equation}
\item 如果对所有点都成立,则是拟凸的
\end{enumerate}

\subsection{方法2:通过子水平集(更实用)}

\textbf{步骤}:
\begin{enumerate}
\item 对于任意 $\alpha \in \mathbb{R}$,构造子水平集 $S_\alpha = \{\mathbf{x} \mid f(\mathbf{x}) \leq \alpha\}$
\item 检查 $S_\alpha$ 是否为凸集
\item 如果所有子水平集都是凸集,则函数是拟凸的
\end{enumerate}

\textbf{优势}:
\begin{itemize}
\item 通常比直接验证定义更容易
\item 可以利用凸集的已知性质
\item 几何直观更强
\end{itemize}

\section{与凸函数的关系}

\subsection{凸函数 $\Rightarrow$ 拟凸函数}

\textbf{定理}:如果 $f$ 是凸函数,则 $f$ 是拟凸函数。

\textbf{证明}:

如果 $f$ 是凸函数,则:

\begin{equation}
f(\theta \mathbf{x}_1 + (1-\theta) \mathbf{x}_2) \leq \theta f(\mathbf{x}_1) + (1-\theta) f(\mathbf{x}_2) \leq \max\{f(\mathbf{x}_1), f(\mathbf{x}_2)\}
</equation>

因此是拟凸的。

\textbf{通过子水平集}:

凸函数的子水平集是凸集,因此所有子水平集都是凸集,所以是拟凸的。

\subsection{拟凸函数 $\not\Rightarrow$ 凸函数}

\textbf{反例}:$f(x) = \sqrt{|x|}$ 是拟凸的,但不是凸的。

\section{总结}

\subsection{等价性定理}

\begin{equation}
f \text{ 是拟凸函数 } \Leftrightarrow \text{ 所有子水平集 } S_\alpha \text{ 都是凸集}
\end{equation}

\subsection{证明思路}

\begin{enumerate}
\item \textbf{方向1}(拟凸 $\Rightarrow$ 子水平集凸):
   \begin{itemize}
   \item 取子水平集中任意两点
   \item 应用拟凸性,得到凸组合的函数值 $\leq \alpha$
   \item 因此凸组合在子水平集中
   \end{itemize}

\item \textbf{方向2}(子水平集凸 $\Rightarrow$ 拟凸):
   \begin{itemize}
   \item 取任意两点,设 $\alpha = \max\{f(\mathbf{x}_1), f(\mathbf{x}_2)\}$
   \item 两点都在 $S_\alpha$ 中
   \item 由于 $S_\alpha$ 是凸集,凸组合在 $S_\alpha$ 中
   \item 因此函数值 $\leq \alpha = \max\{f(\mathbf{x}_1), f(\mathbf{x}_2)\}$
   \end{itemize}
</enumerate>

\subsection{关键理解}

\begin{enumerate}
\item \textbf{拟凸性}:函数值在任意两点连线上不超过最大值

\item \textbf{子水平集凸性}:所有"等值线"围成的区域是凸的

\item \textbf{等价性}:这两个性质是等价的,可以从一个推导出另一个

\item \textbf{应用}:通过检查子水平集来判断拟凸性通常更容易
</enumerate>

理解这个等价性,是掌握拟凸函数理论的关键!

\end{document}

