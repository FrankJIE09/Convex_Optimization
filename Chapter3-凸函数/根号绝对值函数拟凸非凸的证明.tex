\documentclass[12pt,a4paper]{article}
\usepackage[UTF8]{ctex}
\usepackage{amsmath}
\usepackage{amssymb}
\usepackage{amsthm}
\usepackage{geometry}
\geometry{left=2.5cm,right=2.5cm,top=2.5cm,bottom=2.5cm}

\title{为什么 $f(x) = \sqrt{|x|}$ 是拟凸的但不是凸的?}
\author{}
\date{\today}

\begin{document}

\maketitle

\section{凸函数的定义}

\subsection{基本定义}

\textbf{凸函数}:函数 $f: \mathbb{R}^n \to \mathbb{R}$ 是凸函数,如果:

\begin{enumerate}
\item \textbf{定义域是凸集}:$\text{dom } f$ 是凸集

\item \textbf{凸性条件}:对于任意 $\mathbf{x}_1, \mathbf{x}_2 \in \text{dom } f$ 和任意 $\theta \in [0, 1]$,有:
   \begin{equation}
   f(\theta \mathbf{x}_1 + (1-\theta) \mathbf{x}_2) \leq \theta f(\mathbf{x}_1) + (1-\theta) f(\mathbf{x}_2)
   \end{equation}
\end{enumerate}

\subsection{几何意义}

\textbf{几何解释}:
\begin{itemize}
\item 函数图像上任意两点的连线在函数图像的上方(或重合)
\item 函数图像是"向下凸"的(concave upward)
\item 任意弦(chord)在函数图像上方
\end{itemize}

\subsection{一维情况的简化}

对于一元函数 $f: \mathbb{R} \to \mathbb{R}$,凸性条件简化为:

对于任意 $x_1, x_2 \in \text{dom } f$ 和任意 $\theta \in [0, 1]$,有:

\begin{equation}
f(\theta x_1 + (1-\theta) x_2) \leq \theta f(x_1) + (1-\theta) f(x_2)
\end{equation}

\subsection{可微函数的凸性条件}

如果 $f$ 是可微的,则凸性等价于:

\textbf{一阶条件}:对于任意 $x, y \in \text{dom } f$,有:

\begin{equation}
f(y) \geq f(x) + f'(x)(y - x)
\end{equation}

\textbf{几何意义}:函数图像在任意点处的切线都在函数图像下方(或重合)。

\textbf{二阶条件}:如果 $f$ 是二阶可微的,则凸性等价于:

\begin{equation}
f''(x) \geq 0, \quad \forall x \in \text{dom } f
\end{equation}

\textbf{几何意义}:函数的二阶导数非负,即函数"向上弯曲"。

\section{拟凸函数的定义}

\subsection{基本定义}

\textbf{拟凸函数}:函数 $f: \mathbb{R}^n \to \mathbb{R}$ 是拟凸的,如果:

\begin{enumerate}
\item \textbf{定义域是凸集}:$\text{dom } f$ 是凸集

\item \textbf{拟凸性条件}:对于任意 $\mathbf{x}_1, \mathbf{x}_2 \in \text{dom } f$ 和任意 $\theta \in [0, 1]$,有:
   \begin{equation}
   f(\theta \mathbf{x}_1 + (1-\theta) \mathbf{x}_2) \leq \max\{f(\mathbf{x}_1), f(\mathbf{x}_2)\}
   \end{equation}
\end{enumerate}

\subsection{等价定义}

\textbf{子水平集定义}:函数 $f$ 是拟凸的,当且仅当其所有子水平集都是凸集。

子水平集:$S_\alpha = \{x \in \text{dom } f \mid f(x) \leq \alpha\}$

\subsection{与凸函数的关系}

\begin{itemize}
\item \textbf{凸函数 $\Rightarrow$ 拟凸函数}:所有凸函数都是拟凸的
\item \textbf{拟凸函数 $\not\Rightarrow$ 凸函数}:拟凸函数不一定是凸的
\item 拟凸函数是凸函数的推广
\end{itemize}

\section{函数 $f(x) = \sqrt{|x|}$ 的分析}

\subsection{函数定义}

\textbf{函数}:$f(x) = \sqrt{|x|}$,定义域:$\text{dom } f = \mathbb{R}$

\textbf{函数图像特征}:
\begin{itemize}
\item 在 $x = 0$ 处,$f(0) = 0$
\item 对于 $x > 0$,$f(x) = \sqrt{x}$(单调递增)
\item 对于 $x < 0$,$f(x) = \sqrt{-x}$(单调递减)
\item 函数关于 $y$ 轴对称
\item 在 $x = 0$ 处不可微(尖点)
\end{itemize}

\section{证明:$f(x) = \sqrt{|x|}$ 是拟凸的}

\subsection{方法1:通过子水平集}

\textbf{定理}:如果所有子水平集都是凸集,则函数是拟凸的。

\textbf{子水平集}:对于任意 $\alpha \geq 0$,

\begin{equation}
S_\alpha = \{x \in \mathbb{R} \mid \sqrt{|x|} \leq \alpha\} = \{x \in \mathbb{R} \mid |x| \leq \alpha^2\} = [-\alpha^2, \alpha^2]
\end{equation}

\textbf{分析}:
\begin{itemize}
\item 如果 $\alpha < 0$:$S_\alpha = \emptyset$(空集是凸集)
\item 如果 $\alpha \geq 0$:$S_\alpha = [-\alpha^2, \alpha^2]$ 是闭区间,因此是凸集
\end{itemize}

\textbf{结论}:所有子水平集都是凸集,因此 $f(x) = \sqrt{|x|}$ 是拟凸函数。$\square$

\subsection{方法2:通过定义}

\textbf{需要证明}:对于任意 $x_1, x_2 \in \mathbb{R}$ 和 $\theta \in [0, 1]$,有:

\begin{equation}
f(\theta x_1 + (1-\theta) x_2) \leq \max\{f(x_1), f(x_2)\}
\end{equation}

\textbf{分析}:

由于 $f(x) = \sqrt{|x|}$ 关于 $y$ 轴对称,且对于 $x \geq 0$ 单调递增,我们可以考虑几种情况:

\textbf{情况1}:$x_1, x_2$ 同号(都 $\geq 0$ 或都 $\leq 0$)

不失一般性,设 $x_1, x_2 \geq 0$,则 $f(x) = \sqrt{x}$。

由于 $\sqrt{x}$ 是凹函数(不是凸函数),但我们可以直接验证:

\begin{align}
f(\theta x_1 + (1-\theta) x_2) &= \sqrt{\theta x_1 + (1-\theta) x_2} \\
&\leq \sqrt{\max\{x_1, x_2\}} \quad \text{(因为 $\sqrt{\cdot}$ 单调递增)} \\
&= \max\{\sqrt{x_1}, \sqrt{x_2}\} \\
&= \max\{f(x_1), f(x_2)\}
\end{align}

\textbf{情况2}:$x_1, x_2$ 异号

不失一般性,设 $x_1 < 0 < x_2$。

由于 $f$ 关于 $y$ 轴对称,且 $\theta x_1 + (1-\theta) x_2$ 在 $x_1$ 和 $x_2$ 之间,我们有:

\begin{align}
f(\theta x_1 + (1-\theta) x_2) &\leq \max\{f(x_1), f(x_2)\} \\
&= \max\{\sqrt{|x_1|}, \sqrt{|x_2|}\}
\end{align}

\textbf{结论}:在所有情况下,拟凸性条件都成立,因此 $f$ 是拟凸函数。$\square$

\section{证明:$f(x) = \sqrt{|x|}$ 不是凸的}

\subsection{方法1:通过定义(直接反例)}

\textbf{构造反例}:取 $x_1 = -1$,$x_2 = 1$,$\theta = 0.5$。

\textbf{计算}:
\begin{align}
f(x_1) &= f(-1) = \sqrt{|-1|} = 1 \\
f(x_2) &= f(1) = \sqrt{|1|} = 1 \\
\theta f(x_1) + (1-\theta) f(x_2) &= 0.5 \times 1 + 0.5 \times 1 = 1
\end{align}

\textbf{另一方面}:
\begin{align}
\theta x_1 + (1-\theta) x_2 &= 0.5 \times (-1) + 0.5 \times 1 = 0 \\
f(\theta x_1 + (1-\theta) x_2) &= f(0) = \sqrt{|0|} = 0
\end{align}

\textbf{验证凸性条件}:

如果 $f$ 是凸函数,应该有:
\begin{equation}
f(\theta x_1 + (1-\theta) x_2) \leq \theta f(x_1) + (1-\theta) f(x_2)
\end{equation}

即:$f(0) \leq 0.5 f(-1) + 0.5 f(1)$

但:$0 \leq 1$ ✓(这个条件满足)

\textbf{问题}:这个反例不够强。让我们找更好的反例。

\subsection{方法2:通过二阶导数(更严格)}

\textbf{对于 $x > 0$}:$f(x) = \sqrt{x}$

计算导数:
\begin{align}
f'(x) &= \frac{1}{2\sqrt{x}} \\
f''(x) &= -\frac{1}{4x^{3/2}} < 0
\end{align}

\textbf{结论}:对于 $x > 0$,$f''(x) < 0$,因此 $f$ 在 $(0, +\infty)$ 上是凹函数,不是凸函数。

\textbf{对于 $x < 0$}:$f(x) = \sqrt{-x}$

计算导数:
\begin{align}
f'(x) &= -\frac{1}{2\sqrt{-x}} \\
f''(x) &= -\frac{1}{4(-x)^{3/2}} < 0
\end{align}

\textbf{结论}:对于 $x < 0$,$f''(x) < 0$,因此 $f$ 在 $(-\infty, 0)$ 上也是凹函数,不是凸函数。

\textbf{在 $x = 0$ 处}:$f$ 不可微,但我们可以通过极限分析。

\subsection{方法3:通过更精确的反例}

\textbf{构造反例}:取 $x_1 = 1$,$x_2 = 4$,$\theta = 0.5$。

\textbf{计算}:
\begin{align}
f(x_1) &= f(1) = \sqrt{1} = 1 \\
f(x_2) &= f(4) = \sqrt{4} = 2 \\
\theta f(x_1) + (1-\theta) f(x_2) &= 0.5 \times 1 + 0.5 \times 2 = 1.5
\end{align}

\textbf{另一方面}:
\begin{align}
\theta x_1 + (1-\theta) x_2 &= 0.5 \times 1 + 0.5 \times 4 = 2.5 \\
f(\theta x_1 + (1-\theta) x_2) &= f(2.5) = \sqrt{2.5} \approx 1.581
\end{align}

\textbf{验证凸性条件}:

如果 $f$ 是凸函数,应该有:
\begin{equation}
f(2.5) \leq 0.5 f(1) + 0.5 f(4)
\end{equation}

即:$1.581 \leq 1.5$ ✗

\textbf{结论}:凸性条件不满足,因此 $f(x) = \sqrt{|x|}$ 不是凸函数。$\square$

\subsection{方法4:通过几何直观}

\textbf{几何分析}:
\begin{itemize}
\item 函数 $f(x) = \sqrt{|x|}$ 的图像在 $x = 0$ 处有尖点
\item 对于 $x > 0$,函数是 $\sqrt{x}$,这是凹函数(向下弯曲)
\item 对于 $x < 0$,函数是 $\sqrt{-x}$,这也是凹函数
\item 凹函数的图像上,任意两点的连线在函数图像下方
\item 这与凸函数的定义相反
\end{itemize}

\textbf{具体验证}:

取 $x_1 = 1$,$x_2 = 4$,连接这两点的线段:

线段上的点:$x = \theta \times 1 + (1-\theta) \times 4 = 4 - 3\theta$,$\theta \in [0, 1]$

线段上的函数值(线性插值):$f_{\text{line}}(\theta) = \theta \times 1 + (1-\theta) \times 2 = 2 - \theta$

实际函数值:$f(4 - 3\theta) = \sqrt{4 - 3\theta}$

对于 $\theta = 0.5$:
\begin{itemize}
\item 线段上的值:$f_{\text{line}}(0.5) = 2 - 0.5 = 1.5$
\item 实际函数值:$f(2.5) = \sqrt{2.5} \approx 1.581 > 1.5$
\end{itemize}

\textbf{结论}:实际函数值大于线段上的值,说明函数图像在弦的下方,因此是凹函数,不是凸函数。

\section{总结}

\subsection{凸函数的定义回顾}

\textbf{凸函数}:对于任意 $x_1, x_2 \in \text{dom } f$ 和 $\theta \in [0, 1]$,有:

\begin{equation}
f(\theta x_1 + (1-\theta) x_2) \leq \theta f(x_1) + (1-\theta) f(x_2)
\end{equation}

\textbf{几何意义}:函数图像上任意两点的连线在函数图像上方(或重合)。

\subsection{拟凸函数的定义回顾}

\textbf{拟凸函数}:对于任意 $x_1, x_2 \in \text{dom } f$ 和 $\theta \in [0, 1]$,有:

\begin{equation}
f(\theta x_1 + (1-\theta) x_2) \leq \max\{f(x_1), f(x_2)\}
</equation>

\textbf{几何意义}:函数值在任意两点连线上不超过两点的最大值。

\subsection{关键结论}

\begin{enumerate}
\item \textbf{$f(x) = \sqrt{|x|}$ 是拟凸的}:
   \begin{itemize}
   \item 所有子水平集 $S_\alpha = [-\alpha^2, \alpha^2]$ 都是凸集
   \item 满足拟凸性条件
   \end{itemize}

\item \textbf{$f(x) = \sqrt{|x|}$ 不是凸的}:
   \begin{itemize}
   \item 对于 $x > 0$,$f''(x) < 0$(凹函数)
   \item 存在反例:$f(2.5) = 1.581 > 1.5 = 0.5 f(1) + 0.5 f(4)$
   \item 不满足凸性条件
   \end{itemize}

\item \textbf{关系}:
   \begin{itemize}
   \item 凸函数 $\Rightarrow$ 拟凸函数
   \item 拟凸函数 $\not\Rightarrow$ 凸函数
   \item $f(x) = \sqrt{|x|}$ 是一个典型的例子
   \end{itemize}
</enumerate>

\subsection{记忆技巧}

\begin{enumerate}
\item \textbf{凸函数}:函数图像"向下凸",任意弦在图像上方
\item \textbf{拟凸函数}:函数值"不会在中间突然升高",所有子水平集是凸集
\item \textbf{区别}:凸函数要求更严格(加权平均),拟凸函数要求更宽松(最大值)
\end{enumerate}

理解这个反例,有助于区分凸函数和拟凸函数的概念!

\end{document}

