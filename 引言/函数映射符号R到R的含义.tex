\documentclass[12pt,a4paper]{article}
\usepackage[UTF8]{ctex}
\usepackage{amsmath}
\usepackage{amssymb}
\usepackage{amsthm}
\usepackage{geometry}
\geometry{left=2.5cm,right=2.5cm,top=2.5cm,bottom=2.5cm}

\title{函数映射符号 $\mathbb{R} \to \mathbb{R}$ 和 $\mathbb{R}^k \to \mathbb{R}^n$ 的含义}
\subtitle{理解函数的定义域和值域}
\author{}
\date{\today}

\begin{document}

\maketitle

\section{引言}

在数学和优化中,我们经常看到函数表示为 $f: \mathbb{R} \to \mathbb{R}$ 或 $\varphi: \mathbb{R}^k \to \mathbb{R}^n$。理解这些符号的含义对于学习优化理论非常重要。本文将详细解释这些映射符号的含义。

\section{基本符号}

\subsection{$\mathbb{R}$ 的含义}

\textbf{$\mathbb{R}$} 表示实数集(Real numbers),即所有实数的集合。

\textbf{$\mathbb{R}^n$} 表示 $n$ 维实向量空间,即所有 $n$ 维实向量的集合:
\begin{equation}
\mathbb{R}^n = \{(x_1, x_2, \ldots, x_n) \mid x_i \in \mathbb{R}, i = 1, \ldots, n\}
\end{equation}

\subsection{箭头符号 $\to$}

箭头 $\to$ 表示"映射到"或"从...到...",用于表示函数的定义域和值域。

\section{$\mathbb{R} \to \mathbb{R}$ 的含义}

\subsection{定义}

\textbf{$f: \mathbb{R} \to \mathbb{R}$} 表示:

\begin{itemize}
\item $f$ 是一个函数
\item 定义域(domain):$\mathbb{R}$(所有实数)
\item 值域(codomain):$\mathbb{R}$(所有实数)
\item 即:$f$ 将实数映射到实数
\end{itemize}

\subsection{通俗理解}

\begin{itemize}
\item \textbf{输入}:一个实数 $x \in \mathbb{R}$
\item \textbf{输出}:一个实数 $f(x) \in \mathbb{R}$
\item \textbf{例子}:$f(x) = x^2$,$f(x) = \sin x$,$f(x) = e^x$ 等
\end{itemize}

\subsection{具体例子}

\textbf{例子1}:$f: \mathbb{R} \to \mathbb{R}$,$f(x) = x^2$
\begin{itemize}
\item 输入:任意实数 $x$
\item 输出:$x^2$(也是实数)
\item 例如:$f(2) = 4$,$f(-3) = 9$
\end{itemize}

\textbf{例子2}:$f: \mathbb{R} \to \mathbb{R}$,$f(x) = 2x + 1$
\begin{itemize}
\item 输入:任意实数 $x$
\item 输出:$2x + 1$(也是实数)
\item 例如:$f(0) = 1$,$f(5) = 11$
\end{itemize}

\textbf{例子3}:$f: \mathbb{R} \to \mathbb{R}$,$f(x) = \log x$
\begin{itemize}
\item 注意:这里定义域实际上是 $\mathbb{R}_{++} = \{x \mid x > 0\}$,不是整个 $\mathbb{R}$
\item 更准确的表示:$f: \mathbb{R}_{++} \to \mathbb{R}$
\end{itemize}

\section{$\mathbb{R}^k \to \mathbb{R}^n$ 的含义}

\subsection{定义}

\textbf{$\varphi: \mathbb{R}^k \to \mathbb{R}^n$} 表示:

\begin{itemize}
\item $\varphi$ 是一个函数
\item 定义域:$\mathbb{R}^k$($k$ 维实向量空间)
\item 值域:$\mathbb{R}^n$($n$ 维实向量空间)
\item 即:$\varphi$ 将 $k$ 维向量映射到 $n$ 维向量
\end{itemize}

\subsection{通俗理解}

\begin{itemize}
\item \textbf{输入}:一个 $k$ 维向量 $\mathbf{z} \in \mathbb{R}^k$
\item \textbf{输出}:一个 $n$ 维向量 $\varphi(\mathbf{z}) \in \mathbb{R}^n$
\item \textbf{维度}:$k$ 和 $n$ 可以相同,也可以不同
\end{itemize}

\subsection{具体例子}

\textbf{例子1}:$\varphi: \mathbb{R}^2 \to \mathbb{R}^3$,$\varphi(x, y) = (x + y, x - y, 2x)$
\begin{itemize}
\item 输入:2维向量 $(x, y)$
\item 输出:3维向量 $(x + y, x - y, 2x)$
\item 例如:$\varphi(1, 2) = (3, -1, 2)$
\end{itemize}

\textbf{例子2}:$\varphi: \mathbb{R}^3 \to \mathbb{R}^2$,$\varphi(x, y, z) = (x + z, y - z)$
\begin{itemize}
\item 输入:3维向量 $(x, y, z)$
\item 输出:2维向量 $(x + z, y - z)$
\item 例如:$\varphi(1, 2, 3) = (4, -1)$
\end{itemize}

\textbf{例子3}:$\varphi: \mathbb{R}^n \to \mathbb{R}^n$,$\varphi(\mathbf{x}) = \mathbf{A}\mathbf{x} + \mathbf{b}$
\begin{itemize}
\item 输入:$n$ 维向量 $\mathbf{x}$
\item 输出:$n$ 维向量 $\mathbf{A}\mathbf{x} + \mathbf{b}$(仿射变换)
\item 维度相同:都是 $n$ 维
\end{itemize}

\section{在优化问题中的应用}

\subsection{目标函数}

在优化问题中,目标函数通常表示为:
\begin{equation}
f_0: \mathbb{R}^n \to \mathbb{R}
\end{equation}

\textbf{含义}:
\begin{itemize}
\item 输入:$n$ 维优化变量 $\mathbf{x} \in \mathbb{R}^n$
\item 输出:目标函数值 $f_0(\mathbf{x}) \in \mathbb{R}$(一个实数)
\end{itemize}

\textbf{例子}:
\begin{itemize}
\item $f_0(\mathbf{x}) = \|\mathbf{x}\|_2^2$:输入 $n$ 维向量,输出实数
\item $f_0(\mathbf{x}) = \mathbf{c}^T \mathbf{x}$:输入 $n$ 维向量,输出实数
\end{itemize}

\subsection{约束函数}

约束函数也类似:
\begin{itemize}
\item $f_i: \mathbb{R}^n \to \mathbb{R}$:不等式约束函数
\item $h_i: \mathbb{R}^n \to \mathbb{R}$:等式约束函数
\end{itemize}

\textbf{含义}:
\begin{itemize}
\item 输入:$n$ 维优化变量 $\mathbf{x}$
\item 输出:约束函数值(实数)
\end{itemize}

\subsection{变量变换}

在4.1.3节中,变量变换表示为:
\begin{equation}
\varphi: \mathbb{R}^k \to \mathbb{R}^n
\end{equation}

\textbf{含义}:
\begin{itemize}
\item 输入:新变量 $\mathbf{z} \in \mathbb{R}^k$($k$ 维)
\item 输出:原始变量 $\mathbf{x} = \varphi(\mathbf{z}) \in \mathbb{R}^n$($n$ 维)
\item 通过变换 $\mathbf{x} = \varphi(\mathbf{z})$ 将新变量转换为原始变量
\end{itemize}

\textbf{例子}:
\begin{itemize}
\item 如果 $k = n$:$\varphi: \mathbb{R}^n \to \mathbb{R}^n$,例如 $\varphi(\mathbf{z}) = \mathbf{A}\mathbf{z} + \mathbf{b}$
\item 如果 $k < n$:降维变换,例如消除等式约束
\item 如果 $k > n$:升维变换(较少见)
\end{itemize}

\section{常见表示方法}

\subsection{完整表示}

\begin{itemize}
\item $f: A \to B$:函数 $f$ 从集合 $A$ 映射到集合 $B$
\item $A$ 是定义域(domain)
\item $B$ 是值域(codomain)
\end{itemize}

\subsection{定义域的限制}

有时定义域不是整个 $\mathbb{R}^n$,而是子集:

\begin{itemize}
\item $f: \mathbb{R}^n \to \mathbb{R}$:定义域是整个 $\mathbb{R}^n$
\item $f: \mathcal{D} \to \mathbb{R}$:定义域是 $\mathcal{D} \subseteq \mathbb{R}^n$
\item 例如:$f: \mathbb{R}_{++}^n \to \mathbb{R}$(定义域是所有分量都为正的向量)
\end{itemize}

\section{维度关系}

\subsection{$k = n$ 的情况}

\textbf{$\varphi: \mathbb{R}^n \to \mathbb{R}^n$}

\begin{itemize}
\item 输入和输出维度相同
\item 例如:线性变换、旋转、缩放
\item 如果 $\varphi$ 可逆,可以双向转换
\end{itemize}

\textbf{例子}:
\begin{equation}
\varphi(\mathbf{z}) = \mathbf{A}\mathbf{z} + \mathbf{b}
\end{equation}
其中 $\mathbf{A}$ 是 $n \times n$ 矩阵。

\subsection{$k < n$ 的情况}

\textbf{$\varphi: \mathbb{R}^k \to \mathbb{R}^n$}

\begin{itemize}
\item 输入维度小于输出维度
\item 例如:从低维空间嵌入到高维空间
\item 在优化中,常用于消除等式约束
\end{itemize}

\textbf{例子}(消除等式约束):
\begin{itemize}
\item 原始问题:$n$ 个变量,$p$ 个等式约束
\item 等式约束将可行域限制在 $n-p$ 维子空间中
\item 通过 $\varphi: \mathbb{R}^{n-p} \to \mathbb{R}^n$ 参数化这个子空间
\item 新问题只有 $n-p$ 个变量,没有等式约束
\end{itemize}

\textbf{具体例子}:
\begin{itemize}
\item 约束:$x_1 + x_2 + x_3 = 1$(在 $\mathbb{R}^3$ 中)
\item 参数化:$\varphi(z_1, z_2) = (z_1, z_2, 1 - z_1 - z_2)$
\item $\varphi: \mathbb{R}^2 \to \mathbb{R}^3$(从2维到3维)
\end{itemize}

\subsection{$k > n$ 的情况}

\textbf{$\varphi: \mathbb{R}^k \to \mathbb{R}^n$}

\begin{itemize}
\item 输入维度大于输出维度
\item 例如:投影、降维
\item 较少见,但可能用于某些特殊变换
\end{itemize}

\section{具体例子详解}

\subsection{例子1:一元函数}

\textbf{$f: \mathbb{R} \to \mathbb{R}$,$f(x) = x^2 + 1$}

\begin{itemize}
\item 输入:实数 $x$
\item 输出:实数 $x^2 + 1$
\item 例如:$f(0) = 1$,$f(2) = 5$,$f(-3) = 10$
\end{itemize}

\subsection{例子2:多元函数}

\textbf{$f: \mathbb{R}^2 \to \mathbb{R}$,$f(x, y) = x^2 + y^2$}

\begin{itemize}
\item 输入:2维向量 $(x, y)$
\item 输出:实数 $x^2 + y^2$
\item 例如:$f(1, 1) = 2$,$f(3, 4) = 25$
\end{itemize}

\subsection{例子3:向量值函数}

\textbf{$\varphi: \mathbb{R}^2 \to \mathbb{R}^3$,$\varphi(x, y) = (x + y, x - y, xy)$}

\begin{itemize}
\item 输入:2维向量 $(x, y)$
\item 输出:3维向量 $(x + y, x - y, xy)$
\item 例如:$\varphi(1, 2) = (3, -1, 2)$
\end{itemize}

\subsection{例子4:线性变换}

\textbf{$\varphi: \mathbb{R}^n \to \mathbb{R}^m$,$\varphi(\mathbf{x}) = \mathbf{A}\mathbf{x}$}

其中 $\mathbf{A}$ 是 $m \times n$ 矩阵。

\begin{itemize}
\item 输入:$n$ 维向量 $\mathbf{x}$
\item 输出:$m$ 维向量 $\mathbf{A}\mathbf{x}$
\item 例如:如果 $\mathbf{A} = \begin{pmatrix} 1 & 2 \\ 3 & 4 \end{pmatrix}$,则 $\varphi: \mathbb{R}^2 \to \mathbb{R}^2$
\end{itemize}

\section{在4.1.3节中的应用}

\subsection{变量变换}

在4.1.3节中,变量变换表示为:
\begin{equation}
\varphi: \mathbb{R}^k \to \mathbb{R}^n
\end{equation}

\textbf{含义}:
\begin{itemize}
\item 原始问题有 $n$ 个变量 $\mathbf{x} \in \mathbb{R}^n$
\item 通过变换 $\mathbf{x} = \varphi(\mathbf{z})$,引入新变量 $\mathbf{z} \in \mathbb{R}^k$
\item 新问题有 $k$ 个变量
\item 如果 $k < n$,通常是因为消除了某些约束(如等式约束)
\end{itemize}

\textbf{例子}(消除等式约束):
\begin{itemize}
\item 原始问题:3个变量,1个等式约束 $x_1 + x_2 + x_3 = 1$
\item 参数化:$\varphi(z_1, z_2) = (z_1, z_2, 1 - z_1 - z_2)$
\item $\varphi: \mathbb{R}^2 \to \mathbb{R}^3$
\item 新问题:2个变量,0个等式约束
\end{itemize}

\section{记忆技巧}

\subsection{符号理解}

\begin{itemize}
\item $\mathbb{R}$:实数(Real numbers)
\item $\mathbb{R}^n$:$n$ 维实向量空间
\item $\to$:映射到、从...到...
\item $f: A \to B$:函数 $f$ 从 $A$ 映射到 $B$
\end{itemize}

\subsection{维度理解}

\begin{itemize}
\item $f: \mathbb{R}^n \to \mathbb{R}$:$n$ 维输入,1维输出(标量函数)
\item $\varphi: \mathbb{R}^k \to \mathbb{R}^n$:$k$ 维输入,$n$ 维输出(向量值函数)
\item 维度可以相同或不同
\end{itemize}

\section{总结}

\begin{enumerate}
\item \textbf{$\mathbb{R} \to \mathbb{R}$}:
   \begin{itemize}
   \item 一元函数:实数到实数
   \item 输入:一个实数
   \item 输出:一个实数
   \end{itemize}

\item \textbf{$\mathbb{R}^k \to \mathbb{R}^n$}:
   \begin{itemize}
   \item 向量值函数:$k$ 维向量到 $n$ 维向量
   \item 输入:$k$ 维向量
   \item 输出:$n$ 维向量
   \item $k$ 和 $n$ 可以相同或不同
   \end{itemize}

\item \textbf{在优化中}:
   \begin{itemize}
   \item 目标函数:$\mathbb{R}^n \to \mathbb{R}$($n$ 维输入,实数输出)
   \item 约束函数:$\mathbb{R}^n \to \mathbb{R}$($n$ 维输入,实数输出)
   \item 变量变换:$\mathbb{R}^k \to \mathbb{R}^n$($k$ 维新变量到 $n$ 维原始变量)
   \end{itemize}

\item \textbf{关键理解}:
   \begin{itemize}
   \item 箭头左边是定义域(输入)
   \item 箭头右边是值域(输出)
   \item 维度表示向量的维数
   \end{itemize}
\end{enumerate}

理解这些符号,是学习优化理论的基础!

\end{document}


