\documentclass[12pt,a4paper]{article}
\usepackage[UTF8]{ctex}
\usepackage{amsmath}
\usepackage{amssymb}
\usepackage{amsthm}
\usepackage{geometry}
\geometry{left=2.5cm,right=2.5cm,top=2.5cm,bottom=2.5cm}

\title{为什么列空间定义为 $R(\mathbf{A}) = \{\mathbf{A} \mathbf{x} \mid \mathbf{x} \in \mathbb{R}^n\}$?}
\author{}
\date{\today}

\begin{document}

\maketitle

\section{问题提出}

\textbf{列空间的定义}:对于矩阵 $\mathbf{A} \in \mathbb{R}^{m \times n}$,其列空间定义为:
\begin{equation}
R(\mathbf{A}) = \{\mathbf{A} \mathbf{x} \mid \mathbf{x} \in \mathbb{R}^n\}
\end{equation}

\textbf{问题}:为什么可以这样定义?这个定义与"列向量的线性组合"的定义有什么关系?

\section{列空间的标准定义}

\subsection{定义1:列向量的线性组合}

对于矩阵 $\mathbf{A} \in \mathbb{R}^{m \times n}$,设其列向量为 $\mathbf{a}_1, \mathbf{a}_2, \ldots, \mathbf{a}_n$,则:

\begin{equation}
R(\mathbf{A}) = \text{span}\{\mathbf{a}_1, \mathbf{a}_2, \ldots, \mathbf{a}_n\} = \left\{\sum_{i=1}^n \alpha_i \mathbf{a}_i \mid \alpha_1, \ldots, \alpha_n \in \mathbb{R}\right\}
\end{equation}

\textbf{含义}:列空间是所有列向量的线性组合的集合。

\subsection{定义2:矩阵与向量的乘积}

列空间也可以定义为:
\begin{equation}
R(\mathbf{A}) = \{\mathbf{A} \mathbf{x} \mid \mathbf{x} \in \mathbb{R}^n\}
\end{equation}

\textbf{含义}:列空间是所有形如 $\mathbf{A} \mathbf{x}$ 的向量的集合,其中 $\mathbf{x}$ 取遍 $\mathbb{R}^n$。

\section{两种定义的等价性证明}

\subsection{定理}

\textbf{定理}:对于矩阵 $\mathbf{A} \in \mathbb{R}^{m \times n}$,以下两种定义等价:
\begin{enumerate}
\item $R(\mathbf{A}) = \text{span}\{\mathbf{a}_1, \mathbf{a}_2, \ldots, \mathbf{a}_n\}$
\item $R(\mathbf{A}) = \{\mathbf{A} \mathbf{x} \mid \mathbf{x} \in \mathbb{R}^n\}$
\end{enumerate}

\subsection{证明:方向1(定义2 $\subseteq$ 定义1)}

\textbf{目标}:证明 $\{\mathbf{A} \mathbf{x} \mid \mathbf{x} \in \mathbb{R}^n\} \subseteq \text{span}\{\mathbf{a}_1, \ldots, \mathbf{a}_n\}$

\textbf{步骤1}:设 $\mathbf{w} \in \{\mathbf{A} \mathbf{x} \mid \mathbf{x} \in \mathbb{R}^n\}$

根据定义,存在 $\mathbf{x} \in \mathbb{R}^n$,使得 $\mathbf{w} = \mathbf{A} \mathbf{x}$。

\textbf{步骤2}:展开矩阵乘法

设 $\mathbf{A} = \begin{pmatrix} \mathbf{a}_1 & \mathbf{a}_2 & \cdots & \mathbf{a}_n \end{pmatrix}$,$\mathbf{x} = \begin{pmatrix} x_1 \\ x_2 \\ \vdots \\ x_n \end{pmatrix}$,则:

\begin{align}
\mathbf{A} \mathbf{x} &= \begin{pmatrix} \mathbf{a}_1 & \mathbf{a}_2 & \cdots & \mathbf{a}_n \end{pmatrix} \begin{pmatrix} x_1 \\ x_2 \\ \vdots \\ x_n \end{pmatrix} \\
&= x_1 \mathbf{a}_1 + x_2 \mathbf{a}_2 + \cdots + x_n \mathbf{a}_n
\end{align}

\textbf{步骤3}:结论

由于 $\mathbf{w} = x_1 \mathbf{a}_1 + x_2 \mathbf{a}_2 + \cdots + x_n \mathbf{a}_n$,这是列向量的线性组合,因此:
\begin{equation}
\mathbf{w} \in \text{span}\{\mathbf{a}_1, \mathbf{a}_2, \ldots, \mathbf{a}_n\}
\end{equation}

\textbf{因此}:$\{\mathbf{A} \mathbf{x} \mid \mathbf{x} \in \mathbb{R}^n\} \subseteq \text{span}\{\mathbf{a}_1, \ldots, \mathbf{a}_n\}$。$\square$

\subsection{证明:方向2(定义1 $\subseteq$ 定义2)}

\textbf{目标}:证明 $\text{span}\{\mathbf{a}_1, \ldots, \mathbf{a}_n\} \subseteq \{\mathbf{A} \mathbf{x} \mid \mathbf{x} \in \mathbb{R}^n\}$

\textbf{步骤1}:设 $\mathbf{w} \in \text{span}\{\mathbf{a}_1, \ldots, \mathbf{a}_n\}$

根据定义,存在 $\alpha_1, \ldots, \alpha_n \in \mathbb{R}$,使得:
\begin{equation}
\mathbf{w} = \alpha_1 \mathbf{a}_1 + \alpha_2 \mathbf{a}_2 + \cdots + \alpha_n \mathbf{a}_n
\end{equation}

\textbf{步骤2}:构造向量 $\mathbf{x}$

令 $\mathbf{x} = \begin{pmatrix} \alpha_1 \\ \alpha_2 \\ \vdots \\ \alpha_n \end{pmatrix} \in \mathbb{R}^n$。

\textbf{步骤3}:计算 $\mathbf{A} \mathbf{x}$

\begin{align}
\mathbf{A} \mathbf{x} &= \begin{pmatrix} \mathbf{a}_1 & \mathbf{a}_2 & \cdots & \mathbf{a}_n \end{pmatrix} \begin{pmatrix} \alpha_1 \\ \alpha_2 \\ \vdots \\ \alpha_n \end{pmatrix} \\
&= \alpha_1 \mathbf{a}_1 + \alpha_2 \mathbf{a}_2 + \cdots + \alpha_n \mathbf{a}_n \\
&= \mathbf{w}
\end{align}

\textbf{步骤4}:结论

由于 $\mathbf{w} = \mathbf{A} \mathbf{x}$,且 $\mathbf{x} \in \mathbb{R}^n$,因此:
\begin{equation}
\mathbf{w} \in \{\mathbf{A} \mathbf{x} \mid \mathbf{x} \in \mathbb{R}^n\}
\end{equation}

\textbf{因此}:$\text{span}\{\mathbf{a}_1, \ldots, \mathbf{a}_n\} \subseteq \{\mathbf{A} \mathbf{x} \mid \mathbf{x} \in \mathbb{R}^n\}$。$\square$

\subsection{等价性结论}

由于两个方向都成立,因此:
\begin{equation}
R(\mathbf{A}) = \text{span}\{\mathbf{a}_1, \ldots, \mathbf{a}_n\} = \{\mathbf{A} \mathbf{x} \mid \mathbf{x} \in \mathbb{R}^n\}
\end{equation}

两种定义完全等价!

\section{具体例子}

\subsection{例子1:简单矩阵}

\textbf{矩阵}:$\mathbf{A} = \begin{pmatrix} 1 & 2 \\ 3 & 4 \end{pmatrix}$

\textbf{列向量}:$\mathbf{a}_1 = \begin{pmatrix} 1 \\ 3 \end{pmatrix}$,$\mathbf{a}_2 = \begin{pmatrix} 2 \\ 4 \end{pmatrix}$

\textbf{定义1(线性组合)}:
\begin{equation}
R(\mathbf{A}) = \left\{\alpha_1 \begin{pmatrix} 1 \\ 3 \end{pmatrix} + \alpha_2 \begin{pmatrix} 2 \\ 4 \end{pmatrix} \mid \alpha_1, \alpha_2 \in \mathbb{R}\right\}
\end{equation}

\textbf{定义2(矩阵乘积)}:
\begin{equation}
R(\mathbf{A}) = \left\{\begin{pmatrix} 1 & 2 \\ 3 & 4 \end{pmatrix} \begin{pmatrix} x_1 \\ x_2 \end{pmatrix} \mid x_1, x_2 \in \mathbb{R}\right\}
\end{equation}

\textbf{验证等价性}:

对于任意 $\alpha_1, \alpha_2 \in \mathbb{R}$,令 $\mathbf{x} = \begin{pmatrix} \alpha_1 \\ \alpha_2 \end{pmatrix}$,则:
\begin{align}
\mathbf{A} \mathbf{x} &= \begin{pmatrix} 1 & 2 \\ 3 & 4 \end{pmatrix} \begin{pmatrix} \alpha_1 \\ \alpha_2 \end{pmatrix} \\
&= \alpha_1 \begin{pmatrix} 1 \\ 3 \end{pmatrix} + \alpha_2 \begin{pmatrix} 2 \\ 4 \end{pmatrix} \\
&= \alpha_1 \mathbf{a}_1 + \alpha_2 \mathbf{a}_2
\end{align}

因此两种定义给出相同的结果。✓

\subsection{例子2:一般情况}

\textbf{矩阵}:$\mathbf{A} = \begin{pmatrix} 1 & 1 \\ 0 & 0 \end{pmatrix}$

\textbf{列向量}:$\mathbf{a}_1 = \begin{pmatrix} 1 \\ 0 \end{pmatrix}$,$\mathbf{a}_2 = \begin{pmatrix} 1 \\ 0 \end{pmatrix}$

\textbf{定义1}:
\begin{equation}
R(\mathbf{A}) = \left\{\alpha_1 \begin{pmatrix} 1 \\ 0 \end{pmatrix} + \alpha_2 \begin{pmatrix} 1 \\ 0 \end{pmatrix} \mid \alpha_1, \alpha_2 \in \mathbb{R}\right\} = \left\{\alpha \begin{pmatrix} 1 \\ 0 \end{pmatrix} \mid \alpha \in \mathbb{R}\right\}
\end{equation}

\textbf{定义2}:
\begin{equation}
R(\mathbf{A}) = \left\{\begin{pmatrix} 1 & 1 \\ 0 & 0 \end{pmatrix} \begin{pmatrix} x_1 \\ x_2 \end{pmatrix} \mid x_1, x_2 \in \mathbb{R}\right\} = \left\{\begin{pmatrix} x_1 + x_2 \\ 0 \end{pmatrix} \mid x_1, x_2 \in \mathbb{R}\right\} = \left\{\begin{pmatrix} \alpha \\ 0 \end{pmatrix} \mid \alpha \in \mathbb{R}\right\}
\end{equation}

\textbf{验证}:两种定义都给出 $\text{span}\{(1, 0)^T\}$,即通过原点的直线($x$ 轴)。✓

\section{为什么使用 $\{\mathbf{A} \mathbf{x} \mid \mathbf{x} \in \mathbb{R}^n\}$ 这个定义?}

\subsection{优势1:简洁性}

\textbf{优势}:$\{\mathbf{A} \mathbf{x} \mid \mathbf{x} \in \mathbb{R}^n\}$ 这个定义更加简洁,不需要显式写出列向量。

\textbf{例子}:对于 $m \times n$ 矩阵 $\mathbf{A}$,直接写 $\{\mathbf{A} \mathbf{x} \mid \mathbf{x} \in \mathbb{R}^n\}$ 比列出所有列向量更简洁。

\subsection{优势2:与矩阵乘法的联系}

\textbf{优势}:这个定义直接体现了列空间与矩阵乘法的关系。

\textbf{含义}:
\begin{itemize}
\item 列空间是矩阵 $\mathbf{A}$ 的"值域"(range)
\item 即所有可能的输出 $\mathbf{A} \mathbf{x}$ 的集合
\item 这与函数的值域概念一致
\end{itemize}

\subsection{优势3:在优化中的应用}

\textbf{优势}:在优化问题中,经常需要判断一个向量是否在列空间中。

\textbf{应用}:
\begin{itemize}
\item 判断 $\mathbf{b} \in R(\mathbf{A})$ 等价于判断方程 $\mathbf{A}\mathbf{x} = \mathbf{b}$ 是否有解
\item 判断 $\nabla f_0(\mathbf{x}) \in R(\mathbf{A}^T)$ 是最优性条件
\end{itemize}

\textbf{例子}:在等式约束优化中,最优性条件:
\begin{equation}
\nabla f_0(\mathbf{x}) \in R(\mathbf{A}^T) \quad \Leftrightarrow \quad \exists \boldsymbol{\nu} \in \mathbb{R}^p: \nabla f_0(\mathbf{x}) = \mathbf{A}^T \boldsymbol{\nu}
\end{equation}

这个表示形式更加直观和便于使用。

\subsection{优势4:与线性方程组的联系}

\textbf{优势}:列空间的定义与线性方程组的解存在性直接相关。

\textbf{定理}:线性方程组 $\mathbf{A}\mathbf{x} = \mathbf{b}$ 有解,当且仅当 $\mathbf{b} \in R(\mathbf{A})$。

\textbf{等价表述}:$\mathbf{b} \in R(\mathbf{A}) \Leftrightarrow \exists \mathbf{x} \in \mathbb{R}^n: \mathbf{A}\mathbf{x} = \mathbf{b}$

这个表示形式直接体现了"存在性"。

\section{矩阵乘法的几何意义}

\subsection{矩阵作为线性映射}

\textbf{观点}:矩阵 $\mathbf{A}$ 可以看作从 $\mathbb{R}^n$ 到 $\mathbb{R}^m$ 的线性映射:
\begin{equation}
\mathbf{A}: \mathbb{R}^n \to \mathbb{R}^m, \quad \mathbf{x} \mapsto \mathbf{A} \mathbf{x}
\end{equation}

\textbf{列空间}:$R(\mathbf{A})$ 就是这个线性映射的\textbf{值域}(range),即所有可能的输出值的集合。

\textbf{定义}:
\begin{equation}
R(\mathbf{A}) = \{\mathbf{A} \mathbf{x} \mid \mathbf{x} \in \mathbb{R}^n\} = \text{range}(\mathbf{A})
\end{equation}

\subsection{几何直观}

\textbf{输入空间}:$\mathbb{R}^n$(所有可能的输入 $\mathbf{x}$)

\textbf{输出空间}:$\mathbb{R}^m$(所有可能的输出 $\mathbf{A} \mathbf{x}$)

\textbf{列空间}:$\mathbf{A}$ 将 $\mathbb{R}^n$ 映射到的子空间,即 $R(\mathbf{A}) \subseteq \mathbb{R}^m$

\textbf{维度}:$\dim(R(\mathbf{A})) = \text{rank}(\mathbf{A}) \leq \min(m, n)$

\section{总结}

\subsection{两种等价定义}

\begin{enumerate}
\item \textbf{定义1(列向量的线性组合)}:
\begin{equation}
R(\mathbf{A}) = \text{span}\{\mathbf{a}_1, \mathbf{a}_2, \ldots, \mathbf{a}_n\}
\end{equation}

\item \textbf{定义2(矩阵与向量的乘积)}:
\begin{equation}
R(\mathbf{A}) = \{\mathbf{A} \mathbf{x} \mid \mathbf{x} \in \mathbb{R}^n\}
\end{equation}
\end{enumerate}

\subsection{等价性}

两种定义完全等价,因为:
\begin{itemize}
\item 矩阵乘法 $\mathbf{A} \mathbf{x}$ 就是列向量的线性组合
\item 列向量的线性组合可以写成 $\mathbf{A} \mathbf{x}$ 的形式
\end{itemize}

\subsection{为什么使用定义2?}

\begin{enumerate}
\item \textbf{简洁性}:不需要显式列出列向量

\item \textbf{与矩阵乘法的联系}:直接体现列空间是矩阵的值域

\item \textbf{在优化中的应用}:便于判断向量是否在列空间中

\item \textbf{与线性方程组的联系}:直接体现解的存在性
\end{enumerate}

\subsection{关键理解}

\begin{itemize}
\item 列空间是矩阵的"值域"(所有可能的输出)
\item $\mathbf{A} \mathbf{x}$ 表示列向量的线性组合
\item 两种定义从不同角度描述同一个集合
\end{itemize}

理解这个等价性,是理解列空间概念的关键!

\end{document}

