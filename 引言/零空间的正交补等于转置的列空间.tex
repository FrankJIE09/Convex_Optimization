\documentclass[12pt,a4paper]{article}
\usepackage[UTF8]{ctex}
\usepackage{amsmath}
\usepackage{amssymb}
\usepackage{amsthm}
\usepackage{geometry}
\geometry{left=2.5cm,right=2.5cm,top=2.5cm,bottom=2.5cm}

\title{为什么 $N(\mathbf{A})^\perp = R(\mathbf{A}^T)$?}
\subtitle{零空间的正交补等于转置矩阵的列空间}
\author{}
\date{\today}

\begin{document}

\maketitle

\section{问题提出}

\textbf{定理}:对于矩阵 $\mathbf{A} \in \mathbb{R}^{m \times n}$,有:

\begin{equation}
N(\mathbf{A})^\perp = R(\mathbf{A}^T)
\end{equation}

\textbf{问题}:为什么零空间的正交补等于转置矩阵的列空间?

\section{符号说明}

\subsection{零空间}

\textbf{零空间}:$N(\mathbf{A}) = \{\mathbf{v} \in \mathbb{R}^n \mid \mathbf{A}\mathbf{v} = \mathbf{0}\}$

\textbf{含义}:所有被 $\mathbf{A}$ 映射到零向量的向量的集合。

\subsection{列空间}

\textbf{列空间}:$R(\mathbf{A}^T) = \{\mathbf{A}^T \mathbf{y} \mid \mathbf{y} \in \mathbb{R}^m\}$

\textbf{含义}:$\mathbf{A}^T$ 的所有列向量的线性组合的集合。

\textbf{等价表述}:$R(\mathbf{A}^T) = \{\mathbf{A}^T \mathbf{y} \mid \mathbf{y} \in \mathbb{R}^m\}$ 是 $\mathbf{A}$ 的行空间。

\subsection{正交补}

\textbf{正交补}:对于子空间 $V \subseteq \mathbb{R}^n$,其正交补定义为:

\begin{equation}
V^\perp = \{\mathbf{w} \in \mathbb{R}^n \mid \mathbf{w}^T \mathbf{v} = 0 \text{ 对所有 } \mathbf{v} \in V\}
\end{equation}

\textbf{含义}:所有与 $V$ 中每个向量都垂直的向量的集合。

\section{定理的证明}

\subsection{方向1:$R(\mathbf{A}^T) \subseteq N(\mathbf{A})^\perp$}

\textbf{目标}:证明如果 $\mathbf{w} \in R(\mathbf{A}^T)$,则 $\mathbf{w} \in N(\mathbf{A})^\perp$。

\textbf{步骤1}:设 $\mathbf{w} \in R(\mathbf{A}^T)$。

根据列空间的定义,存在 $\mathbf{y} \in \mathbb{R}^m$,使得 $\mathbf{w} = \mathbf{A}^T \mathbf{y}$。

\textbf{步骤2}:取任意 $\mathbf{v} \in N(\mathbf{A})$。

根据零空间的定义,$\mathbf{A}\mathbf{v} = \mathbf{0}$。

\textbf{步骤3}:计算内积。

\begin{align}
\mathbf{w}^T \mathbf{v} &= (\mathbf{A}^T \mathbf{y})^T \mathbf{v} \\
&= \mathbf{y}^T \mathbf{A} \mathbf{v} \\
&= \mathbf{y}^T \mathbf{0} \\
&= 0
\end{align}

\textbf{步骤4}:结论。

由于 $\mathbf{w}^T \mathbf{v} = 0$ 对所有 $\mathbf{v} \in N(\mathbf{A})$ 成立,因此 $\mathbf{w} \in N(\mathbf{A})^\perp$。

\textbf{因此}:$R(\mathbf{A}^T) \subseteq N(\mathbf{A})^\perp$。$\square$

\subsection{方向2:$N(\mathbf{A})^\perp \subseteq R(\mathbf{A}^T)$}

\textbf{目标}:证明如果 $\mathbf{w} \in N(\mathbf{A})^\perp$,则 $\mathbf{w} \in R(\mathbf{A}^T)$。

\textbf{方法}:使用维度论证。

\textbf{步骤1}:维度关系。

\begin{itemize}
\item $\dim N(\mathbf{A}) + \dim R(\mathbf{A}) = n$(秩-零度定理)
\item $\dim R(\mathbf{A}) = \dim R(\mathbf{A}^T)$(行秩 = 列秩)
\item 因此:$\dim N(\mathbf{A}) + \dim R(\mathbf{A}^T) = n$
\end{itemize}

\textbf{步骤2}:正交补的维度。

\begin{itemize}
\item $\dim N(\mathbf{A}) + \dim N(\mathbf{A})^\perp = n$(正交补的性质)
\item 因此:$\dim N(\mathbf{A})^\perp = n - \dim N(\mathbf{A}) = \dim R(\mathbf{A}^T)$
\end{itemize}

\textbf{步骤3}:包含关系。

\begin{itemize}
\item 我们已经证明:$R(\mathbf{A}^T) \subseteq N(\mathbf{A})^\perp$
\item 维度相同:$\dim R(\mathbf{A}^T) = \dim N(\mathbf{A})^\perp$
\item 因此:$R(\mathbf{A}^T) = N(\mathbf{A})^\perp$
\end{itemize}

\textbf{因此}:$N(\mathbf{A})^\perp \subseteq R(\mathbf{A}^T)$。$\square$

\subsection{等价性}

\textbf{结论}:由于 $R(\mathbf{A}^T) \subseteq N(\mathbf{A})^\perp$ 且 $N(\mathbf{A})^\perp \subseteq R(\mathbf{A}^T)$,有:

\begin{equation}
N(\mathbf{A})^\perp = R(\mathbf{A}^T)
\end{equation}

\section{详细解释}

\subsection{为什么 $\mathbf{w} = \mathbf{A}^T \mathbf{y}$ 与 $N(\mathbf{A})$ 垂直?}

\textbf{关键计算}:

\begin{align}
\mathbf{w}^T \mathbf{v} &= (\mathbf{A}^T \mathbf{y})^T \mathbf{v} \\
&= \mathbf{y}^T \mathbf{A} \mathbf{v} \\
&= \mathbf{y}^T \mathbf{0} = 0
\end{align}

\textbf{解释}:
\begin{itemize}
\item $\mathbf{w} = \mathbf{A}^T \mathbf{y}$ 是 $\mathbf{A}$ 的行向量的线性组合
\item $\mathbf{v} \in N(\mathbf{A})$ 意味着 $\mathbf{A}\mathbf{v} = \mathbf{0}$
\item 因此 $\mathbf{w}^T \mathbf{v} = \mathbf{y}^T \mathbf{A}\mathbf{v} = 0$
\end{itemize}

\subsection{几何直观}

\textbf{行空间与零空间的关系}:
\begin{itemize}
\item $\mathbf{A}$ 的行向量张成行空间 $R(\mathbf{A}^T)$
\item 零空间 $N(\mathbf{A})$ 是所有与行向量垂直的向量
\item 行空间与零空间是正交补的关系
\end{itemize}

\textbf{二维情况}:
\begin{itemize}
\item 如果 $\mathbf{A}$ 的行向量是 $(1, 0)$,则行空间是 $x$ 轴
\item 零空间是 $y$ 轴(与 $x$ 轴垂直)
\item $x$ 轴的正交补是 $y$ 轴
\end{itemize}

\section{具体例子}

\subsection{例子1:简单矩阵}

\textbf{矩阵}:$\mathbf{A} = \begin{pmatrix} 1 & 0 \\ 0 & 0 \end{pmatrix}$

\textbf{零空间}:$N(\mathbf{A}) = \{\mathbf{v} \mid \mathbf{A}\mathbf{v} = \mathbf{0}\}$

$\mathbf{A}\mathbf{v} = \begin{pmatrix} v_1 \\ 0 \end{pmatrix} = \mathbf{0}$,因此 $v_1 = 0$,$v_2$ 任意。

\begin{equation}
N(\mathbf{A}) = \{(0, t)^T \mid t \in \mathbb{R}\} = \text{span}\{(0, 1)^T\}
\end{equation}

\textbf{转置矩阵}:$\mathbf{A}^T = \begin{pmatrix} 1 & 0 \\ 0 & 0 \end{pmatrix}$

\textbf{列空间}:$R(\mathbf{A}^T) = \text{span}\{(1, 0)^T, (0, 0)^T\} = \text{span}\{(1, 0)^T\}$

\textbf{正交补}:$N(\mathbf{A})^\perp = \{\mathbf{w} \mid \mathbf{w}^T \mathbf{v} = 0 \text{ 对所有 } \mathbf{v} \in N(\mathbf{A})\}$

对于 $\mathbf{v} = (0, t)^T$,$\mathbf{w}^T \mathbf{v} = w_2 t = 0$ 对所有 $t$ 成立,因此 $w_2 = 0$。

\begin{equation}
N(\mathbf{A})^\perp = \{(w_1, 0)^T \mid w_1 \in \mathbb{R}\} = \text{span}\{(1, 0)^T\}
\end{equation}

\textbf{验证}:$N(\mathbf{A})^\perp = R(\mathbf{A}^T) = \text{span}\{(1, 0)^T\}$ ✓

\subsection{例子2:一般情况}

\textbf{矩阵}:$\mathbf{A} = \begin{pmatrix} 1 & 1 \\ 0 & 0 \end{pmatrix}$

\textbf{零空间}:$N(\mathbf{A}) = \{\mathbf{v} \mid v_1 + v_2 = 0\} = \text{span}\{(1, -1)^T\}$

\textbf{转置矩阵}:$\mathbf{A}^T = \begin{pmatrix} 1 & 0 \\ 1 & 0 \end{pmatrix}$

\textbf{列空间}:$R(\mathbf{A}^T) = \text{span}\{(1, 1)^T\}$

\textbf{正交补}:$N(\mathbf{A})^\perp = \{\mathbf{w} \mid \mathbf{w}^T (1, -1)^T = 0\} = \{\mathbf{w} \mid w_1 - w_2 = 0\} = \text{span}\{(1, 1)^T\}$

\textbf{验证}:$N(\mathbf{A})^\perp = R(\mathbf{A}^T) = \text{span}\{(1, 1)^T\}$ ✓

\section{在最优性条件中的应用}

\subsection{最优性条件}

\textbf{条件}:$\nabla f_0(\mathbf{x}) \perp N(\mathbf{A})$

\textbf{等价表述}:$\nabla f_0(\mathbf{x}) \in N(\mathbf{A})^\perp$

\textbf{使用定理}:$N(\mathbf{A})^\perp = R(\mathbf{A}^T)$

\textbf{因此}:$\nabla f_0(\mathbf{x}) \in R(\mathbf{A}^T)$

\textbf{含义}:存在 $\boldsymbol{\nu} \in \mathbb{R}^p$,使得:

\begin{equation}
\nabla f_0(\mathbf{x}) = \mathbf{A}^T \boldsymbol{\nu}
\end{equation}

\textbf{等价形式}:

\begin{equation}
\nabla f_0(\mathbf{x}) + \mathbf{A}^T \boldsymbol{\nu} = \mathbf{0}
\end{equation}

这就是拉格朗日乘数法的条件!

\section{几何直观总结}

\subsection{基本关系}

\begin{enumerate}
\item \textbf{行空间} $R(\mathbf{A}^T)$:$\mathbf{A}$ 的行向量张成的空间

\item \textbf{零空间} $N(\mathbf{A})$:所有与行向量垂直的向量

\item \textbf{正交补}:行空间与零空间是正交补的关系
\end{enumerate}

\subsection{维度关系}

\begin{enumerate}
\item $\dim N(\mathbf{A}) + \dim R(\mathbf{A}) = n$

\item $\dim R(\mathbf{A}) = \dim R(\mathbf{A}^T)$

\item $\dim N(\mathbf{A}) + \dim R(\mathbf{A}^T) = n$

\item $\dim N(\mathbf{A}) + \dim N(\mathbf{A})^\perp = n$

\item 因此:$\dim R(\mathbf{A}^T) = \dim N(\mathbf{A})^\perp$
</enumerate}

\section{总结}

\subsection{定理陈述}

\begin{equation}
N(\mathbf{A})^\perp = R(\mathbf{A}^T)
\end{equation}

\subsection{证明思路}

\begin{enumerate}
\item \textbf{方向1}:$R(\mathbf{A}^T) \subseteq N(\mathbf{A})^\perp$
   \begin{itemize}
   \item 对于 $\mathbf{w} = \mathbf{A}^T \mathbf{y} \in R(\mathbf{A}^T)$
   \item 对于任意 $\mathbf{v} \in N(\mathbf{A})$,有 $\mathbf{w}^T \mathbf{v} = \mathbf{y}^T \mathbf{A}\mathbf{v} = 0$
   \item 因此 $\mathbf{w} \in N(\mathbf{A})^\perp$
   \end{itemize}

\item \textbf{方向2}:$N(\mathbf{A})^\perp \subseteq R(\mathbf{A}^T)$
   \begin{itemize}
   \item 使用维度论证
   \item $\dim R(\mathbf{A}^T) = \dim N(\mathbf{A})^\perp$
   \item 结合方向1,得到相等
   \end{itemize}
</enumerate}

\subsection{关键理解}

\begin{enumerate}
\item \textbf{行空间与零空间}:是正交补的关系

\item \textbf{几何意义}:行向量张成的空间与零空间垂直

\item \textbf{在优化中}:用于推导拉格朗日乘数条件
</enumerate}

理解这个定理,是理解等式约束优化理论的关键!

\end{document}

