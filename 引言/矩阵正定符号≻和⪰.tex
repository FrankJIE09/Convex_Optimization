\documentclass[12pt,a4paper]{article}
\usepackage[UTF8]{ctex}
\usepackage{amsmath}
\usepackage{amssymb}
\usepackage{amsthm}
\usepackage{geometry}
\geometry{left=2.5cm,right=2.5cm,top=2.5cm,bottom=2.5cm}

\title{矩阵正定符号 $\succ$ 和 $\succeq$ 的含义}
\author{}
\date{\today}

\begin{document}

\maketitle

\section{引言}

在凸优化中,经常看到符号 $\succ$ 和 $\succeq$ 用于表示矩阵的正定性。理解这些符号对于学习凸优化至关重要。本文将详细解释这些符号的含义、区别和使用场景。

\section{符号 $\succ$:严格正定}

\subsection{定义}

对于对称矩阵 $\mathbf{A} \in \mathbb{R}^{n \times n}$,符号 $\mathbf{A} \succ 0$ 表示 $\mathbf{A}$ 是\textbf{严格正定}(Strictly Positive Definite)矩阵。

\subsection{数学定义}

$\mathbf{A} \succ 0$ 当且仅当对于任意非零向量 $\mathbf{x} \in \mathbb{R}^n$,都有:

\begin{equation}
\mathbf{x}^T \mathbf{A} \mathbf{x} > 0
\end{equation}

\subsection{等价条件}

$\mathbf{A} \succ 0$ 等价于以下条件之一:

\begin{enumerate}
\item \textbf{二次型为正}:对于所有非零 $\mathbf{x}$,$\mathbf{x}^T \mathbf{A} \mathbf{x} > 0$

\item \textbf{所有特征值为正}:$\mathbf{A}$ 的所有特征值 $\lambda_i > 0$

\item \textbf{所有顺序主子式为正}:$\mathbf{A}$ 的所有顺序主子式(leading principal minors)都大于0

\item \textbf{存在Cholesky分解}:$\mathbf{A} = \mathbf{L} \mathbf{L}^T$,其中 $\mathbf{L}$ 是下三角矩阵且对角元素为正

\item \textbf{可逆且逆矩阵也是正定的}:$\mathbf{A}$ 可逆,且 $\mathbf{A}^{-1} \succ 0$
\end{enumerate}

\subsection{例子}

\textbf{例子1}:
\begin{equation}
\mathbf{A} = \begin{pmatrix} 2 & 1 \\ 1 & 2 \end{pmatrix} \succ 0
\end{equation}

验证:对于任意 $(x, y) \neq (0, 0)$,
\begin{align}
(x, y) \begin{pmatrix} 2 & 1 \\ 1 & 2 \end{pmatrix} \begin{pmatrix} x \\ y \end{pmatrix} &= 2x^2 + 2xy + 2y^2 \\
&= 2(x^2 + xy + y^2) > 0
\end{align}

\textbf{例子2}:
\begin{equation}
\mathbf{A} = \begin{pmatrix} 1 & 0 \\ 0 & 1 \end{pmatrix} = \mathbf{I} \succ 0
\end{equation}

单位矩阵是正定的。

\section{符号 $\succeq$:半正定}

\subsection{定义}

对于对称矩阵 $\mathbf{A} \in \mathbb{R}^{n \times n}$,符号 $\mathbf{A} \succeq 0$ 表示 $\mathbf{A}$ 是\textbf{半正定}(Positive Semidefinite)矩阵。

\subsection{数学定义}

$\mathbf{A} \succeq 0$ 当且仅当对于任意向量 $\mathbf{x} \in \mathbb{R}^n$,都有:

\begin{equation}
\mathbf{x}^T \mathbf{A} \mathbf{x} \geq 0
\end{equation}

注意:这里允许等号成立($\geq$),而严格正定要求严格大于($>$)。

\subsection{等价条件}

$\mathbf{A} \succeq 0$ 等价于:

\begin{enumerate}
\item \textbf{二次型非负}:对于所有 $\mathbf{x}$,$\mathbf{x}^T \mathbf{A} \mathbf{x} \geq 0$

\item \textbf{所有特征值非负}:$\mathbf{A}$ 的所有特征值 $\lambda_i \geq 0$

\item \textbf{存在矩阵分解}:$\mathbf{A} = \mathbf{B}^T \mathbf{B}$ 对某个矩阵 $\mathbf{B}$
\end{enumerate}

\subsection{例子}

\textbf{例子1}:
\begin{equation}
\mathbf{A} = \begin{pmatrix} 1 & 1 \\ 1 & 1 \end{pmatrix} \succeq 0
\end{equation}

验证:对于任意 $(x, y)$,
\begin{align}
(x, y) \begin{pmatrix} 1 & 1 \\ 1 & 1 \end{pmatrix} \begin{pmatrix} x \\ y \end{pmatrix} &= x^2 + 2xy + y^2 \\
&= (x + y)^2 \geq 0
\end{align}

注意:当 $x = -y$ 时,等号成立,所以是半正定,不是严格正定。

\textbf{例子2}:
\begin{equation}
\mathbf{A} = \begin{pmatrix} 1 & 0 \\ 0 & 0 \end{pmatrix} \succeq 0
\end{equation}

验证:对于任意 $(x, y)$,
\begin{equation}
(x, y) \begin{pmatrix} 1 & 0 \\ 0 & 0 \end{pmatrix} \begin{pmatrix} x \\ y \end{pmatrix} = x^2 \geq 0
\end{equation}

\section{符号对比}

\subsection{关系}

\begin{itemize}
\item $\mathbf{A} \succ 0$ 意味着 $\mathbf{A} \succeq 0$(严格正定 $\Rightarrow$ 半正定)
\item 但 $\mathbf{A} \succeq 0$ 不一定意味着 $\mathbf{A} \succ 0$(半正定 $\not\Rightarrow$ 严格正定)
\end{itemize}

\subsection{区别总结}

\begin{table}[h]
\centering
\begin{tabular}{|l|l|l|}
\hline
\textbf{符号} & \textbf{名称} & \textbf{条件} \\
\hline
$\mathbf{A} \succ 0$ & 严格正定 & $\mathbf{x}^T \mathbf{A} \mathbf{x} > 0$(对所有非零 $\mathbf{x}$) \\
\hline
$\mathbf{A} \succeq 0$ & 半正定 & $\mathbf{x}^T \mathbf{A} \mathbf{x} \geq 0$(对所有 $\mathbf{x}$) \\
\hline
$\mathbf{A} \prec 0$ & 严格负定 & $\mathbf{x}^T \mathbf{A} \mathbf{x} < 0$(对所有非零 $\mathbf{x}$) \\
\hline
$\mathbf{A} \preceq 0$ & 半负定 & $\mathbf{x}^T \mathbf{A} \mathbf{x} \leq 0$(对所有 $\mathbf{x}$) \\
\hline
\end{tabular}
\caption{矩阵正定性符号}
\end{table}

\section{在凸优化中的应用}

\subsection{椭球定义}

在椭球的定义中:
\begin{equation}
\mathcal{E} = \{\mathbf{x} \mid (\mathbf{x} - \mathbf{x}_c)^T \mathbf{P}^{-1} (\mathbf{x} - \mathbf{x}_c) \leq 1\}
\end{equation}

要求 $\mathbf{P} = \mathbf{P}^T \succ 0$,即 $\mathbf{P}$ 是对称严格正定矩阵。

\textbf{为什么需要严格正定?}
\begin{itemize}
\item 如果 $\mathbf{P}$ 只是半正定,$\mathbf{P}^{-1}$ 可能不存在
\item 即使存在,椭球会"退化"(某些方向没有厚度)
\item 严格正定保证椭球是"正常"的(所有方向都有正厚度)
\end{itemize}

\subsection{半定规划(SDP)}

在半定规划中,约束条件通常为:
\begin{equation}
\mathbf{A}(\mathbf{x}) \succeq 0
\end{equation}

这里使用 $\succeq$ 而不是 $\succ$,因为:
\begin{itemize}
\item 允许边界情况(等号成立)
\item 可行域是闭集
\item 更一般的约束形式
\end{itemize}

\subsection{二次型约束}

在二次型约束中:
\begin{itemize}
\item $\mathbf{x}^T \mathbf{Q} \mathbf{x} \leq 1$,其中 $\mathbf{Q} \succ 0$:定义椭球
\item $\mathbf{x}^T \mathbf{Q} \mathbf{x} \leq 1$,其中 $\mathbf{Q} \succeq 0$:定义更一般的椭球(可能退化)
\end{itemize}

\section{几何直观}

\subsection{严格正定矩阵}

对于 $\mathbf{A} \succ 0$:
\begin{itemize}
\item 集合 $\{\mathbf{x} \mid \mathbf{x}^T \mathbf{A} \mathbf{x} \leq 1\}$ 是一个"正常"的椭球
\item 所有方向都有正厚度
\item 椭球是"有界"的
\end{itemize}

\subsection{半正定矩阵}

对于 $\mathbf{A} \succeq 0$(但 $\mathbf{A} \not\succ 0$):
\begin{itemize}
\item 集合 $\{\mathbf{x} \mid \mathbf{x}^T \mathbf{A} \mathbf{x} \leq 1\}$ 可能是"退化"的椭球
\item 某些方向没有厚度(像"椭圆盘")
\item 椭球可能是"无界"的(如果 $\mathbf{A}$ 奇异)
\end{itemize}

\section{具体例子}

\subsection{例子1:严格正定}

\begin{equation}
\mathbf{A} = \begin{pmatrix} 4 & 0 \\ 0 & 9 \end{pmatrix} \succ 0
\end{equation}

特征值:$\lambda_1 = 4 > 0$,$\lambda_2 = 9 > 0$

椭球:$\frac{x^2}{4} + \frac{y^2}{9} \leq 1$(正常椭圆)

\subsection{例子2:半正定(非严格)}

\begin{equation}
\mathbf{A} = \begin{pmatrix} 4 & 0 \\ 0 & 0 \end{pmatrix} \succeq 0
\end{equation}

特征值:$\lambda_1 = 4 > 0$,$\lambda_2 = 0$

椭球:$x^2 \leq 1$(退化为两条平行线之间的区域)

\subsection{例子3:半正定(非严格)}

\begin{equation}
\mathbf{A} = \begin{pmatrix} 1 & 1 \\ 1 & 1 \end{pmatrix} \succeq 0
\end{equation}

特征值:$\lambda_1 = 2 > 0$,$\lambda_2 = 0$

椭球:$(x + y)^2 \leq 1$(退化为两条平行线之间的区域)

\section{记忆技巧}

\subsection{符号形状}

\begin{itemize}
\item $\succ$:像"大于"符号 $>$,表示严格大于
\item $\succeq$:像"大于等于"符号 $\geq$,表示大于或等于
\item $\prec$:像"小于"符号 $<$,表示严格小于
\item $\preceq$:像"小于等于"符号 $\leq$,表示小于或等于
\end{itemize}

\subsection{方向}

\begin{itemize}
\item 向右($\succ, \succeq$):正定(positive)
\item 向左($\prec, \preceq$):负定(negative)
\end{itemize}

\section{与其他符号的关系}

\subsection{与标量不等式的类比}

\begin{itemize}
\item 标量:$a > 0$ 表示 $a$ 是正数
\item 矩阵:$\mathbf{A} \succ 0$ 表示 $\mathbf{A}$ 是正定矩阵
\item 标量:$a \geq 0$ 表示 $a$ 是非负数
\item 矩阵:$\mathbf{A} \succeq 0$ 表示 $\mathbf{A}$ 是半正定矩阵
\end{itemize}

\subsection{矩阵不等式}

对于矩阵 $\mathbf{A}$ 和 $\mathbf{B}$:
\begin{itemize}
\item $\mathbf{A} \succ \mathbf{B}$ 表示 $\mathbf{A} - \mathbf{B} \succ 0$
\item $\mathbf{A} \succeq \mathbf{B}$ 表示 $\mathbf{A} - \mathbf{B} \succeq 0$
\end{itemize}

这定义了矩阵之间的"偏序关系"。

\section{总结}

\begin{enumerate}
\item \textbf{$\succ$(严格正定)}:
   \begin{itemize}
   \item 条件:$\mathbf{x}^T \mathbf{A} \mathbf{x} > 0$(对所有非零 $\mathbf{x}$)
   \item 特征值:所有特征值 $> 0$
   \item 可逆性:$\mathbf{A}$ 可逆
   \end{itemize}

\item \textbf{$\succeq$(半正定)}:
   \begin{itemize}
   \item 条件:$\mathbf{x}^T \mathbf{A} \mathbf{x} \geq 0$(对所有 $\mathbf{x}$)
   \item 特征值:所有特征值 $\geq 0$
   \item 可逆性:$\mathbf{A}$ 可能不可逆
   \end{itemize}

\item \textbf{关系}:
   \begin{itemize}
   \item 严格正定 $\Rightarrow$ 半正定
   \item 半正定 $\not\Rightarrow$ 严格正定
   \end{itemize}

\item \textbf{在凸优化中}:
   \begin{itemize}
   \item 椭球定义中通常要求 $\mathbf{P} \succ 0$
   \item 半定规划中使用 $\succeq 0$
   \item 理解这些符号对于学习凸优化至关重要
   \end{itemize}
\end{enumerate}

掌握这些符号的含义和区别,是深入学习凸优化的基础!

\end{document}


