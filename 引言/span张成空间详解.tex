\documentclass[12pt,a4paper]{article}
\usepackage[UTF8]{ctex}
\usepackage{amsmath}
\usepackage{amssymb}
\usepackage{amsthm}
\usepackage{geometry}
\geometry{left=2.5cm,right=2.5cm,top=2.5cm,bottom=2.5cm}

\title{Span(张成空间)详解}
\author{}
\date{\today}

\begin{document}

\maketitle

\section{引言}

\textbf{Span}(张成空间)是线性代数中的核心概念之一。它描述了一组向量通过线性组合能够"生成"或"张成"的所有向量构成的集合。理解span对于理解列空间、基、维度等概念至关重要。

\section{定义}

\subsection{基本定义}

对于向量集合 $S = \{\mathbf{v}_1, \mathbf{v}_2, \ldots, \mathbf{v}_k\} \subseteq \mathbb{R}^n$,由 $S$ \textbf{张成}(span)的集合定义为:

\begin{equation}
\text{span}(S) = \text{span}\{\mathbf{v}_1, \mathbf{v}_2, \ldots, \mathbf{v}_k\} = \left\{\sum_{i=1}^k \alpha_i \mathbf{v}_i \mid \alpha_1, \alpha_2, \ldots, \alpha_k \in \mathbb{R}\right\}
\end{equation}

即所有可能的\textbf{线性组合}的集合。

\subsection{为什么 $S \subseteq \mathbb{R}^n$?}

\textbf{关键理解}:需要区分三个不同的"维度"概念:

\begin{enumerate}
\item \textbf{向量的维度}:每个向量 $\mathbf{v}_i$ 本身是 $n$ 维向量,即 $\mathbf{v}_i \in \mathbb{R}^n$

\item \textbf{集合的维度}:集合 $S$ 作为 $\mathbb{R}^n$ 的子集,其元素(向量)都是 $n$ 维的

\item \textbf{Span的维度}:$\text{span}(S)$ 作为子空间的维度,可能小于 $n$(取决于向量的线性无关性)
\end{enumerate}

\textbf{详细解释}:

\begin{itemize}
\item \textbf{每个向量是 $\mathbb{R}^n$ 中的元素}:
  \begin{equation}
  \mathbf{v}_i = \begin{pmatrix} v_{i1} \\ v_{i2} \\ \vdots \\ v_{in} \end{pmatrix} \in \mathbb{R}^n
  \end{equation}
  每个 $\mathbf{v}_i$ 有 $n$ 个分量,所以它是 $\mathbb{R}^n$ 中的一个点(或向量)。

\item \textbf{集合 $S$ 是这些向量的集合}:
  \begin{equation}
  S = \{\mathbf{v}_1, \mathbf{v}_2, \ldots, \mathbf{v}_k\}
  \end{equation}
  由于每个 $\mathbf{v}_i \in \mathbb{R}^n$,所以 $S$ 是 $\mathbb{R}^n$ 的子集,即 $S \subseteq \mathbb{R}^n$。

\item \textbf{Span的维度可能小于 $n$}:
  \begin{itemize}
  \item 虽然每个向量是 $n$ 维的,但 $\text{span}(S)$ 的维度取决于这些向量的线性无关性
  \item 如果 $k$ 个向量线性无关,则 $\dim(\text{span}(S)) = k$(但 $k \leq n$)
  \item 如果向量线性相关,则 $\dim(\text{span}(S)) < k$
  \end{itemize}
\end{itemize}

\textbf{具体例子}:

\textbf{例子1}:在 $\mathbb{R}^3$ 中

设 $\mathbf{v}_1 = (1, 0, 0)$,$\mathbf{v}_2 = (0, 1, 0) \in \mathbb{R}^3$,则:
\begin{itemize}
\item 每个向量是 $3$ 维的:$\mathbf{v}_1, \mathbf{v}_2 \in \mathbb{R}^3$
\item 集合 $S = \{\mathbf{v}_1, \mathbf{v}_2\} \subseteq \mathbb{R}^3$
\item 但 $\text{span}(S) = \{(x, y, 0) \mid x, y \in \mathbb{R}\}$ 是 $2$ 维子空间($xy$ 平面)
\end{itemize}

\textbf{关键观察}:
\begin{itemize}
\item 向量的维度 = $3$(每个向量有 $3$ 个分量)
\item Span的维度 = $2$(张成的是平面,不是整个空间)
\item 集合 $S$ 包含在 $\mathbb{R}^3$ 中,但只包含 $2$ 个点
\end{itemize}

\textbf{例子2}:在 $\mathbb{R}^2$ 中

设 $\mathbf{v}_1 = (1, 0)$,$\mathbf{v}_2 = (2, 0) \in \mathbb{R}^2$,则:
\begin{itemize}
\item 每个向量是 $2$ 维的:$\mathbf{v}_1, \mathbf{v}_2 \in \mathbb{R}^2$
\item 集合 $S = \{\mathbf{v}_1, \mathbf{v}_2\} \subseteq \mathbb{R}^2$
\item 但 $\text{span}(S) = \{(x, 0) \mid x \in \mathbb{R}\}$ 是 $1$ 维子空间(一条直线)
\end{itemize}

\textbf{关键观察}:
\begin{itemize}
\item 向量的维度 = $2$(每个向量有 $2$ 个分量)
\item Span的维度 = $1$(只张成一条直线)
\item 虽然有两个向量,但它们线性相关,所以只张成 $1$ 维空间
\end{itemize}

\textbf{总结}:
\begin{itemize}
\item $S \subseteq \mathbb{R}^n$ 表示:集合 $S$ 中的每个向量都是 $n$ 维向量
\item 这并不意味着 $\text{span}(S)$ 的维度等于 $n$
\item $\text{span}(S)$ 的维度取决于 $S$ 中向量的线性无关性,可能为 $1, 2, \ldots, n$
\end{itemize}

\subsection{为什么不是 $\mathbb{R}^{n \times k}$?}

\textbf{关键区别}:集合 $S$ 和矩阵表示是不同的数学对象。

\begin{enumerate}
\item \textbf{集合 $S$ 是向量的集合}:
  \begin{equation}
  S = \{\mathbf{v}_1, \mathbf{v}_2, \ldots, \mathbf{v}_k\}
  \end{equation}
  \begin{itemize}
  \item 每个 $\mathbf{v}_i$ 是一个 $n$ 维向量:$\mathbf{v}_i \in \mathbb{R}^n$
  \item 集合 $S$ 包含 $k$ 个向量,每个都是 $\mathbb{R}^n$ 中的元素
  \item 所以 $S \subseteq \mathbb{R}^n$(集合是 $\mathbb{R}^n$ 的子集)
  \end{itemize}

\item \textbf{矩阵表示是另一种对象}:
  如果我们把这些向量排列成矩阵:
  \begin{equation}
  \mathbf{A} = \begin{pmatrix} \mathbf{v}_1 & \mathbf{v}_2 & \cdots & \mathbf{v}_k \end{pmatrix} = \begin{pmatrix}
  v_{11} & v_{21} & \cdots & v_{k1} \\
  v_{12} & v_{22} & \cdots & v_{k2} \\
  \vdots & \vdots & \ddots & \vdots \\
  v_{1n} & v_{2n} & \cdots & v_{kn}
  \end{pmatrix}
  \end{equation}
  \begin{itemize}
  \item 矩阵 $\mathbf{A}$ 是 $n \times k$ 的矩阵:$\mathbf{A} \in \mathbb{R}^{n \times k}$
  \item 矩阵的每一列是一个向量 $\mathbf{v}_i$
  \item 矩阵 $\mathbf{A}$ 属于矩阵空间 $\mathbb{R}^{n \times k}$
  \end{itemize}
\end{enumerate}

\textbf{关键理解}:

\begin{itemize}
\item \textbf{集合 $S$}:是 $k$ 个向量的集合,$S \subseteq \mathbb{R}^n$
  \begin{itemize}
  \item 集合的元素是向量
  \item 集合本身是 $\mathbb{R}^n$ 的子集
  \end{itemize}

\item \textbf{矩阵 $\mathbf{A}$}:是把这些向量排列成的矩阵,$\mathbf{A} \in \mathbb{R}^{n \times k}$
  \begin{itemize}
  \item 矩阵是一个 $n \times k$ 的数组
  \item 矩阵属于矩阵空间 $\mathbb{R}^{n \times k}$
  \end{itemize}

\item \textbf{它们的关系}:
  \begin{equation}
  R(\mathbf{A}) = \text{span}\{\mathbf{v}_1, \ldots, \mathbf{v}_k\} = \text{span}(S)
  \end{equation}
  矩阵 $\mathbf{A}$ 的列空间等于集合 $S$ 的span。
\end{itemize}

\textbf{具体例子}:

设 $\mathbf{v}_1 = \begin{pmatrix} 1 \\ 0 \end{pmatrix}$,$\mathbf{v}_2 = \begin{pmatrix} 0 \\ 1 \end{pmatrix}$,则:

\begin{enumerate}
\item \textbf{集合表示}:
  \begin{equation}
  S = \left\{\begin{pmatrix} 1 \\ 0 \end{pmatrix}, \begin{pmatrix} 0 \\ 1 \end{pmatrix}\right\} \subseteq \mathbb{R}^2
  \end{equation}
  \begin{itemize}
  \item $S$ 是 $\mathbb{R}^2$ 的子集(包含2个向量)
  \item 每个向量是 $2$ 维的
  \end{itemize}

\item \textbf{矩阵表示}:
  \begin{equation}
  \mathbf{A} = \begin{pmatrix} 1 & 0 \\ 0 & 1 \end{pmatrix} \in \mathbb{R}^{2 \times 2}
  \end{equation}
  \begin{itemize}
  \item $\mathbf{A}$ 是 $2 \times 2$ 的矩阵
  \item $\mathbf{A}$ 属于矩阵空间 $\mathbb{R}^{2 \times 2}$
  \end{itemize}

\item \textbf{关系}:
  \begin{equation}
  R(\mathbf{A}) = \text{span}(S) = \mathbb{R}^2
  \end{equation}
  矩阵的列空间等于集合的span。
\end{enumerate}

\textbf{类比理解}:

想象一下:
\begin{itemize}
\item \textbf{集合 $S$}:就像是一个"袋子",里面装着 $k$ 个 $n$ 维向量
  \begin{itemize}
  \item 这个"袋子"是 $\mathbb{R}^n$ 的一部分(因为里面的每个向量都属于 $\mathbb{R}^n$)
  \item 所以 $S \subseteq \mathbb{R}^n$
  \end{itemize}

\item \textbf{矩阵 $\mathbf{A}$}:就像是把这些向量"排列"成一个表格
  \begin{itemize}
  \item 这个表格有 $n$ 行、$k$ 列
  \item 这个表格属于矩阵空间 $\mathbb{R}^{n \times k}$
  \end{itemize}
\end{itemize}

\textbf{总结}:
\begin{itemize}
\item 集合 $S = \{\mathbf{v}_1, \ldots, \mathbf{v}_k\} \subseteq \mathbb{R}^n$(向量的集合)
\item 矩阵 $\mathbf{A} = [\mathbf{v}_1 | \cdots | \mathbf{v}_k] \in \mathbb{R}^{n \times k}$(矩阵表示)
\item 它们是不同的数学对象,但密切相关:$R(\mathbf{A}) = \text{span}(S)$
\end{itemize}

\subsection{数学表述}

更精确地,$\text{span}(S)$ 是满足以下条件的集合:
\begin{itemize}
\item 包含 $S$ 中的所有向量(取 $\alpha_i = 1$,其他为0)
\item 包含所有线性组合
\item 是包含 $S$ 的最小线性子空间
\end{itemize}

\section{几何意义}

\subsection{一维情况}

\textbf{例子1}:单个非零向量

设 $\mathbf{v} = (1, 0) \in \mathbb{R}^2$,则:
\begin{equation}
\text{span}\{\mathbf{v}\} = \{\alpha \mathbf{v} \mid \alpha \in \mathbb{R}\} = \{(\alpha, 0) \mid \alpha \in \mathbb{R}\}
\end{equation}

\textbf{几何意义}:这是通过原点的直线,方向为 $\mathbf{v}$。

\textbf{可视化}:
\begin{itemize}
\item 当 $\alpha = 0$ 时,得到原点 $(0, 0)$
\item 当 $\alpha = 1$ 时,得到 $\mathbf{v} = (1, 0)$
\item 当 $\alpha = -1$ 时,得到 $-\mathbf{v} = (-1, 0)$
\item 当 $\alpha = 2$ 时,得到 $2\mathbf{v} = (2, 0)$
\end{itemize}

所有这些点都在同一条直线上。

\subsection{二维情况}

\textbf{例子2}:两个线性无关的向量

设 $\mathbf{v}_1 = (1, 0)$,$\mathbf{v}_2 = (0, 1) \in \mathbb{R}^2$,则:
\begin{equation}
\text{span}\{\mathbf{v}_1, \mathbf{v}_2\} = \{\alpha_1 (1, 0) + \alpha_2 (0, 1) \mid \alpha_1, \alpha_2 \in \mathbb{R}\} = \{(\alpha_1, \alpha_2) \mid \alpha_1, \alpha_2 \in \mathbb{R}\} = \mathbb{R}^2
\end{equation}

\textbf{几何意义}:这两个向量张成整个 $\mathbb{R}^2$ 平面。

\textbf{关键观察}:
\begin{itemize}
\item $\mathbf{v}_1$ 和 $\mathbf{v}_2$ 线性无关
\item 它们张成整个二维空间
\item 任意向量 $(x, y) \in \mathbb{R}^2$ 都可以表示为 $x \mathbf{v}_1 + y \mathbf{v}_2$
\end{itemize}

\textbf{例子3}:两个线性相关的向量

设 $\mathbf{v}_1 = (1, 0)$,$\mathbf{v}_2 = (2, 0) \in \mathbb{R}^2$,则:
\begin{equation}
\text{span}\{\mathbf{v}_1, \mathbf{v}_2\} = \{\alpha_1 (1, 0) + \alpha_2 (2, 0) \mid \alpha_1, \alpha_2 \in \mathbb{R}\} = \{(\alpha_1 + 2\alpha_2, 0) \mid \alpha_1, \alpha_2 \in \mathbb{R}\} = \{(x, 0) \mid x \in \mathbb{R}\}
\end{equation}

\textbf{几何意义}:虽然有两个向量,但它们线性相关($\mathbf{v}_2 = 2\mathbf{v}_1$),所以只张成一条直线,而不是整个平面。

\textbf{关键观察}:
\begin{itemize}
\item $\mathbf{v}_1$ 和 $\mathbf{v}_2$ 线性相关
\item 它们只张成一维子空间(一条直线)
\item $\text{span}\{\mathbf{v}_1, \mathbf{v}_2\} = \text{span}\{\mathbf{v}_1\}$(因为 $\mathbf{v}_2$ 是冗余的)
\end{itemize}

\subsection{三维情况}

\textbf{例子4}:三个向量张成平面

设 $\mathbf{v}_1 = (1, 0, 0)$,$\mathbf{v}_2 = (0, 1, 0)$,$\mathbf{v}_3 = (1, 1, 0) \in \mathbb{R}^3$,则:
\begin{equation}
\text{span}\{\mathbf{v}_1, \mathbf{v}_2, \mathbf{v}_3\} = \{(x, y, 0) \mid x, y \in \mathbb{R}\}
\end{equation}

\textbf{几何意义}:虽然有三个向量,但 $\mathbf{v}_3 = \mathbf{v}_1 + \mathbf{v}_2$ 是冗余的,所以只张成 $xy$ 平面(二维子空间)。

\textbf{关键观察}:
\begin{itemize}
\item $\mathbf{v}_1$ 和 $\mathbf{v}_2$ 线性无关
\item $\mathbf{v}_3$ 是 $\mathbf{v}_1$ 和 $\mathbf{v}_2$ 的线性组合
\item $\text{span}\{\mathbf{v}_1, \mathbf{v}_2, \mathbf{v}_3\} = \text{span}\{\mathbf{v}_1, \mathbf{v}_2\}$
\end{itemize}

\section{重要性质}

\subsection{性质1:Span是线性子空间}

\textbf{定理}:对于任意向量集合 $S$,$\text{span}(S)$ 是一个线性子空间。

\textbf{证明}:
\begin{enumerate}
\item \textbf{包含零向量}:取所有系数 $\alpha_i = 0$,得到 $\sum_{i=1}^k 0 \cdot \mathbf{v}_i = \mathbf{0} \in \text{span}(S)$

\item \textbf{对加法封闭}:设 $\mathbf{u}, \mathbf{v} \in \text{span}(S)$,则存在 $\alpha_i, \beta_i \in \mathbb{R}$ 使得:
\begin{align}
\mathbf{u} &= \sum_{i=1}^k \alpha_i \mathbf{v}_i \\
\mathbf{v} &= \sum_{i=1}^k \beta_i \mathbf{v}_i
\end{align}
因此:
\begin{equation}
\mathbf{u} + \mathbf{v} = \sum_{i=1}^k (\alpha_i + \beta_i) \mathbf{v}_i \in \text{span}(S)
\end{equation}

\item \textbf{对数乘封闭}:设 $\mathbf{u} \in \text{span}(S)$,$\alpha \in \mathbb{R}$,则存在 $\beta_i \in \mathbb{R}$ 使得:
\begin{equation}
\mathbf{u} = \sum_{i=1}^k \beta_i \mathbf{v}_i
\end{equation}
因此:
\begin{equation}
\alpha \mathbf{u} = \sum_{i=1}^k (\alpha \beta_i) \mathbf{v}_i \in \text{span}(S)
\end{equation}
\end{enumerate}

\subsection{性质2:最小性}

\textbf{定理}:$\text{span}(S)$ 是包含 $S$ 的最小线性子空间。

\textbf{含义}:
\begin{itemize}
\item 任何包含 $S$ 的线性子空间都包含 $\text{span}(S)$
\item $\text{span}(S)$ 是"最小"的,因为它是所有包含 $S$ 的线性子空间的交集
\end{itemize}

\textbf{证明思路}:
\begin{enumerate}
\item 设 $V$ 是任意包含 $S$ 的线性子空间
\item 由于 $V$ 对线性组合封闭,$V$ 必须包含 $S$ 中所有向量的线性组合
\item 因此 $V \supseteq \text{span}(S)$
\item 由于 $\text{span}(S)$ 本身是包含 $S$ 的线性子空间,所以它是最小的
\end{enumerate}

\subsection{性质3:维度与线性无关性}

\textbf{定理}:设 $S = \{\mathbf{v}_1, \mathbf{v}_2, \ldots, \mathbf{v}_k\}$,则:
\begin{itemize}
\item 如果 $S$ 中的向量线性无关,则 $\dim(\text{span}(S)) = k$
\item 如果 $S$ 中的向量线性相关,则 $\dim(\text{span}(S)) < k$
\end{itemize}

\textbf{例子}:
\begin{itemize}
\item $\text{span}\{(1, 0), (0, 1)\}$:两个向量线性无关,维度为2
\item $\text{span}\{(1, 0), (2, 0)\}$:两个向量线性相关,维度为1
\item $\text{span}\{(1, 0, 0), (0, 1, 0), (1, 1, 0)\}$:三个向量线性相关,维度为2
\end{itemize}

\subsection{性质4:冗余向量的移除}

\textbf{定理}:如果 $\mathbf{v}_k$ 是 $\mathbf{v}_1, \ldots, \mathbf{v}_{k-1}$ 的线性组合,则:
\begin{equation}
\text{span}\{\mathbf{v}_1, \ldots, \mathbf{v}_{k-1}, \mathbf{v}_k\} = \text{span}\{\mathbf{v}_1, \ldots, \mathbf{v}_{k-1}\}
\end{equation}

\textbf{含义}:移除冗余向量不会改变span。

\textbf{证明}:
\begin{enumerate}
\item 显然 $\text{span}\{\mathbf{v}_1, \ldots, \mathbf{v}_{k-1}\} \subseteq \text{span}\{\mathbf{v}_1, \ldots, \mathbf{v}_k\}$

\item 由于 $\mathbf{v}_k = \sum_{i=1}^{k-1} \beta_i \mathbf{v}_i$,对于任意 $\alpha_1, \ldots, \alpha_k$:
\begin{align}
\sum_{i=1}^k \alpha_i \mathbf{v}_i &= \sum_{i=1}^{k-1} \alpha_i \mathbf{v}_i + \alpha_k \mathbf{v}_k \\
&= \sum_{i=1}^{k-1} \alpha_i \mathbf{v}_i + \alpha_k \sum_{i=1}^{k-1} \beta_i \mathbf{v}_i \\
&= \sum_{i=1}^{k-1} (\alpha_i + \alpha_k \beta_i) \mathbf{v}_i \in \text{span}\{\mathbf{v}_1, \ldots, \mathbf{v}_{k-1}\}
\end{align}

\item 因此 $\text{span}\{\mathbf{v}_1, \ldots, \mathbf{v}_k\} \subseteq \text{span}\{\mathbf{v}_1, \ldots, \mathbf{v}_{k-1}\}$
\end{enumerate}

\section{与列空间的关系}

\subsection{列空间的定义}

对于矩阵 $\mathbf{A} \in \mathbb{R}^{m \times n}$,设其列向量为 $\mathbf{a}_1, \mathbf{a}_2, \ldots, \mathbf{a}_n$,则:

\begin{equation}
\text{col}(\mathbf{A}) = R(\mathbf{A}) = \text{span}\{\mathbf{a}_1, \mathbf{a}_2, \ldots, \mathbf{a}_n\}
\end{equation}

\textbf{含义}:列空间就是矩阵列向量的span。

\subsection{等价表述}

列空间也可以表示为:
\begin{equation}
R(\mathbf{A}) = \{\mathbf{A}\mathbf{x} \mid \mathbf{x} \in \mathbb{R}^n\}
\end{equation}

\textbf{等价性证明}:
\begin{align}
\mathbf{A}\mathbf{x} &= \begin{pmatrix} \mathbf{a}_1 & \mathbf{a}_2 & \cdots & \mathbf{a}_n \end{pmatrix} \begin{pmatrix} x_1 \\ x_2 \\ \vdots \\ x_n \end{pmatrix} \\
&= x_1 \mathbf{a}_1 + x_2 \mathbf{a}_2 + \cdots + x_n \mathbf{a}_n
\end{align}

因此 $\{\mathbf{A}\mathbf{x} \mid \mathbf{x} \in \mathbb{R}^n\}$ 就是所有列向量的线性组合,即 $\text{span}\{\mathbf{a}_1, \ldots, \mathbf{a}_n\}$。

\subsection{例子}

\textbf{例子}:矩阵 $\mathbf{A} = \begin{pmatrix} 1 & 2 \\ 3 & 6 \end{pmatrix}$

列向量:$\mathbf{a}_1 = \begin{pmatrix} 1 \\ 3 \end{pmatrix}$,$\mathbf{a}_2 = \begin{pmatrix} 2 \\ 6 \end{pmatrix}$

注意到 $\mathbf{a}_2 = 2 \mathbf{a}_1$,所以:
\begin{equation}
R(\mathbf{A}) = \text{span}\{\mathbf{a}_1, \mathbf{a}_2\} = \text{span}\{\mathbf{a}_1\} = \{\alpha \begin{pmatrix} 1 \\ 3 \end{pmatrix} \mid \alpha \in \mathbb{R}\}
\end{equation}

\textbf{几何意义}:列空间是一条通过原点的直线,方向为 $(1, 3)$。

\textbf{维度}:$\dim(R(\mathbf{A})) = 1$(因为两个列向量线性相关)

\section{基(Basis)与Span}

\subsection{基的定义}

线性子空间 $V$ 的\textbf{基}(Basis)是一组线性无关的向量 $\{\mathbf{v}_1, \mathbf{v}_2, \ldots, \mathbf{v}_k\}$,使得:
\begin{equation}
V = \text{span}\{\mathbf{v}_1, \mathbf{v}_2, \ldots, \mathbf{v}_k\}
\end{equation}

\subsection{基的性质}

\begin{enumerate}
\item \textbf{线性无关性}:基中的向量线性无关

\item \textbf{生成性}:基张成整个子空间

\item \textbf{最小性}:基是张成子空间的最小向量集合(不能再减少)

\item \textbf{唯一表示}:子空间中的任意向量都可以\textbf{唯一地}表示为基向量的线性组合
\end{enumerate}

\subsection{例子}

\textbf{例子1}:$\mathbb{R}^2$ 的标准基

$\{(1, 0), (0, 1)\}$ 是 $\mathbb{R}^2$ 的一个基:
\begin{itemize}
\item 线性无关:$\alpha_1 (1, 0) + \alpha_2 (0, 1) = (0, 0) \Rightarrow \alpha_1 = \alpha_2 = 0$
\item 张成整个空间:$\text{span}\{(1, 0), (0, 1)\} = \mathbb{R}^2$
\item 任意向量 $(x, y)$ 可以唯一表示为 $x(1, 0) + y(0, 1)$
\end{itemize}

\textbf{例子2}:另一个基

$\{(1, 1), (1, -1)\}$ 也是 $\mathbb{R}^2$ 的一个基:
\begin{itemize}
\item 线性无关:$\alpha_1 (1, 1) + \alpha_2 (1, -1) = (0, 0) \Rightarrow \alpha_1 = \alpha_2 = 0$
\item 张成整个空间:$\text{span}\{(1, 1), (1, -1)\} = \mathbb{R}^2$
\item 任意向量 $(x, y)$ 可以表示为 $\frac{x+y}{2}(1, 1) + \frac{x-y}{2}(1, -1)$
\end{itemize}

\textbf{关键观察}:基的选择不唯一,但基中向量的个数(维度)是唯一的。

\section{实际应用}

\subsection{在优化问题中的应用}

\textbf{应用1}:等式约束的消除

在凸优化中,当处理线性等式约束 $\mathbf{A}\mathbf{x} = \mathbf{b}$ 时,可行解可以表示为:
\begin{equation}
\mathbf{x} = \mathbf{x}_0 + \mathbf{v}, \quad \mathbf{v} \in N(\mathbf{A})
\end{equation}

其中 $N(\mathbf{A})$ 是零空间。如果 $\{\mathbf{f}_1, \ldots, \mathbf{f}_k\}$ 是 $N(\mathbf{A})$ 的基,则:
\begin{equation}
N(\mathbf{A}) = \text{span}\{\mathbf{f}_1, \ldots, \mathbf{f}_k\}
\end{equation}

因此可行解可以表示为:
\begin{equation}
\mathbf{x} = \mathbf{x}_0 + \sum_{i=1}^k \alpha_i \mathbf{f}_i
\end{equation}

\textbf{应用2}:列空间与可行性

对于线性方程组 $\mathbf{A}\mathbf{x} = \mathbf{b}$:
\begin{itemize}
\item 有解当且仅当 $\mathbf{b} \in R(\mathbf{A}) = \text{span}\{\mathbf{a}_1, \ldots, \mathbf{a}_n\}$
\item 即 $\mathbf{b}$ 必须是列向量的线性组合
\end{itemize}

\subsection{在机器学习中的应用}

\textbf{应用}:特征空间

在机器学习中,特征向量 $\mathbf{x}_1, \ldots, \mathbf{x}_n$ 张成的空间:
\begin{equation}
\text{span}\{\mathbf{x}_1, \ldots, \mathbf{x}_n\}
\end{equation}

表示这些特征能够表示的所有可能的数据点。如果特征向量线性相关,则存在冗余特征。

\section{常见误区}

\subsection{误区1:Span等于向量集合本身}

\textbf{错误理解}:$\text{span}\{\mathbf{v}_1, \mathbf{v}_2\} = \{\mathbf{v}_1, \mathbf{v}_2\}$

\textbf{正确理解}:$\text{span}\{\mathbf{v}_1, \mathbf{v}_2\}$ 包含 $\mathbf{v}_1$ 和 $\mathbf{v}_2$,但还包含它们的所有线性组合,通常比原集合大得多。

\textbf{例子}:$\text{span}\{(1, 0), (0, 1)\} = \mathbb{R}^2$,而 $\{(1, 0), (0, 1)\}$ 只包含两个点。

\subsection{误区2:向量个数等于维度}

\textbf{错误理解}:如果 $S$ 有 $k$ 个向量,则 $\dim(\text{span}(S)) = k$

\textbf{正确理解}:只有当 $S$ 中的向量线性无关时,维度才等于 $k$。如果向量线性相关,维度小于 $k$。

\textbf{例子}:$\text{span}\{(1, 0), (2, 0)\}$ 有2个向量,但维度为1。

\subsection{误区3:Span依赖于向量的顺序}

\textbf{错误理解}:$\text{span}\{\mathbf{v}_1, \mathbf{v}_2\} \neq \text{span}\{\mathbf{v}_2, \mathbf{v}_1\}$

\textbf{正确理解}:Span是集合的span,不依赖于顺序:
\begin{equation}
\text{span}\{\mathbf{v}_1, \mathbf{v}_2\} = \text{span}\{\mathbf{v}_2, \mathbf{v}_1\}
\end{equation}

\section{总结}

\subsection{关键要点}

\begin{enumerate}
\item \textbf{定义}:Span是一组向量所有线性组合的集合

\item \textbf{性质}:Span是线性子空间,是包含原集合的最小线性子空间

\item \textbf{维度}:Span的维度等于集合中线性无关向量的最大个数

\item \textbf{应用}:列空间、基、特征空间等概念都基于span

\item \textbf{几何意义}:Span表示向量能够"覆盖"的空间范围
\end{enumerate}

\subsection{记忆技巧}

\begin{itemize}
\item \textbf{Span = 所有线性组合}
\item \textbf{Span = 最小包含原集合的子空间}
\item \textbf{列空间 = 列向量的span}
\item \textbf{基 = 线性无关且张成子空间的向量集合}
\end{itemize}

\end{document}

