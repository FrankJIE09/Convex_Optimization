\documentclass[12pt,a4paper]{article}
\usepackage[UTF8]{ctex}
\usepackage{amsmath}
\usepackage{amssymb}
\usepackage{amsthm}
\usepackage{geometry}
\geometry{left=2.5cm,right=2.5cm,top=2.5cm,bottom=2.5cm}

\title{优化问题中下标0的含义}
\subtitle{为什么目标函数用 $f_0$ 表示?}
\author{}
\date{\today}

\begin{document}

\maketitle

\section{引言}

在优化问题中,我们经常看到函数带有下标,如 $f_0$、$f_1$、$f_2$、$h_1$、$h_2$ 等。理解这些下标的含义对于正确理解优化问题的结构非常重要。本文将详细解释下标0的特殊含义。

\section{标准优化问题的表示}

\subsection{标准形式}

优化问题的标准形式为:

\begin{align}
\begin{array}{ll}
\text{minimize} & f_0(\mathbf{x}) \\
\text{subject to} & f_i(\mathbf{x}) \leq 0, \quad i = 1, \ldots, m \\
& h_i(\mathbf{x}) = 0, \quad i = 1, \ldots, p
\end{array}
\end{align}

\subsection{下标的含义}

\begin{itemize}
\item \textbf{$f_0$}:目标函数(objective function)
\item \textbf{$f_1, f_2, \ldots, f_m$}:不等式约束函数(inequality constraint functions)
\item \textbf{$h_1, h_2, \ldots, h_p$}:等式约束函数(equality constraint functions)
\end{itemize}

\section{为什么目标函数用下标0?}

\subsection{历史原因}

\begin{itemize}
\item 下标从0开始是数学和计算机科学中的常见约定
\item 将目标函数编号为0,约束函数从1开始编号
\item 这样可以统一表示:$f_0$ 是"第0个函数"(目标函数),$f_1, \ldots, f_m$ 是约束函数
\end{itemize}

\subsection{统一表示}

使用下标0可以统一表示所有函数:

\begin{itemize}
\item 所有函数:$f_0, f_1, \ldots, f_m, h_1, \ldots, h_p$
\item 目标函数:$f_0$(特殊标记为下标0)
\item 约束函数:$f_1, \ldots, f_m, h_1, \ldots, h_p$(从1开始)
\end{itemize}

\subsection{便于理论分析}

在某些理论分析中,可以将目标函数和约束函数统一处理:

\begin{itemize}
\item 可以写成:$f_i(\mathbf{x}) \leq 0$,$i = 0, 1, \ldots, m$(但 $f_0$ 不用于约束)
\item 或者在某些对偶理论中,$f_0$ 和 $f_i$ 可以统一处理
\end{itemize}

\section{具体例子}

\subsection{例子1:简单优化问题}

\begin{align}
\begin{array}{ll}
\text{minimize} & f_0(x, y) = x^2 + y^2 \\
\text{subject to} & f_1(x, y) = x + y - 1 \leq 0 \\
& f_2(x, y) = -x \leq 0 \\
& h_1(x, y) = x - y = 0
\end{array}
\end{align}

\textbf{解释}:
\begin{itemize}
\item $f_0$:目标函数(要最小化 $x^2 + y^2$)
\item $f_1$:第一个不等式约束($x + y \leq 1$)
\item $f_2$:第二个不等式约束($x \geq 0$,写成 $-x \leq 0$)
\item $h_1$:第一个等式约束($x = y$)
\end{itemize}

\subsection{例子2:线性规划}

\begin{align}
\begin{array}{ll}
\text{minimize} & f_0(\mathbf{x}) = \mathbf{c}^T \mathbf{x} \\
\text{subject to} & f_i(\mathbf{x}) = \mathbf{a}_i^T \mathbf{x} - b_i \leq 0, \quad i = 1, \ldots, m \\
& h_i(\mathbf{x}) = \mathbf{d}_i^T \mathbf{x} - e_i = 0, \quad i = 1, \ldots, p
\end{array}
\end{align}

\textbf{解释}:
\begin{itemize}
\item $f_0$:线性目标函数
\item $f_1, \ldots, f_m$:$m$ 个线性不等式约束
\item $h_1, \ldots, h_p$:$p$ 个线性等式约束
\end{itemize}

\section{其他约定}

\subsection{有时也用其他符号}

在某些文献中,可能使用不同的符号:

\begin{itemize}
\item 目标函数:$f(\mathbf{x})$ 或 $J(\mathbf{x})$(不用下标)
\item 约束函数:$g_i(\mathbf{x}) \leq 0$,$h_i(\mathbf{x}) = 0$
\end{itemize}

但在《Convex Optimization》中,统一使用 $f_0$ 表示目标函数。

\subsection{定义域的交集}

在定义域的定义中:

\begin{equation}
\mathcal{D} = \bigcap_{i=0}^m \text{dom } f_i \cap \bigcap_{i=1}^p \text{dom } h_i
\end{equation}

注意:这里 $i = 0$ 也包括在内,表示目标函数的定义域也参与交集。

\section{记忆技巧}

\subsection{数字0的特殊性}

\begin{itemize}
\item \textbf{0} 在数学中常表示"起点"或"基准"
\item 目标函数是优化问题的"起点"(我们要优化的对象)
\item 约束函数是"附加条件"(从1开始编号)
\end{itemize}

\subsection{类比}

\begin{itemize}
\item 就像数组索引从0开始
\item $f_0$ 是"第0个函数"(目标函数)
\item $f_1, f_2, \ldots$ 是"第1个、第2个、...约束函数"
\end{itemize}

\section{在理论中的统一处理}

\subsection{拉格朗日函数}

在拉格朗日对偶中,可以统一处理:

\begin{equation}
L(\mathbf{x}, \boldsymbol{\lambda}, \boldsymbol{\nu}) = f_0(\mathbf{x}) + \sum_{i=1}^m \lambda_i f_i(\mathbf{x}) + \sum_{i=1}^p \nu_i h_i(\mathbf{x})
\end{equation}

这里 $f_0$ 是目标函数,$f_i$ 和 $h_i$ 是约束函数。

\subsection{对偶函数}

对偶函数定义为:

\begin{equation}
g(\boldsymbol{\lambda}, \boldsymbol{\nu}) = \inf_{\mathbf{x}} L(\mathbf{x}, \boldsymbol{\lambda}, \boldsymbol{\nu})
\end{equation}

注意:这里 $f_0$ 和约束函数在拉格朗日函数中统一处理。

\section{总结}

\begin{enumerate}
\item \textbf{下标0的含义}:
   \begin{itemize}
   \item $f_0$ 表示目标函数(objective function)
   \item 这是优化问题的约定,将目标函数编号为0
   \end{itemize}

\item \textbf{其他下标}:
   \begin{itemize}
   \item $f_1, \ldots, f_m$:不等式约束函数
   \item $h_1, \ldots, h_p$:等式约束函数
   \item 约束函数从1开始编号
   \end{itemize}

\item \textbf{为什么用0?}:
   \begin{itemize}
   \item 历史约定和数学传统
   \item 便于统一表示和理论分析
   \item 将目标函数与约束函数区分开
   \end{itemize}

\item \textbf{记忆方法}:
   \begin{itemize}
   \item 0 = Objective(目标函数)
   \item 1, 2, ... = Constraints(约束函数)
   \end{itemize}
\end{enumerate}

理解下标0的含义,有助于正确理解优化问题的结构和符号!

\end{document}


