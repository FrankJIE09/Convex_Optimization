\documentclass[12pt,a4paper]{article}
\usepackage[UTF8]{ctex}
\usepackage{amsmath}
\usepackage{amssymb}
\usepackage{amsthm}
\usepackage{geometry}
\geometry{left=2.5cm,right=2.5cm,top=2.5cm,bottom=2.5cm}

\title{向量空间(Vector Space)详解}
\author{}
\date{\today}

\begin{document}

\maketitle

\section{引言}

向量空间是线性代数的基础概念,也是理解凸优化中许多概念(如子空间、仿射集合等)的基础。本节详细解释向量空间的定义、性质和例子。

\section{向量空间的定义}

\subsection{基本定义}

\textbf{向量空间}:设 $V$ 是一个集合,$\mathbb{F}$ 是一个数域(通常是 $\mathbb{R}$ 或 $\mathbb{C}$)。如果 $V$ 上定义了两种运算:

\begin{enumerate}
\item \textbf{向量加法}:$+ : V \times V \to V$,$(\mathbf{u}, \mathbf{v}) \mapsto \mathbf{u} + \mathbf{v}$

\item \textbf{标量乘法}:$\cdot : \mathbb{F} \times V \to V$,$(\alpha, \mathbf{v}) \mapsto \alpha \mathbf{v}$
\end{enumerate}

并且满足以下8条公理,则 $V$ 称为向量空间(或线性空间):

\subsection{向量空间公理}

\textbf{加法公理}:
\begin{enumerate}
\item \textbf{交换律}:$\mathbf{u} + \mathbf{v} = \mathbf{v} + \mathbf{u}$,对所有 $\mathbf{u}, \mathbf{v} \in V$

\item \textbf{结合律}:$(\mathbf{u} + \mathbf{v}) + \mathbf{w} = \mathbf{u} + (\mathbf{v} + \mathbf{w})$,对所有 $\mathbf{u}, \mathbf{v}, \mathbf{w} \in V$

\item \textbf{零元存在}:存在 $\mathbf{0} \in V$,使得 $\mathbf{v} + \mathbf{0} = \mathbf{v}$ 对所有 $\mathbf{v} \in V$

\item \textbf{负元存在}:对每个 $\mathbf{v} \in V$,存在 $-\mathbf{v} \in V$,使得 $\mathbf{v} + (-\mathbf{v}) = \mathbf{0}$
\end{enumerate}

\textbf{标量乘法公理}:
\begin{enumerate}
\setcounter{enumi}{4}
\item \textbf{单位元}:$1 \cdot \mathbf{v} = \mathbf{v}$ 对所有 $\mathbf{v} \in V$

\item \textbf{结合律}:$(\alpha \beta) \mathbf{v} = \alpha (\beta \mathbf{v})$ 对所有 $\alpha, \beta \in \mathbb{F}$ 和 $\mathbf{v} \in V$
\end{enumerate}

\textbf{分配律}:
\begin{enumerate}
\setcounter{enumi}{6}
\item \textbf{标量对向量}:$\alpha (\mathbf{u} + \mathbf{v}) = \alpha \mathbf{u} + \alpha \mathbf{v}$ 对所有 $\alpha \in \mathbb{F}$ 和 $\mathbf{u}, \mathbf{v} \in V$

\item \textbf{向量对标量}:$(\alpha + \beta) \mathbf{v} = \alpha \mathbf{v} + \beta \mathbf{v}$ 对所有 $\alpha, \beta \in \mathbb{F}$ 和 $\mathbf{v} \in V$
\end{enumerate}

\section{常见的向量空间}

\subsection{$\mathbb{R}^n$:$n$ 维实向量空间}

\textbf{定义}:所有 $n$ 维实向量的集合

\begin{equation}
\mathbb{R}^n = \{(x_1, x_2, \ldots, x_n) \mid x_i \in \mathbb{R}, i = 1, \ldots, n\}
\end{equation}

\textbf{运算}:
\begin{itemize}
\item 向量加法:$(x_1, \ldots, x_n) + (y_1, \ldots, y_n) = (x_1 + y_1, \ldots, x_n + y_n)$
\item 标量乘法:$\alpha (x_1, \ldots, x_n) = (\alpha x_1, \ldots, \alpha x_n)$
\end{itemize}

\textbf{例子}:
\begin{itemize}
\item $\mathbb{R}^2$:平面上的所有点
\item $\mathbb{R}^3$:三维空间中的所有点
\item $\mathbb{R}^n$:$n$ 维空间中的所有点
\end{itemize}

\subsection{矩阵空间 $\mathbb{R}^{m \times n}$}

\textbf{定义}:所有 $m \times n$ 实矩阵的集合

\begin{equation}
\mathbb{R}^{m \times n} = \{\mathbf{A} = [a_{ij}] \mid a_{ij} \in \mathbb{R}, i = 1, \ldots, m, j = 1, \ldots, n\}
\end{equation}

\textbf{运算}:
\begin{itemize}
\item 矩阵加法:$[\mathbf{A} + \mathbf{B}]_{ij} = a_{ij} + b_{ij}$
\item 标量乘法:$[\alpha \mathbf{A}]_{ij} = \alpha a_{ij}$
\end{itemize}

\textbf{维度}:$mn$(需要 $mn$ 个实数来确定一个矩阵)

\subsection{对称矩阵空间 $\mathbb{S}^n$}

\textbf{定义}:所有 $n \times n$ 对称矩阵的集合

\begin{equation}
\mathbb{S}^n = \{\mathbf{X} \in \mathbb{R}^{n \times n} \mid \mathbf{X} = \mathbf{X}^T\}
\end{equation}

\textbf{性质}:
\begin{itemize}
\item 这是 $\mathbb{R}^{n \times n}$ 的子空间
\item 维度:$\frac{n(n+1)}{2}$(因为对称矩阵只需要存储上三角部分)
\end{itemize}

\textbf{运算}:
\begin{itemize}
\item 矩阵加法:对称矩阵的和仍是对称矩阵
\item 标量乘法:对称矩阵的标量倍仍是对称矩阵
\end{itemize}

\subsection{多项式空间}

\textbf{定义}:所有次数不超过 $n$ 的多项式的集合

\begin{equation}
\mathcal{P}_n = \{p(x) = a_0 + a_1 x + \cdots + a_n x^n \mid a_i \in \mathbb{R}\}
\end{equation}

\textbf{运算}:
\begin{itemize}
\item 多项式加法:$(p + q)(x) = p(x) + q(x)$
\item 标量乘法:$(\alpha p)(x) = \alpha p(x)$
\end{itemize}

\textbf{维度}:$n + 1$(由 $n + 1$ 个系数确定)

\subsection{函数空间}

\textbf{定义}:所有从区间 $[a, b]$ 到 $\mathbb{R}$ 的连续函数的集合

\begin{equation}
C[a, b] = \{f : [a, b] \to \mathbb{R} \mid f \text{ 连续}\}
\end{equation}

\textbf{运算}:
\begin{itemize}
\item 函数加法:$(f + g)(x) = f(x) + g(x)$
\item 标量乘法:$(\alpha f)(x) = \alpha f(x)$
\end{itemize}

\textbf{注意}:这是无限维向量空间。

\section{向量空间的性质}

\subsection{基本性质}

\begin{enumerate}
\item \textbf{零向量唯一}:向量空间中有且仅有一个零向量

\item \textbf{负向量唯一}:每个向量的负向量是唯一的

\item \textbf{零标量}:$0 \cdot \mathbf{v} = \mathbf{0}$ 对所有 $\mathbf{v} \in V$

\item \textbf{零向量}:$\alpha \cdot \mathbf{0} = \mathbf{0}$ 对所有 $\alpha \in \mathbb{F}$

\item \textbf{消去律}:如果 $\alpha \mathbf{v} = \mathbf{0}$ 且 $\alpha \neq 0$,则 $\mathbf{v} = \mathbf{0}$
\end{enumerate}

\subsection{线性组合}

\textbf{线性组合}:对于向量 $\mathbf{v}_1, \ldots, \mathbf{v}_k \in V$ 和标量 $\alpha_1, \ldots, \alpha_k \in \mathbb{F}$,表达式:

\begin{equation}
\alpha_1 \mathbf{v}_1 + \alpha_2 \mathbf{v}_2 + \cdots + \alpha_k \mathbf{v}_k
\end{equation}

称为这些向量的线性组合。

\textbf{性质}:
\begin{itemize}
\item 线性组合的结果仍在向量空间中
\item 这是向量空间的基本运算
\end{itemize}

\section{子空间}

\subsection{定义}

\textbf{子空间}:向量空间 $V$ 的子集 $W$ 是子空间,如果:

\begin{enumerate}
\item $\mathbf{0} \in W$(包含零向量)

\item 对加法封闭:如果 $\mathbf{u}, \mathbf{v} \in W$,则 $\mathbf{u} + \mathbf{v} \in W$

\item 对标量乘法封闭:如果 $\mathbf{v} \in W$ 且 $\alpha \in \mathbb{F}$,则 $\alpha \mathbf{v} \in W$
\end{enumerate}

\textbf{等价表述}:$W$ 是子空间,当且仅当 $W$ 对线性组合封闭。

\subsection{例子}

\textbf{例子1}:$\mathbb{R}^2$ 中的子空间
\begin{itemize}
\item 过原点的直线:$\{(x, y) \mid ax + by = 0\}$
\item 整个平面:$\mathbb{R}^2$ 本身
\item 只包含原点:$\{\mathbf{0}\}$
\end{itemize}

\textbf{例子2}:$\mathbb{R}^3$ 中的子空间
\begin{itemize}
\item 过原点的直线
\item 过原点的平面
\item 整个空间:$\mathbb{R}^3$ 本身
\item 只包含原点:$\{\mathbf{0}\}$
\end{itemize}

\textbf{例子3}:$\mathbb{S}^n$ 是 $\mathbb{R}^{n \times n}$ 的子空间
\begin{itemize}
\item 对称矩阵的集合是矩阵空间的子空间
\item 对加法和标量乘法封闭
\end{itemize}

\section{基和维度}

\subsection{线性无关}

\textbf{线性无关}:向量 $\mathbf{v}_1, \ldots, \mathbf{v}_k$ 是线性无关的,如果:

\begin{equation}
\alpha_1 \mathbf{v}_1 + \cdots + \alpha_k \mathbf{v}_k = \mathbf{0} \Rightarrow \alpha_1 = \cdots = \alpha_k = 0
\end{equation}

\textbf{线性相关}:如果存在不全为零的标量使得线性组合为零,则向量线性相关。

\subsection{基}

\textbf{基}:向量空间 $V$ 的基是一组线性无关的向量 $\{\mathbf{v}_1, \ldots, \mathbf{v}_n\}$,使得 $V$ 中的每个向量都可以唯一地表示为这些向量的线性组合。

\textbf{标准基}:
\begin{itemize}
\item $\mathbb{R}^n$ 的标准基:$\mathbf{e}_1 = (1, 0, \ldots, 0)^T$,$\mathbf{e}_2 = (0, 1, \ldots, 0)^T$,$\ldots$,$\mathbf{e}_n = (0, 0, \ldots, 1)^T$
\item 每个向量 $\mathbf{x} = (x_1, \ldots, x_n)^T$ 可以表示为:$\mathbf{x} = x_1 \mathbf{e}_1 + \cdots + x_n \mathbf{e}_n$
\end{itemize}

\subsection{维度}

\textbf{维度}:向量空间 $V$ 的维度是基中向量的个数,记作 $\dim V$。

\textbf{性质}:
\begin{itemize}
\item 所有基都有相同数量的向量
\item 维度是向量空间的"大小"的度量
\end{itemize}

\textbf{例子}:
\begin{itemize}
\item $\dim \mathbb{R}^n = n$
\item $\dim \mathbb{R}^{m \times n} = mn$
\item $\dim \mathbb{S}^n = \frac{n(n+1)}{2}$
\item $\dim \mathcal{P}_n = n + 1$
\end{itemize}

\section{在凸优化中的应用}

\subsection{子空间}

\textbf{应用}:在凸优化中,子空间是特殊的凸集和仿射集合。

\textbf{性质}:
\begin{itemize}
\item 子空间是凸集
\item 子空间是仿射集合(通过原点)
\item 如果仿射集合通过原点,则它是子空间
\end{itemize}

\subsection{矩阵空间}

\textbf{应用}:$\mathbb{S}^n$ 是凸优化中重要的向量空间。

\textbf{性质}:
\begin{itemize}
\item $\mathbb{S}^n$ 是向量空间
\item $\mathbb{S}_+^n$(半正定矩阵)是 $\mathbb{S}^n$ 的凸锥,但不是子空间
\end{itemize}

\section{总结}

\subsection{向量空间的定义}

\begin{enumerate}
\item \textbf{集合} $V$ 和数域 $\mathbb{F}$

\item \textbf{两种运算}:向量加法和标量乘法

\item \textbf{8条公理}:保证运算的"良好"性质
</enumerate}

\subsection{关键概念}

\begin{enumerate}
\item \textbf{线性组合}:向量的加权和

\item \textbf{线性无关}:没有冗余的向量组

\item \textbf{基}:生成整个空间的线性无关向量组

\item \textbf{维度}:基中向量的个数
</enumerate}

\subsection{常见例子}

\begin{enumerate}
\item \textbf{$\mathbb{R}^n$}:$n$ 维实向量空间

\item \textbf{$\mathbb{R}^{m \times n}$}:矩阵空间

\item \textbf{$\mathbb{S}^n$}:对称矩阵空间

\item \textbf{多项式空间}:有限维

\item \textbf{函数空间}:无限维
</enumerate}

理解向量空间,是理解线性代数和凸优化理论的基础!

\end{document}

