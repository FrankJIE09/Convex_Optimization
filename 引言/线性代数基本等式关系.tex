\documentclass[12pt,a4paper]{article}
\usepackage[UTF8]{ctex}
\usepackage{amsmath}
\usepackage{amssymb}
\usepackage{amsthm}
\usepackage{geometry}
\geometry{left=2.5cm,right=2.5cm,top=2.5cm,bottom=2.5cm}

\title{线性代数基本等式关系}
\subtitle{与 $N(\mathbf{A})^\perp = R(\mathbf{A}^T)$ 相关的等式}
\author{}
\date{\today}

\begin{document}

\maketitle

\section{引言}

线性代数中有许多重要的等式关系,这些关系在优化理论中经常用到。本节列出与 $N(\mathbf{A})^\perp = R(\mathbf{A}^T)$ 相关的所有重要等式。

\section{正交补关系}

\subsection{基本等式}

\textbf{等式1}:$N(\mathbf{A})^\perp = R(\mathbf{A}^T)$

\textbf{含义}:零空间的正交补等于转置矩阵的列空间(行空间)。

\textbf{等式2}:$R(\mathbf{A})^\perp = N(\mathbf{A}^T)$

\textbf{含义}:列空间的正交补等于转置矩阵的零空间。

\textbf{证明思路}:对 $\mathbf{A}^T$ 应用等式1,得到 $N(\mathbf{A}^T)^\perp = R(\mathbf{A})$,取正交补得到 $R(\mathbf{A})^\perp = N(\mathbf{A}^T)$。

\subsection{对偶关系}

\textbf{对称性}:
\begin{itemize}
\item $N(\mathbf{A})^\perp = R(\mathbf{A}^T)$
\item $R(\mathbf{A})^\perp = N(\mathbf{A}^T)$
\end{itemize}

\textbf{几何意义}:
\begin{itemize}
\item 行空间与零空间是正交补
\item 列空间与转置的零空间是正交补
\end{itemize}

\section{维度关系}

\subsection{秩-零度定理}

\textbf{定理}(秩-零度定理):对于矩阵 $\mathbf{A} \in \mathbb{R}^{m \times n}$:

\begin{equation}
\dim N(\mathbf{A}) + \dim R(\mathbf{A}) = n
\end{equation}

\textbf{含义}:
\begin{itemize}
\item 零空间的维度 + 列空间的维度 = 列数
\item 这是线性代数的基础定理
\end{itemize}

\subsection{行秩等于列秩}

\textbf{定理}:$\dim R(\mathbf{A}) = \dim R(\mathbf{A}^T)$

\textbf{含义}:矩阵的行秩等于列秩,都等于矩阵的秩。

\textbf{记号}:$\text{rank}(\mathbf{A}) = \dim R(\mathbf{A}) = \dim R(\mathbf{A}^T)$

\subsection{综合维度关系}

\textbf{等式3}:$\dim N(\mathbf{A}) + \dim R(\mathbf{A}) = n$

\textbf{等式4}:$\dim N(\mathbf{A}^T) + \dim R(\mathbf{A}^T) = m$

\textbf{等式5}:$\dim R(\mathbf{A}) = \dim R(\mathbf{A}^T) = \text{rank}(\mathbf{A})$

\textbf{等式6}:$\dim N(\mathbf{A}) + \dim R(\mathbf{A}^T) = n$

\textbf{推导}:
\begin{align}
\dim N(\mathbf{A}) + \dim R(\mathbf{A}^T) &= \dim N(\mathbf{A}) + \dim R(\mathbf{A}) \\
&= n
\end{align}

\textbf{等式7}:$\dim N(\mathbf{A}) + \dim N(\mathbf{A})^\perp = n$

\textbf{推导}:由于 $N(\mathbf{A})^\perp = R(\mathbf{A}^T)$,有:

\begin{align}
\dim N(\mathbf{A}) + \dim N(\mathbf{A})^\perp &= \dim N(\mathbf{A}) + \dim R(\mathbf{A}^T) \\
&= n
\end{align}

\section{正交补的性质}

\subsection{基本性质}

\textbf{性质1}:$(V^\perp)^\perp = V$(对于子空间 $V$)

\textbf{性质2}:$V \cap V^\perp = \{\mathbf{0}\}$

\textbf{性质3}:$\dim V + \dim V^\perp = n$(在 $\mathbb{R}^n$ 中)

\subsection{应用到零空间和列空间}

\textbf{等式8}:$(N(\mathbf{A})^\perp)^\perp = N(\mathbf{A})$

\textbf{等式9}:$(R(\mathbf{A})^\perp)^\perp = R(\mathbf{A})$

\textbf{等式10}:$N(\mathbf{A}) \cap R(\mathbf{A}^T) = \{\mathbf{0}\}$

\textbf{含义}:零空间与行空间只相交于原点。

\section{矩阵转置的关系}

\subsection{转置的基本性质}

\textbf{等式11}:$(\mathbf{A}^T)^T = \mathbf{A}$

\textbf{等式12}:$N(\mathbf{A}^T)^\perp = R(\mathbf{A})$

\textbf{推导}:对 $\mathbf{A}^T$ 应用等式1,得到 $N(\mathbf{A}^T)^\perp = R((\mathbf{A}^T)^T) = R(\mathbf{A})$。

\textbf{等式13}:$R(\mathbf{A}^T)^\perp = N(\mathbf{A})$

\textbf{推导}:对等式1取正交补,得到 $R(\mathbf{A}^T)^\perp = (N(\mathbf{A})^\perp)^\perp = N(\mathbf{A})$。

\section{四个基本子空间}

\subsection{四个子空间}

对于矩阵 $\mathbf{A} \in \mathbb{R}^{m \times n}$,有四个基本子空间:

\begin{enumerate}
\item \textbf{列空间}:$R(\mathbf{A}) \subseteq \mathbb{R}^m$

\item \textbf{零空间}:$N(\mathbf{A}) \subseteq \mathbb{R}^n$

\item \textbf{行空间}:$R(\mathbf{A}^T) \subseteq \mathbb{R}^n$

\item \textbf{左零空间}:$N(\mathbf{A}^T) \subseteq \mathbb{R}^m$
\end{enumerate}

\subsection{正交补关系}

\textbf{在 $\mathbb{R}^n$ 中}:
\begin{itemize}
\item $N(\mathbf{A})^\perp = R(\mathbf{A}^T)$
\item $R(\mathbf{A}^T)^\perp = N(\mathbf{A})$
\end{itemize}

\textbf{在 $\mathbb{R}^m$ 中}:
\begin{itemize}
\item $R(\mathbf{A})^\perp = N(\mathbf{A}^T)$
\item $N(\mathbf{A}^T)^\perp = R(\mathbf{A})$
\end{itemize}

\section{具体例子}

\subsection{例子:矩阵 $\mathbf{A} = \begin{pmatrix} 1 & 2 \\ 3 & 6 \end{pmatrix}$}

\textbf{列空间}:$R(\mathbf{A}) = \text{span}\{(1, 3)^T\}$(一维)

\textbf{零空间}:$N(\mathbf{A}) = \{\mathbf{v} \mid \mathbf{A}\mathbf{v} = \mathbf{0}\} = \text{span}\{(2, -1)^T\}$(一维)

\textbf{行空间}:$R(\mathbf{A}^T) = \text{span}\{(1, 2)^T\}$(一维)

\textbf{左零空间}:$N(\mathbf{A}^T) = \text{span}\{(3, -1)^T\}$(一维)

\textbf{验证维度}:
\begin{itemize}
\item $\dim N(\mathbf{A}) + \dim R(\mathbf{A}) = 1 + 1 = 2 = n$ ✓
\item $\dim N(\mathbf{A}^T) + \dim R(\mathbf{A}^T) = 1 + 1 = 2 = m$ ✓
\end{itemize}

\textbf{验证正交补}:
\begin{itemize}
\item $N(\mathbf{A})^\perp = R(\mathbf{A}^T)$:$(2, -1)^T$ 与 $(1, 2)^T$ 垂直 ✓
\item $R(\mathbf{A})^\perp = N(\mathbf{A}^T)$:$(1, 3)^T$ 与 $(3, -1)^T$ 垂直 ✓
\end{itemize}

\subsection{例子2:满秩矩阵 $\mathbf{A} = \begin{pmatrix} 1 & 2 \\ 3 & 4 \end{pmatrix}$}

\textbf{秩}:$\text{rank}(\mathbf{A}) = 2$(满秩,因为 $\det(\mathbf{A}) = 1 \times 4 - 2 \times 3 = -2 \neq 0$)

\textbf{列空间}:$R(\mathbf{A}) = \text{span}\{(1, 3)^T, (2, 4)^T\} = \mathbb{R}^2$(二维,整个空间)

\textbf{原因}:两个列向量线性无关(不成比例),因此张成整个 $\mathbb{R}^2$。

\textbf{零空间}:$N(\mathbf{A}) = \{\mathbf{v} \mid \mathbf{A}\mathbf{v} = \mathbf{0}\}$

求解 $\begin{pmatrix} 1 & 2 \\ 3 & 4 \end{pmatrix} \begin{pmatrix} v_1 \\ v_2 \end{pmatrix} = \begin{pmatrix} 0 \\ 0 \end{pmatrix}$:

\begin{align}
v_1 + 2v_2 &= 0 \\
3v_1 + 4v_2 &= 0
\end{align}

从第一个方程:$v_1 = -2v_2$。代入第二个:$3(-2v_2) + 4v_2 = -2v_2 = 0$,因此 $v_2 = 0$,$v_1 = 0$。

\begin{equation}
N(\mathbf{A}) = \{\mathbf{0}\} \quad \text{(只包含零向量)}
\end{equation}

\textbf{行空间}:$R(\mathbf{A}^T) = \text{span}\{(1, 2)^T, (3, 4)^T\} = \mathbb{R}^2$(二维,整个空间)

\textbf{左零空间}:$N(\mathbf{A}^T) = \{\mathbf{0}\}$(只包含零向量)

\textbf{验证维度}:
\begin{itemize}
\item $\dim N(\mathbf{A}) + \dim R(\mathbf{A}) = 0 + 2 = 2 = n$ ✓
\item $\dim N(\mathbf{A}^T) + \dim R(\mathbf{A}^T) = 0 + 2 = 2 = m$ ✓
\item $\text{rank}(\mathbf{A}) = 2 = \min(m, n)$(满秩)✓
\end{itemize}

\textbf{验证正交补}:
\begin{itemize}
\item $N(\mathbf{A})^\perp = \{\mathbf{0}\}^\perp = \mathbb{R}^2 = R(\mathbf{A}^T)$ ✓
\item $R(\mathbf{A})^\perp = \mathbb{R}^2^\perp = \{\mathbf{0}\} = N(\mathbf{A}^T)$ ✓
\end{itemize}

\textbf{关键观察}:
\begin{itemize}
\item 满秩矩阵:零空间和左零空间都只包含零向量
\item 列空间和行空间都是整个空间
\item 正交补关系仍然成立
\end{itemize}

\section{在优化中的应用}

\subsection{等式约束优化}

\textbf{问题}:
\begin{align}
\begin{array}{ll}
\text{minimize} & f_0(\mathbf{x}) \\
\text{subject to} & \mathbf{A}\mathbf{x} = \mathbf{b}
\end{array}
\end{align}

\textbf{最优性条件}:$\nabla f_0(\mathbf{x}) \perp N(\mathbf{A})$

\textbf{使用等式}:$N(\mathbf{A})^\perp = R(\mathbf{A}^T)$

\textbf{等价条件}:$\nabla f_0(\mathbf{x}) \in R(\mathbf{A}^T)$

\textbf{拉格朗日条件}:存在 $\boldsymbol{\nu}$,使得 $\nabla f_0(\mathbf{x}) + \mathbf{A}^T \boldsymbol{\nu} = \mathbf{0}$

\section{所有相关等式总结}

\subsection{正交补关系}

\begin{enumerate}
\item $N(\mathbf{A})^\perp = R(\mathbf{A}^T)$

\item $R(\mathbf{A})^\perp = N(\mathbf{A}^T)$

\item $N(\mathbf{A}^T)^\perp = R(\mathbf{A})$

\item $R(\mathbf{A}^T)^\perp = N(\mathbf{A})$
\end{enumerate}

\subsection{维度关系}

\begin{enumerate}
\item $\dim N(\mathbf{A}) + \dim R(\mathbf{A}) = n$

\item $\dim N(\mathbf{A}^T) + \dim R(\mathbf{A}^T) = m$

\item $\dim R(\mathbf{A}) = \dim R(\mathbf{A}^T) = \text{rank}(\mathbf{A})$

\item $\dim N(\mathbf{A}) + \dim R(\mathbf{A}^T) = n$

\item $\dim N(\mathbf{A}) + \dim N(\mathbf{A})^\perp = n$
\end{enumerate}

\subsection{正交补的性质}

\begin{enumerate}
\item $(N(\mathbf{A})^\perp)^\perp = N(\mathbf{A})$

\item $(R(\mathbf{A})^\perp)^\perp = R(\mathbf{A})$

\item $N(\mathbf{A}) \cap R(\mathbf{A}^T) = \{\mathbf{0}\}$

\item $R(\mathbf{A}) \cap N(\mathbf{A}^T) = \{\mathbf{0}\}$
\end{enumerate}

\section{记忆技巧}

\subsection{对称性}

\textbf{模式}:
\begin{itemize}
\item 零空间 $\leftrightarrow$ 行空间(在 $\mathbb{R}^n$ 中)
\item 列空间 $\leftrightarrow$ 左零空间(在 $\mathbb{R}^m$ 中)
\end{itemize}

\subsection{维度}

\textbf{规则}:
\begin{itemize}
\item 零空间维度 + 列空间维度 = 列数
\item 左零空间维度 + 行空间维度 = 行数
\item 行秩 = 列秩 = 秩
\end{itemize}

\section{总结}

\subsection{核心等式}

\begin{enumerate}
\item \textbf{正交补}:$N(\mathbf{A})^\perp = R(\mathbf{A}^T)$

\item \textbf{维度}:$\dim N(\mathbf{A}) + \dim R(\mathbf{A}) = n$

\item \textbf{秩}:$\dim R(\mathbf{A}) = \dim R(\mathbf{A}^T) = \text{rank}(\mathbf{A})$
</enumerate}

\subsection{应用}

\begin{enumerate}
\item \textbf{优化理论}:推导拉格朗日乘数条件

\item \textbf{线性方程组}:理解解的结构

\item \textbf{几何理解}:理解四个基本子空间的关系
</enumerate}

掌握这些等式关系,是理解线性代数和优化理论的基础!

\end{document}

