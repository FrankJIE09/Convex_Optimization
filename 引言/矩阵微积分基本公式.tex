\documentclass[12pt,a4paper]{article}
\usepackage[UTF8]{ctex}
\usepackage{amsmath}
\usepackage{amssymb}
\usepackage{amsthm}
\usepackage{geometry}
\geometry{left=2.5cm,right=2.5cm,top=2.5cm,bottom=2.5cm}

\title{矩阵微积分基本公式}
\subtitle{$\nabla \mathbf{a}^T \mathbf{x} = \mathbf{a}$ 及类似公式}
\author{}
\date{\today}

\begin{document}

\maketitle

\section{引言}

矩阵微积分是优化理论中的重要工具。理解基本公式如 $\nabla \mathbf{a}^T \mathbf{x} = \mathbf{a}$ 及其类似公式,有助于快速计算梯度。

\section{基本公式:$\nabla \mathbf{a}^T \mathbf{x} = \mathbf{a}$}

\subsection{公式陈述}

\textbf{公式}:对于向量 $\mathbf{a} \in \mathbb{R}^n$ 和 $\mathbf{x} \in \mathbb{R}^n$,有:

\begin{equation}
\nabla (\mathbf{a}^T \mathbf{x}) = \mathbf{a}
\end{equation}

\subsection{证明}

\textbf{展开}:$\mathbf{a}^T \mathbf{x} = a_1 x_1 + a_2 x_2 + \cdots + a_n x_n$

\textbf{计算梯度}:

\begin{align}
\nabla (\mathbf{a}^T \mathbf{x}) &= \begin{pmatrix}
\frac{\partial}{\partial x_1}(\mathbf{a}^T \mathbf{x}) \\
\frac{\partial}{\partial x_2}(\mathbf{a}^T \mathbf{x}) \\
\vdots \\
\frac{\partial}{\partial x_n}(\mathbf{a}^T \mathbf{x})
\end{pmatrix} \\
&= \begin{pmatrix}
\frac{\partial}{\partial x_1}(a_1 x_1 + a_2 x_2 + \cdots + a_n x_n) \\
\frac{\partial}{\partial x_2}(a_1 x_1 + a_2 x_2 + \cdots + a_n x_n) \\
\vdots \\
\frac{\partial}{\partial x_n}(a_1 x_1 + a_2 x_2 + \cdots + a_n x_n)
\end{pmatrix} \\
&= \begin{pmatrix}
a_1 \\
a_2 \\
\vdots \\
a_n
\end{pmatrix} = \mathbf{a}
\end{align}

\textbf{关键观察}:
\begin{itemize}
\item 对 $x_i$ 求偏导,只有 $a_i x_i$ 项有贡献
\item 偏导数是 $a_i$
\item 因此梯度是 $\mathbf{a}$
\end{itemize}

\section{类似的矩阵微积分公式}

\subsection{线性函数的梯度}

\textbf{公式1}:$\nabla (\mathbf{a}^T \mathbf{x}) = \mathbf{a}$

\textbf{公式2}:$\nabla (\mathbf{a}^T \mathbf{x} + b) = \mathbf{a}$

\textbf{说明}:常数项 $b$ 的梯度是 $\mathbf{0}$,因此结果相同。

\textbf{公式3}:$\nabla (\mathbf{x}^T \mathbf{a}) = \mathbf{a}$

\textbf{说明}:$\mathbf{x}^T \mathbf{a} = \mathbf{a}^T \mathbf{x}$(标量的转置等于自身),因此结果相同。

\subsection{二次型的梯度}

\textbf{公式4}:$\nabla (\mathbf{x}^T \mathbf{A} \mathbf{x}) = (\mathbf{A} + \mathbf{A}^T)\mathbf{x}$

\textbf{证明思路}:

展开 $\mathbf{x}^T \mathbf{A} \mathbf{x} = \sum_{i,j} a_{ij} x_i x_j$,对 $x_k$ 求偏导:

\begin{align}
\frac{\partial}{\partial x_k}(\mathbf{x}^T \mathbf{A} \mathbf{x}) &= \sum_{i,j} a_{ij} \frac{\partial}{\partial x_k}(x_i x_j) \\
&= \sum_{i,j} a_{ij} (\delta_{ik} x_j + x_i \delta_{jk}) \\
&= \sum_j a_{kj} x_j + \sum_i a_{ik} x_i \\
&= [\mathbf{A}\mathbf{x}]_k + [\mathbf{A}^T \mathbf{x}]_k
\end{align}

因此 $\nabla (\mathbf{x}^T \mathbf{A} \mathbf{x}) = \mathbf{A}\mathbf{x} + \mathbf{A}^T \mathbf{x} = (\mathbf{A} + \mathbf{A}^T)\mathbf{x}$。

\textbf{公式5}:如果 $\mathbf{A}$ 是对称矩阵($\mathbf{A} = \mathbf{A}^T$),则:

\begin{equation}
\nabla (\mathbf{x}^T \mathbf{A} \mathbf{x}) = 2\mathbf{A}\mathbf{x}
\end{equation}

\textbf{公式6}:对于 $\frac{1}{2}\mathbf{x}^T \mathbf{A} \mathbf{x}$,如果 $\mathbf{A}$ 对称:

\begin{equation}
\nabla\left(\frac{1}{2}\mathbf{x}^T \mathbf{A} \mathbf{x}\right) = \mathbf{A}\mathbf{x}
\end{equation}

\subsection{范数的梯度}

\textbf{公式7}:$\nabla (\|\mathbf{x}\|_2^2) = \nabla (\mathbf{x}^T \mathbf{x}) = 2\mathbf{x}$

\textbf{证明}:$\|\mathbf{x}\|_2^2 = \mathbf{x}^T \mathbf{x}$,使用公式5($\mathbf{A} = \mathbf{I}$ 是对称的)。

\textbf{公式8}:$\nabla (\|\mathbf{x}\|_2) = \frac{\mathbf{x}}{\|\mathbf{x}\|_2}$(当 $\mathbf{x} \neq \mathbf{0}$)

\textbf{证明}:

\begin{align}
\nabla (\|\mathbf{x}\|_2) &= \nabla \left(\sqrt{\mathbf{x}^T \mathbf{x}}\right) \\
&= \frac{1}{2\sqrt{\mathbf{x}^T \mathbf{x}}} \nabla (\mathbf{x}^T \mathbf{x}) \\
&= \frac{1}{2\|\mathbf{x}\|_2} \cdot 2\mathbf{x} \\
&= \frac{\mathbf{x}}{\|\mathbf{x}\|_2}
\end{align}

\subsection{矩阵向量乘法的梯度}

\textbf{公式9}:$\nabla (\mathbf{A}\mathbf{x}) = \mathbf{A}^T$

\textbf{注意}:这是雅可比矩阵(Jacobian matrix),不是梯度。

\textbf{说明}:
\begin{itemize}
\item 对于向量值函数 $\mathbf{f}(\mathbf{x}) = \mathbf{A}\mathbf{x}$,梯度是雅可比矩阵
\item 雅可比矩阵是 $\mathbf{A}^T$(转置)
\end{itemize}

\section{复合函数的梯度}

\subsection{链式法则}

\textbf{链式法则}:对于复合函数 $f(g(\mathbf{x}))$:

\begin{equation}
\nabla f(g(\mathbf{x})) = f'(g(\mathbf{x})) \nabla g(\mathbf{x})
\end{equation}

\textbf{例子1}:$f(\mathbf{x}) = \log(\mathbf{a}^T \mathbf{x})$

\begin{align}
\nabla \log(\mathbf{a}^T \mathbf{x}) &= \frac{1}{\mathbf{a}^T \mathbf{x}} \nabla (\mathbf{a}^T \mathbf{x}) \\
&= \frac{1}{\mathbf{a}^T \mathbf{x}} \mathbf{a}
\end{align}

\textbf{例子2}:$f(\mathbf{x}) = e^{\mathbf{a}^T \mathbf{x}}$

\begin{align}
\nabla e^{\mathbf{a}^T \mathbf{x}} &= e^{\mathbf{a}^T \mathbf{x}} \nabla (\mathbf{a}^T \mathbf{x}) \\
&= e^{\mathbf{a}^T \mathbf{x}} \mathbf{a}
\end{align}

\textbf{例子3}:$f(\mathbf{x}) = (b - \mathbf{a}^T \mathbf{x})^2$

\begin{align}
\nabla (b - \mathbf{a}^T \mathbf{x})^2 &= 2(b - \mathbf{a}^T \mathbf{x}) \nabla (b - \mathbf{a}^T \mathbf{x}) \\
&= 2(b - \mathbf{a}^T \mathbf{x}) (-\mathbf{a}) \\
&= -2(b - \mathbf{a}^T \mathbf{x}) \mathbf{a}
\end{align}

\section{在例4.6中的应用}

\subsection{梯度计算}

\textbf{函数}:$f_0(\mathbf{x}) = -\sum_{i=1}^m \log(b_i - \mathbf{a}_i^T \mathbf{x})$

\textbf{每一项}:$-\log(b_i - \mathbf{a}_i^T \mathbf{x})$

\textbf{使用链式法则}:

\begin{align}
\nabla [-\log(b_i - \mathbf{a}_i^T \mathbf{x})] &= -\frac{1}{b_i - \mathbf{a}_i^T \mathbf{x}} \nabla (b_i - \mathbf{a}_i^T \mathbf{x}) \\
&= -\frac{1}{b_i - \mathbf{a}_i^T \mathbf{x}} (-\mathbf{a}_i) \\
&= \frac{1}{b_i - \mathbf{a}_i^T \mathbf{x}} \mathbf{a}_i
\end{align}

\textbf{总梯度}:

\begin{equation}
\nabla f_0(\mathbf{x}) = \sum_{i=1}^m \frac{1}{b_i - \mathbf{a}_i^T \mathbf{x}} \mathbf{a}_i
\end{equation}

\textbf{关键}:这里使用了 $\nabla (b_i - \mathbf{a}_i^T \mathbf{x}) = -\mathbf{a}_i$(基于公式 $\nabla \mathbf{a}^T \mathbf{x} = \mathbf{a}$)。

\section{常用公式总结}

\subsection{基本公式}

\begin{enumerate}
\item $\nabla (\mathbf{a}^T \mathbf{x}) = \mathbf{a}$

\item $\nabla (\mathbf{a}^T \mathbf{x} + b) = \mathbf{a}$

\item $\nabla (\mathbf{x}^T \mathbf{a}) = \mathbf{a}$

\item $\nabla (\mathbf{x}^T \mathbf{A} \mathbf{x}) = (\mathbf{A} + \mathbf{A}^T)\mathbf{x}$

\item 如果 $\mathbf{A} = \mathbf{A}^T$,则 $\nabla (\mathbf{x}^T \mathbf{A} \mathbf{x}) = 2\mathbf{A}\mathbf{x}$

\item $\nabla\left(\frac{1}{2}\mathbf{x}^T \mathbf{A} \mathbf{x}\right) = \mathbf{A}\mathbf{x}$(当 $\mathbf{A}$ 对称时)

\item $\nabla (\|\mathbf{x}\|_2^2) = 2\mathbf{x}$

\item $\nabla (\|\mathbf{x}\|_2) = \frac{\mathbf{x}}{\|\mathbf{x}\|_2}$(当 $\mathbf{x} \neq \mathbf{0}$ 时)
\end{enumerate}

\subsection{复合函数}

\begin{enumerate}
\item $\nabla \log(\mathbf{a}^T \mathbf{x}) = \frac{\mathbf{a}}{\mathbf{a}^T \mathbf{x}}$

\item $\nabla e^{\mathbf{a}^T \mathbf{x}} = e^{\mathbf{a}^T \mathbf{x}} \mathbf{a}$

\item $\nabla (b - \mathbf{a}^T \mathbf{x})^2 = -2(b - \mathbf{a}^T \mathbf{x}) \mathbf{a}$
\end{enumerate}

\section{记忆技巧}

\subsection{线性函数}

\textbf{规则}:线性函数 $\mathbf{a}^T \mathbf{x}$ 的梯度是系数向量 $\mathbf{a}$。

\textbf{类比}:类似于 $f(x) = ax$ 的导数是 $a$。

\subsection{二次型}

\textbf{规则}:二次型 $\mathbf{x}^T \mathbf{A} \mathbf{x}$ 的梯度需要考虑对称性。

\textbf{对称矩阵}:如果 $\mathbf{A}$ 对称,梯度是 $2\mathbf{A}\mathbf{x}$。

\textbf{带系数}:$\frac{1}{2}\mathbf{x}^T \mathbf{A} \mathbf{x}$ 的梯度是 $\mathbf{A}\mathbf{x}$(当 $\mathbf{A}$ 对称时)。

\section{总结}

\subsection{核心公式}

\begin{enumerate}
\item \textbf{线性函数}:$\nabla \mathbf{a}^T \mathbf{x} = \mathbf{a}$

\item \textbf{二次型}:$\nabla \mathbf{x}^T \mathbf{A} \mathbf{x} = (\mathbf{A} + \mathbf{A}^T)\mathbf{x}$

\item \textbf{对称矩阵}:$\nabla \mathbf{x}^T \mathbf{A} \mathbf{x} = 2\mathbf{A}\mathbf{x}$(当 $\mathbf{A}$ 对称时)
</enumerate}

\subsection{应用}

\begin{enumerate}
\item \textbf{快速计算梯度}:使用这些公式可以快速计算梯度

\item \textbf{链式法则}:结合链式法则处理复合函数

\item \textbf{优化问题}:在求解最优性条件时非常有用
</enumerate}

掌握这些矩阵微积分公式,是快速计算梯度的关键!

\end{document}

