\documentclass[12pt,a4paper]{article}
\usepackage[UTF8]{ctex}
\usepackage{amsmath}
\usepackage{amssymb}
\usepackage{amsthm}
\usepackage{geometry}
\geometry{left=2.5cm,right=2.5cm,top=2.5cm,bottom=2.5cm}

\title{inf(下确界)详解}
\subtitle{理解为什么优化问题中使用 inf 而不是 min}
\author{}
\date{\today}

\begin{document}

\maketitle

\section{引言}

在优化问题中,我们经常看到使用 $\inf$(下确界,infimum)而不是 $\min$(最小值,minimum)。理解这两者的区别对于正确理解优化理论至关重要。本文将详细解释 $\inf$ 的概念、性质,以及为什么在优化问题中使用它。

\section{inf(下确界)的定义}

\subsection{基本定义}

\textbf{定义}:对于实数集合 $S \subseteq \mathbb{R}$,$S$ 的\textbf{下确界}(infimum),记为 $\inf S$,是满足以下条件的最大实数 $m$:

\begin{enumerate}
\item $m$ 是 $S$ 的下界:对于所有 $s \in S$,都有 $s \geq m$
\item $m$ 是最大的下界:对于任何 $m' > m$,存在 $s \in S$ 使得 $s < m'$
\end{enumerate}

\subsection{通俗理解}

\begin{itemize}
\item \textbf{下确界}是集合的"最大下界"
\item 它是所有下界中最大的那个
\item 下确界可能属于集合,也可能不属于集合
\item 如果下确界属于集合,它就是最小值
\end{itemize}

\section{inf 与 min 的区别}

\subsection{min(最小值)的定义}

\textbf{定义}:$\min S$ 是集合 $S$ 中的最小元素,即:
\begin{itemize}
\item $\min S \in S$(最小值必须在集合中)
\item 对于所有 $s \in S$,都有 $s \geq \min S$
\end{itemize}

\subsection{关键区别}

\begin{table}[h]
\centering
\begin{tabular}{|l|l|l|}
\hline
\textbf{性质} & \textbf{min} & \textbf{inf} \\
\hline
是否必须在集合中 & 是 & 不一定 \\
\hline
存在性 & 可能不存在 & 总是存在(可能是 $\pm\infty$) \\
\hline
关系 & 如果 $\min S$ 存在,则 $\min S = \inf S$ & $\inf S$ 总是存在 \\
\hline
\end{tabular}
\caption{min 与 inf 的区别}
\end{table}

\subsection{例子说明}

\textbf{例子1:min 和 inf 相同的情况}

设 $S = \{1, 2, 3, 4, 5\}$:
\begin{itemize}
\item $\min S = 1$(最小值存在)
\item $\inf S = 1$(下确界等于最小值)
\item 因为 $1 \in S$,所以 $\min S = \inf S$
\end{itemize}

\textbf{例子2:min 不存在但 inf 存在的情况}

设 $S = \{x \in \mathbb{R} \mid x > 0\}$(所有正实数):
\begin{itemize}
\item $\min S$ 不存在(因为对于任何 $x > 0$,都存在 $x/2 > 0$ 且 $x/2 < x$)
\item $\inf S = 0$(0是最大的下界)
\item 注意:$0 \notin S$,所以 $\min S$ 不存在,但 $\inf S = 0$
\end{itemize}

\textbf{例子3:无下界的情况}

设 $S = \{x \in \mathbb{R} \mid x < 0\}$(所有负实数):
\begin{itemize}
\item $\min S$ 不存在
\item $\inf S = -\infty$(没有下界,下确界是 $-\infty$)
\end{itemize}

\textbf{例子4:空集的情况}

设 $S = \emptyset$(空集):
\begin{itemize}
\item $\min S$ 不存在
\item $\inf S = +\infty$(按照约定,空集的下确界是 $+\infty$)
\end{itemize}

\section{为什么优化问题中使用 inf?}

\subsection{问题}

在优化问题中,最优值定义为:
\begin{equation}
p^* = \inf \{f_0(\mathbf{x}) \mid \mathbf{x} \in \mathcal{X}\}
\end{equation}

\textbf{为什么使用 $\inf$ 而不是 $\min$?}

\subsection{原因1:最优值可能未达到}

\textbf{情况}:即使最优值 $p^*$ 是有限的,也可能不存在最优点 $\mathbf{x}^*$ 使得 $f_0(\mathbf{x}^*) = p^*$。

\textbf{例子}:考虑问题
\begin{equation}
\text{minimize } f_0(x) = \frac{1}{x}, \quad \text{subject to } x > 0
\end{equation}

\textbf{分析}:
\begin{itemize}
\item 可行集:$\mathcal{X} = \{x \mid x > 0\}$
\item 目标函数值集合:$S = \{1/x \mid x > 0\} = \{y \mid y > 0\}$
\item 最优值:$p^* = \inf S = 0$
\item 但是:不存在 $x > 0$ 使得 $1/x = 0$
\item 因此:$\min S$ 不存在,但 $\inf S = 0$ 存在
\end{itemize}

\textbf{结论}:使用 $\inf$ 可以处理最优值未达到的情况。

\subsection{原因2:处理无下界问题}

\textbf{情况}:如果问题无下界,最优值应该是 $-\infty$。

\textbf{例子}:考虑问题
\begin{equation}
\text{minimize } f_0(x) = -x, \quad \text{subject to } x \geq 0
\end{equation}

\textbf{分析}:
\begin{itemize}
\item 可行集:$\mathcal{X} = \{x \mid x \geq 0\}$
\item 目标函数值集合:$S = \{-x \mid x \geq 0\} = \{y \mid y \leq 0\}$
\item 最优值:$p^* = \inf S = -\infty$
\item $\min S$ 不存在(因为 $S$ 无下界)
\item 但 $\inf S = -\infty$ 有明确定义
\end{itemize}

\subsection{原因3:处理不可行问题}

\textbf{情况}:如果问题不可行,可行集为空。

\textbf{分析}:
\begin{itemize}
\item 可行集:$\mathcal{X} = \emptyset$
\item 目标函数值集合:$S = \emptyset$
\item 最优值:$p^* = \inf \emptyset = +\infty$(按照约定)
\item $\min \emptyset$ 没有定义
\item 但 $\inf \emptyset = +\infty$ 有明确定义
\end{itemize}

\subsection{原因4:统一处理所有情况}

使用 $\inf$ 可以统一处理:
\begin{itemize}
\item 最优值达到的情况:$\inf = \min$
\item 最优值未达到的情况:$\inf$ 存在,$\min$ 不存在
\item 无下界的情况:$\inf = -\infty$
\item 不可行的情况:$\inf = +\infty$
\end{itemize}

\section{sup(上确界)}

\subsection{定义}

\textbf{上确界}(supremum),记为 $\sup$,是下确界的对偶概念:

对于集合 $S \subseteq \mathbb{R}$,$\sup S$ 是满足以下条件的最小实数 $M$:
\begin{enumerate}
\item $M$ 是 $S$ 的上界:对于所有 $s \in S$,都有 $s \leq M$
\item $M$ 是最小的上界:对于任何 $M' < M$,存在 $s \in S$ 使得 $s > M'$
\end{enumerate}

\subsection{与 max 的关系}

\begin{itemize}
\item 如果 $\max S$ 存在,则 $\max S = \sup S$
\item $\sup S$ 总是存在(可能是 $\pm\infty$)
\item $\max S$ 可能不存在
\end{itemize}

\section{具体例子}

\subsection{例子1:闭区间}

设 $S = [0, 1] = \{x \mid 0 \leq x \leq 1\}$:
\begin{itemize}
\item $\min S = 0$,$\inf S = 0$(相同)
\item $\max S = 1$,$\sup S = 1$(相同)
\end{itemize}

\subsection{例子2:开区间}

设 $S = (0, 1) = \{x \mid 0 < x < 1\}$:
\begin{itemize}
\item $\min S$ 不存在(因为对于任何 $x > 0$,都存在 $x/2 \in S$ 且 $x/2 < x$)
\item $\inf S = 0$(0是最大的下界,但 $0 \notin S$)
\item $\max S$ 不存在
\item $\sup S = 1$(1是最小的上界,但 $1 \notin S$)
\end{itemize}

\subsection{例子3:优化问题中的例子}

\textbf{问题1}:
\begin{equation}
\text{minimize } f_0(x) = x^2, \quad \text{subject to } x > 0
\end{equation}

\textbf{分析}:
\begin{itemize}
\item 可行集:$\mathcal{X} = \{x \mid x > 0\}$
\item 目标函数值集合:$S = \{x^2 \mid x > 0\} = \{y \mid y > 0\}$
\item $p^* = \inf S = 0$
\item 最优值未达到(因为不存在 $x > 0$ 使得 $x^2 = 0$)
\end{itemize}

\textbf{问题2}:
\begin{equation}
\text{minimize } f_0(x) = x^2, \quad \text{subject to } x \geq 0
\end{equation}

\textbf{分析}:
\begin{itemize}
\item 可行集:$\mathcal{X} = \{x \mid x \geq 0\}$
\item 目标函数值集合:$S = \{x^2 \mid x \geq 0\} = \{y \mid y \geq 0\}$
\item $p^* = \inf S = 0 = \min S$
\item 最优值达到($x^* = 0$ 是最优点)
\end{itemize}

\section{在优化问题中的应用}

\subsection{最优值的定义}

在优化问题中,最优值定义为:
\begin{equation}
p^* = \inf \{f_0(\mathbf{x}) \mid \mathbf{x} \in \mathcal{X}\}
\end{equation}

\textbf{各种情况}:

\begin{enumerate}
\item \textbf{最优值达到}:
   \begin{itemize}
   \item 存在 $\mathbf{x}^* \in \mathcal{X}$ 使得 $f_0(\mathbf{x}^*) = p^*$
   \item 此时 $p^* = \min \{f_0(\mathbf{x}) \mid \mathbf{x} \in \mathcal{X}\}$
   \end{itemize}

\item \textbf{最优值未达到}:
   \begin{itemize}
   \item $p^*$ 是有限的,但不存在 $\mathbf{x}^*$ 使得 $f_0(\mathbf{x}^*) = p^*$
   \item 此时 $\min$ 不存在,但 $\inf = p^*$ 存在
   \end{itemize}

\item \textbf{无下界}:
   \begin{itemize}
   \item $p^* = -\infty$
   \item 存在可行点序列使得目标函数值趋于 $-\infty$
   \end{itemize}

\item \textbf{不可行}:
   \begin{itemize}
   \item $p^* = +\infty$(按照约定,空集的下确界是 $+\infty$)
   \end{itemize}
\end{enumerate}

\section{数学性质}

\subsection{基本性质}

\begin{enumerate}
\item \textbf{单调性}:如果 $S_1 \subseteq S_2$,则 $\inf S_2 \leq \inf S_1$

\item \textbf{下确界总是存在}:对于任何非空集合 $S$,$\inf S$ 总是存在(可能是 $\pm\infty$)

\item \textbf{与最小值的关系}:如果 $\min S$ 存在,则 $\min S = \inf S$

\item \textbf{空集约定}:$\inf \emptyset = +\infty$,$\sup \emptyset = -\infty$
\end{enumerate}

\subsection{计算规则}

\begin{itemize}
\item $\inf (S_1 + S_2) = \inf S_1 + \inf S_2$(如果 $S_1 + S_2 = \{s_1 + s_2 \mid s_1 \in S_1, s_2 \in S_2\}$)

\item $\inf (\alpha S) = \alpha \inf S$(如果 $\alpha > 0$)

\item $\inf (-S) = -\sup S$(其中 $-S = \{-s \mid s \in S\}$)
\end{itemize}

\section{记忆技巧}

\subsection{inf vs min}

\begin{itemize}
\item \textbf{inf}:\textbf{inf}imum = \textbf{inf}erior(下界),是"最大下界"
\item \textbf{min}:\textbf{min}imum = 最小值,必须在集合中
\item \textbf{关系}:如果最小值存在,则 $\min = \inf$;但 $\inf$ 总是存在
\end{itemize}

\subsection{sup vs max}

\begin{itemize}
\item \textbf{sup}:\textbf{sup}remum = \textbf{sup}erior(上界),是"最小上界"
\item \textbf{max}:\textbf{max}imum = 最大值,必须在集合中
\item \textbf{关系}:如果最大值存在,则 $\max = \sup$;但 $\sup$ 总是存在
\end{itemize}

\section{总结}

\begin{enumerate}
\item \textbf{inf(下确界)}:
   \begin{itemize}
   \item 定义:集合的最大下界
   \item 性质:总是存在(可能是 $\pm\infty$)
   \item 不一定属于集合
   \end{itemize}

\item \textbf{min(最小值)}:
   \begin{itemize}
   \item 定义:集合中的最小元素
   \item 性质:可能不存在
   \item 必须在集合中
   \end{itemize}

\item \textbf{关系}:
   \begin{itemize}
   \item 如果 $\min$ 存在,则 $\min = \inf$
   \item 但 $\inf$ 总是存在,即使 $\min$ 不存在
   \end{itemize}

\item \textbf{在优化问题中}:
   \begin{itemize}
   \item 使用 $\inf$ 可以统一处理所有情况
   \item 包括最优值未达到、无下界、不可行等情况
   \item 这是优化理论的标准做法
   \end{itemize}
\end{enumerate}

理解 $\inf$ 和 $\min$ 的区别,对于正确理解优化问题的解的概念至关重要!

\end{document}

