\documentclass[12pt,a4paper]{article}
\usepackage[UTF8]{ctex}
\usepackage{amsmath}
\usepackage{amssymb}
\usepackage{amsthm}
\usepackage{geometry}
\geometry{left=2.5cm,right=2.5cm,top=2.5cm,bottom=2.5cm}

\title{例4.5:无约束二次优化详解}
\author{}
\date{\today}

\begin{document}

\maketitle

\section{引言}

例4.5展示了如何求解无约束二次优化问题。这个例子不仅展示了最优性条件的应用,还说明了不同情况下最优解的存在性和唯一性。

\section{符号说明:$\mathbb{S}_+^n$ 是什么?}

\subsection{对称矩阵集合}

\textbf{$\mathbb{S}^n$}:$n \times n$ 对称矩阵的集合

\begin{equation}
\mathbb{S}^n = \{\mathbf{X} \in \mathbb{R}^{n \times n} \mid \mathbf{X} = \mathbf{X}^T\}
\end{equation}

\textbf{性质}:
\begin{itemize}
\item 这是一个向量空间
\item 维度:$\frac{n(n+1)}{2}$(因为对称矩阵只需要存储上三角部分)
\item 所有对称矩阵的集合
\end{itemize}

\subsection{半正定矩阵集合}

\textbf{$\mathbb{S}_+^n$}:$n \times n$ 对称半正定矩阵的集合

\begin{equation}
\mathbb{S}_+^n = \{\mathbf{X} \in \mathbb{S}^n \mid \mathbf{X} \succeq 0\}
\end{equation}

\textbf{等价定义}:

\begin{equation}
\mathbb{S}_+^n = \{\mathbf{X} \in \mathbb{S}^n \mid \mathbf{v}^T \mathbf{X} \mathbf{v} \geq 0, \forall \mathbf{v} \in \mathbb{R}^n\}
\end{equation}

\textbf{含义}:
\begin{itemize}
\item 所有特征值 $\geq 0$ 的对称矩阵
\item 对于任意向量 $\mathbf{v}$,有 $\mathbf{v}^T \mathbf{X} \mathbf{v} \geq 0$
\item 半正定矩阵的集合
\end{itemize}

\subsection{正定矩阵集合}

\textbf{$\mathbb{S}_{++}^n$}:$n \times n$ 对称正定矩阵的集合

\begin{equation}
\mathbb{S}_{++}^n = \{\mathbf{X} \in \mathbb{S}^n \mid \mathbf{X} \succ 0\}
\end{equation}

\textbf{等价定义}:

\begin{equation}
\mathbb{S}_{++}^n = \{\mathbf{X} \in \mathbb{S}^n \mid \mathbf{v}^T \mathbf{X} \mathbf{v} > 0, \forall \mathbf{v} \neq \mathbf{0}\}
\end{equation}

\textbf{含义}:
\begin{itemize}
\item 所有特征值 $> 0$ 的对称矩阵
\item 对于任意非零向量 $\mathbf{v}$,有 $\mathbf{v}^T \mathbf{X} \mathbf{v} > 0$
\item 正定矩阵的集合
\end{itemize}

\subsection{符号类比}

\textbf{类比}:
\begin{itemize}
\item $\mathbb{R}_+$:非负实数集合 $\{x \in \mathbb{R} \mid x \geq 0\}$
\item $\mathbb{R}_{++}$:正实数集合 $\{x \in \mathbb{R} \mid x > 0\}$
\item $\mathbb{S}_+^n$:半正定矩阵集合(类似 $\mathbb{R}_+$)
\item $\mathbb{S}_{++}^n$:正定矩阵集合(类似 $\mathbb{R}_{++}$)
\end{itemize}

\subsection{性质}

\textbf{$\mathbb{S}_+^n$ 的性质}:
\begin{enumerate}
\item \textbf{凸锥}:如果 $\theta_1, \theta_2 \geq 0$ 且 $\mathbf{A}, \mathbf{B} \in \mathbb{S}_+^n$,则 $\theta_1 \mathbf{A} + \theta_2 \mathbf{B} \in \mathbb{S}_+^n$

\item \textbf{闭集}:$\mathbb{S}_+^n$ 是闭集

\item \textbf{包含关系}:$\mathbb{S}_{++}^n \subset \mathbb{S}_+^n \subset \mathbb{S}^n$
\end{enumerate}

\section{例题4.5:无约束二次优化}

\subsection{问题陈述}

\textbf{问题}:最小化二次函数

\begin{equation}
f_0(\mathbf{x}) = \frac{1}{2}\mathbf{x}^T \mathbf{P} \mathbf{x} + \mathbf{q}^T \mathbf{x} + r
\end{equation}

其中:
\begin{itemize}
\item $\mathbf{P} \in \mathbb{S}_+^n$(对称半正定矩阵,保证 $f_0$ 是凸函数)
\item $\mathbf{q} \in \mathbb{R}^n$(向量)
\item $r \in \mathbb{R}$(标量)
\end{itemize}

\textbf{无约束}:没有约束条件,$\mathbf{x} \in \mathbb{R}^n$。

\subsection{为什么 $\mathbf{P} \in \mathbb{S}_+^n$ 保证凸性?}

\textbf{二次函数的Hessian矩阵}:

\begin{equation}
\nabla^2 f_0(\mathbf{x}) = \mathbf{P}
\end{equation}

\textbf{凸性条件}:

函数 $f_0$ 是凸函数,当且仅当 $\nabla^2 f_0(\mathbf{x}) \succeq 0$ 对所有 $\mathbf{x}$ 成立。

由于 $\nabla^2 f_0(\mathbf{x}) = \mathbf{P}$ 是常数矩阵,因此:

\begin{equation}
f_0 \text{ 是凸函数 } \Leftrightarrow \mathbf{P} \succeq 0 \Leftrightarrow \mathbf{P} \in \mathbb{S}_+^n
\end{equation}

\subsection{最优性条件}

\textbf{梯度}:

\begin{equation}
\nabla f_0(\mathbf{x}) = \mathbf{P}\mathbf{x} + \mathbf{q}
\end{equation}

\textbf{最优性条件}:

对于无约束问题,最优性条件是:

\begin{equation}
\nabla f_0(\mathbf{x}) = \mathbf{0}
\end{equation}

即:

\begin{equation}
\mathbf{P}\mathbf{x} + \mathbf{q} = \mathbf{0}
\end{equation}

这是关于 $\mathbf{x}$ 的线性方程组。

\section{三种情况分析}

\subsection{情况1:无解($\mathbf{q} \notin R(\mathbf{P})$)}

\textbf{条件}:$\mathbf{q} \notin R(\mathbf{P})$,其中 $R(\mathbf{P})$ 是矩阵 $\mathbf{P}$ 的列空间(range space)。

\textbf{含义}:
\begin{itemize}
\item 方程 $\mathbf{P}\mathbf{x} + \mathbf{q} = \mathbf{0}$ 无解
\item 即 $\mathbf{P}\mathbf{x} = -\mathbf{q}$ 无解
\item 这意味着 $-\mathbf{q}$ 不在 $\mathbf{P}$ 的列空间中
\end{itemize}

\textbf{结果}:$f_0$ 无下界(unbounded below)。

\textbf{为什么?}

考虑方向 $\mathbf{d} \in N(\mathbf{P})$($\mathbf{P}$ 的零空间),即 $\mathbf{P}\mathbf{d} = \mathbf{0}$。

对于 $\mathbf{x}(t) = \mathbf{x}_0 + t\mathbf{d}$,有:

\begin{align}
f_0(\mathbf{x}(t)) &= \frac{1}{2}(\mathbf{x}_0 + t\mathbf{d})^T \mathbf{P}(\mathbf{x}_0 + t\mathbf{d}) + \mathbf{q}^T(\mathbf{x}_0 + t\mathbf{d}) + r \\
&= \frac{1}{2}\mathbf{x}_0^T \mathbf{P}\mathbf{x}_0 + t\mathbf{x}_0^T \mathbf{P}\mathbf{d} + \frac{t^2}{2}\mathbf{d}^T \mathbf{P}\mathbf{d} + \mathbf{q}^T\mathbf{x}_0 + t\mathbf{q}^T\mathbf{d} + r \\
&= \frac{1}{2}\mathbf{x}_0^T \mathbf{P}\mathbf{x}_0 + \mathbf{q}^T\mathbf{x}_0 + r + t\mathbf{q}^T\mathbf{d}
\end{align}

如果 $\mathbf{q}^T\mathbf{d} < 0$,则当 $t \to +\infty$ 时,$f_0(\mathbf{x}(t)) \to -\infty$。

由于 $\mathbf{q} \notin R(\mathbf{P})$,存在 $\mathbf{d} \in N(\mathbf{P})$ 使得 $\mathbf{q}^T\mathbf{d} \neq 0$,因此 $f_0$ 无下界。

\subsection{情况2:唯一解($\mathbf{P} \succ 0$)}

\textbf{条件}:$\mathbf{P} \succ 0$(正定,即 $\mathbf{P} \in \mathbb{S}_{++}^n$)

\textbf{含义}:
\begin{itemize}
\item $\mathbf{P}$ 是可逆的
\item $f_0$ 是严格凸函数
\item 方程 $\mathbf{P}\mathbf{x} + \mathbf{q} = \mathbf{0}$ 有唯一解
\end{itemize}

\textbf{最优解}:

\begin{equation}
\mathbf{x}^* = -\mathbf{P}^{-1}\mathbf{q}
\end{equation}

\textbf{验证}:

\begin{align}
\nabla f_0(\mathbf{x}^*) &= \mathbf{P}(-\mathbf{P}^{-1}\mathbf{q}) + \mathbf{q} \\
&= -\mathbf{q} + \mathbf{q} = \mathbf{0} \quad \checkmark
\end{align}

\textbf{几何意义}:
\begin{itemize}
\item 严格凸函数有唯一的全局最小值
\item 最优解是二次函数的"底部"
\end{itemize}

\subsection{情况3:多个解($\mathbf{P}$ 奇异,但 $\mathbf{q} \in R(\mathbf{P})$)}

\textbf{条件}:
\begin{itemize}
\item $\mathbf{P}$ 是奇异的(不可逆,即 $\mathbf{P} \succeq 0$ 但 $\mathbf{P} \not\succ 0$)
\item $\mathbf{q} \in R(\mathbf{P})$($\mathbf{q}$ 在 $\mathbf{P}$ 的列空间中)
\end{itemize}

\textbf{含义}:
\begin{itemize}
\item 方程 $\mathbf{P}\mathbf{x} + \mathbf{q} = \mathbf{0}$ 有解,但不唯一
\item 最优解集合是一个仿射集合
\end{itemize}

\textbf{最优解集合}:

\begin{equation}
\mathcal{X}_{\text{opt}} = -\mathbf{P}^{\dagger}\mathbf{q} + N(\mathbf{P})
\end{equation}

其中:
\begin{itemize}
\item $\mathbf{P}^{\dagger}$:$\mathbf{P}$ 的伪逆(pseudo-inverse)
\item $N(\mathbf{P})$:$\mathbf{P}$ 的零空间(null space)
\end{itemize}

\textbf{解释}:
\begin{itemize}
\item $-\mathbf{P}^{\dagger}\mathbf{q}$:一个特解
\item $N(\mathbf{P})$:齐次方程 $\mathbf{P}\mathbf{x} = \mathbf{0}$ 的解空间
\item 最优解集合 = 特解 + 齐次解空间
\end{itemize}

\textbf{为什么是最优解?}

对于任意 $\mathbf{x} \in \mathcal{X}_{\text{opt}}$,有 $\mathbf{P}\mathbf{x} + \mathbf{q} = \mathbf{0}$,因此 $\nabla f_0(\mathbf{x}) = \mathbf{0}$,满足最优性条件。

\section{具体例子}

\subsection{例子1:正定情况}

\textbf{问题}:$\min_{x, y} x^2 + 2y^2 + x + y$

\textbf{矩阵形式}:

\begin{equation}
f_0(\mathbf{x}) = \frac{1}{2}\mathbf{x}^T \begin{pmatrix} 2 & 0 \\ 0 & 4 \end{pmatrix} \mathbf{x} + \begin{pmatrix} 1 \\ 1 \end{pmatrix}^T \mathbf{x}
\end{equation}

\textbf{分析}:
\begin{itemize}
\item $\mathbf{P} = \begin{pmatrix} 2 & 0 \\ 0 & 4 \end{pmatrix} \succ 0$(正定)
\item 唯一最优解:$\mathbf{x}^* = -\mathbf{P}^{-1}\mathbf{q} = -\begin{pmatrix} 1/2 & 0 \\ 0 & 1/4 \end{pmatrix}\begin{pmatrix} 1 \\ 1 \end{pmatrix} = \begin{pmatrix} -1/2 \\ -1/4 \end{pmatrix}$
\end{itemize}

\subsection{例子2:奇异情况}

\textbf{问题}:$\min_{x, y} x^2 + x + y$

\textbf{矩阵形式}:

\begin{equation}
f_0(\mathbf{x}) = \frac{1}{2}\mathbf{x}^T \begin{pmatrix} 2 & 0 \\ 0 & 0 \end{pmatrix} \mathbf{x} + \begin{pmatrix} 1 \\ 1 \end{pmatrix}^T \mathbf{x}
\end{equation}

\textbf{分析}:
\begin{itemize}
\item $\mathbf{P} = \begin{pmatrix} 2 & 0 \\ 0 & 0 \end{pmatrix} \succeq 0$(半正定,但奇异)
\item $\mathbf{q} = \begin{pmatrix} 1 \\ 1 \end{pmatrix}$
\item 检查:$\mathbf{q} \in R(\mathbf{P})$?$R(\mathbf{P}) = \text{span}\{(1, 0)^T\}$,但 $\mathbf{q}$ 的第二个分量是 $1 \neq 0$,所以 $\mathbf{q} \notin R(\mathbf{P})$
\item 结果:无下界
\end{itemize}

\textbf{验证}:沿方向 $(0, 1)^T$(在 $N(\mathbf{P})$ 中),函数值可以任意小。

\subsection{例子3:多个最优解}

\textbf{问题}:$\min_{x, y} x^2 + x$

\textbf{矩阵形式}:

\begin{equation}
f_0(\mathbf{x}) = \frac{1}{2}\mathbf{x}^T \begin{pmatrix} 2 & 0 \\ 0 & 0 \end{pmatrix} \mathbf{x} + \begin{pmatrix} 1 \\ 0 \end{pmatrix}^T \mathbf{x}
\end{equation}

\textbf{分析}:
\begin{itemize}
\item $\mathbf{P} = \begin{pmatrix} 2 & 0 \\ 0 & 0 \end{pmatrix}$(奇异)
\item $\mathbf{q} = \begin{pmatrix} 1 \\ 0 \end{pmatrix} \in R(\mathbf{P})$
\item 最优性条件:$2x + 1 = 0$,即 $x = -1/2$,$y$ 任意
\item 最优解集合:$\{(x, y) \mid x = -1/2, y \in \mathbb{R}\}$
\end{itemize}

\section{总结}

\subsection{$\mathbb{S}_+^n$ 的含义}

\begin{enumerate}
\item \textbf{定义}:$n \times n$ 对称半正定矩阵的集合

\item \textbf{符号}:$\mathbb{S}_+^n = \{\mathbf{X} \in \mathbb{S}^n \mid \mathbf{X} \succeq 0\}$

\item \textbf{性质}:凸锥、闭集
\end{enumerate}

\subsection{例题4.5的关键点}

\begin{enumerate}
\item \textbf{问题}:无约束二次优化

\item \textbf{最优性条件}:$\mathbf{P}\mathbf{x} + \mathbf{q} = \mathbf{0}$

\item \textbf{三种情况}:
   \begin{itemize}
   \item 无解:$\mathbf{q} \notin R(\mathbf{P})$ → 无下界
   \item 唯一解:$\mathbf{P} \succ 0$ → $\mathbf{x}^* = -\mathbf{P}^{-1}\mathbf{q}$
   \item 多个解:$\mathbf{P}$ 奇异但 $\mathbf{q} \in R(\mathbf{P})$ → 仿射解集
   \end{itemize}
\end{enumerate}

\subsection{关键理解}

\begin{enumerate}
\item \textbf{$\mathbf{P} \in \mathbb{S}_+^n$}:保证函数是凸的

\item \textbf{最优性条件}:梯度为零

\item \textbf{解的存在性}:取决于 $\mathbf{q}$ 是否在 $\mathbf{P}$ 的列空间中

\item \textbf{解的唯一性}:取决于 $\mathbf{P}$ 是否正定
\end{enumerate}

理解这个例子,有助于掌握无约束优化和矩阵理论!

\end{document}

