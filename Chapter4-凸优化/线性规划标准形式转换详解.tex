\documentclass[12pt,a4paper]{article}
\usepackage[UTF8]{ctex}
\usepackage{amsmath}
\usepackage{amssymb}
\usepackage{amsthm}
\usepackage{geometry}
\geometry{left=2.5cm,right=2.5cm,top=2.5cm,bottom=2.5cm}

\title{线性规划标准形式转换详解}
\author{}
\date{\today}

\begin{document}

\maketitle

\section{问题提出}

在将一般线性规划问题转化为标准形式时,需要处理无符号约束的变量。书中提到:

\textbf{第二步}:将变量 $\mathbf{x}$ 表示为两个非负变量的差,即 $\mathbf{x} = \mathbf{x}^+ - \mathbf{x}^-$,其中 $\mathbf{x}^+, \mathbf{x}^- \succeq \mathbf{0}$。

\textbf{问题}:这个构造是怎么来的?为什么可以这样表示?如何理解这个转换过程?

\section{一般线性规划问题}

\subsection{原始问题}

考虑一般形式的线性规划问题:
\begin{align}
\text{minimize} \quad & \mathbf{c}^T \mathbf{x} + d \\
\text{subject to} \quad & \mathbf{G}\mathbf{x} \preceq \mathbf{h} \\
& \mathbf{A}\mathbf{x} = \mathbf{b}
\end{align}

其中:
\begin{itemize}
\item $\mathbf{x} \in \mathbb{R}^n$:决策变量(\textbf{无符号约束},即 $\mathbf{x}$ 可以取任意实数值)
\item $\mathbf{c} \in \mathbb{R}^n$:目标函数系数向量
\item $d \in \mathbb{R}$:目标函数常数项
\item $\mathbf{G} \in \mathbb{R}^{m \times n}$:不等式约束矩阵
\item $\mathbf{h} \in \mathbb{R}^m$:不等式约束常数向量
\item $\mathbf{A} \in \mathbb{R}^{p \times n}$:等式约束矩阵
\item $\mathbf{b} \in \mathbb{R}^p$:等式约束常数向量
\end{itemize}

\textbf{关键点}:变量 $\mathbf{x}$ 没有非负约束,可以取任意实数值(包括负数)。

\section{标准形式的要求}

\subsection{标准形式}

线性规划的标准形式要求:
\begin{align}
\text{minimize} \quad & \mathbf{c}^T \mathbf{x} \\
\text{subject to} \quad & \mathbf{A}\mathbf{x} = \mathbf{b} \\
& \mathbf{x} \succeq \mathbf{0}
\end{align}

\textbf{关键要求}:
\begin{itemize}
\item 所有变量必须非负:$\mathbf{x} \succeq \mathbf{0}$
\item 只有等式约束(没有不等式约束)
\item 目标函数没有常数项(或常数项可以忽略)
\end{itemize}

\subsection{为什么要求所有变量非负?}

\textbf{问题}:为什么标准形式要求所有变量非负?这个要求从何而来?

\textbf{答案}:这个要求主要来自以下几个方面:

\subsubsection{原因1:单纯形法的需要}

\textbf{单纯形法}(Simplex Method)是求解线性规划最经典的方法,它要求:

\begin{enumerate}
\item \textbf{基本可行解的定义}:
   \begin{itemize}
   \item 基本解:将非基变量设为0,求解基变量
   \item 基本可行解:基本解满足非负约束
   \item 如果变量可以为负,基本解的定义会变得复杂
   \end{itemize}

\item \textbf{顶点搜索}:
   \begin{itemize}
   \item 单纯形法在多面体的顶点之间移动
   \item 顶点对应基本可行解
   \item 非负约束使得顶点的识别更加清晰
   \end{itemize}

\item \textbf{进基和出基操作}:
   \begin{itemize}
   \item 进基变量从0开始增加
   \item 出基变量减少到0
   \item 如果变量可以为负,这些操作会变得复杂
   \end{itemize}
\end{enumerate}

\textbf{例子}:在单纯形法中,我们通过将非基变量设为0来定义基本解。如果变量可以为负,我们需要考虑"设为0"是否合理,以及如何定义基本解。

\subsubsection{原因2:基本可行解的理论}

\textbf{基本可行解}(Basic Feasible Solution, BFS)是线性规划理论的核心概念:

\begin{itemize}
\item \textbf{定义}:基本可行解是满足非负约束的基本解
\item \textbf{重要性}:如果线性规划有最优解,则至少有一个最优解是基本可行解
\item \textbf{有限性}:基本可行解的数量是有限的(最多 $\binom{n}{m}$ 个)
\end{itemize}

\textbf{如果变量可以为负}:
\begin{itemize}
\item 基本解的定义需要修改
\item 基本可行解的概念变得模糊
\item 理论分析变得更加复杂
\end{itemize}

\subsubsection{原因3:实际应用的自然性}

\textbf{许多实际问题}中,变量天然就是非负的:

\begin{itemize}
\item \textbf{生产量}:不能为负
\item \textbf{资源分配}:分配量不能为负
\item \textbf{运输量}:运输量不能为负
\item \textbf{投资额}:在某些模型中,投资额不能为负
\end{itemize}

\textbf{例子}:
\begin{itemize}
\item 生产计划问题:$x_i$ 表示第 $i$ 种产品的生产量,显然 $x_i \geq 0$
\item 运输问题:$x_{ij}$ 表示从地点 $i$ 到地点 $j$ 的运输量,显然 $x_{ij} \geq 0$
\end{itemize}

\subsubsection{原因4:理论分析的便利性}

\textbf{非负约束}使得理论分析更加便利:

\begin{enumerate}
\item \textbf{可行域的性质}:
   \begin{itemize}
   \item 可行域是有界或无界的多面体
   \item 非负约束使得可行域在第一象限(或非负象限)
   \item 几何结构更加清晰
   \end{itemize}

\item \textbf{对偶理论}:
   \begin{itemize}
   \item 对偶变量的符号有明确含义
   \item 互补松弛条件更加清晰
   \end{itemize}

\item \textbf{最优性条件}:
   \begin{itemize}
   \item KKT条件中的非负性条件更加明确
   \item 最优解的识别更加容易
   \end{itemize}
\end{enumerate}

\subsubsection{原因5:历史原因}

\textbf{线性规划的发展历史}:

\begin{itemize}
\item 线性规划最初用于解决实际问题(如生产计划、运输问题)
\item 这些问题中的变量天然就是非负的
\item 标准形式是在这些实际应用的基础上发展起来的
\item 单纯形法(1947年)就是针对这种标准形式设计的
\end{itemize}

\textbf{因此}:标准形式要求所有变量非负,是历史发展和实际应用共同作用的结果。

\subsubsection{总结}

\textbf{标准形式要求所有变量非负的原因}:

\begin{enumerate}
\item \textbf{算法需要}:单纯形法等算法需要非负约束
\item \textbf{理论需要}:基本可行解等理论概念需要非负约束
\item \textbf{实际需要}:许多实际问题中的变量天然就是非负的
\item \textbf{分析便利}:非负约束使得理论分析更加便利
\item \textbf{历史原因}:标准形式是在实际应用的基础上发展起来的
\end{enumerate}

\textbf{关键理解}:标准形式是一个\textbf{约定},是为了算法和理论的便利性而设定的。任何线性规划问题都可以通过变量替换(如 $x = x^+ - x^-$)转化为标准形式。

\section{转换步骤}

\subsection{步骤1:引入松弛变量}

\textbf{目标}:将不等式约束 $\mathbf{G}\mathbf{x} \preceq \mathbf{h}$ 转化为等式约束。

\textbf{方法}:引入松弛变量 $\mathbf{s} \in \mathbb{R}^m$,$\mathbf{s} \succeq \mathbf{0}$。

\textbf{转换}:
\begin{align}
\mathbf{G}\mathbf{x} \preceq \mathbf{h} \quad &\Leftrightarrow \quad \mathbf{G}\mathbf{x} + \mathbf{s} = \mathbf{h}, \quad \mathbf{s} \succeq \mathbf{0}
\end{align}

\textbf{解释}:
\begin{itemize}
\item 如果 $\mathbf{G}\mathbf{x} \preceq \mathbf{h}$,则存在 $\mathbf{s} \succeq \mathbf{0}$ 使得 $\mathbf{G}\mathbf{x} + \mathbf{s} = \mathbf{h}$
\item 如果存在 $\mathbf{s} \succeq \mathbf{0}$ 使得 $\mathbf{G}\mathbf{x} + \mathbf{s} = \mathbf{h}$,则 $\mathbf{G}\mathbf{x} = \mathbf{h} - \mathbf{s} \preceq \mathbf{h}$
\end{itemize}

\textbf{转换后的问题}:
\begin{align}
\text{minimize} \quad & \mathbf{c}^T \mathbf{x} + d \\
\text{subject to} \quad & \mathbf{G}\mathbf{x} + \mathbf{s} = \mathbf{h} \\
& \mathbf{A}\mathbf{x} = \mathbf{b} \\
& \mathbf{s} \succeq \mathbf{0}
\end{align}

\textbf{注意}:此时 $\mathbf{x}$ 仍然没有非负约束。

\subsection{步骤2:处理无符号约束的变量}

\textbf{问题}:标准形式要求所有变量非负,但 $\mathbf{x}$ 可以取任意实数值。

\textbf{关键观察}:任意实数可以表示为两个非负数的差。

\textbf{定理}:对于任意 $\mathbf{x} \in \mathbb{R}^n$,存在 $\mathbf{x}^+, \mathbf{x}^- \in \mathbb{R}^n$,$\mathbf{x}^+, \mathbf{x}^- \succeq \mathbf{0}$,使得:
\begin{equation}
\mathbf{x} = \mathbf{x}^+ - \mathbf{x}^-
\end{equation}

\textbf{构造方法}:
\begin{align}
x_i^+ &= \max\{x_i, 0\} = \begin{cases} x_i & \text{如果 } x_i \geq 0 \\ 0 & \text{如果 } x_i < 0 \end{cases} \\
x_i^- &= \max\{-x_i, 0\} = \begin{cases} 0 & \text{如果 } x_i \geq 0 \\ -x_i & \text{如果 } x_i < 0 \end{cases}
\end{align}

\textbf{验证}:
\begin{itemize}
\item 如果 $x_i \geq 0$:$x_i^+ = x_i$,$x_i^- = 0$,因此 $x_i = x_i^+ - x_i^- = x_i - 0 = x_i$ ✓
\item 如果 $x_i < 0$:$x_i^+ = 0$,$x_i^- = -x_i > 0$,因此 $x_i = x_i^+ - x_i^- = 0 - (-x_i) = x_i$ ✓
\end{itemize}

\textbf{性质}:
\begin{itemize}
\item $\mathbf{x}^+ \succeq \mathbf{0}$,$\mathbf{x}^- \succeq \mathbf{0}$(由定义)
\item $x_i^+ \cdot x_i^- = 0$(至少有一个为0)
\end{itemize}

\subsection{步骤3:代入约束和目标函数}

\textbf{目标函数}:
\begin{align}
\mathbf{c}^T \mathbf{x} + d &= \mathbf{c}^T (\mathbf{x}^+ - \mathbf{x}^-) + d \\
&= \mathbf{c}^T \mathbf{x}^+ - \mathbf{c}^T \mathbf{x}^- + d
\end{align}

\textbf{不等式约束}(已转换为等式):
\begin{align}
\mathbf{G}\mathbf{x} + \mathbf{s} = \mathbf{h} \quad &\Rightarrow \quad \mathbf{G}(\mathbf{x}^+ - \mathbf{x}^-) + \mathbf{s} = \mathbf{h} \\
&\Rightarrow \quad \mathbf{G}\mathbf{x}^+ - \mathbf{G}\mathbf{x}^- + \mathbf{s} = \mathbf{h}
\end{align}

\textbf{等式约束}:
\begin{align}
\mathbf{A}\mathbf{x} = \mathbf{b} \quad &\Rightarrow \quad \mathbf{A}(\mathbf{x}^+ - \mathbf{x}^-) = \mathbf{b} \\
&\Rightarrow \quad \mathbf{A}\mathbf{x}^+ - \mathbf{A}\mathbf{x}^- = \mathbf{b}
\end{align}

\textbf{变量约束}:
\begin{itemize}
\item $\mathbf{x}^+ \succeq \mathbf{0}$
\item $\mathbf{x}^- \succeq \mathbf{0}$
\item $\mathbf{s} \succeq \mathbf{0}$
\end{itemize}

\subsection{最终标准形式}

\textbf{转换后的问题}:
\begin{align}
\text{minimize} \quad & \mathbf{c}^T \mathbf{x}^+ - \mathbf{c}^T \mathbf{x}^- + d \\
\text{subject to} \quad & \mathbf{G}\mathbf{x}^+ - \mathbf{G}\mathbf{x}^- + \mathbf{s} = \mathbf{h} \\
& \mathbf{A}\mathbf{x}^+ - \mathbf{A}\mathbf{x}^- = \mathbf{b} \\
& \mathbf{x}^+ \succeq \mathbf{0}, \quad \mathbf{x}^- \succeq \mathbf{0}, \quad \mathbf{s} \succeq \mathbf{0}
\end{align}

\textbf{变量}:$\mathbf{x}^+ \in \mathbb{R}^n$,$\mathbf{x}^- \in \mathbb{R}^n$,$\mathbf{s} \in \mathbb{R}^m$

\textbf{总变量数}:$2n + m$(原始问题有 $n$ 个变量)

\section{等价性证明}

\subsection{从原始问题到标准形式}

\textbf{定理}:如果 $\mathbf{x}$ 是原始问题的可行解,则存在 $\mathbf{x}^+, \mathbf{x}^-, \mathbf{s}$ 使得 $(\mathbf{x}^+, \mathbf{x}^-, \mathbf{s})$ 是标准形式问题的可行解,且目标函数值相同。

\textbf{证明}:
\begin{enumerate}
\item 设 $\mathbf{x}$ 满足 $\mathbf{G}\mathbf{x} \preceq \mathbf{h}$,$\mathbf{A}\mathbf{x} = \mathbf{b}$

\item 构造 $\mathbf{x}^+, \mathbf{x}^-$:
   \begin{align}
   x_i^+ &= \max\{x_i, 0\} \\
   x_i^- &= \max\{-x_i, 0\}
   \end{align}
   显然 $\mathbf{x}^+ \succeq \mathbf{0}$,$\mathbf{x}^- \succeq \mathbf{0}$,且 $\mathbf{x} = \mathbf{x}^+ - \mathbf{x}^-$

\item 构造 $\mathbf{s}$:
   \begin{equation}
   \mathbf{s} = \mathbf{h} - \mathbf{G}\mathbf{x} = \mathbf{h} - \mathbf{G}(\mathbf{x}^+ - \mathbf{x}^-)
   \end{equation}
   由于 $\mathbf{G}\mathbf{x} \preceq \mathbf{h}$,有 $\mathbf{s} \succeq \mathbf{0}$

\item 验证约束:
   \begin{itemize}
   \item $\mathbf{G}\mathbf{x}^+ - \mathbf{G}\mathbf{x}^- + \mathbf{s} = \mathbf{G}(\mathbf{x}^+ - \mathbf{x}^-) + \mathbf{s} = \mathbf{G}\mathbf{x} + \mathbf{s} = \mathbf{h}$ ✓
   \item $\mathbf{A}\mathbf{x}^+ - \mathbf{A}\mathbf{x}^- = \mathbf{A}(\mathbf{x}^+ - \mathbf{x}^-) = \mathbf{A}\mathbf{x} = \mathbf{b}$ ✓
   \end{itemize}

\item 验证目标函数:
   \begin{equation}
   \mathbf{c}^T \mathbf{x}^+ - \mathbf{c}^T \mathbf{x}^- + d = \mathbf{c}^T (\mathbf{x}^+ - \mathbf{x}^-) + d = \mathbf{c}^T \mathbf{x} + d
   \end{equation}
\end{enumerate}

\subsection{从标准形式到原始问题}

\textbf{定理}:如果 $(\mathbf{x}^+, \mathbf{x}^-, \mathbf{s})$ 是标准形式问题的可行解,则 $\mathbf{x} = \mathbf{x}^+ - \mathbf{x}^-$ 是原始问题的可行解,且目标函数值相同。

\textbf{证明}:
\begin{enumerate}
\item 设 $(\mathbf{x}^+, \mathbf{x}^-, \mathbf{s})$ 满足所有约束

\item 定义 $\mathbf{x} = \mathbf{x}^+ - \mathbf{x}^-$

\item 验证约束:
   \begin{itemize}
   \item $\mathbf{G}\mathbf{x} = \mathbf{G}(\mathbf{x}^+ - \mathbf{x}^-) = \mathbf{G}\mathbf{x}^+ - \mathbf{G}\mathbf{x}^- = \mathbf{h} - \mathbf{s} \preceq \mathbf{h}$ ✓
   \item $\mathbf{A}\mathbf{x} = \mathbf{A}(\mathbf{x}^+ - \mathbf{x}^-) = \mathbf{A}\mathbf{x}^+ - \mathbf{A}\mathbf{x}^- = \mathbf{b}$ ✓
   \end{itemize}

\item 验证目标函数:
   \begin{equation}
   \mathbf{c}^T \mathbf{x} + d = \mathbf{c}^T (\mathbf{x}^+ - \mathbf{x}^-) + d = \mathbf{c}^T \mathbf{x}^+ - \mathbf{c}^T \mathbf{x}^- + d
   \end{equation}
\end{enumerate}

\textbf{结论}:两个问题等价。

\section{具体例子}

\subsection{例子1:简单问题}

\textbf{原始问题}:
\begin{align}
\text{minimize} \quad & x_1 + 2x_2 \\
\text{subject to} \quad & x_1 + x_2 \leq 3 \\
& x_1 - x_2 = 1
\end{align}

其中 $x_1, x_2$ 无符号约束(可以取任意实数值)。

\textbf{步骤1}:引入松弛变量 $s \geq 0$
\begin{align}
\text{minimize} \quad & x_1 + 2x_2 \\
\text{subject to} \quad & x_1 + x_2 + s = 3 \\
& x_1 - x_2 = 1 \\
& s \geq 0
\end{align}

\textbf{步骤2}:将 $x_1, x_2$ 表示为非负变量的差
\begin{align}
x_1 &= x_1^+ - x_1^- \\
x_2 &= x_2^+ - x_2^-
\end{align}
其中 $x_1^+, x_1^-, x_2^+, x_2^- \geq 0$。

\textbf{步骤3}:代入约束和目标函数

\textbf{目标函数}:
\begin{equation}
x_1 + 2x_2 = (x_1^+ - x_1^-) + 2(x_2^+ - x_2^-) = x_1^+ - x_1^- + 2x_2^+ - 2x_2^-
\end{equation}

\textbf{约束1}:
\begin{align}
x_1 + x_2 + s = 3 \quad &\Rightarrow \quad (x_1^+ - x_1^-) + (x_2^+ - x_2^-) + s = 3 \\
&\Rightarrow \quad x_1^+ - x_1^- + x_2^+ - x_2^- + s = 3
\end{align}

\textbf{约束2}:
\begin{align}
x_1 - x_2 = 1 \quad &\Rightarrow \quad (x_1^+ - x_1^-) - (x_2^+ - x_2^-) = 1 \\
&\Rightarrow \quad x_1^+ - x_1^- - x_2^+ + x_2^- = 1
\end{align}

\textbf{标准形式}:
\begin{align}
\text{minimize} \quad & x_1^+ - x_1^- + 2x_2^+ - 2x_2^- \\
\text{subject to} \quad & x_1^+ - x_1^- + x_2^+ - x_2^- + s = 3 \\
& x_1^+ - x_1^- - x_2^+ + x_2^- = 1 \\
& x_1^+, x_1^-, x_2^+, x_2^-, s \geq 0
\end{align}

\textbf{变量数}:从2个增加到5个($x_1^+, x_1^-, x_2^+, x_2^-, s$)

\subsection{例子2:数值验证}

\textbf{原始问题的一个可行解}:$x_1 = 2$,$x_2 = 1$

\textbf{验证约束}:
\begin{itemize}
\item $x_1 + x_2 = 2 + 1 = 3 \leq 3$ ✓
\item $x_1 - x_2 = 2 - 1 = 1$ ✓
\end{itemize}

\textbf{构造标准形式的解}:
\begin{align}
x_1^+ &= \max\{2, 0\} = 2 \\
x_1^- &= \max\{-2, 0\} = 0 \\
x_2^+ &= \max\{1, 0\} = 1 \\
x_2^- &= \max\{-1, 0\} = 0 \\
s &= 3 - (2 + 1) = 0
\end{align}

\textbf{验证标准形式的约束}:
\begin{itemize}
\item $x_1^+ - x_1^- + x_2^+ - x_2^- + s = 2 - 0 + 1 - 0 + 0 = 3$ ✓
\item $x_1^+ - x_1^- - x_2^+ + x_2^- = 2 - 0 - 1 + 0 = 1$ ✓
\item 所有变量非负 ✓
\end{itemize}

\textbf{目标函数值}:
\begin{itemize}
\item 原始:$x_1 + 2x_2 = 2 + 2 \times 1 = 4$
\item 标准形式:$x_1^+ - x_1^- + 2x_2^+ - 2x_2^- = 2 - 0 + 2 \times 1 - 0 = 4$ ✓
\end{itemize}

\subsection{例子3:包含负数的解}

\textbf{原始问题的一个可行解}:$x_1 = 0$,$x_2 = -1$

\textbf{验证约束}:
\begin{itemize}
\item $x_1 + x_2 = 0 + (-1) = -1 \leq 3$ ✓
\item $x_1 - x_2 = 0 - (-1) = 1$ ✓
\end{itemize}

\textbf{构造标准形式的解}:
\begin{align}
x_1^+ &= \max\{0, 0\} = 0 \\
x_1^- &= \max\{-0, 0\} = 0 \\
x_2^+ &= \max\{-1, 0\} = 0 \\
x_2^- &= \max\{-(-1), 0\} = 1 \\
s &= 3 - (0 + (-1)) = 4
\end{align}

\textbf{验证标准形式的约束}:
\begin{itemize}
\item $x_1^+ - x_1^- + x_2^+ - x_2^- + s = 0 - 0 + 0 - 1 + 4 = 3$ ✓
\item $x_1^+ - x_1^- - x_2^+ + x_2^- = 0 - 0 - 0 + 1 = 1$ ✓
\item 所有变量非负 ✓
\end{itemize}

\textbf{目标函数值}:
\begin{itemize}
\item 原始:$x_1 + 2x_2 = 0 + 2 \times (-1) = -2$
\item 标准形式:$x_1^+ - x_1^- + 2x_2^+ - 2x_2^- = 0 - 0 + 0 - 2 \times 1 = -2$ ✓
\end{itemize}

\section{为什么这样构造?}

\subsection{数学原理}

\textbf{关键事实}:任意实数可以唯一地表示为正部和负部的差。

\textbf{正部}(Positive Part):$x^+ = \max\{x, 0\}$
\begin{itemize}
\item 如果 $x \geq 0$:$x^+ = x$,$x^- = 0$
\item 如果 $x < 0$:$x^+ = 0$,$x^- = -x > 0$
\end{itemize}

\textbf{负部}(Negative Part):$x^- = \max\{-x, 0\}$

\textbf{关系}:$x = x^+ - x^-$

\subsection{几何意义}

\textbf{一维情况}:
\begin{itemize}
\item 如果 $x \geq 0$:$x^+ = x$(在正轴上),$x^- = 0$(在原点)
\item 如果 $x < 0$:$x^+ = 0$(在原点),$x^- = -x$(在正轴上,表示 $|x|$)
\end{itemize}

\textbf{多维情况}:对每个分量分别应用上述构造。

\subsection{优势}

\begin{enumerate}
\item \textbf{统一形式}:所有变量都非负,符合标准形式要求

\item \textbf{保持线性性}:
   \begin{itemize}
   \item 约束仍然是线性的
   \item 目标函数仍然是线性的
   \end{itemize}

\item \textbf{等价性}:转换前后的问题完全等价

\item \textbf{算法兼容}:可以使用标准形式的算法(如单纯形法)
\end{enumerate}

\subsection{代价}

\begin{enumerate}
\item \textbf{变量数增加}:从 $n$ 个变量增加到 $2n + m$ 个变量

\item \textbf{可能冗余}:对于 $x_i \geq 0$ 的情况,$x_i^- = 0$ 是冗余的

\item \textbf{解的唯一性}:标准形式的解不唯一(可以有不同的 $x^+, x^-$ 组合表示同一个 $x$)
\end{enumerate}

\section{总结}

\subsection{转换步骤总结}

\begin{enumerate}
\item \textbf{步骤1}:引入松弛变量 $\mathbf{s} \succeq \mathbf{0}$,将不等式 $\mathbf{G}\mathbf{x} \preceq \mathbf{h}$ 转化为等式 $\mathbf{G}\mathbf{x} + \mathbf{s} = \mathbf{h}$

\item \textbf{步骤2}:将无符号约束的变量 $\mathbf{x}$ 表示为 $\mathbf{x} = \mathbf{x}^+ - \mathbf{x}^-$,其中 $\mathbf{x}^+, \mathbf{x}^- \succeq \mathbf{0}$

\item \textbf{步骤3}:代入所有约束和目标函数,得到标准形式
\end{enumerate}

\subsection{关键公式}

\begin{enumerate}
\item \textbf{变量分解}:$\mathbf{x} = \mathbf{x}^+ - \mathbf{x}^-$,其中:
   \begin{align}
   x_i^+ &= \max\{x_i, 0\} \\
   x_i^- &= \max\{-x_i, 0\}
   \end{align}

\item \textbf{目标函数}:$\mathbf{c}^T \mathbf{x} + d = \mathbf{c}^T \mathbf{x}^+ - \mathbf{c}^T \mathbf{x}^- + d$

\item \textbf{约束}:
   \begin{align}
   \mathbf{G}\mathbf{x}^+ - \mathbf{G}\mathbf{x}^- + \mathbf{s} &= \mathbf{h} \\
   \mathbf{A}\mathbf{x}^+ - \mathbf{A}\mathbf{x}^- &= \mathbf{b}
   \end{align}
\end{enumerate}

\subsection{关键理解}

\begin{itemize}
\item \textbf{为什么可以这样表示}:任意实数可以表示为两个非负数的差
\item \textbf{如何构造}:使用正部和负部的概念
\item \textbf{等价性}:转换前后的问题完全等价
\item \textbf{代价}:变量数增加,但可以使用标准算法
\end{itemize}

理解这个构造过程,对于理解线性规划的标准形式转换非常重要!

\end{document}

