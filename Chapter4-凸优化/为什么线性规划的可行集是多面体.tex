\documentclass[12pt,a4paper]{article}
\usepackage[UTF8]{ctex}
\usepackage{amsmath}
\usepackage{amssymb}
\usepackage{amsthm}
\usepackage{geometry}
\geometry{left=2.5cm,right=2.5cm,top=2.5cm,bottom=2.5cm}

\title{为什么线性规划的可行集是多面体?}
\author{}
\date{\today}

\begin{document}

\maketitle

\section{问题提出}

在第4.3节中,书中提到:"线性规划的可行集是多面体 $P$"。

\textbf{问题}:为什么线性规划的可行集是多面体?多面体是什么?线性规划的约束如何构成多面体?

\section{多面体的定义}

\subsection{标准定义}

\textbf{多面体}(Polyhedron)定义为有限个线性等式和不等式的解集:

\begin{equation}
P = \{\mathbf{x} \in \mathbb{R}^n \mid \mathbf{a}_j^T \mathbf{x} \leq b_j, j = 1, \ldots, m, \mathbf{c}_j^T \mathbf{x} = d_j, j = 1, \ldots, p\}
\end{equation}

\textbf{等价表述}:多面体是有限个半空间(halfspace)和超平面(hyperplane)的交集。

\textbf{紧凑表示}:
\begin{equation}
P = \{\mathbf{x} \in \mathbb{R}^n \mid \mathbf{A}\mathbf{x} \preceq \mathbf{b}, \mathbf{C}\mathbf{x} = \mathbf{d}\}
\end{equation}

其中:
\begin{itemize}
\item $\mathbf{A} \in \mathbb{R}^{m \times n}$:不等式约束的系数矩阵
\item $\mathbf{b} \in \mathbb{R}^m$:不等式约束的常数向量
\item $\mathbf{C} \in \mathbb{R}^{p \times n}$:等式约束的系数矩阵
\item $\mathbf{d} \in \mathbb{R}^p$:等式约束的常数向量
\item $\preceq$ 表示分量不等式:$\mathbf{u} \preceq \mathbf{v}$ 意味着 $u_i \leq v_i$ 对所有 $i$
\end{itemize}

\subsection{几何意义}

\textbf{半空间}:每个线性不等式 $\mathbf{a}_j^T \mathbf{x} \leq b_j$ 定义一个半空间:
\begin{equation}
H_j = \{\mathbf{x} \in \mathbb{R}^n \mid \mathbf{a}_j^T \mathbf{x} \leq b_j\}
\end{equation}

\textbf{超平面}:每个线性等式 $\mathbf{c}_j^T \mathbf{x} = d_j$ 定义一个超平面:
\begin{equation}
L_j = \{\mathbf{x} \in \mathbb{R}^n \mid \mathbf{c}_j^T \mathbf{x} = d_j\}
\end{equation}

\textbf{多面体}:所有这些半空间和超平面的交集:
\begin{equation}
P = \left(\bigcap_{j=1}^m H_j\right) \cap \left(\bigcap_{j=1}^p L_j\right)
\end{equation}

\section{线性规划的标准形式}

\subsection{一般形式}

\textbf{线性规划}(Linear Program, LP)的标准形式:
\begin{align}
\text{minimize} \quad & \mathbf{c}^T \mathbf{x} + d \\
\text{subject to} \quad & \mathbf{G}\mathbf{x} \preceq \mathbf{h} \\
& \mathbf{A}\mathbf{x} = \mathbf{b}
\end{align}

其中:
\begin{itemize}
\item $\mathbf{x} \in \mathbb{R}^n$:决策变量
\item $\mathbf{c} \in \mathbb{R}^n$:目标函数系数向量
\item $d \in \mathbb{R}$:目标函数常数项
\item $\mathbf{G} \in \mathbb{R}^{m \times n}$:不等式约束矩阵
\item $\mathbf{h} \in \mathbb{R}^m$:不等式约束常数向量
\item $\mathbf{A} \in \mathbb{R}^{p \times n}$:等式约束矩阵
\item $\mathbf{b} \in \mathbb{R}^p$:等式约束常数向量
\end{itemize}

\subsection{可行集}

\textbf{可行集}(Feasible Set)定义为所有满足约束的点的集合:
\begin{equation}
\mathcal{F} = \{\mathbf{x} \in \mathbb{R}^n \mid \mathbf{G}\mathbf{x} \preceq \mathbf{h}, \mathbf{A}\mathbf{x} = \mathbf{b}\}
\end{equation}

\section{为什么可行集是多面体?}

\subsection{关键观察}

\textbf{观察1}:线性规划的约束都是线性的(仿射的)。

\begin{itemize}
\item 不等式约束:$\mathbf{G}\mathbf{x} \preceq \mathbf{h}$($m$ 个线性不等式)
\item 等式约束:$\mathbf{A}\mathbf{x} = \mathbf{b}$($p$ 个线性等式)
\end{itemize}

\textbf{观察2}:每个线性不等式定义一个半空间。

对于第 $i$ 个不等式约束 $\mathbf{g}_i^T \mathbf{x} \leq h_i$(其中 $\mathbf{g}_i^T$ 是 $\mathbf{G}$ 的第 $i$ 行),它定义半空间:
\begin{equation}
H_i = \{\mathbf{x} \in \mathbb{R}^n \mid \mathbf{g}_i^T \mathbf{x} \leq h_i\}
\end{equation}

\textbf{观察3}:每个线性等式定义一个超平面。

对于第 $j$ 个等式约束 $\mathbf{a}_j^T \mathbf{x} = b_j$(其中 $\mathbf{a}_j^T$ 是 $\mathbf{A}$ 的第 $j$ 行),它定义超平面:
\begin{equation}
L_j = \{\mathbf{x} \in \mathbb{R}^n \mid \mathbf{a}_j^T \mathbf{x} = b_j\}
\end{equation}

\textbf{观察4}:可行集是这些半空间和超平面的交集。

\begin{align}
\mathcal{F} &= \{\mathbf{x} \in \mathbb{R}^n \mid \mathbf{G}\mathbf{x} \preceq \mathbf{h}, \mathbf{A}\mathbf{x} = \mathbf{b}\} \\
&= \{\mathbf{x} \in \mathbb{R}^n \mid \mathbf{g}_i^T \mathbf{x} \leq h_i, i = 1, \ldots, m, \mathbf{a}_j^T \mathbf{x} = b_j, j = 1, \ldots, p\} \\
&= \left(\bigcap_{i=1}^m \{\mathbf{x} \in \mathbb{R}^n \mid \mathbf{g}_i^T \mathbf{x} \leq h_i\}\right) \cap \left(\bigcap_{j=1}^p \{\mathbf{x} \in \mathbb{R}^n \mid \mathbf{a}_j^T \mathbf{x} = b_j\}\right) \\
&= \left(\bigcap_{i=1}^m H_i\right) \cap \left(\bigcap_{j=1}^p L_j\right)
\end{align}

\subsection{结论}

\textbf{定理}:线性规划的可行集是多面体。

\textbf{证明}:
\begin{enumerate}
\item 可行集 $\mathcal{F}$ 由有限个线性不等式和等式定义
\item 每个线性不等式定义一个半空间
\item 每个线性等式定义一个超平面
\item 可行集是这些半空间和超平面的交集
\item 根据多面体的定义,可行集是多面体
\end{enumerate}

\textbf{因此}:$\mathcal{F} = P$,其中 $P$ 是多面体。$\square$

\section{具体例子}

\subsection{例子1:二维线性规划}

\textbf{问题}:
\begin{align}
\text{minimize} \quad & x_1 + x_2 \\
\text{subject to} \quad & x_1 \geq 0 \\
& x_2 \geq 0 \\
& x_1 + x_2 \leq 1
\end{align}

\textbf{约束分析}:
\begin{itemize}
\item $x_1 \geq 0$:半空间 $H_1 = \{\mathbf{x} \mid x_1 \geq 0\}$
\item $x_2 \geq 0$:半空间 $H_2 = \{\mathbf{x} \mid x_2 \geq 0\}$
\item $x_1 + x_2 \leq 1$:半空间 $H_3 = \{\mathbf{x} \mid x_1 + x_2 \leq 1\}$
\end{itemize}

\textbf{可行集}:
\begin{equation}
\mathcal{F} = H_1 \cap H_2 \cap H_3 = \{\mathbf{x} \in \mathbb{R}^2 \mid x_1 \geq 0, x_2 \geq 0, x_1 + x_2 \leq 1\}
\end{equation}

\textbf{几何意义}:这是一个三角形(包括边界),是多面体。

\textbf{可视化}:
\begin{itemize}
\item $x_1 \geq 0$:$x_1$ 轴右侧的半平面
\item $x_2 \geq 0$:$x_2$ 轴上方的半平面
\item $x_1 + x_2 \leq 1$:直线 $x_1 + x_2 = 1$ 下方的半平面
\item 交集:第一象限中,直线 $x_1 + x_2 = 1$ 下方的三角形区域
\end{itemize}

\subsection{例子2:带等式约束的线性规划}

\textbf{问题}:
\begin{align}
\text{minimize} \quad & x_1 + 2x_2 \\
\text{subject to} \quad & x_1 + x_2 = 1 \\
& x_1 \geq 0 \\
& x_2 \geq 0
\end{align}

\textbf{约束分析}:
\begin{itemize}
\item $x_1 + x_2 = 1$:超平面 $L_1 = \{\mathbf{x} \mid x_1 + x_2 = 1\}$(一条直线)
\item $x_1 \geq 0$:半空间 $H_1 = \{\mathbf{x} \mid x_1 \geq 0\}$
\item $x_2 \geq 0$:半空间 $H_2 = \{\mathbf{x} \mid x_2 \geq 0\}$
\end{itemize}

\textbf{可行集}:
\begin{equation}
\mathcal{F} = L_1 \cap H_1 \cap H_2 = \{\mathbf{x} \in \mathbb{R}^2 \mid x_1 + x_2 = 1, x_1 \geq 0, x_2 \geq 0\}
\end{equation}

\textbf{几何意义}:这是直线 $x_1 + x_2 = 1$ 在第一象限中的线段,从 $(1, 0)$ 到 $(0, 1)$。

\textbf{注意}:这是一个有界多面体(线段),也称为多胞形(polytope)。

\subsection{例子3:三维线性规划}

\textbf{问题}:
\begin{align}
\text{minimize} \quad & x_1 + x_2 + x_3 \\
\text{subject to} \quad & x_1 \geq 0 \\
& x_2 \geq 0 \\
& x_3 \geq 0 \\
& x_1 + x_2 + x_3 \leq 1
\end{align}

\textbf{可行集}:
\begin{equation}
\mathcal{F} = \{\mathbf{x} \in \mathbb{R}^3 \mid x_1 \geq 0, x_2 \geq 0, x_3 \geq 0, x_1 + x_2 + x_3 \leq 1\}
\end{equation}

\textbf{几何意义}:这是一个四面体(包括边界),由4个半空间定义:
\begin{itemize}
\item $x_1 \geq 0$:平面 $x_1 = 0$ 的一侧
\item $x_2 \geq 0$:平面 $x_2 = 0$ 的一侧
\item $x_3 \geq 0$:平面 $x_3 = 0$ 的一侧
\item $x_1 + x_2 + x_3 \leq 1$:平面 $x_1 + x_2 + x_3 = 1$ 的一侧
\end{itemize}

\section{多面体的性质}

\subsection{凸性}

\textbf{定理}:多面体是凸集。

\textbf{证明}:
\begin{enumerate}
\item 半空间是凸集
\item 超平面是凸集
\item 凸集的交集是凸集
\item 因此多面体(半空间和超平面的交集)是凸集
\end{enumerate}

\textbf{推论}:线性规划的可行集是凸集。

\subsection{有界性}

\textbf{定义}:
\begin{itemize}
\item \textbf{有界多面体}:如果多面体是有界的,称为\textbf{多胞形}(Polytope)
\item \textbf{无界多面体}:如果多面体是无界的,仍然称为多面体
\end{itemize}

\textbf{例子}:
\begin{itemize}
\item 有界:$\{\mathbf{x} \in \mathbb{R}^2 \mid x_1 \geq 0, x_2 \geq 0, x_1 + x_2 \leq 1\}$(三角形)
\item 无界:$\{\mathbf{x} \in \mathbb{R}^2 \mid x_1 \geq 0, x_2 \geq 0\}$(第一象限)
\end{itemize}

\subsection{顶点}

\textbf{定义}:多面体的\textbf{顶点}(Vertex)或\textbf{极值点}(Extreme Point)是不能表示为多面体中两个不同点的严格凸组合的点。

\textbf{重要性}:对于线性规划,如果存在最优解,则至少有一个最优解在顶点处。

\section{为什么这个性质重要?}

\subsection{算法优势}

\textbf{单纯形法}:
\begin{itemize}
\item 利用可行集是多面体的性质
\item 在多面体的顶点之间移动
\item 顶点数量有限,因此算法可以在有限步内终止
\end{itemize}

\textbf{内点法}:
\begin{itemize}
\item 利用可行集是凸集的性质
\item 在可行集内部移动
\item 可以高效求解大规模线性规划问题
\end{itemize}

\subsection{理论意义}

\textbf{对偶理论}:
\begin{itemize}
\item 多面体的结构使得对偶理论更加完善
\item 强对偶性在多面体可行集上更容易建立
\end{itemize}

\textbf{最优性条件}:
\begin{itemize}
\item 多面体的顶点结构使得最优性条件更加清晰
\item 基本可行解对应顶点
\end{itemize}

\section{总结}

\subsection{关键要点}

\begin{enumerate}
\item \textbf{多面体的定义}:
   \begin{itemize}
   \item 有限个线性等式和不等式的解集
   \item 有限个半空间和超平面的交集
   \end{itemize}

\item \textbf{线性规划的约束}:
   \begin{itemize}
   \item 线性不等式:$\mathbf{G}\mathbf{x} \preceq \mathbf{h}$(定义半空间)
   \item 线性等式:$\mathbf{A}\mathbf{x} = \mathbf{b}$(定义超平面)
   \end{itemize}

\item \textbf{可行集是多面体}:
   \begin{itemize}
   \item 可行集是这些半空间和超平面的交集
   \item 根据定义,可行集是多面体
   \end{itemize}

\item \textbf{多面体的性质}:
   \begin{itemize}
   \item 凸集
   \item 可能有界(多胞形)或无界
   \item 有有限个顶点(如果有界)
   \end{itemize}
\end{enumerate}

\subsection{关键公式}

\begin{enumerate}
\item \textbf{多面体定义}:
\begin{equation}
P = \{\mathbf{x} \in \mathbb{R}^n \mid \mathbf{A}\mathbf{x} \preceq \mathbf{b}, \mathbf{C}\mathbf{x} = \mathbf{d}\}
\end{equation}

\item \textbf{线性规划可行集}:
\begin{equation}
\mathcal{F} = \{\mathbf{x} \in \mathbb{R}^n \mid \mathbf{G}\mathbf{x} \preceq \mathbf{h}, \mathbf{A}\mathbf{x} = \mathbf{b}\}
\end{equation}

\item \textbf{等价性}:$\mathcal{F} = P$(可行集是多面体)
\end{enumerate}

理解线性规划的可行集是多面体,对于理解线性规划的理论和算法都非常重要!

\end{document}

