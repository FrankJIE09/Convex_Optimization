\documentclass[12pt,a4paper]{article}
\usepackage[UTF8]{ctex}
\usepackage{amsmath}
\usepackage{amssymb}
\usepackage{amsthm}
\usepackage{geometry}
\geometry{left=2.5cm,right=2.5cm,top=2.5cm,bottom=2.5cm}

\title{第4.2.2节:局部最优与全局最优详解}
\author{}
\date{\today}

\begin{document}

\maketitle

\section{引言}

第4.2.2节证明了凸优化问题的一个关键性质:\textbf{任何局部最优点也是全局最优点}。这是凸优化最重要的优势之一,本节详细解释这个定理及其证明。

\section{局部最优与全局最优的定义}

\subsection{局部最优}

\textbf{局部最优点}:点 $\mathbf{x}$ 是局部最优的,如果:
\begin{enumerate}
\item $\mathbf{x}$ 是可行的
\item 存在 $R > 0$,使得在 $\mathbf{x}$ 的 $R$-邻域内,$\mathbf{x}$ 是最优的
\end{enumerate}

\textbf{数学表述}:

\begin{equation}
f_0(\mathbf{x}) = \inf\{f_0(\mathbf{z}) \mid \mathbf{z} \text{ 可行}, \|\mathbf{z} - \mathbf{x}\|_2 \leq R\}
\end{equation}

\textbf{含义}:
\begin{itemize}
\item 在 $\mathbf{x}$ 附近的可行点中,$\mathbf{x}$ 使目标函数最小
\item "附近"由半径 $R$ 定义
\item 只考虑局部区域,不考虑整个可行域
\end{itemize}

\subsection{全局最优}

\textbf{全局最优点}:点 $\mathbf{x}$ 是全局最优的,如果:
\begin{enumerate}
\item $\mathbf{x}$ 是可行的
\item 在所有可行点中,$\mathbf{x}$ 使目标函数最小
\end{enumerate}

\textbf{数学表述}:

\begin{equation}
f_0(\mathbf{x}) = \inf\{f_0(\mathbf{z}) \mid \mathbf{z} \text{ 可行}\}
\end{equation}

\textbf{含义}:
\begin{itemize}
\item 在整个可行域中,$\mathbf{x}$ 使目标函数最小
\item 考虑所有可行点,不仅仅是局部区域
\end{itemize}

\section{凸优化的关键定理}

\subsection{定理陈述}

\textbf{定理}:对于凸优化问题,任何局部最优点也是全局最优点。

\textbf{数学表述}:

如果 $\mathbf{x}$ 是凸优化问题的局部最优点,则 $\mathbf{x}$ 也是全局最优点。

\textbf{重要性}:
\begin{itemize}
\item 这是凸优化的核心优势
\item 意味着我们只需要找到局部最优,就自动得到全局最优
\item 大大简化了优化问题的求解
\end{itemize}

\section{证明:局部最优 $\Rightarrow$ 全局最优}

\subsection{证明思路}

\textbf{方法}:反证法

\textbf{假设}:
\begin{enumerate}
\item $\mathbf{x}$ 是局部最优的(已知)
\item $\mathbf{x}$ 不是全局最优的(假设,要导出矛盾)
\end{enumerate}

\textbf{目标}:证明假设2导致矛盾,因此 $\mathbf{x}$ 必须是全局最优的。

\subsection{详细证明步骤}

\textbf{步骤1}:假设 $\mathbf{x}$ 是局部最优的。

根据局部最优的定义,存在 $R > 0$,使得:

\begin{equation}
f_0(\mathbf{x}) = \inf\{f_0(\mathbf{z}) \mid \mathbf{z} \text{ 可行}, \|\mathbf{z} - \mathbf{x}\|_2 \leq R\}
\end{equation}

\textbf{步骤2}:假设 $\mathbf{x}$ 不是全局最优的。

即:存在可行点 $\mathbf{y}$,使得 $f_0(\mathbf{y}) < f_0(\mathbf{x})$。

\textbf{步骤3}:关键观察。

\textbf{观察1}:$\|\mathbf{y} - \mathbf{x}\|_2 > R$

\textbf{为什么?}

如果 $\|\mathbf{y} - \mathbf{x}\|_2 \leq R$,则 $\mathbf{y}$ 在 $\mathbf{x}$ 的 $R$-邻域内。

由于 $\mathbf{x}$ 是局部最优的,在 $R$-邻域内,$f_0(\mathbf{x}) \leq f_0(\mathbf{y})$。

但我们已经假设 $f_0(\mathbf{y}) < f_0(\mathbf{x})$,矛盾!

因此必须有 $\|\mathbf{y} - \mathbf{x}\|_2 > R$。

\textbf{步骤4}:构造一个矛盾的点。

\textbf{构造}:在连接 $\mathbf{x}$ 和 $\mathbf{y}$ 的线段上,选择一个点 $\mathbf{z}$,使得 $\|\mathbf{z} - \mathbf{x}\|_2 < R$。

\textbf{具体构造}:

\begin{equation}
\mathbf{z} = (1 - \theta)\mathbf{x} + \theta\mathbf{y}
\end{equation}

其中:

\begin{equation}
\theta = \frac{R}{2\|\mathbf{y} - \mathbf{x}\|_2}
\end{equation}

\textbf{为什么这样选择?}

\begin{align}
\|\mathbf{z} - \mathbf{x}\|_2 &= \|(1 - \theta)\mathbf{x} + \theta\mathbf{y} - \mathbf{x}\|_2 \\
&= \|\theta(\mathbf{y} - \mathbf{x})\|_2 \\
&= \theta \|\mathbf{y} - \mathbf{x}\|_2 \\
&= \frac{R}{2\|\mathbf{y} - \mathbf{x}\|_2} \cdot \|\mathbf{y} - \mathbf{x}\|_2 \\
&= \frac{R}{2} < R
\end{equation}

因此 $\mathbf{z}$ 在 $\mathbf{x}$ 的 $R$-邻域内。

\textbf{步骤5}:验证 $\mathbf{z}$ 的可行性。

由于可行集是凸集,且 $\mathbf{x}, \mathbf{y}$ 都是可行的,凸组合 $\mathbf{z} = (1 - \theta)\mathbf{x} + \theta\mathbf{y}$ 也是可行的。✓

\textbf{步骤6}:计算 $f_0(\mathbf{z})$。

由于 $f_0$ 是凸函数,有:

\begin{align}
f_0(\mathbf{z}) &= f_0((1 - \theta)\mathbf{x} + \theta\mathbf{y}) \\
&\leq (1 - \theta) f_0(\mathbf{x}) + \theta f_0(\mathbf{y}) \\
&< (1 - \theta) f_0(\mathbf{x}) + \theta f_0(\mathbf{x}) \quad \text{(因为 $f_0(\mathbf{y}) < f_0(\mathbf{x})$)} \\
&= f_0(\mathbf{x})
\end{equation}

\textbf{步骤7}:导出矛盾。

\begin{itemize}
\item $\mathbf{z}$ 是可行的(步骤5)
\item $\|\mathbf{z} - \mathbf{x}\|_2 = R/2 < R$(步骤4)
\item $f_0(\mathbf{z}) < f_0(\mathbf{x})$(步骤6)
\end{itemize}

但根据局部最优的定义,在 $R$-邻域内,$f_0(\mathbf{x}) \leq f_0(\mathbf{z})$,矛盾!

\textbf{步骤8}:结论。

假设"$\mathbf{x}$ 不是全局最优的"导致矛盾,因此 $\mathbf{x}$ 必须是全局最优的。$\square$

\section{证明的几何直观}

\subsection{几何解释}

\textbf{情况}:
\begin{itemize}
\item $\mathbf{x}$ 是局部最优点(在 $R$-邻域内最优)
\item $\mathbf{y}$ 是更好的可行点($f_0(\mathbf{y}) < f_0(\mathbf{x})$)
\item $\mathbf{y}$ 在 $R$-邻域外($\|\mathbf{y} - \mathbf{x}\|_2 > R$)
\end{itemize}

\textbf{构造的点 $\mathbf{z}$}:
\begin{itemize}
\item 在连接 $\mathbf{x}$ 和 $\mathbf{y}$ 的线段上
\item 距离 $\mathbf{x}$ 为 $R/2 < R$(在 $R$-邻域内)
\item 由于凸性,$f_0(\mathbf{z}) < f_0(\mathbf{x})$
\end{itemize}

\textbf{矛盾}:
\begin{itemize}
\item $\mathbf{z}$ 在 $R$-邻域内,应该 $f_0(\mathbf{x}) \leq f_0(\mathbf{z})$
\item 但 $f_0(\mathbf{z}) < f_0(\mathbf{x})$
\item 矛盾!
\end{itemize}

\subsection{图示说明}

\textbf{示意图}:
\begin{itemize}
\item 可行集:凸集(阴影区域)
\item 点 $\mathbf{x}$:局部最优点
\item 点 $\mathbf{y}$:更好的点(在可行域中,但距离 $\mathbf{x}$ 较远)
\item 点 $\mathbf{z}$:在 $\mathbf{x}$ 和 $\mathbf{y}$ 之间的线段上,在 $R$-邻域内
\item $R$-邻域:以 $\mathbf{x}$ 为中心、半径为 $R$ 的球
\end{itemize}

\section{关键理解}

\subsection{为什么需要凸性?}

\textbf{可行集的凸性}:
\begin{itemize}
\item 如果可行集不是凸集,$\mathbf{z}$ 可能不可行
\item 凸性保证了 $\mathbf{z}$ 的可行性
\end{itemize}

\textbf{目标函数的凸性}:
\begin{itemize}
\item 如果 $f_0$ 不是凸函数,$f_0(\mathbf{z})$ 可能不小于 $f_0(\mathbf{x})$
\item 凸性保证了 $f_0(\mathbf{z}) < f_0(\mathbf{x})$
\end{itemize}

\subsection{为什么这个性质重要?}

\begin{enumerate}
\item \textbf{简化求解}:
   \begin{itemize}
   \item 不需要担心陷入局部最优
   \item 找到局部最优就自动得到全局最优
   \end{itemize}

\item \textbf{算法保证}:
   \begin{itemize}
   \item 许多优化算法只能保证找到局部最优
   \item 对于凸优化问题,这已经足够了
   \end{itemize}

\item \textbf{理论优势}:
   \begin{itemize}
   \item 可以放心地使用局部搜索算法
   \item 不需要全局优化方法
   \end{itemize}
\end{enumerate}

\section{与拟凸优化的对比}

\subsection{拟凸优化的情况}

\textbf{注意}:书中提到,拟凸优化问题不具有这个性质。

\textbf{原因}:
\begin{itemize}
\item 拟凸函数的子水平集是凸集
\item 但拟凸函数本身不一定是凸的
\item 可能存在多个局部最优点,其中只有一个是全局最优的
\end{itemize}

\textbf{例子}:单峰函数可能是拟凸的,但可能有多个局部最小值。

\section{具体例子}

\subsection{例子1:凸函数}

\textbf{问题}:$\min_{x} x^2$,无约束

\textbf{分析}:
\begin{itemize}
\item $f(x) = x^2$ 是凸函数
\item 局部最优点:$x = 0$(在任意邻域内都是最小的)
\item 全局最优点:$x = 0$
\item 局部最优 = 全局最优 ✓
\end{itemize}

\subsection{例子2:非凸函数}

\textbf{问题}:$\min_{x} x^4 - 2x^2$,无约束

\textbf{分析}:
\begin{itemize}
\item $f(x) = x^4 - 2x^2$ 不是凸函数
\item 局部最优点:$x = -1$ 和 $x = 1$(都是局部最小)
\item 全局最优点:$x = -1$ 和 $x = 1$(两者都是全局最小,值相同)
\item 但存在其他局部最优点(如 $x = 0$ 是局部最大)
\end{itemize}

\textbf{注意}:这个例子中,虽然局部最优 = 全局最优,但函数不是凸的。对于非凸函数,这个性质不总是成立。

\section{总结}

\subsection{关键定理}

\begin{enumerate}
\item \textbf{定理}:对于凸优化问题,局部最优 = 全局最优

\item \textbf{证明方法}:反证法
   \begin{itemize}
   \item 假设存在更好的可行点 $\mathbf{y}$
   \item 构造一个在 $R$-邻域内的点 $\mathbf{z}$
   \item 利用凸性证明 $f_0(\mathbf{z}) < f_0(\mathbf{x})$
   \item 导出矛盾
   \end{itemize}

\item \textbf{关键条件}:
   \begin{itemize}
   \item 可行集是凸集
   \item 目标函数是凸函数
   \end{itemize}
\end{enumerate}

\subsection{重要性}

\begin{enumerate}
\item \textbf{简化求解}:只需要找到局部最优

\item \textbf{算法保证}:局部搜索算法足够

\item \textbf{理论优势}:凸优化的核心优势之一
\end{enumerate}

理解这个定理,是理解凸优化优势的关键!

\end{document}

