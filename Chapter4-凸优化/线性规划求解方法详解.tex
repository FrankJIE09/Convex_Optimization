\documentclass[12pt,a4paper]{article}
\usepackage[UTF8]{ctex}
\usepackage{amsmath}
\usepackage{amssymb}
\usepackage{amsthm}
\usepackage{geometry}
\geometry{left=2.5cm,right=2.5cm,top=2.5cm,bottom=2.5cm}

\title{线性规划求解方法详解}
\subtitle{从问题建立到最优解求解的完整过程}
\author{}
\date{\today}

\begin{document}

\maketitle

\section{引言}

线性规划问题建立后,如何求解最优解?本节详细介绍线性规划的主要求解方法,特别是单纯形法的完整步骤和计算过程。

\section{线性规划标准形式回顾}

\subsection{标准形式}

\begin{align}
\begin{array}{ll}
\text{minimize} & \mathbf{c}^T \mathbf{x} \\
\text{subject to} & \mathbf{A}\mathbf{x} = \mathbf{b} \\
& \mathbf{x} \geq \mathbf{0}
\end{array}
\end{align}

其中:
\begin{itemize}
\item $\mathbf{x} \in \mathbb{R}^n$:决策变量
\item $\mathbf{c} \in \mathbb{R}^n$:目标函数系数向量
\item $\mathbf{A} \in \mathbb{R}^{m \times n}$:约束矩阵($m < n$,通常)
\item $\mathbf{b} \in \mathbb{R}^m$:约束右端向量
\end{itemize}

\subsection{关键性质}

\begin{enumerate}
\item \textbf{可行域是凸多面体}:由有限个半空间的交集构成
\item \textbf{最优解在顶点}:如果存在最优解,至少有一个最优解是基本可行解(顶点)
\item \textbf{基本可行解有限}:最多有 $\binom{n}{m}$ 个基本可行解
\end{enumerate}

\section{单纯形法(Simplex Method)}

\subsection{算法思想}

单纯形法是求解线性规划最经典的方法,由George Dantzig在1947年提出。

\textbf{核心思想}:
\begin{enumerate}
\item 从可行域的一个顶点(基本可行解)开始
\item 沿着可行域的边移动到相邻顶点
\item 每次移动都使目标函数值减小(或不变)
\item 当无法继续改进时,达到最优解
\end{enumerate}

\textbf{为什么有效?}
\begin{itemize}
\item 可行域是凸多面体,顶点数量有限
\item 最优解在顶点,只需检查有限个顶点
\item 沿着边移动,保证始终在可行域内
\end{itemize}

\subsection{基本概念}

\subsubsection{基变量与非基变量}

设 $\mathbf{A}$ 的秩为 $m$(约束线性无关)。

\textbf{基(Basis)}:从 $\mathbf{A}$ 的 $n$ 列中选择 $m$ 个线性无关的列,构成基矩阵 $\mathbf{B}$。

\textbf{基变量(Basic Variables)}:对应基矩阵的变量,记为 $\mathbf{x}_B$。

\textbf{非基变量(Non-basic Variables)}:其余 $n-m$ 个变量,设为0,记为 $\mathbf{x}_N = \mathbf{0}$。

\subsubsection{基本解}

将约束 $\mathbf{A}\mathbf{x} = \mathbf{b}$ 分块:

\begin{equation}
\mathbf{A}\mathbf{x} = [\mathbf{B} \mid \mathbf{N}] \begin{bmatrix} \mathbf{x}_B \\ \mathbf{x}_N \end{bmatrix} = \mathbf{B}\mathbf{x}_B + \mathbf{N}\mathbf{x}_N = \mathbf{b}
\end{equation}

设 $\mathbf{x}_N = \mathbf{0}$,则:

\begin{equation}
\mathbf{B}\mathbf{x}_B = \mathbf{b} \quad \Rightarrow \quad \mathbf{x}_B = \mathbf{B}^{-1}\mathbf{b}
\end{equation}

\textbf{基本解}:$\mathbf{x} = \begin{bmatrix} \mathbf{x}_B \\ \mathbf{0} \end{bmatrix}$,其中 $\mathbf{x}_B = \mathbf{B}^{-1}\mathbf{b}$。

\textbf{基本可行解}:如果 $\mathbf{x}_B \geq \mathbf{0}$,则基本解是可行的。

\subsection{单纯形法的完整步骤}

\subsubsection{步骤1:初始基本可行解}

\textbf{方法1:标准形式已有单位矩阵}

如果问题已经是标准形式,且约束矩阵包含单位矩阵,可以直接得到初始基。

\textbf{方法2:两阶段法}

如果问题不是标准形式,使用两阶段法:
\begin{enumerate}
\item \textbf{第一阶段}:引入人工变量,构造辅助问题,找到初始基本可行解
\item \textbf{第二阶段}:使用第一阶段得到的基,求解原问题
\end{enumerate}

\subsubsection{步骤2:计算检验数(Reduced Cost)}

\textbf{目标函数}:$z = \mathbf{c}^T \mathbf{x} = \mathbf{c}_B^T \mathbf{x}_B + \mathbf{c}_N^T \mathbf{x}_N$

将 $\mathbf{x}_B = \mathbf{B}^{-1}\mathbf{b} - \mathbf{B}^{-1}\mathbf{N}\mathbf{x}_N$ 代入:

\begin{align}
z &= \mathbf{c}_B^T (\mathbf{B}^{-1}\mathbf{b} - \mathbf{B}^{-1}\mathbf{N}\mathbf{x}_N) + \mathbf{c}_N^T \mathbf{x}_N \\
&= \mathbf{c}_B^T \mathbf{B}^{-1}\mathbf{b} + (\mathbf{c}_N^T - \mathbf{c}_B^T \mathbf{B}^{-1}\mathbf{N})\mathbf{x}_N
\end{align}

\textbf{检验数}:$\bar{\mathbf{c}}_N^T = \mathbf{c}_N^T - \mathbf{c}_B^T \mathbf{B}^{-1}\mathbf{N}$

\textbf{含义}:
\begin{itemize}
\item 如果 $\bar{c}_j < 0$(对于最小化问题),增加 $x_j$ 可以减小目标函数
\item 如果所有 $\bar{c}_j \geq 0$,当前解是最优解
\end{itemize}

\subsubsection{步骤3:选择进基变量(Entering Variable)}

\textbf{规则}:选择检验数最小(最负)的非基变量进基。

\begin{equation}
j^* = \arg\min_{j \in N} \{\bar{c}_j \mid \bar{c}_j < 0\}
\end{equation}

其中 $N$ 是非基变量索引集。

\textbf{为什么?}:这样可以最大程度地减小目标函数。

\subsubsection{步骤4:选择出基变量(Leaving Variable)}

\textbf{目标}:保持可行性,即 $\mathbf{x}_B \geq \mathbf{0}$。

设进基变量为 $x_{j^*}$,考虑增加 $x_{j^*}$ 的影响:

\begin{equation}
\mathbf{x}_B = \mathbf{B}^{-1}\mathbf{b} - \mathbf{B}^{-1}\mathbf{a}_{j^*} x_{j^*}
\end{equation}

设 $\mathbf{d} = \mathbf{B}^{-1}\mathbf{a}_{j^*}$,则:

\begin{equation}
\mathbf{x}_B = \mathbf{B}^{-1}\mathbf{b} - \mathbf{d} x_{j^*}
\end{equation}

为了保持 $\mathbf{x}_B \geq \mathbf{0}$,需要:

\begin{equation}
x_{j^*} \leq \min_{i: d_i > 0} \frac{(\mathbf{B}^{-1}\mathbf{b})_i}{d_i}
</equation>

\textbf{出基变量选择规则}:

\begin{equation}
i^* = \arg\min_{i: d_i > 0} \frac{(\mathbf{B}^{-1}\mathbf{b})_i}{d_i}
</equation>

如果所有 $d_i \leq 0$,则问题无界(unbounded)。

\subsubsection{步骤5:更新基矩阵}

\textbf{操作}:
\begin{enumerate}
\item 将 $x_{j^*}$ 加入基变量
\item 将 $x_{i^*}$ 移出基变量
\item 更新基矩阵 $\mathbf{B}$
\item 更新 $\mathbf{B}^{-1}$(使用初等行变换或直接计算)
</enumerate>

\subsubsection{步骤6:判断最优性}

\textbf{最优性条件}:如果所有检验数 $\bar{c}_j \geq 0$,则当前解是最优解。

\textbf{停止条件}:
\begin{itemize}
\item 所有检验数 $\geq 0$:达到最优解
\item 问题无界:无法找到最优解
\item 达到最大迭代次数:算法终止
</enumerate>

\subsection{单纯形表(Simplex Tableau)}

单纯形表是单纯形法的表格形式,便于手工计算。

\textbf{表格结构}:

\begin{table}[h]
\centering
\begin{tabular}{|c|c|c|c|}
\hline
 & $x_B$ & $x_N$ & RHS \\
\hline
$z$ & $\mathbf{c}_B^T \mathbf{B}^{-1}\mathbf{B} - \mathbf{c}_B^T$ & $\bar{\mathbf{c}}_N^T$ & $\mathbf{c}_B^T \mathbf{B}^{-1}\mathbf{b}$ \\
\hline
$\mathbf{x}_B$ & $\mathbf{B}^{-1}\mathbf{B} = \mathbf{I}$ & $\mathbf{B}^{-1}\mathbf{N}$ & $\mathbf{B}^{-1}\mathbf{b}$ \\
\hline
\end{tabular}
\caption{单纯形表结构}
\end{table}

\textbf{操作}:
\begin{enumerate}
\item 选择进基列(检验数最负)
\item 选择出基行(最小比值)
\item 主元消元(pivoting)
\item 更新表格
\end{enumerate}

\section{完整数值例子}

\subsection{例子:生产计划问题}

\textbf{问题}:
\begin{align}
\max_{x_1, x_2} \quad & 3x_1 + 4x_2 \\
\text{s.t.} \quad & 2x_1 + x_2 \leq 8 \\
& x_1 + 2x_2 \leq 6 \\
& x_1, x_2 \geq 0
\end{align}

\subsubsection{步骤1:转化为标准形式}

引入松弛变量 $s_1, s_2 \geq 0$:

\begin{align}
\max_{x_1, x_2, s_1, s_2} \quad & 3x_1 + 4x_2 \\
\text{s.t.} \quad & 2x_1 + x_2 + s_1 = 8 \\
& x_1 + 2x_2 + s_2 = 6 \\
& x_1, x_2, s_1, s_2 \geq 0
\end{align}

转换为最小化形式:

\begin{align}
\min_{x_1, x_2, s_1, s_2} \quad & -3x_1 - 4x_2 \\
\text{s.t.} \quad & 2x_1 + x_2 + s_1 = 8 \\
& x_1 + 2x_2 + s_2 = 6 \\
& x_1, x_2, s_1, s_2 \geq 0
\end{align}

\textbf{矩阵形式}:
\begin{align}
\mathbf{c} &= \begin{pmatrix} -3 \\ -4 \\ 0 \\ 0 \end{pmatrix}, \quad
\mathbf{A} = \begin{pmatrix} 2 & 1 & 1 & 0 \\ 1 & 2 & 0 & 1 \end{pmatrix}, \quad
\mathbf{b} = \begin{pmatrix} 8 \\ 6 \end{pmatrix}
\end{align}

\subsubsection{步骤2:初始基本可行解}

选择 $s_1, s_2$ 作为初始基变量(对应单位矩阵):

\begin{align}
\mathbf{B} &= \begin{pmatrix} 1 & 0 \\ 0 & 1 \end{pmatrix} = \mathbf{I}, \quad
\mathbf{x}_B = \begin{pmatrix} s_1 \\ s_2 \end{pmatrix} = \begin{pmatrix} 8 \\ 6 \end{pmatrix}
\end{align}

初始解:$(x_1, x_2, s_1, s_2) = (0, 0, 8, 6)$,目标值 $z = 0$。

\subsubsection{步骤3:第一次迭代}

\textbf{计算检验数}:

\begin{align}
\mathbf{c}_B &= \begin{pmatrix} 0 \\ 0 \end{pmatrix}, \quad
\mathbf{c}_N = \begin{pmatrix} -3 \\ -4 \end{pmatrix} \\
\mathbf{B}^{-1} &= \mathbf{I} \\
\bar{\mathbf{c}}_N^T &= \mathbf{c}_N^T - \mathbf{c}_B^T \mathbf{B}^{-1}\mathbf{N} \\
&= \begin{pmatrix} -3 & -4 \end{pmatrix} - \begin{pmatrix} 0 & 0 \end{pmatrix} \begin{pmatrix} 2 & 1 \\ 1 & 2 \end{pmatrix} \\
&= \begin{pmatrix} -3 & -4 \end{pmatrix}
\end{align}

检验数:$\bar{c}_1 = -3$,$\bar{c}_2 = -4$。

\textbf{选择进基变量}:$x_2$(检验数最小,$\bar{c}_2 = -4$)。

\textbf{选择出基变量}:

\begin{align}
\mathbf{d} &= \mathbf{B}^{-1}\mathbf{a}_2 = \begin{pmatrix} 1 \\ 2 \end{pmatrix} \\
\text{比值} &= \min\left\{\frac{8}{1}, \frac{6}{2}\right\} = \min\{8, 3\} = 3
\end{align}

$s_2$ 出基(对应比值3的行)。

\textbf{更新基}:
\begin{itemize}
\item 进基:$x_2$
\item 出基:$s_2$
\item 新基:$\{s_1, x_2\}$
\end{itemize}

\textbf{更新基矩阵}:

\begin{align}
\mathbf{B} &= \begin{pmatrix} 1 & 1 \\ 0 & 2 \end{pmatrix} \\
\mathbf{B}^{-1} &= \begin{pmatrix} 1 & -1/2 \\ 0 & 1/2 \end{pmatrix}
\end{align}

\textbf{计算新解}:

\begin{align}
\mathbf{x}_B &= \mathbf{B}^{-1}\mathbf{b} = \begin{pmatrix} 1 & -1/2 \\ 0 & 1/2 \end{pmatrix} \begin{pmatrix} 8 \\ 6 \end{pmatrix} = \begin{pmatrix} 5 \\ 3 \end{pmatrix}
\end{align}

新解:$(x_1, x_2, s_1, s_2) = (0, 3, 5, 0)$,目标值 $z = -12$。

\subsubsection{步骤4:第二次迭代}

\textbf{计算检验数}:

\begin{align}
\mathbf{c}_B &= \begin{pmatrix} 0 \\ -4 \end{pmatrix}, \quad
\mathbf{c}_N = \begin{pmatrix} -3 \\ 0 \end{pmatrix} \\
\mathbf{N} &= \begin{pmatrix} 2 & 0 \\ 1 & 1 \end{pmatrix} \\
\bar{\mathbf{c}}_N^T &= \begin{pmatrix} -3 & 0 \end{pmatrix} - \begin{pmatrix} 0 & -4 \end{pmatrix} \begin{pmatrix} 1 & -1/2 \\ 0 & 1/2 \end{pmatrix} \begin{pmatrix} 2 & 0 \\ 1 & 1 \end{pmatrix} \\
&= \begin{pmatrix} -3 & 0 \end{pmatrix} - \begin{pmatrix} 0 & -4 \end{pmatrix} \begin{pmatrix} 2 & -1/2 \\ 1/2 & 1/2 \end{pmatrix} \\
&= \begin{pmatrix} -3 & 0 \end{pmatrix} - \begin{pmatrix} -2 & -2 \end{pmatrix} \\
&= \begin{pmatrix} -1 & 2 \end{pmatrix}
</align>

检验数:$\bar{c}_1 = -1$,$\bar{c}_{s_2} = 2$。

\textbf{选择进基变量}:$x_1$(检验数 $\bar{c}_1 = -1 < 0$)。

\textbf{选择出基变量}:

\begin{align}
\mathbf{d} &= \mathbf{B}^{-1}\mathbf{a}_1 = \begin{pmatrix} 1 & -1/2 \\ 0 & 1/2 \end{pmatrix} \begin{pmatrix} 2 \\ 1 \end{pmatrix} = \begin{pmatrix} 3/2 \\ 1/2 \end{pmatrix} \\
\text{比值} &= \min\left\{\frac{5}{3/2}, \frac{3}{1/2}\right\} = \min\{10/3, 6\} = 10/3
</align>

$s_1$ 出基(对应比值 $10/3$ 的行)。

\textbf{更新基}:
\begin{itemize}
\item 进基:$x_1$
\item 出基:$s_1$
\item 新基:$\{x_1, x_2\}$
</{itemize}

\textbf{更新基矩阵}:

\begin{align}
\mathbf{B} &= \begin{pmatrix} 2 & 1 \\ 1 & 2 \end{pmatrix} \\
\mathbf{B}^{-1} &= \frac{1}{3} \begin{pmatrix} 2 & -1 \\ -1 & 2 \end{pmatrix}
</align>

\textbf{计算新解}:

\begin{align}
\mathbf{x}_B &= \mathbf{B}^{-1}\mathbf{b} = \frac{1}{3} \begin{pmatrix} 2 & -1 \\ -1 & 2 \end{pmatrix} \begin{pmatrix} 8 \\ 6 \end{pmatrix} = \begin{pmatrix} 10/3 \\ 4/3 \end{pmatrix}
</align>

新解:$(x_1, x_2, s_1, s_2) = (10/3, 4/3, 0, 0)$,目标值 $z = -46/3$。

\subsubsection{步骤5:第三次迭代(最优性检查)}

\textbf{计算检验数}:

\begin{align}
\mathbf{c}_B &= \begin{pmatrix} -3 \\ -4 \end{pmatrix}, \quad
\mathbf{c}_N = \begin{pmatrix} 0 \\ 0 \end{pmatrix} \\
\mathbf{N} &= \begin{pmatrix} 1 & 0 \\ 0 & 1 \end{pmatrix} \\
\bar{\mathbf{c}}_N^T &= \begin{pmatrix} 0 & 0 \end{pmatrix} - \begin{pmatrix} -3 & -4 \end{pmatrix} \frac{1}{3} \begin{pmatrix} 2 & -1 \\ -1 & 2 \end{pmatrix} \begin{pmatrix} 1 & 0 \\ 0 & 1 \end{pmatrix} \\
&= \begin{pmatrix} 0 & 0 \end{pmatrix} - \frac{1}{3} \begin{pmatrix} -3 & -4 \end{pmatrix} \begin{pmatrix} 2 & -1 \\ -1 & 2 \end{pmatrix} \\
&= \begin{pmatrix} 0 & 0 \end{pmatrix} - \frac{1}{3} \begin{pmatrix} -2 & -5 \end{pmatrix} \\
&= \begin{pmatrix} 2/3 & 5/3 \end{pmatrix}
</align>

所有检验数 $\geq 0$,达到最优解!

\textbf{最优解}:
\begin{itemize}
\item $x_1^* = 10/3 \approx 3.33$
\item $x_2^* = 4/3 \approx 1.33$
\item 最优值:$z^* = -46/3$(最小化),即原问题的最大值为 $46/3 \approx 15.33$
</{itemize}

\section{内点法(Interior Point Method)}

\subsection{算法思想}

内点法从可行域内部开始,通过障碍函数方法逐步接近最优解。

\textbf{核心思想}:
\begin{enumerate}
\item 从可行域内部的一个点开始
\item 使用障碍函数将约束转化为目标函数的一部分
\item 通过求解一系列无约束优化问题,逐步接近最优解
\item 最终收敛到最优解
</enumerate}

\subsection{障碍函数方法}

\textbf{原始问题}:
\begin{align}
\min_{\mathbf{x}} \quad & \mathbf{c}^T \mathbf{x} \\
\text{s.t.} \quad & \mathbf{A}\mathbf{x} = \mathbf{b} \\
& \mathbf{x} \geq \mathbf{0}
</align>

\textbf{障碍问题}:
\begin{align}
\min_{\mathbf{x}} \quad & \mathbf{c}^T \mathbf{x} - \mu \sum_{i=1}^n \ln x_i \\
\text{s.t.} \quad & \mathbf{A}\mathbf{x} = \mathbf{b}
</align>

其中 $\mu > 0$ 是障碍参数。

\textbf{算法步骤}:
\begin{enumerate}
\item 选择初始 $\mu > 0$ 和可行内点 $\mathbf{x}^0$
\item 求解障碍问题,得到 $\mathbf{x}(\mu)$
\item 减小 $\mu$:$\mu \leftarrow \theta \mu$($0 < \theta < 1$)
\item 重复直到 $\mu$ 足够小
</enumerate>

\subsection{原始-对偶内点法}

\textbf{KKT条件}:
\begin{align}
\mathbf{A}\mathbf{x} &= \mathbf{b} \\
\mathbf{A}^T \mathbf{y} + \mathbf{s} &= \mathbf{c} \\
\mathbf{x} \circ \mathbf{s} &= \mu \mathbf{1} \\
\mathbf{x}, \mathbf{s} &\geq \mathbf{0}
</align>

其中 $\mathbf{y}$ 是对偶变量,$\mathbf{s}$ 是松弛变量,$\circ$ 表示逐元素乘积。

\textbf{牛顿法求解}:线性化KKT系统,求解修正方向。

\section{算法比较}

\subsection{单纯形法 vs 内点法}

\begin{table}[h]
\centering
\begin{tabular}{|l|l|l|}
\hline
\textbf{性质} & \textbf{单纯形法} & \textbf{内点法} \\
\hline
起点 & 顶点 & 可行域内部 \\
\hline
路径 & 沿着边移动 & 穿过内部 \\
\hline
迭代次数 & 通常 $O(m)$ & 通常 $O(\sqrt{n})$ \\
\hline
每次迭代 & 快速(基更新) & 较慢(求解线性系统) \\
\hline
适用性 & 小到中等规模 & 大规模问题 \\
\hline
\end{tabular}
\caption{单纯形法与内点法对比}
\end{table}

\section{实际求解工具}

\subsection{Python示例}

\textbf{使用scipy.optimize.linprog}:

\begin{verbatim}
from scipy.optimize import linprog

c = [-3, -4]  # 目标函数系数(最小化)
A = [[2, 1], [1, 2]]  # 约束矩阵
b = [8, 6]  # 约束右端项

# 求解
result = linprog(c, A_ub=A, b_ub=b, method='simplex')
print(result.x)  # 最优解
print(result.fun)  # 最优值
\end{verbatim}

\subsection{MATLAB示例}

\begin{verbatim}
c = [-3; -4];
A = [2 1; 1 2];
b = [8; 6];

[x, fval] = linprog(c, A, b);
\end{verbatim}

\section{总结}

\begin{enumerate}
\item \textbf{单纯形法}:
   \begin{itemize}
   \item 从顶点开始,沿着边移动
   \item 适合小到中等规模问题
   \item 理论最坏情况是指数时间,但实际表现很好
   \end{itemize}

\item \textbf{内点法}:
   \begin{itemize}
   \item 从可行域内部开始
   \item 适合大规模问题
   \item 多项式时间算法
   \end{itemize}

\item \textbf{选择建议}:
   \begin{itemize}
   \item 小规模问题:单纯形法
   \item 大规模问题:内点法
   \item 现代求解器通常自动选择
   \end{itemize}
</enumerate>

理解线性规划的求解方法,对于应用优化理论解决实际问题非常重要!

\end{document}
<|tool▁calls▁begin|><|tool▁call▁begin|>
read_file
