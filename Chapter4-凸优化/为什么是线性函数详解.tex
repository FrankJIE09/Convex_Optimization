\documentclass[12pt,a4paper]{article}
\usepackage[UTF8]{ctex}
\usepackage{amsmath}
\usepackage{amssymb}
\usepackage{amsthm}
\usepackage{geometry}
\geometry{left=2.5cm,right=2.5cm,top=2.5cm,bottom=2.5cm}

\title{为什么是线性函数?}
\subtitle{理解 $\nabla f_0(\mathbf{x})^T \mathbf{v}$ 的线性性}
\author{}
\date{\today}

\begin{document}

\maketitle

\section{问题提出}

在等式约束的最优性条件中,我们遇到表达式:

\begin{equation}
\nabla f_0(\mathbf{x})^T \mathbf{v}
\end{equation}

\textbf{问题}:为什么这个表达式是线性函数(关于 $\mathbf{v}$ 的线性函数)?

\section{线性函数的定义}

\subsection{基本定义}

\textbf{线性函数}:函数 $L : \mathbb{R}^n \to \mathbb{R}$ 是线性的,如果:

\begin{enumerate}
\item \textbf{可加性}:$L(\mathbf{u} + \mathbf{v}) = L(\mathbf{u}) + L(\mathbf{v})$ 对所有 $\mathbf{u}, \mathbf{v} \in \mathbb{R}^n$

\item \textbf{齐次性}:$L(\alpha \mathbf{v}) = \alpha L(\mathbf{v})$ 对所有 $\alpha \in \mathbb{R}$ 和 $\mathbf{v} \in \mathbb{R}^n$
\end{enumerate}

\textbf{等价表述}:$L(\alpha \mathbf{u} + \beta \mathbf{v}) = \alpha L(\mathbf{u}) + \beta L(\mathbf{v})$

\subsection{矩阵表示}

\textbf{定理}:函数 $L : \mathbb{R}^n \to \mathbb{R}$ 是线性的,当且仅当存在向量 $\mathbf{a} \in \mathbb{R}^n$,使得:

\begin{equation}
L(\mathbf{v}) = \mathbf{a}^T \mathbf{v}
\end{equation}

\textbf{证明思路}:
\begin{itemize}
\item 如果 $L$ 是线性的,则 $L(\mathbf{v}) = L(\sum_{i=1}^n v_i \mathbf{e}_i) = \sum_{i=1}^n v_i L(\mathbf{e}_i)$
\item 设 $a_i = L(\mathbf{e}_i)$,则 $L(\mathbf{v}) = \sum_{i=1}^n a_i v_i = \mathbf{a}^T \mathbf{v}$
\end{itemize}

\section{为什么 $\nabla f_0(\mathbf{x})^T \mathbf{v}$ 是线性函数?}

\subsection{表达式分析}

\textbf{表达式}:$L(\mathbf{v}) = \nabla f_0(\mathbf{x})^T \mathbf{v}$

\textbf{关键观察}:
\begin{itemize}
\item $\nabla f_0(\mathbf{x})$ 是固定的向量(当 $\mathbf{x}$ 固定时)
\item $\mathbf{v}$ 是变量
\item 表达式是向量内积的形式
\end{itemize}

\subsection{验证线性性}

\textbf{可加性}:

\begin{align}
L(\mathbf{u} + \mathbf{v}) &= \nabla f_0(\mathbf{x})^T (\mathbf{u} + \mathbf{v}) \\
&= \nabla f_0(\mathbf{x})^T \mathbf{u} + \nabla f_0(\mathbf{x})^T \mathbf{v} \\
&= L(\mathbf{u}) + L(\mathbf{v}) \quad \checkmark
\end{align}

\textbf{齐次性}:

\begin{align}
L(\alpha \mathbf{v}) &= \nabla f_0(\mathbf{x})^T (\alpha \mathbf{v}) \\
&= \alpha (\nabla f_0(\mathbf{x})^T \mathbf{v}) \\
&= \alpha L(\mathbf{v}) \quad \checkmark
\end{align}

\textbf{结论}:$L(\mathbf{v}) = \nabla f_0(\mathbf{x})^T \mathbf{v}$ 是线性函数。$\square$

\section{内积的线性性}

\subsection{内积的性质}

\textbf{内积的线性性}:对于固定向量 $\mathbf{a}$,函数 $L(\mathbf{v}) = \mathbf{a}^T \mathbf{v}$ 是线性的。

\textbf{原因}:
\begin{itemize}
\item 内积对第二个变量是线性的
\item $\mathbf{a}^T (\mathbf{u} + \mathbf{v}) = \mathbf{a}^T \mathbf{u} + \mathbf{a}^T \mathbf{v}$
\item $\mathbf{a}^T (\alpha \mathbf{v}) = \alpha (\mathbf{a}^T \mathbf{v})$
\end{itemize}

\subsection{具体例子}

\textbf{例子}:$\mathbf{a} = (1, 2)^T$,$L(\mathbf{v}) = \mathbf{a}^T \mathbf{v} = v_1 + 2v_2$

\textbf{验证线性性}:

\begin{align}
L(\mathbf{u} + \mathbf{v}) &= (u_1 + v_1) + 2(u_2 + v_2) \\
&= (u_1 + 2u_2) + (v_1 + 2v_2) \\
&= L(\mathbf{u}) + L(\mathbf{v}) \quad \checkmark
\end{align}

\begin{align}
L(\alpha \mathbf{v}) &= \alpha v_1 + 2(\alpha v_2) \\
&= \alpha (v_1 + 2v_2) \\
&= \alpha L(\mathbf{v}) \quad \checkmark
\end{align}

\section{为什么这个性质重要?}

\subsection{在最优性条件中的应用}

\textbf{最优性条件}:$\nabla f_0(\mathbf{x})^T \mathbf{v} \geq 0$ 对所有 $\mathbf{v} \in N(\mathbf{A})$ 成立。

\textbf{关键观察}:由于 $L(\mathbf{v}) = \nabla f_0(\mathbf{x})^T \mathbf{v}$ 是线性函数,且 $N(\mathbf{A})$ 是子空间,有:

\begin{itemize}
\item 如果 $L(\mathbf{v}) \geq 0$ 对所有 $\mathbf{v} \in N(\mathbf{A})$ 成立
\item 则对 $-\mathbf{v} \in N(\mathbf{A})$(因为 $N(\mathbf{A})$ 是子空间),也有 $L(-\mathbf{v}) \geq 0$
\item 由于线性性:$L(-\mathbf{v}) = -L(\mathbf{v}) \geq 0$,因此 $L(\mathbf{v}) \leq 0$
\item 结合两个不等式:$L(\mathbf{v}) = 0$
\end{itemize}

\textbf{结论}:线性函数在子空间上非负,则它必须在该子空间上为零。

\subsection{几何意义}

\textbf{几何解释}:
\begin{itemize}
\item 线性函数 $L(\mathbf{v}) = \mathbf{a}^T \mathbf{v}$ 表示超平面
\item 如果线性函数在子空间上非负,则子空间必须在该超平面上
\item 即:子空间垂直于 $\mathbf{a}$
\end{itemize}

\section{其他线性函数的例子}

\subsection{例子1:$\mathbf{a}^T \mathbf{x}$}

\textbf{函数}:$f(\mathbf{x}) = \mathbf{a}^T \mathbf{x}$

\textbf{线性性}:
\begin{itemize}
\item $f(\mathbf{u} + \mathbf{v}) = \mathbf{a}^T (\mathbf{u} + \mathbf{v}) = \mathbf{a}^T \mathbf{u} + \mathbf{a}^T \mathbf{v} = f(\mathbf{u}) + f(\mathbf{v})$
\item $f(\alpha \mathbf{x}) = \mathbf{a}^T (\alpha \mathbf{x}) = \alpha (\mathbf{a}^T \mathbf{x}) = \alpha f(\mathbf{x})$
\end{itemize}

\textbf{结论}:$\mathbf{a}^T \mathbf{x}$ 是线性函数。

\subsection{例子2:矩阵向量乘法}

\textbf{函数}:$f(\mathbf{x}) = \mathbf{A}\mathbf{x}$(向量值函数)

\textbf{线性性}:
\begin{itemize}
\item $f(\mathbf{u} + \mathbf{v}) = \mathbf{A}(\mathbf{u} + \mathbf{v}) = \mathbf{A}\mathbf{u} + \mathbf{A}\mathbf{v} = f(\mathbf{u}) + f(\mathbf{v})$
\item $f(\alpha \mathbf{x}) = \mathbf{A}(\alpha \mathbf{x}) = \alpha (\mathbf{A}\mathbf{x}) = \alpha f(\mathbf{x})$
\end{itemize}

\textbf{结论}:$\mathbf{A}\mathbf{x}$ 是线性函数(向量值)。

\subsection{例子3:固定向量的内积}

\textbf{函数}:$L(\mathbf{v}) = \mathbf{a}^T \mathbf{v}$,其中 $\mathbf{a}$ 固定

\textbf{线性性}:
\begin{itemize}
\item $L(\mathbf{u} + \mathbf{v}) = \mathbf{a}^T (\mathbf{u} + \mathbf{v}) = \mathbf{a}^T \mathbf{u} + \mathbf{a}^T \mathbf{v} = L(\mathbf{u}) + L(\mathbf{v})$
\item $L(\alpha \mathbf{v}) = \mathbf{a}^T (\alpha \mathbf{v}) = \alpha (\mathbf{a}^T \mathbf{v}) = \alpha L(\mathbf{v})$
\end{itemize}

\textbf{结论}:固定向量的内积是线性函数。

\section{线性函数 vs 仿射函数}

\subsection{线性函数}

\textbf{定义}:$L(\mathbf{v}) = \mathbf{a}^T \mathbf{v}$

\textbf{性质}:
\begin{itemize}
\item $L(\mathbf{0}) = 0$(通过原点)
\item 可加性和齐次性
\end{itemize}

\subsection{仿射函数}

\textbf{定义}:$A(\mathbf{v}) = \mathbf{a}^T \mathbf{v} + b$

\textbf{性质}:
\begin{itemize}
\item $A(\mathbf{0}) = b$(不一定通过原点)
\item 线性函数 + 常数
\end{itemize}

\subsection{区别}

\begin{itemize}
\item \textbf{线性函数}:$L(\mathbf{v}) = \mathbf{a}^T \mathbf{v}$(通过原点)
\item \textbf{仿射函数}:$A(\mathbf{v}) = \mathbf{a}^T \mathbf{v} + b$(不一定通过原点)
\end{itemize}

\textbf{注意}:在最优性条件中,$\nabla f_0(\mathbf{x})^T \mathbf{v}$ 是线性函数(关于 $\mathbf{v}$),因为它是内积形式,没有常数项。

\section{总结}

\subsection{为什么是线性函数?}

\begin{enumerate}
\item \textbf{形式}:$\nabla f_0(\mathbf{x})^T \mathbf{v}$ 是内积形式

\item \textbf{固定向量}:$\nabla f_0(\mathbf{x})$ 是固定的(当 $\mathbf{x}$ 固定时)

\item \textbf{内积的线性性}:内积对第二个变量是线性的

\item \textbf{验证}:满足可加性和齐次性
</enumerate}

\subsection{关键理解}

\begin{enumerate}
\item \textbf{线性函数}:形式为 $L(\mathbf{v}) = \mathbf{a}^T \mathbf{v}$ 的函数

\item \textbf{内积}:$\mathbf{a}^T \mathbf{v}$ 是线性函数(关于 $\mathbf{v}$)

\item \textbf{在最优性条件中}:$\nabla f_0(\mathbf{x})^T \mathbf{v}$ 是线性函数,这允许我们使用线性函数的性质
</enumerate}

\subsection{应用}

\begin{enumerate}
\item \textbf{简化最优性条件}:利用线性函数的性质

\item \textbf{几何理解}:线性函数在子空间上的性质

\item \textbf{理论推导}:从"非负"推出"为零"
</enumerate}

理解线性函数的概念,是理解最优性条件推导的关键!

\end{document}

