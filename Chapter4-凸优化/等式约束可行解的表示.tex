\documentclass[12pt,a4paper]{article}
\usepackage[UTF8]{ctex}
\usepackage{amsmath}
\usepackage{amssymb}
\usepackage{amsthm}
\usepackage{geometry}
\geometry{left=2.5cm,right=2.5cm,top=2.5cm,bottom=2.5cm}

\title{为什么每个可行解都可以写成 $y = x + v$,其中 $v \in N(\mathbf{A})$?}
\author{}
\date{\today}

\begin{document}

\maketitle

\section{问题提出}

在只有等式约束的优化问题中:

\begin{align}
\begin{array}{ll}
\text{minimize} & f_0(\mathbf{x}) \\
\text{subject to} & \mathbf{A}\mathbf{x} = \mathbf{b}
\end{array}
\end{align}

\textbf{问题}:为什么每个可行解 $\mathbf{y}$ 都可以写成 $\mathbf{y} = \mathbf{x} + \mathbf{v}$,其中 $\mathbf{v} \in N(\mathbf{A})$?

\section{零空间(Null Space)回顾}

\subsection{定义}

\textbf{零空间}:矩阵 $\mathbf{A} \in \mathbb{R}^{m \times n}$ 的零空间定义为:

\begin{equation}
N(\mathbf{A}) = \{\mathbf{v} \in \mathbb{R}^n \mid \mathbf{A}\mathbf{v} = \mathbf{0}\}
\end{equation}

\textbf{含义}:
\begin{itemize}
\item 所有被 $\mathbf{A}$ 映射到零向量的向量的集合
\item 是 $\mathbb{R}^n$ 的子空间
\item 也称为核空间(kernel)
\end{itemize}

\section{关键观察}

\subsection{可行解的性质}

\textbf{可行解}:$\mathbf{x}$ 是可行的,如果 $\mathbf{A}\mathbf{x} = \mathbf{b}$。

\textbf{另一个可行解}:$\mathbf{y}$ 也是可行的,如果 $\mathbf{A}\mathbf{y} = \mathbf{b}$。

\subsection{关键计算}

\textbf{计算差向量}:

\begin{align}
\mathbf{A}(\mathbf{y} - \mathbf{x}) &= \mathbf{A}\mathbf{y} - \mathbf{A}\mathbf{x} \\
&= \mathbf{b} - \mathbf{b} \\
&= \mathbf{0}
\end{align}

\textbf{结论}:$\mathbf{A}(\mathbf{y} - \mathbf{x}) = \mathbf{0}$

\textbf{根据零空间定义}:$\mathbf{y} - \mathbf{x} \in N(\mathbf{A})$

\textbf{设}:$\mathbf{v} = \mathbf{y} - \mathbf{x}$,则 $\mathbf{v} \in N(\mathbf{A})$

\textbf{因此}:$\mathbf{y} = \mathbf{x} + \mathbf{v}$,其中 $\mathbf{v} \in N(\mathbf{A})$。$\square$

\section{详细证明}

\subsection{方向1:如果 $\mathbf{y}$ 是可行解,则 $\mathbf{y} = \mathbf{x} + \mathbf{v}$,其中 $\mathbf{v} \in N(\mathbf{A})$}

\textbf{已知}:
\begin{itemize}
\item $\mathbf{x}$ 是可行解:$\mathbf{A}\mathbf{x} = \mathbf{b}$
\item $\mathbf{y}$ 是可行解:$\mathbf{A}\mathbf{y} = \mathbf{b}$
\end{itemize}

\textbf{证明}:

\begin{align}
\mathbf{A}(\mathbf{y} - \mathbf{x}) &= \mathbf{A}\mathbf{y} - \mathbf{A}\mathbf{x} \\
&= \mathbf{b} - \mathbf{b} \\
&= \mathbf{0}
\end{align}

因此 $\mathbf{y} - \mathbf{x} \in N(\mathbf{A})$。

设 $\mathbf{v} = \mathbf{y} - \mathbf{x}$,则 $\mathbf{v} \in N(\mathbf{A})$,且 $\mathbf{y} = \mathbf{x} + \mathbf{v}$。$\square$

\subsection{方向2:如果 $\mathbf{y} = \mathbf{x} + \mathbf{v}$,其中 $\mathbf{v} \in N(\mathbf{A})$,则 $\mathbf{y}$ 是可行解}

\textbf{已知}:
\begin{itemize}
\item $\mathbf{x}$ 是可行解:$\mathbf{A}\mathbf{x} = \mathbf{b}$
\item $\mathbf{v} \in N(\mathbf{A})$:$\mathbf{A}\mathbf{v} = \mathbf{0}$
\item $\mathbf{y} = \mathbf{x} + \mathbf{v}$
\end{itemize}

\textbf{证明}:

\begin{align}
\mathbf{A}\mathbf{y} &= \mathbf{A}(\mathbf{x} + \mathbf{v}) \\
&= \mathbf{A}\mathbf{x} + \mathbf{A}\mathbf{v} \\
&= \mathbf{b} + \mathbf{0} \\
&= \mathbf{b}
\end{align}

因此 $\mathbf{y}$ 是可行解。$\square$

\subsection{等价性}

\textbf{结论}:$\mathbf{y}$ 是可行解 $\Leftrightarrow$ $\mathbf{y} = \mathbf{x} + \mathbf{v}$,其中 $\mathbf{v} \in N(\mathbf{A})$

\section{几何直观}

\subsection{可行集是仿射集合}

\textbf{可行集}:$\{\mathbf{x} \mid \mathbf{A}\mathbf{x} = \mathbf{b}\}$

\textbf{性质}:这是仿射集合(affine set)。

\textbf{几何意义}:
\begin{itemize}
\item 如果 $\mathbf{x}$ 是一个可行解
\item 则所有可行解都在通过 $\mathbf{x}$ 且平行于 $N(\mathbf{A})$ 的仿射集合中
\item 这个仿射集合是 $\mathbf{x} + N(\mathbf{A}) = \{\mathbf{x} + \mathbf{v} \mid \mathbf{v} \in N(\mathbf{A})\}$
\end{itemize}

\subsection{具体例子}

\textbf{例子}:$\mathbf{A} = \begin{pmatrix} 1 & 1 \end{pmatrix}$,$\mathbf{b} = 1$

\textbf{约束}:$x + y = 1$

\textbf{零空间}:$N(\mathbf{A}) = \{\mathbf{v} \mid \mathbf{A}\mathbf{v} = 0\} = \{\mathbf{v} \mid v_1 + v_2 = 0\}$

即:$N(\mathbf{A}) = \{(t, -t)^T \mid t \in \mathbb{R}\}$(一条直线)

\textbf{可行解}:$\mathbf{x} = (1, 0)^T$ 是可行解($1 + 0 = 1$)

\textbf{所有可行解}:

\begin{equation}
\mathbf{y} = \mathbf{x} + \mathbf{v} = \begin{pmatrix} 1 \\ 0 \end{pmatrix} + \begin{pmatrix} t \\ -t \end{pmatrix} = \begin{pmatrix} 1 + t \\ -t \end{pmatrix}
\end{equation}

\textbf{验证}:$y_1 + y_2 = (1 + t) + (-t) = 1$ ✓

\textbf{几何意义}:
\begin{itemize}
\item 可行集是直线 $x + y = 1$
\item 零空间是直线 $v_1 + v_2 = 0$(通过原点)
\item 可行集 = 一个可行解 + 零空间
\end{itemize}

\section{在最优性条件中的应用}

\subsection{最优性条件}

\textbf{条件}:对于可行点 $\mathbf{x}$,$\mathbf{x}$ 是最优的,当且仅当:

\begin{equation}
\nabla f_0(\mathbf{x})^T (\mathbf{y} - \mathbf{x}) \geq 0
\end{equation}

对所有可行 $\mathbf{y}$ 成立。

\subsection{使用可行解的表示}

\textbf{关键步骤}:由于每个可行解 $\mathbf{y}$ 可以写成 $\mathbf{y} = \mathbf{x} + \mathbf{v}$,其中 $\mathbf{v} \in N(\mathbf{A})$,有:

\begin{align}
\nabla f_0(\mathbf{x})^T (\mathbf{y} - \mathbf{x}) &= \nabla f_0(\mathbf{x})^T (\mathbf{x} + \mathbf{v} - \mathbf{x}) \\
&= \nabla f_0(\mathbf{x})^T \mathbf{v}
\end{align}

\textbf{最优性条件变为}:

\begin{equation}
\nabla f_0(\mathbf{x})^T \mathbf{v} \geq 0 \quad \text{对所有 } \mathbf{v} \in N(\mathbf{A})
\end{equation}

\subsection{进一步简化}

\textbf{关键观察}:如果线性函数在子空间上非负,则它必须在该子空间上为零。

\textbf{原因}:
\begin{itemize}
\item 如果 $\nabla f_0(\mathbf{x})^T \mathbf{v} \geq 0$ 对所有 $\mathbf{v} \in N(\mathbf{A})$ 成立
\item 则对 $-\mathbf{v} \in N(\mathbf{A})$(因为 $N(\mathbf{A})$ 是子空间),也有 $\nabla f_0(\mathbf{x})^T (-\mathbf{v}) \geq 0$
\item 即 $-\nabla f_0(\mathbf{x})^T \mathbf{v} \geq 0$,因此 $\nabla f_0(\mathbf{x})^T \mathbf{v} \leq 0$
\item 结合两个不等式:$\nabla f_0(\mathbf{x})^T \mathbf{v} = 0$
\end{itemize}

\textbf{结论}:$\nabla f_0(\mathbf{x}) \perp N(\mathbf{A})$(梯度垂直于零空间)

\section{为什么这个表示有用?}

\subsection{简化最优性条件}

\textbf{优势}:
\begin{itemize}
\item 将"对所有可行 $\mathbf{y}$"的条件
\item 转化为"对所有 $\mathbf{v} \in N(\mathbf{A})$"的条件
\item 零空间是子空间,更容易处理
\end{itemize}

\subsection{几何理解}

\textbf{几何意义}:
\begin{itemize}
\item 可行集是仿射集合:$\mathbf{x} + N(\mathbf{A})$
\item 最优性条件:梯度垂直于可行集的方向(零空间)
\item 这意味着梯度在可行集上是常数(或零)
\end{itemize}

\section{总结}

\subsection{关键结论}

\begin{enumerate}
\item \textbf{可行解的表示}:每个可行解 $\mathbf{y}$ 可以写成 $\mathbf{y} = \mathbf{x} + \mathbf{v}$,其中 $\mathbf{v} \in N(\mathbf{A})$

\item \textbf{原因}:$\mathbf{A}(\mathbf{y} - \mathbf{x}) = \mathbf{A}\mathbf{y} - \mathbf{A}\mathbf{x} = \mathbf{b} - \mathbf{b} = \mathbf{0}$

\item \textbf{几何意义}:可行集是仿射集合 $\mathbf{x} + N(\mathbf{A})$
</enumerate}

\subsection{在最优性条件中的应用}

\begin{enumerate}
\item \textbf{原始条件}:$\nabla f_0(\mathbf{x})^T (\mathbf{y} - \mathbf{x}) \geq 0$ 对所有可行 $\mathbf{y}$

\item \textbf{简化条件}:$\nabla f_0(\mathbf{x})^T \mathbf{v} \geq 0$ 对所有 $\mathbf{v} \in N(\mathbf{A})$

\item \textbf{进一步简化}:$\nabla f_0(\mathbf{x})^T \mathbf{v} = 0$ 对所有 $\mathbf{v} \in N(\mathbf{A})$

\item \textbf{等价表述}:$\nabla f_0(\mathbf{x}) \perp N(\mathbf{A})$
</enumerate}

理解这个表示,是理解等式约束优化问题的关键!

\end{document}

