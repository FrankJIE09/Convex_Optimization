\documentclass[12pt,a4paper]{article}
\usepackage[UTF8]{ctex}
\usepackage{amsmath}
\usepackage{amssymb}
\usepackage{amsthm}
\usepackage{geometry}
\geometry{left=2.5cm,right=2.5cm,top=2.5cm,bottom=2.5cm}

\title{例4.6:解析中心(Analytic Centering)详解}
\author{}
\date{\today}

\begin{document}

\maketitle

\section{引言}

例4.6介绍了解析中心(Analytic Centering)问题,这是一个重要的无约束凸优化问题。解析中心在多面体内部找到一个"中心"点,在优化理论和应用中都有重要地位。

\section{问题陈述}

\subsection{目标函数}

\textbf{函数}:$f_0 : \mathbb{R}^n \to \mathbb{R}$

\begin{equation}
f_0(\mathbf{x}) = -\sum_{i=1}^m \log(b_i - \mathbf{a}_i^T \mathbf{x})
\end{equation}

其中:
\begin{itemize}
\item $\mathbf{a}_i^T$ 是矩阵 $\mathbf{A}$ 的第 $i$ 行
\item $b_i$ 是向量 $\mathbf{b}$ 的第 $i$ 个分量
\item $m$ 是约束的个数
\end{itemize}

\subsection{定义域}

\textbf{定义域}:

\begin{equation}
\text{dom } f_0 = \{\mathbf{x} \in \mathbb{R}^n \mid \mathbf{A}\mathbf{x} \prec \mathbf{b}\}
\end{equation}

\textbf{含义}:
\begin{itemize}
\item $\mathbf{A}\mathbf{x} \prec \mathbf{b}$ 表示 $\mathbf{a}_i^T \mathbf{x} < b_i$ 对所有 $i = 1, \ldots, m$
\item 这是严格不等式约束
\item 定义域是多面体的内部(不包括边界)
\end{itemize}

\subsection{为什么是凸函数?}

\textbf{分析}:

函数 $f_0(\mathbf{x}) = -\sum_{i=1}^m \log(b_i - \mathbf{a}_i^T \mathbf{x})$ 是凸函数,因为:

\begin{enumerate}
\item \textbf{每一项}:$-\log(b_i - \mathbf{a}_i^T \mathbf{x})$ 是凸函数
   \begin{itemize}
   \item $g_i(\mathbf{x}) = b_i - \mathbf{a}_i^T \mathbf{x}$ 是仿射函数(凹函数)
   \item $-\log(\cdot)$ 是凸函数(对数函数的负值)
   \item 凸函数的非负组合是凸函数
   \end{itemize}

\item \textbf{凸函数的和}:凸函数的和仍然是凸函数
\end{enumerate}

\section{最优性条件}

\subsection{梯度计算}

\textbf{计算梯度}:

对于 $f_0(\mathbf{x}) = -\sum_{i=1}^m \log(b_i - \mathbf{a}_i^T \mathbf{x})$,计算梯度:

\begin{align}
\nabla f_0(\mathbf{x}) &= -\sum_{i=1}^m \nabla \log(b_i - \mathbf{a}_i^T \mathbf{x}) \\
&= -\sum_{i=1}^m \frac{1}{b_i - \mathbf{a}_i^T \mathbf{x}} \nabla(b_i - \mathbf{a}_i^T \mathbf{x}) \\
&= -\sum_{i=1}^m \frac{1}{b_i - \mathbf{a}_i^T \mathbf{x}} (-\mathbf{a}_i) \\
&= \sum_{i=1}^m \frac{1}{b_i - \mathbf{a}_i^T \mathbf{x}} \mathbf{a}_i
\end{align}

\textbf{详细推导}:

对于每一项 $-\log(b_i - \mathbf{a}_i^T \mathbf{x})$:

\begin{align}
\frac{\partial}{\partial x_j} [-\log(b_i - \mathbf{a}_i^T \mathbf{x})] &= -\frac{1}{b_i - \mathbf{a}_i^T \mathbf{x}} \cdot \frac{\partial}{\partial x_j}(b_i - \mathbf{a}_i^T \mathbf{x}) \\
&= -\frac{1}{b_i - \mathbf{a}_i^T \mathbf{x}} \cdot (-a_{ij}) \\
&= \frac{a_{ij}}{b_i - \mathbf{a}_i^T \mathbf{x}}
\end{align}

因此:

\begin{equation}
\nabla [-\log(b_i - \mathbf{a}_i^T \mathbf{x})] = \frac{1}{b_i - \mathbf{a}_i^T \mathbf{x}} \mathbf{a}_i
\end{equation}

对所有项求和:

\begin{equation}
\nabla f_0(\mathbf{x}) = \sum_{i=1}^m \frac{1}{b_i - \mathbf{a}_i^T \mathbf{x}} \mathbf{a}_i
\end{equation}

\subsection{最优性条件}

\textbf{无约束问题}:由于这是无约束优化问题,最优性条件是:

\begin{equation}
\nabla f_0(\mathbf{x}) = \mathbf{0}
\end{equation}

\textbf{具体条件}:

\begin{enumerate}
\item \textbf{可行性}:$\mathbf{A}\mathbf{x} \prec \mathbf{b}$($\mathbf{x} \in \text{dom } f_0$)

\item \textbf{最优性}:
   \begin{equation}
   \nabla f_0(\mathbf{x}) = \sum_{i=1}^m \frac{1}{b_i - \mathbf{a}_i^T \mathbf{x}} \mathbf{a}_i = \mathbf{0}
   \end{equation}
\end{enumerate}

\section{几何意义:解析中心}

\subsection{什么是解析中心?}

\textbf{解析中心}:在多面体 $\{\mathbf{x} \mid \mathbf{A}\mathbf{x} \prec \mathbf{b}\}$ 内部,使所有约束"距离"的乘积最大的点。

\textbf{更准确地说}:

\begin{itemize}
\item 对于每个约束 $i$,距离是 $b_i - \mathbf{a}_i^T \mathbf{x}$(到超平面的距离)
\item 解析中心最大化 $\prod_{i=1}^m (b_i - \mathbf{a}_i^T \mathbf{x})$
\item 等价地,最小化 $-\sum_{i=1}^m \log(b_i - \mathbf{a}_i^T \mathbf{x})$
\end{itemize}

\textbf{几何直观}:
\begin{itemize}
\item 解析中心是"最居中"的点
\item 它尽可能远离所有约束边界
\item 类似于多面体的"重心"
\end{itemize}

\section{解的存在性分析}

\subsection{情况1:无解($f_0$ 无下界)}

\textbf{条件}:方程 $\nabla f_0(\mathbf{x}) = \mathbf{0}$ 无解。

\textbf{结果}:$f_0$ 无下界,没有最优解。

\textbf{原因}:
\begin{itemize}
\item 定义域可能是无界的
\item 函数值可以趋于 $-\infty$
\end{itemize}

\subsection{情况2:多个解(仿射集合)}

\textbf{条件}:方程 $\nabla f_0(\mathbf{x}) = \mathbf{0}$ 有多个解。

\textbf{结果}:所有解构成一个仿射集合。

\textbf{原因}:
\begin{itemize}
\item 如果定义域是多面体的内部,且多面体是"扁平的"
\item 可能存在多个点满足最优性条件
\end{itemize}

\subsection{情况3:唯一解}

\textbf{条件}:开多面体 $\{\mathbf{x} \mid \mathbf{A}\mathbf{x} \prec \mathbf{b}\}$ 非空且有界。

\textbf{结果}:存在唯一的解析中心。

\textbf{原因}:
\begin{itemize}
\item 有界性保证了函数有下界
\item 严格凸性(在某些条件下)保证了唯一性
\end{itemize}

\section{具体例子}

\subsection{例子1:二维情况}

\textbf{约束}:
\begin{align}
x + y &< 1 \\
x &< 2 \\
y &< 2
\end{align}

\textbf{矩阵形式}:$\mathbf{A} = \begin{pmatrix} 1 & 1 \\ 1 & 0 \\ 0 & 1 \end{pmatrix}$,$\mathbf{b} = \begin{pmatrix} 1 \\ 2 \\ 2 \end{pmatrix}$

\textbf{目标函数}:

\begin{equation}
f_0(x, y) = -\log(1 - x - y) - \log(2 - x) - \log(2 - y)
\end{equation}

\textbf{梯度}:

\begin{align}
\frac{\partial f_0}{\partial x} &= \frac{1}{1 - x - y} + \frac{1}{2 - x} = 0 \\
\frac{\partial f_0}{\partial y} &= \frac{1}{1 - x - y} + \frac{1}{2 - y} = 0
\end{align}

\textbf{解析中心}:需要求解这个非线性方程组。

\subsection{例子2:一维情况}

\textbf{约束}:$a < x < b$(区间)

\textbf{目标函数}:

\begin{equation}
f_0(x) = -\log(x - a) - \log(b - x)
\end{equation}

\textbf{梯度}:

\begin{equation}
f_0'(x) = -\frac{1}{x - a} + \frac{1}{b - x} = \frac{-(b - x) + (x - a)}{(x - a)(b - x)} = \frac{2x - (a + b)}{(x - a)(b - x)}
\end{equation}

\textbf{最优性条件}:$f_0'(x) = 0$,即 $2x - (a + b) = 0$

\textbf{解析中心}:$x^* = \frac{a + b}{2}$(区间的中点)

\textbf{验证}:这确实是区间的中心点。

\section{为什么叫"解析中心"?}

\subsection{名称来源}

\textbf{解析中心}(Analytic Centering):
\begin{itemize}
\item \textbf{解析}:使用解析方法(求导)找到中心
\item \textbf{中心}:在多面体内部找到"中心"点
\item 区别于几何中心(重心)
\end{itemize}

\subsection{与几何中心的区别}

\textbf{几何中心}(重心):
\begin{itemize}
\item 所有顶点的平均
\item 依赖于顶点的位置
\end{itemize}

\textbf{解析中心}:
\begin{itemize}
\item 最大化到所有边界的"距离"的乘积
\item 只依赖于约束,不依赖于顶点
\item 在多面体内部(不包括边界)
\end{itemize}

\section{应用}

\subsection{在优化中的应用}

\textbf{内点法}:
\begin{itemize}
\item 解析中心可以作为内点法的起始点
\item 提供多面体内部的"好"的初始点
\end{itemize}

\textbf{障碍函数}:
\begin{itemize}
\item $-\log(b_i - \mathbf{a}_i^T \mathbf{x})$ 是障碍函数
\item 当接近边界时,函数值趋于 $+\infty$
\item 保证解在多面体内部
\end{itemize}

\section{总结}

\subsection{问题特点}

\begin{enumerate}
\item \textbf{目标函数}:$f_0(\mathbf{x}) = -\sum_{i=1}^m \log(b_i - \mathbf{a}_i^T \mathbf{x})$

\item \textbf{定义域}:$\{\mathbf{x} \mid \mathbf{A}\mathbf{x} \prec \mathbf{b}\}$(多面体内部)

\item \textbf{凸函数}:对数障碍函数的负和

\item \textbf{无约束}:在定义域内无约束
\end{enumerate}

\subsection{最优性条件}

\begin{enumerate}
\item \textbf{可行性}:$\mathbf{A}\mathbf{x} \prec \mathbf{b}$

\item \textbf{最优性}:$\sum_{i=1}^m \frac{1}{b_i - \mathbf{a}_i^T \mathbf{x}} \mathbf{a}_i = \mathbf{0}$
\end{enumerate}

\subsection{解的存在性}

\begin{enumerate}
\item \textbf{无解}:$f_0$ 无下界

\item \textbf{多个解}:构成仿射集合

\item \textbf{唯一解}:当多面体有界时
</enumerate}

理解解析中心问题,有助于理解内点法和障碍函数方法!

\end{document}

