\documentclass[12pt,a4paper]{article}
\usepackage[UTF8]{ctex}
\usepackage{amsmath}
\usepackage{amssymb}
\usepackage{amsthm}
\usepackage{geometry}
\geometry{left=2.5cm,right=2.5cm,top=2.5cm,bottom=2.5cm}

\title{第4.1.4节:参数与Oracle问题描述详解}
\subtitle{理解优化问题的两种描述方式}
\author{}
\date{\today}

\begin{document}

\maketitle

\section{引言}

在优化问题中,除了问题的数学形式,还有一个重要问题:\textbf{如何描述或指定目标函数和约束函数?}第4.1.4节介绍了两种主要的问题描述方式:参数描述(Parameter Description)和Oracle描述(Oracle Description)。理解这两种方式的区别对于学习优化算法非常重要。

\section{问题:如何描述函数?}

\subsection{标准形式回顾}

优化问题的标准形式为:

\begin{align}
\begin{array}{ll}
\text{minimize} & f_0(\mathbf{x}) \\
\text{subject to} & f_i(\mathbf{x}) \leq 0, \quad i = 1, \ldots, m \\
& h_i(\mathbf{x}) = 0, \quad i = 1, \ldots, p
\end{array}
\end{align}

\textbf{问题}:这些函数 $f_0, f_1, \ldots, f_m, h_1, \ldots, h_p$ 是如何给出的?

\subsection{两种主要方式}

\begin{enumerate}
\item \textbf{参数描述}(Parameter Description):通过公式和参数给出
\item \textbf{Oracle描述}(Oracle Description):通过"黑盒"函数给出
\end{enumerate}

\section{参数描述(Parameter Description)}

\subsection{定义}

\textbf{参数描述}:函数通过\textbf{解析形式}或\textbf{闭式表达式}给出,即用公式表示,公式中包含变量 $\mathbf{x}$ 和某些\textbf{参数}(parameters)。

\subsection{特点}

\begin{itemize}
\item 函数有明确的数学表达式
\item 通过给出参数的值来指定具体的问题实例
\item 可以直接看到函数的结构
\end{itemize}

\subsection{例子1:二次函数}

\textbf{目标函数}:$f_0(\mathbf{x}) = \frac{1}{2}\mathbf{x}^T \mathbf{P} \mathbf{x} + \mathbf{q}^T \mathbf{x} + r$

\textbf{参数}:
\begin{itemize}
\item $\mathbf{P} \in \mathbb{S}^n$:对称矩阵($n \times n$)
\item $\mathbf{q} \in \mathbb{R}^n$:向量
\item $r \in \mathbb{R}$:标量
\end{itemize}

\textbf{如何指定问题?}

给出参数的值:
\begin{itemize}
\item 例如:$\mathbf{P} = \begin{pmatrix} 2 & 1 \\ 1 & 2 \end{pmatrix}$,$\mathbf{q} = (1, 0)^T$,$r = 3$
\item 这样就完全确定了目标函数
\item 不同的参数值对应不同的问题实例
\end{itemize}

\textbf{理解}:
\begin{itemize}
\item 函数形式是固定的:二次函数
\item 通过改变参数,得到不同的具体问题
\item 参数是"问题数据"(problem data)
\end{itemize}

\subsection{例子2:线性规划}

\textbf{问题}:
\begin{align}
\text{minimize} \quad & \mathbf{c}^T \mathbf{x} \\
\text{subject to} \quad & \mathbf{A}\mathbf{x} \leq \mathbf{b}
\end{align}

\textbf{参数}:
\begin{itemize}
\item $\mathbf{c} \in \mathbb{R}^n$:目标函数的系数向量
\item $\mathbf{A} \in \mathbb{R}^{m \times n}$:约束矩阵
\item $\mathbf{b} \in \mathbb{R}^m$:约束右端项向量
\end{itemize}

\textbf{如何指定问题?}

给出 $\mathbf{c}$、$\mathbf{A}$、$\mathbf{b}$ 的具体数值,就确定了问题实例。

\subsection{例子3:最小二乘问题}

\textbf{问题}:
\begin{equation}
\text{minimize } \|\mathbf{A}\mathbf{x} - \mathbf{b}\|_2^2
\end{equation}

\textbf{参数}:
\begin{itemize}
\item $\mathbf{A} \in \mathbb{R}^{m \times n}$:数据矩阵
\item $\mathbf{b} \in \mathbb{R}^m$:观测向量
\end{itemize}

\textbf{如何指定问题?}

给出 $\mathbf{A}$ 和 $\mathbf{b}$ 的具体数值。

\section{Oracle描述(Oracle Description)}

\subsection{定义}

\textbf{Oracle描述}(也称为"黑盒"模型或子程序模型):我们\textbf{不知道}函数的显式表达式,但可以在任意点 $\mathbf{x} \in \text{dom } f$ 处\textbf{评估}函数值 $f(\mathbf{x})$(通常还包括一些导数)。

\subsection{特点}

\begin{itemize}
\item 函数是"黑盒":看不到内部结构
\item 只能通过"查询"(query)获得函数值
\item 查询通常有成本(如时间成本)
\item 我们有一些先验信息(如凸性、有界性等)
\end{itemize}

\subsection{Oracle的含义}

\textbf{Oracle}(预言机)是一个"黑盒"函数,可以:
\begin{itemize}
\item 接受输入:点 $\mathbf{x}$
\item 返回输出:函数值 $f(\mathbf{x})$ 和/或导数 $\nabla f(\mathbf{x})$ 等
\item 不提供:函数的显式表达式或内部结构
\end{itemize}

\textbf{类比}:
\begin{itemize}
\item 就像调用一个子程序(subroutine)
\item 输入参数,得到返回值
\item 但看不到源代码
\end{itemize}

\subsection{具体例子}

\textbf{无约束优化问题}:

\begin{equation}
\text{minimize } f(\mathbf{x})
\end{equation}

\textbf{Oracle模型}:
\begin{itemize}
\item 有一个子程序可以计算 $f(\mathbf{x})$ 和 $\nabla f(\mathbf{x})$
\item 我们可以在任意 $\mathbf{x} \in \text{dom } f$ 处调用这个子程序
\item 调用时传入 $\mathbf{x}$,返回 $f(\mathbf{x})$ 和 $\nabla f(\mathbf{x})$
\item 但我们看不到函数的源代码或表达式
\end{itemize}

\textbf{先验信息}:
\begin{itemize}
\item 函数是凸的
\item 函数是可微的
\item 函数值有界
\item 等等
\end{itemize}

\subsection{Oracle模型的特点}

\begin{enumerate}
\item \textbf{不知道函数表达式}:
   \begin{itemize}
   \item 我们不知道 $f$ 的具体公式
   \item 只能通过查询了解函数
   \end{itemize}

\item \textbf{只能查询}:
   \begin{itemize}
   \item 在查询过的点,我们知道函数值
   \item 在未查询的点,我们不知道函数值
   \end{itemize}

\item \textbf{查询有成本}:
   \begin{itemize}
   \item 每次查询可能需要时间
   \item 可能涉及复杂的计算
   \item 我们希望尽量减少查询次数
   \end{itemize}

\item \textbf{有先验信息}:
   \begin{itemize}
   \item 知道函数是凸的
   \item 知道函数是可微的
   \item 知道函数值的界限
   \end{itemize}
\end{enumerate}

\section{两种方式的对比}

\subsection{参数描述 vs Oracle描述}

\begin{table}[h]
\centering
\begin{tabular}{|l|l|l|}
\hline
\textbf{性质} & \textbf{参数描述} & \textbf{Oracle描述} \\
\hline
函数表达式 & 已知(有公式) & 未知(黑盒) \\
\hline
函数结构 & 可见 & 不可见 \\
\hline
评估方式 & 直接计算 & 调用Oracle \\
\hline
灵活性 & 固定形式 & 任意函数 \\
\hline
算法利用 & 可利用结构 & 只能利用先验信息 \\
\hline
\end{tabular}
\caption{参数描述与Oracle描述的对比}
\end{table}

\subsection{具体例子对比}

\textbf{例子:最小化 $f(x) = x^2 + 2x + 1$}

\textbf{参数描述}:
\begin{itemize}
\item 函数形式:$f(x) = ax^2 + bx + c$
\item 参数:$a = 1$,$b = 2$,$c = 1$
\item 我们知道这是二次函数
\item 可以直接利用二次函数的性质
\end{itemize}

\textbf{Oracle描述}:
\begin{itemize}
\item 有一个子程序:输入 $x$,返回 $f(x)$ 和 $f'(x)$
\item 我们不知道 $f$ 是二次函数
\item 只能通过多次查询来了解函数
\item 算法只能利用先验信息(如凸性)
\end{itemize}

\section{实际应用}

\subsection{参数描述的应用}

\begin{itemize}
\item \textbf{标准优化问题}:线性规划、二次规划等
\item \textbf{问题建模}:通过参数描述问题结构
\item \textbf{算法设计}:可以利用函数的具体结构
\end{itemize}

\textbf{优势}:
\begin{itemize}
\item 可以利用函数的结构设计高效算法
\item 可以进行符号计算
\item 可以进行理论分析
\end{itemize}

\subsection{Oracle描述的应用}

\begin{itemize}
\item \textbf{复杂函数}:难以写出显式表达式
\item \textbf{仿真模型}:通过仿真计算函数值
\item \textbf{实验数据}:通过实验测量函数值
\item \textbf{机器学习}:损失函数可能很复杂
\end{itemize}

\textbf{优势}:
\begin{itemize}
\item 灵活性高:可以处理任意函数
\item 适用于复杂系统
\item 可以处理数值计算得到的函数
\end{itemize}

\section{两者的关系}

\subsection{从参数描述到Oracle}

\textbf{转换}:如果给定参数描述,可以构造一个Oracle:

\begin{itemize}
\item Oracle接受输入 $\mathbf{x}$
\item 使用参数和公式计算 $f(\mathbf{x})$
\item 返回函数值和导数
\end{itemize}

\textbf{例子}:
\begin{itemize}
\item 参数描述:$f(\mathbf{x}) = \mathbf{x}^T \mathbf{P} \mathbf{x} + \mathbf{q}^T \mathbf{x} + r$
\item Oracle:输入 $\mathbf{x}$,计算并返回 $f(\mathbf{x})$ 和 $\nabla f(\mathbf{x}) = 2\mathbf{P}\mathbf{x} + \mathbf{q}$
\end{itemize}

\subsection{从Oracle到参数描述}

\textbf{通常不可能}:
\begin{itemize}
\item 如果只有Oracle,通常无法得到参数描述
\item 因为我们不知道函数的结构
\item 只能通过查询了解函数
\end{itemize}

\subsection{实际中的混合}

\textbf{实际情况}:
\begin{itemize}
\item 大多数算法使用Oracle模型
\item 但对于特定的参数化问题族,可以设计更高效的算法
\item 例如:针对二次函数的特殊算法比通用算法更高效
\end{itemize}

\section{在算法中的应用}

\subsection{通用算法}

\begin{itemize}
\item 使用Oracle模型
\item 适用于各种函数
\item 但可能不是最优的
\end{itemize}

\textbf{例子}:梯度下降法
\begin{itemize}
\item 只需要Oracle提供 $f(\mathbf{x})$ 和 $\nabla f(\mathbf{x})$
\item 不关心函数的具体形式
\item 适用于任何可微函数
\end{itemize}

\subsection{专用算法}

\begin{itemize}
\item 针对特定参数化问题族
\item 利用函数结构
\item 通常更高效
\end{itemize}

\textbf{例子}:二次规划的内点法
\begin{itemize}
\item 专门针对二次函数设计
\item 利用二次函数的结构
\item 比通用算法更高效
\end{itemize}

\section{具体例子}

\subsection{例子1:参数描述}

\textbf{问题}:最小化 $f(x) = x^2 - 4x + 3$

\textbf{参数描述}:
\begin{itemize}
\item 函数形式:$f(x) = ax^2 + bx + c$
\item 参数:$a = 1$,$b = -4$,$c = 3$
\item 我们知道这是二次函数,可以直接求解
\end{itemize}

\textbf{求解}:
\begin{itemize}
\item $f'(x) = 2x - 4 = 0$,得到 $x^* = 2$
\item $f(2) = 4 - 8 + 3 = -1$
\end{itemize}

\subsection{例子2:Oracle描述}

\textbf{问题}:最小化某个未知函数 $f(x)$

\textbf{Oracle描述}:
\begin{itemize}
\item 有一个子程序:输入 $x$,返回 $f(x)$ 和 $f'(x)$
\item 我们不知道 $f$ 的表达式
\item 只知道 $f$ 是凸的、可微的
\end{itemize}

\textbf{求解过程}:
\begin{enumerate}
\item 查询 $x_0 = 0$:得到 $f(0) = 3$,$f'(0) = -4$
\item 查询 $x_1 = 2$:得到 $f(2) = -1$,$f'(2) = 0$
\item 由于 $f'(2) = 0$ 且 $f$ 是凸的,$x^* = 2$ 是最优点
\end{enumerate}

\textbf{注意}:我们通过查询找到了最优点,但可能仍然不知道 $f$ 的表达式。

\section{总结}

\begin{enumerate}
\item \textbf{参数描述}:
   \begin{itemize}
   \item 函数通过公式和参数给出
   \item 可以看到函数结构
   \item 可以利用结构设计算法
   \end{itemize}

\item \textbf{Oracle描述}:
   \begin{itemize}
   \item 函数是"黑盒",只能查询
   \item 不知道函数表达式
   \item 只能利用先验信息
   \end{itemize}

\item \textbf{关系}:
   \begin{itemize}
   \item 参数描述可以转换为Oracle
   \item Oracle通常不能转换为参数描述
   \item 实际中经常混合使用
   \end{itemize}

\item \textbf{在算法中}:
   \begin{itemize}
   \item 通用算法使用Oracle模型
   \item 专用算法利用参数描述的结构
   \end{itemize}
\end{enumerate}

理解这两种描述方式的区别,有助于理解优化算法的设计和使用!

\end{document}


