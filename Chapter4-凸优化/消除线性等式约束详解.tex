\documentclass[12pt,a4paper]{article}
\usepackage[UTF8]{ctex}
\usepackage{amsmath}
\usepackage{amssymb}
\usepackage{amsthm}
\usepackage{geometry}
\geometry{left=2.5cm,right=2.5cm,top=2.5cm,bottom=2.5cm}

\title{消除线性等式约束详解}
\subtitle{理解 $R(F) = N(A)$ 的含义}
\author{}
\date{\today}

\begin{document}

\maketitle

\section{问题提出}

在《Convex Optimization》第4.1.3节中,消除线性等式约束的方法涉及:

\begin{equation}
R(F) = N(A)
\end{equation}

\textbf{问题}:这个等式的含义是什么?各个参数 $R$、$F$、$N$、$A$ 分别代表什么?如何理解消除等式约束的过程?

\section{基本概念回顾}

\subsection{线性等式约束}

考虑线性等式约束:
\begin{equation}
\mathbf{A}\mathbf{x} = \mathbf{b}
\end{equation}

其中:
\begin{itemize}
\item $\mathbf{A} \in \mathbb{R}^{p \times n}$:系数矩阵($p$ 个等式,$n$ 个变量)
\item $\mathbf{x} \in \mathbb{R}^n$:优化变量
\item $\mathbf{b} \in \mathbb{R}^p$:常数向量
\end{itemize}

\subsection{矩阵的列空间(Range/Column Space)}

\textbf{定义}:矩阵 $\mathbf{F} \in \mathbb{R}^{n \times k}$ 的\textbf{列空间}(Range),记为 $R(\mathbf{F})$ 或 $\text{col}(\mathbf{F})$,定义为:

\begin{equation}
R(\mathbf{F}) = \{\mathbf{F}\mathbf{z} \mid \mathbf{z} \in \mathbb{R}^k\} = \text{span}\{\mathbf{f}_1, \mathbf{f}_2, \ldots, \mathbf{f}_k\}
\end{equation}

其中 $\mathbf{f}_i$ 是 $\mathbf{F}$ 的第 $i$ 列。

\textbf{几何意义}:
\begin{itemize}
\item $R(\mathbf{F})$ 是所有 $\mathbf{F}\mathbf{z}$ 的集合($\mathbf{z}$ 取遍 $\mathbb{R}^k$)
\item 是 $\mathbf{F}$ 的列向量张成的线性子空间
\item 维度:$\dim(R(\mathbf{F})) = \text{rank}(\mathbf{F})$
\end{itemize}

\subsection{矩阵的零空间(Null Space)}

\textbf{定义}:矩阵 $\mathbf{A} \in \mathbb{R}^{p \times n}$ 的\textbf{零空间}(Null Space),记为 $N(\mathbf{A})$ 或 $\text{null}(\mathbf{A})$,定义为:

\begin{equation}
N(\mathbf{A}) = \{\mathbf{x} \in \mathbb{R}^n \mid \mathbf{A}\mathbf{x} = \mathbf{0}\}
\end{equation}

\textbf{几何意义}:
\begin{itemize}
\item $N(\mathbf{A})$ 是所有使 $\mathbf{A}\mathbf{x} = \mathbf{0}$ 的向量 $\mathbf{x}$ 的集合
\item 是齐次线性方程组 $\mathbf{A}\mathbf{x} = \mathbf{0}$ 的解空间
\item 维度:$\dim(N(\mathbf{A})) = n - \text{rank}(\mathbf{A})$
\end{itemize}

\section{$R(F) = N(A)$ 的含义}

\subsection{等式的含义}

\textbf{$R(\mathbf{F}) = N(\mathbf{A})$} 表示:

矩阵 $\mathbf{F}$ 的列空间等于矩阵 $\mathbf{A}$ 的零空间。

\textbf{具体含义}:
\begin{itemize}
\item $R(\mathbf{F})$:$\mathbf{F}$ 的列向量张成的子空间
\item $N(\mathbf{A})$:$\mathbf{A}$ 的零空间(所有使 $\mathbf{A}\mathbf{x} = \mathbf{0}$ 的向量)
\item 等式:这两个子空间完全相同
\end{itemize}

\subsection{为什么需要这个条件?}

\textbf{目标}:参数化线性方程组 $\mathbf{A}\mathbf{x} = \mathbf{b}$ 的所有解。

\textbf{关键观察}:
\begin{itemize}
\item 如果 $\mathbf{x}_0$ 是 $\mathbf{A}\mathbf{x} = \mathbf{b}$ 的一个特解
\item 那么所有解可以表示为:$\mathbf{x} = \mathbf{x}_0 + \mathbf{v}$,其中 $\mathbf{v} \in N(\mathbf{A})$
\item 因为:$\mathbf{A}(\mathbf{x}_0 + \mathbf{v}) = \mathbf{A}\mathbf{x}_0 + \mathbf{A}\mathbf{v} = \mathbf{b} + \mathbf{0} = \mathbf{b}$
\end{itemize}

\textbf{参数化}:
\begin{itemize}
\item 我们需要用参数 $\mathbf{z} \in \mathbb{R}^k$ 表示 $N(\mathbf{A})$ 中的所有向量
\item 如果 $R(\mathbf{F}) = N(\mathbf{A})$,那么 $N(\mathbf{A})$ 中的任意向量可以表示为 $\mathbf{F}\mathbf{z}$
\item 因此,所有解可以表示为:$\mathbf{x} = \mathbf{x}_0 + \mathbf{F}\mathbf{z}$
\end{itemize}

\section{详细推导}

\subsection{步骤1:找到特解}

设 $\mathbf{x}_0$ 是 $\mathbf{A}\mathbf{x} = \mathbf{b}$ 的一个特解(particular solution),即:
\begin{equation}
\mathbf{A}\mathbf{x}_0 = \mathbf{b}
\end{equation}

\textbf{如何找特解?}
\begin{itemize}
\item 可以使用高斯消元法
\item 或者使用伪逆:$\mathbf{x}_0 = \mathbf{A}^\dagger \mathbf{b}$(如果 $\mathbf{A}$ 有解)
\end{itemize}

\subsection{步骤2:找到零空间的基}

\textbf{目标}:找到矩阵 $\mathbf{F}$,使得 $R(\mathbf{F}) = N(\mathbf{A})$。

\textbf{方法}:
\begin{itemize}
\item 找到 $N(\mathbf{A})$ 的一组基 $\{\mathbf{v}_1, \mathbf{v}_2, \ldots, \mathbf{v}_k\}$
\item 将这些基向量作为 $\mathbf{F}$ 的列:$\mathbf{F} = [\mathbf{v}_1, \mathbf{v}_2, \ldots, \mathbf{v}_k]$
\item 则 $R(\mathbf{F}) = \text{span}\{\mathbf{v}_1, \mathbf{v}_2, \ldots, \mathbf{v}_k\} = N(\mathbf{A})$
\end{itemize}

\textbf{维度}:
\begin{itemize}
\item $k = \dim(N(\mathbf{A})) = n - \text{rank}(\mathbf{A})$
\item 如果 $\mathbf{F}$ 是列满秩的(full rank),则 $k = n - \text{rank}(\mathbf{A})$
\end{itemize}

\subsection{步骤3:参数化所有解}

\textbf{通解}:$\mathbf{A}\mathbf{x} = \mathbf{b}$ 的所有解可以表示为:

\begin{equation}
\mathbf{x} = \mathbf{x}_0 + \mathbf{F}\mathbf{z}, \quad \mathbf{z} \in \mathbb{R}^k
\end{equation}

\textbf{验证}:
\begin{align}
\mathbf{A}\mathbf{x} &= \mathbf{A}(\mathbf{x}_0 + \mathbf{F}\mathbf{z}) \\
&= \mathbf{A}\mathbf{x}_0 + \mathbf{A}\mathbf{F}\mathbf{z} \\
&= \mathbf{b} + \mathbf{0} \quad \text{(因为 $\mathbf{F}\mathbf{z} \in N(\mathbf{A})$)} \\
&= \mathbf{b}
\end{align}

\textbf{完备性}:对于 $\mathbf{A}\mathbf{x} = \mathbf{b}$ 的任意解 $\mathbf{x}$,存在 $\mathbf{z}$ 使得 $\mathbf{x} = \mathbf{x}_0 + \mathbf{F}\mathbf{z}$。

\section{具体例子}

\subsection{例子1:简单的2维情况}

\textbf{问题}:消除等式约束 $x_1 + x_2 = 1$

\textbf{步骤1:写成矩阵形式}

\begin{equation}
\mathbf{A}\mathbf{x} = \mathbf{b} \quad \Rightarrow \quad [1 \quad 1] \begin{pmatrix} x_1 \\ x_2 \end{pmatrix} = 1
\end{equation}

其中 $\mathbf{A} = [1 \quad 1]$,$\mathbf{b} = 1$。

\textbf{步骤2:找特解}

一个特解:$\mathbf{x}_0 = (1, 0)^T$(满足 $1 + 0 = 1$)

\textbf{步骤3:找零空间}

$N(\mathbf{A}) = \{\mathbf{x} \mid [1 \quad 1]\mathbf{x} = 0\} = \{\mathbf{x} \mid x_1 + x_2 = 0\}$

零空间的一组基:$\mathbf{v}_1 = (1, -1)^T$(因为 $1 + (-1) = 0$)

\textbf{步骤4:构造 $\mathbf{F}$}

\begin{equation}
\mathbf{F} = [1 \quad -1]^T = \begin{pmatrix} 1 \\ -1 \end{pmatrix}
\end{equation}

验证:$R(\mathbf{F}) = \{\alpha(1, -1)^T \mid \alpha \in \mathbb{R}\} = N(\mathbf{A})$ ✓

\textbf{步骤5:参数化}

所有解:$\mathbf{x} = \mathbf{x}_0 + \mathbf{F}z = \begin{pmatrix} 1 \\ 0 \end{pmatrix} + z\begin{pmatrix} 1 \\ -1 \end{pmatrix} = \begin{pmatrix} 1 + z \\ -z \end{pmatrix}$

验证:$x_1 + x_2 = (1 + z) + (-z) = 1$ ✓

\textbf{步骤6:消除约束}

原始问题:
\begin{align}
\text{minimize} \quad & f_0(x_1, x_2) \\
\text{subject to} \quad & x_1 + x_2 = 1
\end{align}

变换后(用 $z$ 代替 $(x_1, x_2)$):
\begin{align}
\text{minimize} \quad & f_0(1 + z, -z) = \tilde{f}_0(z) \\
\text{subject to} \quad & \text{(无等式约束)}
\end{align}

新问题只有1个变量 $z$,没有等式约束!

\subsection{例子2:3维情况}

\textbf{问题}:消除等式约束 $x_1 + x_2 + x_3 = 1$

\textbf{矩阵形式}:

\begin{equation}
[1 \quad 1 \quad 1] \begin{pmatrix} x_1 \\ x_2 \\ x_3 \end{pmatrix} = 1
\end{equation}

其中 $\mathbf{A} = [1 \quad 1 \quad 1]$,$\mathbf{b} = 1$。

\textbf{特解}:$\mathbf{x}_0 = (1, 0, 0)^T$

\textbf{零空间}:$N(\mathbf{A}) = \{\mathbf{x} \mid x_1 + x_2 + x_3 = 0\}$

零空间的基(需要2个线性无关的向量):
\begin{itemize}
\item $\mathbf{v}_1 = (1, -1, 0)^T$
\item $\mathbf{v}_2 = (1, 0, -1)^T$
\end{itemize}

\textbf{构造 $\mathbf{F}$}:

\begin{equation}
\mathbf{F} = \begin{pmatrix} 1 & 1 \\ -1 & 0 \\ 0 & -1 \end{pmatrix}
\end{equation}

验证:$R(\mathbf{F}) = \text{span}\{(1, -1, 0)^T, (1, 0, -1)^T\} = N(\mathbf{A})$ ✓

\textbf{参数化}:

\begin{equation}
\mathbf{x} = \mathbf{x}_0 + \mathbf{F}\mathbf{z} = \begin{pmatrix} 1 \\ 0 \\ 0 \end{pmatrix} + \begin{pmatrix} 1 & 1 \\ -1 & 0 \\ 0 & -1 \end{pmatrix} \begin{pmatrix} z_1 \\ z_2 \end{pmatrix} = \begin{pmatrix} 1 + z_1 + z_2 \\ -z_1 \\ -z_2 \end{pmatrix}
\end{equation}

验证:$x_1 + x_2 + x_3 = (1 + z_1 + z_2) + (-z_1) + (-z_2) = 1$ ✓

\textbf{消除约束后}:

原始问题:3个变量,1个等式约束

新问题:2个变量,0个等式约束

\section{各个参数的含义总结}

\subsection{参数列表}

\begin{table}[h]
\centering
\begin{tabular}{|l|l|l|}
\hline
\textbf{符号} & \textbf{名称} & \textbf{含义} \\
\hline
$\mathbf{A}$ & 系数矩阵 & $p \times n$ 矩阵,等式约束的系数 \\
\hline
$\mathbf{b}$ & 常数向量 & $p$ 维向量,等式约束的右端项 \\
\hline
$\mathbf{x}_0$ & 特解 & 满足 $\mathbf{A}\mathbf{x}_0 = \mathbf{b}$ 的一个解 \\
\hline
$N(\mathbf{A})$ & 零空间 & 所有使 $\mathbf{A}\mathbf{x} = \mathbf{0}$ 的向量 \\
\hline
$\mathbf{F}$ & 参数化矩阵 & $n \times k$ 矩阵,$R(\mathbf{F}) = N(\mathbf{A})$ \\
\hline
$R(\mathbf{F})$ & 列空间 & $\mathbf{F}$ 的列向量张成的子空间 \\
\hline
$\mathbf{z}$ & 新变量 & $k$ 维向量,$k = n - \text{rank}(\mathbf{A})$ \\
\hline
\end{tabular}
\caption{各个参数的含义}
\end{table}

\subsection{关键关系}

\begin{enumerate}
\item \textbf{$R(\mathbf{F}) = N(\mathbf{A})$}:
   \begin{itemize}
   \item $\mathbf{F}$ 的列空间等于 $\mathbf{A}$ 的零空间
   \item 这意味着 $\mathbf{F}$ 的列是 $N(\mathbf{A})$ 的一组基
   \end{itemize}

\item \textbf{维度关系}:
   \begin{itemize}
   \item $\dim(N(\mathbf{A})) = n - \text{rank}(\mathbf{A})$
   \item 如果 $\mathbf{F}$ 列满秩,则 $k = n - \text{rank}(\mathbf{A})$
   \end{itemize}

\item \textbf{参数化}:
   \begin{itemize}
   \item 所有解:$\mathbf{x} = \mathbf{x}_0 + \mathbf{F}\mathbf{z}$,$\mathbf{z} \in \mathbb{R}^k$
   \item 新变量数:$k = n - \text{rank}(\mathbf{A})$(减少了 $\text{rank}(\mathbf{A})$ 个变量)
   \end{itemize}
\end{enumerate}

\section{为什么这样消除约束?}

\subsection{优势}

\begin{enumerate}
\item \textbf{减少变量数}:
   \begin{itemize}
   \item 原始:$n$ 个变量
   \item 新问题:$k = n - \text{rank}(\mathbf{A})$ 个变量
   \item 减少了 $\text{rank}(\mathbf{A})$ 个变量
   \end{itemize}

\item \textbf{消除等式约束}:
   \begin{itemize}
   \item 原始:$p$ 个等式约束
   \item 新问题:0个等式约束
   \end{itemize}

\item \textbf{保持等价性}:
   \begin{itemize}
   \item 可行集一一对应
   \item 最优点一一对应
   \end{itemize}
\end{enumerate}

\subsection{几何直观}

\begin{itemize}
\item 等式约束 $\mathbf{A}\mathbf{x} = \mathbf{b}$ 定义了一个仿射集合($n - \text{rank}(\mathbf{A})$ 维)
\item 这个仿射集合可以表示为:$\mathbf{x}_0 + N(\mathbf{A})$
\item 通过参数化 $N(\mathbf{A})$,我们用 $k$ 个自由参数表示这个仿射集合
\item 新问题在这些自由参数上优化,不再有等式约束
\end{itemize}

\section{实际计算}

\subsection{如何构造 $\mathbf{F}$?}

\textbf{方法1:通过零空间基}

\begin{enumerate}
\item 求解 $\mathbf{A}\mathbf{x} = \mathbf{0}$,得到 $N(\mathbf{A})$ 的一组基
\item 将这些基向量作为 $\mathbf{F}$ 的列
\end{enumerate}

\textbf{方法2:通过QR分解}

\begin{enumerate}
\item 对 $\mathbf{A}^T$ 进行QR分解:$\mathbf{A}^T = \mathbf{Q}\mathbf{R}$
\item 取 $\mathbf{Q}$ 的后 $n - \text{rank}(\mathbf{A})$ 列作为 $\mathbf{F}$ 的列
\end{enumerate}

\textbf{方法3:通过SVD}

\begin{enumerate}
\item 对 $\mathbf{A}$ 进行SVD:$\mathbf{A} = \mathbf{U}\boldsymbol{\Sigma}\mathbf{V}^T$
\item 取 $\mathbf{V}$ 的后 $n - \text{rank}(\mathbf{A})$ 列作为 $\mathbf{F}$ 的列
\end{enumerate}

\section{总结}

\begin{enumerate}
\item \textbf{$R(\mathbf{F}) = N(\mathbf{A})$ 的含义}:
   \begin{itemize}
   \item $\mathbf{F}$ 的列空间等于 $\mathbf{A}$ 的零空间
   \item $\mathbf{F}$ 的列是 $N(\mathbf{A})$ 的一组基
   \end{itemize}

\item \textbf{参数化过程}:
   \begin{itemize}
   \item 找特解:$\mathbf{x}_0$ 满足 $\mathbf{A}\mathbf{x}_0 = \mathbf{b}$
   \item 找零空间基:构造 $\mathbf{F}$ 使得 $R(\mathbf{F}) = N(\mathbf{A})$
   \item 参数化:$\mathbf{x} = \mathbf{x}_0 + \mathbf{F}\mathbf{z}$
   \end{itemize}

\item \textbf{效果}:
   \begin{itemize}
   \item 变量数:从 $n$ 减少到 $k = n - \text{rank}(\mathbf{A})$
   \item 等式约束:从 $p$ 个减少到0个
   \item 问题等价:可行集和最优点一一对应
   \end{itemize}

\item \textbf{关键理解}:
   \begin{itemize}
   \item 等式约束将可行域限制在仿射集合中
   \item 通过参数化这个仿射集合,我们消除了等式约束
   \item $R(\mathbf{F}) = N(\mathbf{A})$ 确保参数化覆盖所有解
   \end{itemize}
\end{enumerate}

理解消除线性等式约束的方法,对于简化优化问题和理解优化理论都非常重要!

\end{document}


