\documentclass[12pt,a4paper]{article}
\usepackage[UTF8]{ctex}
\usepackage{amsmath}
\usepackage{amssymb}
\usepackage{amsthm}
\usepackage{geometry}
\geometry{left=2.5cm,right=2.5cm,top=2.5cm,bottom=2.5cm}

\title{第4.2.4节:消除等式约束详解}
\author{}
\date{\today}

\begin{document}

\maketitle

\section{引言}

第4.2.4节讨论了等价凸问题(Equivalent convex problems),其中最重要的内容之一是消除等式约束。对于凸优化问题,等式约束必须是线性的(仿射的),即 $\mathbf{A}\mathbf{x} = \mathbf{b}$。通过参数化可行集,我们可以将带等式约束的问题转化为无等式约束的问题,同时保持问题的凸性。

\section{消除等式约束的基本方法}

\subsection{原始问题}

考虑带等式约束的凸优化问题:
\begin{align}
\text{minimize} \quad & f_0(\mathbf{x}) \\
\text{subject to} \quad & f_i(\mathbf{x}) \leq 0, \quad i = 1, \ldots, m \\
& \mathbf{A}\mathbf{x} = \mathbf{b}
\end{align}

其中:
\begin{itemize}
\item $\mathbf{A} \in \mathbb{R}^{p \times n}$:等式约束的系数矩阵
\item $\mathbf{b} \in \mathbb{R}^p$:等式约束的常数向量
\item $f_0, f_1, \ldots, f_m$:凸函数
\end{itemize}

\subsection{关键观察}

\textbf{关键点}:对于凸优化问题,等式约束必须是线性的(仿射的)。

\textbf{原因}:如果等式约束不是仿射的,可行集通常不是凸集,这与凸优化问题的定义矛盾。

\subsection{消除等式约束的步骤}

\textbf{步骤1}:找特解 $\mathbf{x}_0$

找一个满足 $\mathbf{A}\mathbf{x}_0 = \mathbf{b}$ 的特解。

\textbf{步骤2}:构造参数化矩阵 $\mathbf{F}$

找一个矩阵 $\mathbf{F} \in \mathbb{R}^{n \times k}$,使得:
\begin{equation}
R(\mathbf{F}) = N(\mathbf{A})
\end{equation}

其中 $R(\mathbf{F})$ 是 $\mathbf{F}$ 的列空间,$N(\mathbf{A})$ 是 $\mathbf{A}$ 的零空间。

\textbf{步骤3}:参数化可行集}

所有满足 $\mathbf{A}\mathbf{x} = \mathbf{b}$ 的 $\mathbf{x}$ 可以表示为:
\begin{equation}
\mathbf{x} = \mathbf{F}\mathbf{z} + \mathbf{x}_0, \quad \mathbf{z} \in \mathbb{R}^k
\end{equation}

其中 $k = n - \text{rank}(\mathbf{A})$。

\textbf{步骤4}:变换问题}

将原始问题转化为:
\begin{align}
\text{minimize} \quad & \tilde{f}_0(\mathbf{z}) = f_0(\mathbf{F}\mathbf{z} + \mathbf{x}_0) \\
\text{subject to} \quad & \tilde{f}_i(\mathbf{z}) = f_i(\mathbf{F}\mathbf{z} + \mathbf{x}_0) \leq 0, \quad i = 1, \ldots, m
\end{align}

其中变量是 $\mathbf{z} \in \mathbb{R}^k$,\textbf{没有等式约束}!

\section{为什么这样消除约束?}

\subsection{可行性}

\textbf{验证}:对于任意 $\mathbf{z} \in \mathbb{R}^k$,有:
\begin{align}
\mathbf{A}(\mathbf{F}\mathbf{z} + \mathbf{x}_0) &= \mathbf{A}\mathbf{F}\mathbf{z} + \mathbf{A}\mathbf{x}_0 \\
&= \mathbf{0} + \mathbf{b} = \mathbf{b}
\end{align}

因为 $\mathbf{A}\mathbf{F} = \mathbf{0}$($\mathbf{F}$ 的列在零空间中)且 $\mathbf{A}\mathbf{x}_0 = \mathbf{b}$。

\textbf{结论}:所有 $\mathbf{F}\mathbf{z} + \mathbf{x}_0$ 都满足等式约束。

\subsection{完备性}

\textbf{验证}:对于任意满足 $\mathbf{A}\mathbf{x} = \mathbf{b}$ 的 $\mathbf{x}$,有:
\begin{align}
\mathbf{A}(\mathbf{x} - \mathbf{x}_0) &= \mathbf{A}\mathbf{x} - \mathbf{A}\mathbf{x}_0 \\
&= \mathbf{b} - \mathbf{b} = \mathbf{0}
\end{align}

因此 $\mathbf{x} - \mathbf{x}_0 \in N(\mathbf{A}) = R(\mathbf{F})$,存在 $\mathbf{z}$ 使得:
\begin{equation}
\mathbf{x} - \mathbf{x}_0 = \mathbf{F}\mathbf{z} \quad \Rightarrow \quad \mathbf{x} = \mathbf{F}\mathbf{z} + \mathbf{x}_0
\end{equation}

\textbf{结论}:所有可行解都可以用 $\mathbf{F}\mathbf{z} + \mathbf{x}_0$ 表示。

\section{凸性的保持}

\subsection{关键定理}

\textbf{定理}:如果 $f$ 是凸函数,$\mathbf{F}\mathbf{z} + \mathbf{x}_0$ 是仿射函数(关于 $\mathbf{z}$),则复合函数 $\tilde{f}(\mathbf{z}) = f(\mathbf{F}\mathbf{z} + \mathbf{x}_0)$ 也是凸函数。

\textbf{证明}:

对于任意 $\mathbf{z}_1, \mathbf{z}_2$ 和 $\theta \in [0, 1]$:
\begin{align}
\tilde{f}(\theta \mathbf{z}_1 + (1-\theta)\mathbf{z}_2) &= f(\mathbf{F}(\theta \mathbf{z}_1 + (1-\theta)\mathbf{z}_2) + \mathbf{x}_0) \\
&= f(\theta(\mathbf{F}\mathbf{z}_1 + \mathbf{x}_0) + (1-\theta)(\mathbf{F}\mathbf{z}_2 + \mathbf{x}_0)) \\
&\leq \theta f(\mathbf{F}\mathbf{z}_1 + \mathbf{x}_0) + (1-\theta) f(\mathbf{F}\mathbf{z}_2 + \mathbf{x}_0) \\
&= \theta \tilde{f}(\mathbf{z}_1) + (1-\theta) \tilde{f}(\mathbf{z}_2)
\end{align}

其中第二个等号使用了仿射函数的线性性质,第三个等号使用了 $f$ 的凸性。

\textbf{结论}:消除等式约束后,新问题仍然是凸优化问题!

\section{具体例子}

\subsection{例子1:简单约束}

\textbf{原始问题}:
\begin{align}
\text{minimize} \quad & x_1^2 + x_2^2 \\
\text{subject to} \quad & x_1 + x_2 = 1
\end{align}

\textbf{步骤1}:找特解

$\mathbf{A} = [1 \quad 1]$,$\mathbf{b} = 1$

特解:$\mathbf{x}_0 = (1, 0)^T$(满足 $1 + 0 = 1$)

\textbf{步骤2}:构造 $\mathbf{F}$

零空间:$N(\mathbf{A}) = \{\mathbf{x} \mid x_1 + x_2 = 0\} = \text{span}\{(1, -1)^T\}$

因此:$\mathbf{F} = \begin{pmatrix} 1 \\ -1 \end{pmatrix}$

\textbf{步骤3}:参数化

$\mathbf{x} = \mathbf{F}z + \mathbf{x}_0 = \begin{pmatrix} 1 \\ -1 \end{pmatrix} z + \begin{pmatrix} 1 \\ 0 \end{pmatrix} = \begin{pmatrix} 1 + z \\ -z \end{pmatrix}$

\textbf{步骤4}:变换问题

\begin{align}
\tilde{f}_0(z) &= f_0(1 + z, -z) \\
&= (1 + z)^2 + (-z)^2 \\
&= 1 + 2z + z^2 + z^2 \\
&= 1 + 2z + 2z^2
\end{align}

\textbf{新问题}:
\begin{align}
\text{minimize} \quad & 1 + 2z + 2z^2 \\
\text{subject to} \quad & \text{(无约束)}
\end{align}

\textbf{求解}:$\tilde{f}_0'(z) = 2 + 4z = 0$,因此 $z^* = -1/2$

\textbf{原始问题的最优解}:
\begin{equation}
\mathbf{x}^* = \begin{pmatrix} 1 + (-1/2) \\ -(-1/2) \end{pmatrix} = \begin{pmatrix} 1/2 \\ 1/2 \end{pmatrix}
\end{equation}

\textbf{验证}:$x_1^* + x_2^* = 1/2 + 1/2 = 1$ ✓

\subsection{例子2:多个等式约束}

\textbf{原始问题}:
\begin{align}
\text{minimize} \quad & x_1^2 + x_2^2 + x_3^2 \\
\text{subject to} \quad & x_1 + x_2 + x_3 = 1 \\
& x_1 - x_2 = 0
\end{align}

\textbf{矩阵形式}:
\begin{equation}
\mathbf{A} = \begin{pmatrix} 1 & 1 & 1 \\ 1 & -1 & 0 \end{pmatrix}, \quad \mathbf{b} = \begin{pmatrix} 1 \\ 0 \end{pmatrix}
\end{equation}

\textbf{特解}:$\mathbf{x}_0 = (0, 0, 1)^T$(满足两个约束)

\textbf{零空间}:$N(\mathbf{A}) = \{\mathbf{x} \mid x_1 + x_2 + x_3 = 0, x_1 - x_2 = 0\}$

从 $x_1 - x_2 = 0$ 得 $x_1 = x_2$,代入第一个方程:$2x_1 + x_3 = 0$,即 $x_3 = -2x_1$

因此:$N(\mathbf{A}) = \text{span}\{(1, 1, -2)^T\}$

\textbf{构造 $\mathbf{F}$}:$\mathbf{F} = \begin{pmatrix} 1 \\ 1 \\ -2 \end{pmatrix}$

\textbf{参数化}:$\mathbf{x} = \mathbf{F}z + \mathbf{x}_0 = \begin{pmatrix} z \\ z \\ 1 - 2z \end{pmatrix}$

\textbf{新问题}:
\begin{align}
\text{minimize} \quad & z^2 + z^2 + (1 - 2z)^2 = 2z^2 + 1 - 4z + 4z^2 = 6z^2 - 4z + 1 \\
\text{subject to} \quad & \text{(无约束)}
\end{align}

\textbf{求解}:$\tilde{f}_0'(z) = 12z - 4 = 0$,因此 $z^* = 1/3$

\textbf{原始问题的最优解}:
\begin{equation}
\mathbf{x}^* = \begin{pmatrix} 1/3 \\ 1/3 \\ 1 - 2/3 \end{pmatrix} = \begin{pmatrix} 1/3 \\ 1/3 \\ 1/3 \end{pmatrix}
\end{equation}

\section{优势与局限性}

\subsection{优势}

\begin{enumerate}
\item \textbf{减少变量数}:
\begin{itemize}
\item 原始:$n$ 个变量
\item 新问题:$k = n - \text{rank}(\mathbf{A})$ 个变量
\item 减少了 $\text{rank}(\mathbf{A})$ 个变量
\end{itemize}

\item \textbf{消除等式约束}:
\begin{itemize}
\item 原始:$p$ 个等式约束
\item 新问题:0个等式约束
\end{itemize}

\item \textbf{保持凸性}:
\begin{itemize}
\item 如果原始问题是凸的,新问题也是凸的
\item 这是因为凸函数与仿射函数的复合仍然是凸函数
\end{itemize}

\item \textbf{等价性}:
\begin{itemize}
\item 可行集一一对应
\item 最优点一一对应
\end{itemize}
\end{enumerate}

\subsection{局限性}

\textbf{注意}:虽然理论上可以消除等式约束,但在实际中,有时保留等式约束更好:

\begin{enumerate}
\item \textbf{问题理解}:保留等式约束可能使问题更容易理解和分析

\item \textbf{算法效率}:
\begin{itemize}
\item 消除等式约束可能破坏问题的稀疏性
\item 可能破坏问题的其他有用结构
\item 某些算法(如内点法)可以高效处理等式约束
\end{itemize}

\item \textbf{数值稳定性}:
\begin{itemize}
\item 消除等式约束可能引入数值误差
\item 保留等式约束可能更稳定
\end{itemize}

\item \textbf{大规模问题}:
\begin{itemize}
\item 当变量维度很大时,消除等式约束可能不实际
\item 构造 $\mathbf{F}$ 和计算 $\mathbf{F}\mathbf{z} + \mathbf{x}_0$ 可能很昂贵
\end{itemize}
\end{enumerate}

\section{如何构造 $\mathbf{F}$?}

\subsection{方法1:通过零空间基}

\textbf{步骤}:
\begin{enumerate}
\item 求解 $\mathbf{A}\mathbf{x} = \mathbf{0}$,得到 $N(\mathbf{A})$ 的一组基
\item 将这些基向量作为 $\mathbf{F}$ 的列
\end{enumerate}

\textbf{例子}:如果 $N(\mathbf{A}) = \text{span}\{\mathbf{v}_1, \mathbf{v}_2\}$,则:
\begin{equation}
\mathbf{F} = \begin{pmatrix} \mathbf{v}_1 & \mathbf{v}_2 \end{pmatrix}
\end{equation}

\subsection{方法2:通过QR分解}

\textbf{步骤}:
\begin{enumerate}
\item 对 $\mathbf{A}^T$ 进行QR分解:$\mathbf{A}^T = \mathbf{Q}\mathbf{R}$
\item 取 $\mathbf{Q}$ 的后 $n - \text{rank}(\mathbf{A})$ 列作为 $\mathbf{F}$ 的列
\end{enumerate}

\textbf{原因}:$\mathbf{Q}$ 的列是正交的,后 $n - \text{rank}(\mathbf{A})$ 列张成零空间。

\subsection{方法3:通过SVD}

\textbf{步骤}:
\begin{enumerate}
\item 对 $\mathbf{A}$ 进行SVD:$\mathbf{A} = \mathbf{U}\boldsymbol{\Sigma}\mathbf{V}^T$
\item 取 $\mathbf{V}$ 的后 $n - \text{rank}(\mathbf{A})$ 列作为 $\mathbf{F}$ 的列
\end{enumerate}

\textbf{原因}:$\mathbf{V}$ 的列是正交的,后 $n - \text{rank}(\mathbf{A})$ 列对应零奇异值,张成零空间。

\section{从新问题恢复原始问题的解}

\subsection{恢复过程}

\textbf{步骤}:
\begin{enumerate}
\item 求解新问题,得到最优解 $\mathbf{z}^*$
\item 计算原始问题的最优解:$\mathbf{x}^* = \mathbf{F}\mathbf{z}^* + \mathbf{x}_0$
\end{enumerate}

\textbf{验证}:
\begin{itemize}
\item $\mathbf{A}\mathbf{x}^* = \mathbf{A}(\mathbf{F}\mathbf{z}^* + \mathbf{x}_0) = \mathbf{A}\mathbf{F}\mathbf{z}^* + \mathbf{A}\mathbf{x}_0 = \mathbf{0} + \mathbf{b} = \mathbf{b}$ ✓
\item 等式约束满足
\end{itemize}

\section{总结}

\subsection{关键要点}

\begin{enumerate}
\item \textbf{消除等式约束的方法}:
\begin{itemize}
\item 找特解:$\mathbf{x}_0$ 满足 $\mathbf{A}\mathbf{x}_0 = \mathbf{b}$
\item 构造参数化矩阵:$\mathbf{F}$ 使得 $R(\mathbf{F}) = N(\mathbf{A})$
\item 参数化:$\mathbf{x} = \mathbf{F}\mathbf{z} + \mathbf{x}_0$
\item 变换问题:用 $\mathbf{z}$ 代替 $\mathbf{x}$
\end{itemize}

\item \textbf{凸性的保持}:
\begin{itemize}
\item 凸函数与仿射函数的复合仍然是凸函数
\item 消除等式约束后,新问题仍然是凸优化问题
\end{itemize}

\item \textbf{优势}:
\begin{itemize}
\item 减少变量数:从 $n$ 减少到 $k = n - \text{rank}(\mathbf{A})$
\item 消除等式约束:从 $p$ 个减少到0个
\item 保持等价性:可行集和最优点一一对应
\end{itemize}

\item \textbf{注意事项}:
\begin{itemize}
\item 有时保留等式约束更好(理解、效率、稳定性)
\item 对于大规模问题,消除等式约束可能不实际
\end{itemize}
\end{enumerate}

\subsection{关键公式}

\begin{enumerate}
\item \textbf{参数化}:$\mathbf{x} = \mathbf{F}\mathbf{z} + \mathbf{x}_0$

\item \textbf{条件}:$R(\mathbf{F}) = N(\mathbf{A})$

\item \textbf{维度}:$k = n - \text{rank}(\mathbf{A})$

\item \textbf{变换后的目标函数}:$\tilde{f}_0(\mathbf{z}) = f_0(\mathbf{F}\mathbf{z} + \mathbf{x}_0)$
\end{enumerate}

理解消除等式约束的方法,对于简化优化问题和理解优化理论都非常重要!

\end{document}

