\documentclass[12pt,a4paper]{article}
\usepackage[UTF8]{ctex}
\usepackage{amsmath}
\usepackage{amssymb}
\usepackage{amsthm}
\usepackage{geometry}
\geometry{left=2.5cm,right=2.5cm,top=2.5cm,bottom=2.5cm}

\title{为什么凸优化问题中等式约束必须是仿射的?}
\subtitle{基于《Convex Optimization》第4.2节}
\author{}
\date{\today}

\begin{document}

\maketitle

\section{问题提出}

在凸优化问题的标准形式中:

\begin{align}
\begin{array}{ll}
\text{minimize} & f_0(\mathbf{x}) \\
\text{subject to} & f_i(\mathbf{x}) \leq 0, \quad i = 1, \ldots, m \\
& \mathbf{a}_i^T \mathbf{x} = b_i, \quad i = 1, \ldots, p
\end{array}
\end{align}

要求:
\begin{itemize}
\item 目标函数 $f_0$ 是凸函数
\item 不等式约束函数 $f_i$ 是凸函数
\item \textbf{等式约束函数必须是仿射的}:$h_i(\mathbf{x}) = \mathbf{a}_i^T \mathbf{x} - b_i$
\end{itemize}

\textbf{问题}:为什么等式约束必须是仿射的?为什么不能是凸函数?

\section{核心原因:可行集必须是凸集}

\subsection{凸优化问题的关键要求}

\textbf{凸优化问题的定义}:在凸集上最小化凸函数。

因此,\textbf{可行集必须是凸集}!

\subsection{可行集的构成}

可行集是以下集合的交集:
\begin{itemize}
\item 定义域:$\mathcal{D} = \bigcap_{i=0}^m \text{dom } f_i$(凸集)
\item 不等式约束:$\{\mathbf{x} \mid f_i(\mathbf{x}) \leq 0\}$(凸函数的子水平集,是凸集)
\item 等式约束:$\{\mathbf{x} \mid h_i(\mathbf{x}) = 0\}$(必须是凸集)
\end{itemize}

由于凸集的交集是凸集,所以可行集是凸集。

\section{为什么非仿射等式约束会导致非凸可行集?}

\subsection{关键观察}

\textbf{定理}:如果 $h$ 是凸函数,那么集合 $\{\mathbf{x} \mid h(\mathbf{x}) = 0\}$ 通常是\textbf{非凸}的(除非 $h$ 是仿射函数)。

\subsection{证明思路}

\textbf{情况1:$h$ 是严格凸函数}

如果 $h$ 是严格凸的,且 $h(\mathbf{x}_1) = 0$,$h(\mathbf{x}_2) = 0$,那么对于 $\theta \in (0, 1)$:

\begin{equation}
h(\theta \mathbf{x}_1 + (1-\theta) \mathbf{x}_2) < \theta h(\mathbf{x}_1) + (1-\theta) h(\mathbf{x}_2) = 0
\end{equation}

因此 $\theta \mathbf{x}_1 + (1-\theta) \mathbf{x}_2$ 不在集合 $\{\mathbf{x} \mid h(\mathbf{x}) = 0\}$ 中,所以这个集合不是凸集。

\textbf{情况2:$h$ 是凸函数(非严格)}

即使 $h$ 不是严格凸的,如果它不是仿射的,集合 $\{\mathbf{x} \mid h(\mathbf{x}) = 0\}$ 通常也不是凸集。

\textbf{情况3:$h$ 是仿射函数}

如果 $h(\mathbf{x}) = \mathbf{a}^T \mathbf{x} - b$ 是仿射的,那么:
\begin{equation}
h(\theta \mathbf{x}_1 + (1-\theta) \mathbf{x}_2) = \theta h(\mathbf{x}_1) + (1-\theta) h(\mathbf{x}_2) = 0
\end{equation}

因此集合 $\{\mathbf{x} \mid h(\mathbf{x}) = 0\}$ 是凸集(实际上是仿射集合)。

\section{具体反例}

\subsection{反例1:二次等式约束}

\textbf{问题}:
\begin{align}
\text{minimize} \quad & x_1^2 + x_2^2 \\
\text{subject to} \quad & x_1^2 + x_2^2 = 1
\end{align}

\textbf{分析}:
\begin{itemize}
\item 等式约束 $h(x_1, x_2) = x_1^2 + x_2^2 - 1 = 0$ 是凸函数(因为 $x_1^2 + x_2^2$ 是凸的)
\item 可行集:$\{(x_1, x_2) \mid x_1^2 + x_2^2 = 1\}$(单位圆周)
\item 单位圆周\textbf{不是凸集}!
\end{itemize}

\textbf{验证非凸性}:

取两点:$(1, 0)$ 和 $(-1, 0)$,都在可行集上。

它们的凸组合:$\theta(1, 0) + (1-\theta)(-1, 0) = (2\theta - 1, 0)$

当 $\theta = 0.5$ 时,得到 $(0, 0)$,但 $(0, 0)$ 不在单位圆周上(因为 $0^2 + 0^2 = 0 \neq 1$)。

因此可行集不是凸集,这不是凸优化问题!

\subsection{反例2:指数等式约束}

\textbf{问题}:
\begin{align}
\text{minimize} \quad & x \\
\text{subject to} \quad & e^x = 1
\end{align}

\textbf{分析}:
\begin{itemize}
\item 等式约束 $h(x) = e^x - 1 = 0$ 是凸函数(因为 $e^x$ 是凸的)
\item 可行集:$\{x \mid e^x = 1\} = \{0\}$(单点集)
\item 单点集是凸集,但这个例子说明了一般情况
\end{itemize}

\textbf{更一般的例子}:

如果 $h(x) = e^x + e^{-x} - 2 = 0$,则可行集是 $\{0\}$(单点集,是凸集)。

但如果 $h(x) = e^x - 2 = 0$,则可行集是 $\{\ln 2\}$(单点集,是凸集)。

\textbf{关键}:虽然这些例子中可行集是凸集(因为是单点集),但如果我们考虑更复杂的凸函数,通常可行集不是凸集。

\subsection{反例3:更复杂的例子}

\textbf{问题}:
\begin{align}
\text{minimize} \quad & x_1 + x_2 \\
\text{subject to} \quad & x_1^2 + x_2^2 = 4 \\
& x_1 \geq 0, x_2 \geq 0
\end{align}

\textbf{分析}:
\begin{itemize}
\item 等式约束 $h(x_1, x_2) = x_1^2 + x_2^2 - 4 = 0$ 是凸函数
\item 可行集:第一象限中的四分之一圆周
\item 这个集合\textbf{不是凸集}(因为圆周不是凸的)
\end{itemize}

\section{为什么仿射等式约束可以?}

\subsection{仿射函数的性质}

\textbf{定理}:如果 $h$ 是仿射函数,即 $h(\mathbf{x}) = \mathbf{a}^T \mathbf{x} - b$,那么集合 $\{\mathbf{x} \mid h(\mathbf{x}) = 0\}$ 是仿射集合(因此是凸集)。

\textbf{证明}:

设 $\mathbf{x}_1, \mathbf{x}_2$ 满足 $h(\mathbf{x}_1) = 0$,$h(\mathbf{x}_2) = 0$,即:
\begin{align}
\mathbf{a}^T \mathbf{x}_1 - b &= 0 \\
\mathbf{a}^T \mathbf{x}_2 - b &= 0
\end{align}

对于 $\theta \in \mathbb{R}$(注意:这里是任意实数,不只是 $[0, 1]$),考虑仿射组合:
\begin{align}
h(\theta \mathbf{x}_1 + (1-\theta) \mathbf{x}_2) &= \mathbf{a}^T (\theta \mathbf{x}_1 + (1-\theta) \mathbf{x}_2) - b \\
&= \theta \mathbf{a}^T \mathbf{x}_1 + (1-\theta) \mathbf{a}^T \mathbf{x}_2 - b \\
&= \theta b + (1-\theta) b - b \\
&= 0
\end{align}

因此 $\theta \mathbf{x}_1 + (1-\theta) \mathbf{x}_2$ 也在集合中,所以集合是仿射集合(因此是凸集)。$\square$

\subsection{几何直观}

\begin{itemize}
\item 仿射等式约束:$\mathbf{a}^T \mathbf{x} = b$ 定义了一个\textbf{超平面}
\item 超平面是仿射集合,因此是凸集
\item 多个超平面的交集仍然是仿射集合(因此是凸集)
\end{itemize}

\textbf{例子}:
\begin{itemize}
\item 在 $\mathbb{R}^2$ 中:$x_1 + x_2 = 1$ 定义了一条直线(凸集)
\item 在 $\mathbb{R}^3$ 中:$x_1 + x_2 + x_3 = 1$ 定义了一个平面(凸集)
\end{itemize}

\section{为什么不能放宽要求?}

\subsection{如果允许非仿射等式约束}

如果允许等式约束是凸函数(非仿射),那么:

\begin{itemize}
\item 可行集可能不是凸集
\item 问题不再是凸优化问题
\item 失去凸优化的所有优势(全局最优、对偶理论等)
\end{itemize}

\subsection{凸优化的优势依赖于可行集的凸性}

凸优化的关键优势:
\begin{enumerate}
\item \textbf{局部最优 = 全局最优}
\item \textbf{对偶理论}:强对偶性
\item \textbf{算法保证}:可以找到全局最优解
\end{enumerate}

这些优势都依赖于可行集是凸集。如果可行集不是凸集,这些优势就失去了。

\section{特殊情况:单点集}

\subsection{观察}

如果等式约束 $h(\mathbf{x}) = 0$ 的可行集是\textbf{单点集}(只有一个点),那么:

\begin{itemize}
\item 单点集是凸集
\item 但这种情况很特殊,通常不实用
\item 而且,如果可行集是单点集,优化问题就退化了(没有选择)
\end{itemize}

\subsection{为什么仍然要求仿射?}

即使在某些特殊情况下,非仿射等式约束可能产生凸可行集(如单点集),我们仍然要求等式约束是仿射的,因为:

\begin{enumerate}
\item \textbf{一般性}:非仿射等式约束通常产生非凸可行集
\item \textbf{理论一致性}:保持凸优化理论的一致性
\item \textbf{算法适用性}:凸优化算法假设可行集是凸集
\end{enumerate}

\section{实际应用}

\subsection{如何处理非仿射等式约束?}

如果遇到非仿射等式约束,通常需要:

\begin{enumerate}
\item \textbf{重新建模}:看能否转化为仿射约束
\item \textbf{松弛}:将等式约束松弛为不等式约束
\item \textbf{使用非凸优化}:如果必须保留,则使用非凸优化方法
\end{enumerate}

\subsection{例子:重新建模}

\textbf{原问题}(非凸):
\begin{align}
\text{minimize} \quad & f_0(\mathbf{x}) \\
\text{subject to} \quad & x_1^2 + x_2^2 = 1
\end{align}

\textbf{重新建模}(可能的方法):
\begin{itemize}
\item 使用参数化:$x_1 = \cos \theta$,$x_2 = \sin \theta$
\item 或者松弛为:$x_1^2 + x_2^2 \leq 1$(如果问题允许)
\end{itemize}

\section{总结}

\begin{enumerate}
\item \textbf{核心原因}:
   \begin{itemize}
   \item 凸优化要求可行集是凸集
   \item 非仿射等式约束通常产生非凸可行集
   \item 仿射等式约束产生仿射集合(凸集)
   \end{itemize}

\item \textbf{数学证明}:
   \begin{itemize}
   \item 如果 $h$ 是严格凸函数,$\{h = 0\}$ 不是凸集
   \item 如果 $h$ 是仿射函数,$\{h = 0\}$ 是仿射集合(凸集)
   \end{itemize}

\item \textbf{实际意义}:
   \begin{itemize}
   \item 保持凸优化的理论优势
   \item 确保算法可以应用
   \item 保证全局最优性
   \end{itemize}

\item \textbf{关键理解}:
   \begin{itemize}
   \item 凸优化 = 在凸集上最小化凸函数
   \item 可行集必须是凸集
   \item 等式约束必须是仿射的,才能保证可行集是凸集
   \end{itemize}
\end{enumerate}

理解为什么等式约束必须是仿射的,对于正确理解凸优化的定义和理论至关重要!

\end{document}


