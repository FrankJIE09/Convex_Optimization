\documentclass[12pt,a4paper]{article}
\usepackage[UTF8]{ctex}
\usepackage{amsmath,amssymb,amsthm}
\usepackage{geometry}
\usepackage{graphicx}
\usepackage{hyperref}
\usepackage{tikz}
\usepackage{pgfplots}

\geometry{left=2.5cm,right=2.5cm,top=3cm,bottom=3cm}

\title{线性规划(Linear Programming)详解}
\author{}
\date{\today}

\theoremstyle{definition}
\newtheorem{definition}{定义}[section]
\newtheorem{theorem}{定理}[section]
\newtheorem{example}{例子}[section]

\begin{document}

\maketitle

\tableofcontents
\newpage

\section{引言}

线性规划(Linear Programming, LP)是数学优化中最基础、最重要的分支之一。它研究在满足一组线性约束条件下,优化一个线性目标函数的问题。线性规划在运筹学、经济学、工程管理等领域有着极其广泛的应用。

\section{线性规划的标准形式}

\subsection{基本定义}

\begin{definition}[线性规划]
线性规划问题是指求解以下形式的优化问题:
\begin{align}
\min_{x \in \mathbb{R}^n} \quad & c^T x \label{eq:lp_objective}\\
\text{s.t.} \quad & A x = b \label{eq:lp_eq_constraint}\\
& x \geq 0 \label{eq:lp_nonneg}
\end{align}
其中:
\begin{itemize}
\item $x \in \mathbb{R}^n$ 是决策变量
\item $c \in \mathbb{R}^n$ 是目标函数系数向量
\item $A \in \mathbb{R}^{m \times n}$ 是约束矩阵
\item $b \in \mathbb{R}^m$ 是约束右端向量
\end{itemize}
\end{definition}

\subsection{其他形式}

线性规划还有其他等价形式:

\subsubsection{不等式形式}
\begin{align}
\min_{x} \quad & c^T x\\
\text{s.t.} \quad & A x \leq b\\
& x \geq 0
\end{align}

\subsubsection{最大化问题}
\begin{align}
\max_{x} \quad & c^T x\\
\text{s.t.} \quad & A x = b\\
& x \geq 0
\end{align}

注意:$\max c^T x = -\min (-c^T x)$,因此最大化问题可以转化为最小化问题。

\section{可行域与最优解}

\subsection{可行域}

\begin{definition}[可行域]
线性规划的可行域$\mathcal{F}$定义为所有满足约束条件的点的集合:
\begin{equation}
\mathcal{F} = \{x \in \mathbb{R}^n \mid Ax = b, x \geq 0\}
\end{equation}
\end{definition}

可行域是一个多面体(polyhedron),由有限个半空间的交集构成。

\subsection{顶点与基本可行解}

\subsubsection{基本解的概念}

在标准形式的线性规划问题中,我们有$m$个等式约束和$n$个变量。通常$n > m$,因此系统$Ax = b$有无穷多个解。

\textbf{重要说明:矩阵$A$与变量$x$的关系}
\begin{itemize}
\item \textbf{$A$是固定的约束矩阵}:矩阵$A$是问题的输入参数,在问题定义时就确定了,\textbf{不会因为$x$的变化而改变}
\item \textbf{$A$描述约束关系}:$A$定义了约束条件$Ax = b$的系数,描述了变量$x$如何线性组合以满足约束
\item \textbf{$x$是决策变量}:$x$是我们要优化的变量,它的值在求解过程中会变化
\item \textbf{约束是$Ax = b$}:在这个等式约束中,$A$和$b$都是\textbf{固定的},只有$x$是\textbf{变量}
\end{itemize}

\textbf{具体例子}:
在约束$2x_1 + 3x_2 = 6$中:
\begin{itemize}
\item $A = [2, 3]$是\textbf{固定的系数},不会改变
\item $b = 6$是\textbf{固定的右端项},不会改变
\item $x = [x_1, x_2]^T$是\textbf{变量},可以取不同的值
\item 当$x_1 = 1, x_2 = 4/3$时,$Ax = 2 \times 1 + 3 \times 4/3 = 6 = b$,满足约束
\item 当$x_1 = 2, x_2 = 1$时,$Ax = 2 \times 2 + 3 \times 1 = 7 \neq b$,不满足约束
\item 注意:无论$x$取什么值,$A = [2, 3]$始终不变
\end{itemize}

\begin{definition}[基本解]
对于线性规划问题$\min\{c^T x \mid Ax = b, x \geq 0\}$,设$A$的秩为$m$(即约束线性无关)。从$A$的$n$列中选择$m$个线性无关的列,构成基矩阵$B$,对应的变量称为基变量(basic variables),其余$n-m$个变量称为非基变量(non-basic variables)。

基本解是通过将非基变量设为0,然后求解$Bx_B = b$得到的解。

\textbf{关键理解}:
\begin{itemize}
\item 矩阵$A$是\textbf{固定的},是问题的输入,不会改变
\item 我们只是从$A$的$n$列中\textbf{选择}哪些列作为基,但$A$本身始终不变
\item 基矩阵$B$是$A$的一个子矩阵(由选定的列组成),但$A$仍然是原来的矩阵
\item 不同的基选择对应不同的基本解,但$A$始终是同一个矩阵
\item $x$的值会变化,但$A$不会因为$x$的变化而改变
\end{itemize}
\end{definition}

\subsubsection{基本可行解}

\begin{definition}[基本可行解]
如果基本解$x$满足非负约束$x \geq 0$,则称为基本可行解(Basic Feasible Solution, BFS)。
\end{definition}

基本可行解对应可行域的顶点(extreme point)。

\subsubsection{基本可行解的性质}

\begin{itemize}
\item \textbf{唯一性}:每个基本可行解对应一个唯一的基
\item \textbf{有限性}:基本可行解的数量是有限的,最多为$\binom{n}{m} = \frac{n!}{m!(n-m)!}$
\item \textbf{最优性}:如果线性规划有最优解,则至少有一个最优解是基本可行解
\end{itemize}

\begin{theorem}[线性规划基本定理]
如果线性规划有最优解,则至少有一个最优解是基本可行解(顶点)。
\end{theorem}

这个定理是单纯形法的理论基础:我们只需要在有限个基本可行解中搜索最优解,而不需要检查整个可行域。

\subsubsection{具体数值例子}

\begin{example}[基本可行解的构造]
考虑以下线性规划问题:
\begin{align}
\min_{x_1, x_2, x_3, x_4} \quad & -2x_1 - 3x_2\\
\text{s.t.} \quad & x_1 + x_2 + x_3 = 4\\
& x_1 + 2x_2 + x_4 = 6\\
& x_1, x_2, x_3, x_4 \geq 0
\end{align}

这里$n = 4$(4个变量),$m = 2$(2个等式约束)。

\textbf{步骤1:识别基矩阵}。

矩阵$A = \begin{bmatrix} 1 & 1 & 1 & 0\\ 1 & 2 & 0 & 1 \end{bmatrix}$是\textbf{固定的},是问题的输入参数,\textbf{不会因为$x$的变化而改变}。$A$的每一列对应一个变量:
\begin{itemize}
\item 第1列对应$x_1$:$\begin{bmatrix} 1\\ 1 \end{bmatrix}$(固定,不受$x_1$影响)
\item 第2列对应$x_2$:$\begin{bmatrix} 1\\ 2 \end{bmatrix}$(固定,不受$x_2$影响)
\item 第3列对应$x_3$:$\begin{bmatrix} 1\\ 0 \end{bmatrix}$(固定,不受$x_3$影响)
\item 第4列对应$x_4$:$\begin{bmatrix} 0\\ 1 \end{bmatrix}$(固定,不受$x_4$影响)
\end{itemize}

\textbf{关键理解}:
\begin{itemize}
\item $A$是约束矩阵,描述了约束关系,是问题的\textbf{输入参数},在问题定义时就确定了
\item 无论$x$取什么值,$A$始终是$\begin{bmatrix} 1 & 1 & 1 & 0\\ 1 & 2 & 0 & 1 \end{bmatrix}$
\item 约束$Ax = b$中,$A$和$b$都是固定的,只有$x$是变量
\item 例如:当$x = (2, 2, 0, 0)$时,$Ax = \begin{bmatrix} 1 & 1 & 1 & 0\\ 1 & 2 & 0 & 1 \end{bmatrix} \begin{bmatrix} 2\\ 2\\ 0\\ 0 \end{bmatrix} = \begin{bmatrix} 4\\ 6 \end{bmatrix} = b$
\item 当$x = (4, 0, 0, 2)$时,$Ax = \begin{bmatrix} 1 & 1 & 1 & 0\\ 1 & 2 & 0 & 1 \end{bmatrix} \begin{bmatrix} 4\\ 0\\ 0\\ 2 \end{bmatrix} = \begin{bmatrix} 4\\ 6 \end{bmatrix} = b$
\item 注意:无论$x$是什么值,$A$始终不变!
\end{itemize}

从$A$的4列中选择2个线性无关的列作为基。注意:我们是在\textbf{选择}使用$A$的哪些列,但$A$本身始终不变。无论我们选择哪两列作为基,$A$仍然是$\begin{bmatrix} 1 & 1 & 1 & 0\\ 1 & 2 & 0 & 1 \end{bmatrix}$。

\textbf{情况1:选择列1和列2作为基}
\begin{itemize}
\item 基矩阵:$B = \begin{bmatrix} 1 & 1\\ 1 & 2 \end{bmatrix}$
\item 基变量:$x_1, x_2$
\item 非基变量:$x_3 = 0, x_4 = 0$
\item 求解:$\begin{bmatrix} 1 & 1\\ 1 & 2 \end{bmatrix} \begin{bmatrix} x_1\\ x_2 \end{bmatrix} = \begin{bmatrix} 4\\ 6 \end{bmatrix}$
\item 解:$x_1 = 2, x_2 = 2, x_3 = 0, x_4 = 0$
\item 检查可行性:$x_1 = 2 \geq 0$ ✓,$x_2 = 2 \geq 0$ ✓
\item \textbf{这是基本可行解}:$(2, 2, 0, 0)$
\end{itemize}

\textbf{情况2:选择列1和列3作为基}
\begin{itemize}
\item 基矩阵:$B = \begin{bmatrix} 1 & 1\\ 1 & 0 \end{bmatrix}$
\item 基变量:$x_1, x_3$
\item 非基变量:$x_2 = 0, x_4 = 0$
\item 求解:$\begin{bmatrix} 1 & 1\\ 1 & 0 \end{bmatrix} \begin{bmatrix} x_1\\ x_3 \end{bmatrix} = \begin{bmatrix} 4\\ 6 \end{bmatrix}$
\item 解:$x_1 = 6, x_3 = -2, x_2 = 0, x_4 = 0$
\item 检查可行性:$x_3 = -2 < 0$ ✗
\item \textbf{这不是基本可行解}(违反非负约束)
\end{itemize}

\textbf{情况3:选择列1和列4作为基}
\begin{itemize}
\item 基矩阵:$B = \begin{bmatrix} 1 & 0\\ 1 & 1 \end{bmatrix}$
\item 基变量:$x_1, x_4$
\item 非基变量:$x_2 = 0, x_3 = 0$
\item 求解:$\begin{bmatrix} 1 & 0\\ 1 & 1 \end{bmatrix} \begin{bmatrix} x_1\\ x_4 \end{bmatrix} = \begin{bmatrix} 4\\ 6 \end{bmatrix}$
\item 解:$x_1 = 4, x_4 = 2, x_2 = 0, x_3 = 0$
\item 检查可行性:所有变量$\geq 0$ ✓
\item \textbf{这是基本可行解}:$(4, 0, 0, 2)$
\end{itemize}

\textbf{情况4:选择列2和列3作为基}
\begin{itemize}
\item 基矩阵:$B = \begin{bmatrix} 1 & 1\\ 2 & 0 \end{bmatrix}$
\item 基变量:$x_2, x_3$
\item 非基变量:$x_1 = 0, x_4 = 0$
\item 求解:$\begin{bmatrix} 1 & 1\\ 2 & 0 \end{bmatrix} \begin{bmatrix} x_2\\ x_3 \end{bmatrix} = \begin{bmatrix} 4\\ 6 \end{bmatrix}$
\item 解:$x_2 = 3, x_3 = 1, x_1 = 0, x_4 = 0$
\item 检查可行性:所有变量$\geq 0$ ✓
\item \textbf{这是基本可行解}:$(0, 3, 1, 0)$
\end{itemize}

\textbf{情况5:选择列2和列4作为基}
\begin{itemize}
\item 基矩阵:$B = \begin{bmatrix} 1 & 0\\ 2 & 1 \end{bmatrix}$
\item 基变量:$x_2, x_4$
\item 非基变量:$x_1 = 0, x_3 = 0$
\item 求解:$\begin{bmatrix} 1 & 0\\ 2 & 1 \end{bmatrix} \begin{bmatrix} x_2\\ x_4 \end{bmatrix} = \begin{bmatrix} 4\\ 6 \end{bmatrix}$
\item 解:$x_2 = 4, x_4 = -2, x_1 = 0, x_3 = 0$
\item 检查可行性:$x_4 = -2 < 0$ ✗
\item \textbf{这不是基本可行解}
\end{itemize}

\textbf{情况6:选择列3和列4作为基}
\begin{itemize}
\item 基矩阵:$B = \begin{bmatrix} 1 & 0\\ 0 & 1 \end{bmatrix} = I$
\item 基变量:$x_3, x_4$
\item 非基变量:$x_1 = 0, x_2 = 0$
\item 求解:$\begin{bmatrix} 1 & 0\\ 0 & 1 \end{bmatrix} \begin{bmatrix} x_3\\ x_4 \end{bmatrix} = \begin{bmatrix} 4\\ 6 \end{bmatrix}$
\item 解:$x_3 = 4, x_4 = 6, x_1 = 0, x_2 = 0$
\item 检查可行性:所有变量$\geq 0$ ✓
\item \textbf{这是基本可行解}:$(0, 0, 4, 6)$
\end{itemize}

\textbf{总结}:
\begin{itemize}
\item 总共有$\binom{4}{2} = 6$种可能的基选择
\item 其中4个是基本可行解:$(2, 2, 0, 0)$、$(4, 0, 0, 2)$、$(0, 3, 1, 0)$、$(0, 0, 4, 6)$
\item 2个不是基本可行解(违反非负约束)
\item 这些基本可行解对应可行域的4个顶点
\end{itemize}

\textbf{目标函数值}:
\begin{itemize}
\item $(2, 2, 0, 0)$:$z = -2 \times 2 - 3 \times 2 = -10$
\item $(4, 0, 0, 2)$:$z = -2 \times 4 - 3 \times 0 = -8$
\item $(0, 3, 1, 0)$:$z = -2 \times 0 - 3 \times 3 = -9$
\item $(0, 0, 4, 6)$:$z = -2 \times 0 - 3 \times 0 = 0$
\end{itemize}

最优解是$(2, 2, 0, 0)$,目标值为$-10$(最小化问题)。
\end{example}

\subsubsection{退化基本可行解}

\begin{definition}[退化基本可行解]
如果基本可行解中某个基变量的值为0,则称为退化基本可行解(degenerate BFS)。
\end{definition}

退化可能导致单纯形法在相同顶点之间循环,需要特殊处理。

\section{对偶理论}

\subsection{对偶问题}

每个线性规划问题都有一个对应的对偶问题:

\begin{definition}[对偶线性规划]
原问题(Primal):
\begin{align}
\min_{x} \quad & c^T x\\
\text{s.t.} \quad & A x = b\\
& x \geq 0
\end{align}

对偶问题(Dual):
\begin{align}
\max_{y} \quad & b^T y\\
\text{s.t.} \quad & A^T y \leq c
\end{align}
\end{definition}

\subsection{对偶定理}

\begin{theorem}[弱对偶定理]
设$x$是原问题的可行解,$y$是对偶问题的可行解,则:
\begin{equation}
c^T x \geq b^T y
\end{equation}
\end{theorem}

\begin{theorem}[强对偶定理]
如果原问题和对偶问题都有可行解,则它们都有最优解,且最优值相等:
\begin{equation}
c^T x^* = b^T y^*
\end{equation}
\end{theorem}

\subsection{互补松弛条件}

\begin{theorem}[互补松弛条件]
设$x^*$和$y^*$分别是原问题和对偶问题的最优解,则:
\begin{align}
x_i^*(c_i - A_i^T y^*) &= 0, \quad i = 1, \ldots, n\\
(A^T y^* - c)_j x_j^* &= 0, \quad j = 1, \ldots, n
\end{align}
\end{theorem}

\section{求解方法}

\subsection{单纯形法(Simplex Method)}

单纯形法是求解线性规划最经典的算法,由George Dantzig在1947年提出。

\subsubsection{算法思想}
\begin{enumerate}
\item 从初始基本可行解(顶点)开始
\item 沿着可行域的边移动到相邻顶点
\item 每次移动都使目标函数值减小(或不变)
\item 当无法继续改进时,达到最优解
\end{enumerate}

\subsubsection{算法步骤}
\begin{enumerate}
\item 将问题转化为标准形式
\item 找到初始基本可行解
\item 计算检验数(reduced cost)
\item 选择进基变量(entering variable)
\item 选择出基变量(leaving variable)
\item 更新基矩阵
\item 重复直到所有检验数非负
\end{enumerate}

\subsection{内点法(Interior Point Method)}

内点法从可行域内部开始,通过障碍函数方法逐步接近最优解。

\subsubsection{原始-对偶内点法}
\begin{enumerate}
\item 引入松弛变量将不等式转化为等式
\item 构造障碍函数
\item 求解KKT系统
\item 更新步长和障碍参数
\item 重复直到收敛
\end{enumerate}

\subsection{椭球法(Ellipsoid Method)}

椭球法是第一个多项式时间的线性规划算法,但实际效率不如单纯形法和内点法。

\section{应用领域}

\subsection{运输问题}

\begin{example}[运输问题]
有$m$个供应点和$n$个需求点,从供应点$i$到需求点$j$的单位运输成本为$c_{ij}$。求最小成本的运输方案。

\begin{align}
\min_{x_{ij}} \quad & \sum_{i=1}^{m}\sum_{j=1}^{n} c_{ij} x_{ij}\\
\text{s.t.} \quad & \sum_{j=1}^{n} x_{ij} = s_i, \quad i = 1, \ldots, m\\
& \sum_{i=1}^{m} x_{ij} = d_j, \quad j = 1, \ldots, n\\
& x_{ij} \geq 0
\end{align}
其中$s_i$是供应量,$d_j$是需求量。
\end{example}

\subsection{指派问题}

\begin{example}[指派问题]
将$n$个任务分配给$n$个人,每人完成一个任务,求最小总成本。

\begin{align}
\min_{x_{ij}} \quad & \sum_{i=1}^{n}\sum_{j=1}^{n} c_{ij} x_{ij}\\
\text{s.t.} \quad & \sum_{j=1}^{n} x_{ij} = 1, \quad i = 1, \ldots, n\\
& \sum_{i=1}^{n} x_{ij} = 1, \quad j = 1, \ldots, n\\
& x_{ij} \in \{0, 1\}
\end{align}
\end{example}

\subsection{资源分配}

线性规划广泛应用于生产计划、资源分配、投资组合等优化问题。

\section{计算复杂度}

\subsection{理论复杂度}

\begin{itemize}
\item \textbf{最坏情况}:单纯形法是指数时间的(Klee-Minty反例)
\item \textbf{平均情况}:单纯形法通常是多项式时间的
\item \textbf{内点法}:多项式时间,$O(n^{3.5}L)$,其中$L$是输入规模
\item \textbf{椭球法}:多项式时间,但实际效率低
\end{itemize}

\subsection{实际性能}

现代LP求解器(如CPLEX、Gurobi)可以高效求解:
\begin{itemize}
\item 小规模问题($n < 1000$):毫秒级
\item 中规模问题($1000 \leq n < 10^5$):秒到分钟级
\item 大规模问题($n \geq 10^5$):可能需要特殊技术
\end{itemize}

\section{软件工具}

\subsection{Python}
\begin{itemize}
\item \textbf{scipy.optimize.linprog}:SciPy内置LP求解器
\item \textbf{pulp}:Python线性规划建模工具
\item \textbf{cvxpy}:凸优化建模工具(支持LP)
\end{itemize}

\subsection{MATLAB}
\begin{itemize}
\item \textbf{linprog}:MATLAB内置LP求解器
\item \textbf{CPLEX}:商业求解器
\item \textbf{Gurobi}:商业求解器
\end{itemize}

\subsection{开源求解器}
\begin{itemize}
\item \textbf{GLPK}:GNU线性规划工具包
\item \textbf{CLP}:COIN-OR线性规划求解器
\item \textbf{HiGHS}:高性能LP求解器
\end{itemize}

\section{总结}

线性规划是优化理论的基础:
\begin{itemize}
\item \textbf{广泛应用}:运筹学、经济学、工程等领域
\item \textbf{成熟算法}:单纯形法、内点法等
\item \textbf{高效求解}:现代求解器可以处理大规模问题
\item \textbf{理论基础}:对偶理论、互补松弛等
\end{itemize}

线性规划是理解更复杂优化问题(如QP、SOCP等)的基础。

\section{具体数值例子}

\begin{example}[生产计划问题]
某工厂生产两种产品A和B,需要优化生产计划以最大化利润。

\textbf{问题数据}:
\begin{itemize}
\item 产品A:每单位利润3元,需要2小时机器时间,1小时人工时间
\item 产品B:每单位利润4元,需要1小时机器时间,2小时人工时间
\item 可用资源:机器时间8小时,人工时间6小时
\end{itemize}

\textbf{数学模型}:
设$x_1$为产品A的产量,$x_2$为产品B的产量。

\begin{align}
\max_{x_1, x_2} \quad & 3x_1 + 4x_2\\
\text{s.t.} \quad & 2x_1 + x_2 \leq 8 \quad \text{(机器时间约束)}\\
& x_1 + 2x_2 \leq 6 \quad \text{(人工时间约束)}\\
& x_1 \geq 0, \quad x_2 \geq 0
\end{align}

\textbf{求解过程}:

1. \textbf{可行域顶点}:
   \begin{itemize}
   \item 顶点1:$(0, 0)$,目标值 = 0
   \item 顶点2:$(0, 3)$,目标值 = 12
   \item 顶点3:$(4, 0)$,目标值 = 12
   \item 顶点4:$(10/3, 4/3)$(两约束的交点),目标值 = $3 \times 10/3 + 4 \times 4/3 = 10 + 16/3 = 46/3 \approx 15.33$
   \end{itemize}

2. \textbf{最优解}:$x_1^* = 10/3 \approx 3.33$,$x_2^* = 4/3 \approx 1.33$

3. \textbf{最优值}:$z^* = 46/3 \approx 15.33$元

\textbf{验证}:
\begin{itemize}
\item 机器时间:$2 \times 10/3 + 4/3 = 24/3 = 8$ ✓
\item 人工时间:$10/3 + 2 \times 4/3 = 18/3 = 6$ ✓
\end{itemize}
\end{example}

\section{参考文献}

\begin{itemize}
\item Dantzig, G. B. (1963). \textit{Linear programming and extensions}. Princeton university press.
\item Chvatal, V. (1983). \textit{Linear programming}. Macmillan.
\item Nocedal, J., \& Wright, S. J. (2006). \textit{Numerical optimization}. Springer.
\end{itemize}

\end{document}

