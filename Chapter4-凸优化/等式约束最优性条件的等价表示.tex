\documentclass[12pt,a4paper]{article}
\usepackage[UTF8]{ctex}
\usepackage{amsmath}
\usepackage{amssymb}
\usepackage{amsthm}
\usepackage{geometry}
\geometry{left=2.5cm,right=2.5cm,top=2.5cm,bottom=2.5cm}

\title{等式约束最优性条件的等价表示}
\author{}
\date{\today}

\begin{document}

\maketitle

\section{问题提出}

\textbf{最优性条件}:对于等式约束优化问题
\begin{align}
\min \quad & f_0(\mathbf{x}) \\
\text{s.t.} \quad & \mathbf{A}\mathbf{x} = \mathbf{b}
\end{align}

其中 $\mathbf{A} \in \mathbb{R}^{p \times n}$,$\mathbf{b} \in \mathbb{R}^p$。

最优性条件可以表示为:
\begin{equation}
\nabla f_0(\mathbf{x}) \in R(\mathbf{A}^T)
\end{equation}

即存在 $\mathbf{v} \in \mathbb{R}^p$,使得:
\begin{equation}
\nabla f_0(\mathbf{x}) + \mathbf{A}^T \mathbf{v} = \mathbf{0}
\end{equation}

\textbf{问题}:这个条件是怎么来的?为什么可以这样表示?

\section{推导过程}

\subsection{步骤1:基本最优性条件}

\textbf{基本最优性条件}:对于可行点 $\mathbf{x}$,$\mathbf{x}$ 是最优的,当且仅当:
\begin{equation}
\nabla f_0(\mathbf{x})^T (\mathbf{y} - \mathbf{x}) \geq 0
\end{equation}
对所有可行 $\mathbf{y}$ 成立。

\textbf{含义}:没有可行方向能使函数值减小。

\subsection{步骤2:利用可行解的表示}

\textbf{关键观察}:每个可行解 $\mathbf{y}$ 可以写成:
\begin{equation}
\mathbf{y} = \mathbf{x} + \mathbf{v}, \quad \mathbf{v} \in N(\mathbf{A})
\end{equation}

\textbf{原因}:$\mathbf{A}(\mathbf{y} - \mathbf{x}) = \mathbf{A}\mathbf{y} - \mathbf{A}\mathbf{x} = \mathbf{b} - \mathbf{b} = \mathbf{0}$,所以 $\mathbf{y} - \mathbf{x} \in N(\mathbf{A})$。

\textbf{最优性条件变为}:
\begin{align}
\nabla f_0(\mathbf{x})^T (\mathbf{y} - \mathbf{x}) &= \nabla f_0(\mathbf{x})^T (\mathbf{x} + \mathbf{v} - \mathbf{x}) \\
&= \nabla f_0(\mathbf{x})^T \mathbf{v} \geq 0
\end{align}

对所有 $\mathbf{v} \in N(\mathbf{A})$ 成立。

\subsection{步骤3:线性函数的性质}

\textbf{关键观察}:$\nabla f_0(\mathbf{x})^T \mathbf{v}$ 是关于 $\mathbf{v}$ 的线性函数。

\textbf{原因}:
\begin{itemize}
\item $\nabla f_0(\mathbf{x})$ 是固定的向量(当 $\mathbf{x}$ 固定时)
\item 内积 $\nabla f_0(\mathbf{x})^T \mathbf{v}$ 是线性函数
\end{itemize}

\textbf{线性函数在子空间上的性质}:如果线性函数在子空间上非负,则它必须在该子空间上为零。

\textbf{证明}:
\begin{enumerate}
\item 如果 $\nabla f_0(\mathbf{x})^T \mathbf{v} \geq 0$ 对所有 $\mathbf{v} \in N(\mathbf{A})$ 成立

\item 由于 $N(\mathbf{A})$ 是子空间,如果 $\mathbf{v} \in N(\mathbf{A})$,则 $-\mathbf{v} \in N(\mathbf{A})$

\item 对 $-\mathbf{v} \in N(\mathbf{A})$,也有 $\nabla f_0(\mathbf{x})^T (-\mathbf{v}) \geq 0$

\item 即 $-\nabla f_0(\mathbf{x})^T \mathbf{v} \geq 0$,因此 $\nabla f_0(\mathbf{x})^T \mathbf{v} \leq 0$

\item 结合两个不等式:$\nabla f_0(\mathbf{x})^T \mathbf{v} = 0$
\end{enumerate}

\textbf{结论}:
\begin{equation}
\nabla f_0(\mathbf{x})^T \mathbf{v} = 0 \quad \text{对所有 } \mathbf{v} \in N(\mathbf{A})
\end{equation}

\subsection{步骤4:正交关系}

\textbf{等价表述}:$\nabla f_0(\mathbf{x}) \perp N(\mathbf{A})$(梯度垂直于零空间)

\textbf{含义}:梯度与零空间中的每个向量都垂直。

\textbf{数学表述}:
\begin{equation}
\nabla f_0(\mathbf{x}) \in N(\mathbf{A})^\perp
\end{equation}

其中 $N(\mathbf{A})^\perp$ 是 $N(\mathbf{A})$ 的正交补。

\subsection{步骤5:使用正交补关系}

\textbf{关键定理}:$N(\mathbf{A})^\perp = R(\mathbf{A}^T)$

\textbf{证明思路}:
\begin{enumerate}
\item 对于任意 $\mathbf{w} \in R(\mathbf{A}^T)$,存在 $\mathbf{y} \in \mathbb{R}^p$ 使得 $\mathbf{w} = \mathbf{A}^T \mathbf{y}$

\item 对于任意 $\mathbf{v} \in N(\mathbf{A})$,有 $\mathbf{A}\mathbf{v} = \mathbf{0}$

\item 计算内积:
\begin{align}
\mathbf{w}^T \mathbf{v} &= (\mathbf{A}^T \mathbf{y})^T \mathbf{v} \\
&= \mathbf{y}^T \mathbf{A} \mathbf{v} \\
&= \mathbf{y}^T \mathbf{0} = 0
\end{align}

\item 因此 $R(\mathbf{A}^T) \subseteq N(\mathbf{A})^\perp$

\item 通过维度论证,可以证明 $N(\mathbf{A})^\perp \subseteq R(\mathbf{A}^T)$

\item 因此 $N(\mathbf{A})^\perp = R(\mathbf{A}^T)$
\end{enumerate}

\textbf{应用}:由于 $\nabla f_0(\mathbf{x}) \in N(\mathbf{A})^\perp$,且 $N(\mathbf{A})^\perp = R(\mathbf{A}^T)$,有:
\begin{equation}
\nabla f_0(\mathbf{x}) \in R(\mathbf{A}^T)
\end{equation}

\subsection{步骤6:列空间的表示}

\textbf{列空间的定义}:$R(\mathbf{A}^T) = \{\mathbf{A}^T \mathbf{y} \mid \mathbf{y} \in \mathbb{R}^p\}$

\textbf{含义}:$R(\mathbf{A}^T)$ 是所有形如 $\mathbf{A}^T \mathbf{y}$ 的向量的集合,其中 $\mathbf{y} \in \mathbb{R}^p$。

\textbf{结论}:由于 $\nabla f_0(\mathbf{x}) \in R(\mathbf{A}^T)$,存在 $\mathbf{v} \in \mathbb{R}^p$,使得:
\begin{equation}
\nabla f_0(\mathbf{x}) = \mathbf{A}^T \mathbf{v}
\end{equation}

\textbf{等价形式}:
\begin{equation}
\nabla f_0(\mathbf{x}) + \mathbf{A}^T (-\mathbf{v}) = \mathbf{0}
\end{equation}

\textbf{重新标记}:令 $\boldsymbol{\nu} = -\mathbf{v}$,则:
\begin{equation}
\nabla f_0(\mathbf{x}) + \mathbf{A}^T \boldsymbol{\nu} = \mathbf{0}
\end{equation}

其中 $\boldsymbol{\nu} \in \mathbb{R}^p$。

\textbf{注意}:在优化理论中,通常使用 $\boldsymbol{\nu}$ 表示拉格朗日乘数(Lagrange multiplier),这就是拉格朗日乘数法中的最优性条件。

\section{完整推导总结}

\subsection{推导链条}

\begin{enumerate}
\item \textbf{基本最优性条件}:
\begin{equation}
\nabla f_0(\mathbf{x})^T (\mathbf{y} - \mathbf{x}) \geq 0 \quad \text{对所有可行 } \mathbf{y}
\end{equation}

\item \textbf{利用可行解的表示}:
\begin{equation}
\nabla f_0(\mathbf{x})^T \mathbf{v} \geq 0 \quad \text{对所有 } \mathbf{v} \in N(\mathbf{A})
\end{equation}

\item \textbf{线性函数的性质}:
\begin{equation}
\nabla f_0(\mathbf{x})^T \mathbf{v} = 0 \quad \text{对所有 } \mathbf{v} \in N(\mathbf{A})
\end{equation}

\item \textbf{正交关系}:
\begin{equation}
\nabla f_0(\mathbf{x}) \perp N(\mathbf{A}) \quad \Leftrightarrow \quad \nabla f_0(\mathbf{x}) \in N(\mathbf{A})^\perp
\end{equation}

\item \textbf{正交补关系}:
\begin{equation}
N(\mathbf{A})^\perp = R(\mathbf{A}^T)
\end{equation}

\item \textbf{列空间表示}:
\begin{equation}
\nabla f_0(\mathbf{x}) \in R(\mathbf{A}^T) \quad \Leftrightarrow \quad \exists \boldsymbol{\nu} \in \mathbb{R}^p: \nabla f_0(\mathbf{x}) + \mathbf{A}^T \boldsymbol{\nu} = \mathbf{0}
\end{equation}
\end{enumerate}

\section{几何理解}

\subsection{几何意义}

\textbf{可行集}:$\{\mathbf{x} \mid \mathbf{A}\mathbf{x} = \mathbf{b}\} = \mathbf{x}_0 + N(\mathbf{A})$

其中 $\mathbf{x}_0$ 是任意可行解,$N(\mathbf{A})$ 是零空间。

\textbf{最优性条件}:
\begin{itemize}
\item 梯度 $\nabla f_0(\mathbf{x})$ 必须垂直于可行集的方向(零空间)
\item 这意味着梯度在可行集上是常数(或零)
\item 梯度必须属于零空间的正交补,即行空间 $R(\mathbf{A}^T)$
\end{itemize}

\subsection{可视化}

在 $\mathbb{R}^3$ 中,如果约束是 $x + y = 1$(一个平面),则:
\begin{itemize}
\item 可行集是平面 $x + y = 1$
\item 零空间 $N(\mathbf{A})$ 是垂直于平面的方向(一维子空间)
\item 行空间 $R(\mathbf{A}^T)$ 是平面的法向量方向
\item 最优性条件:梯度必须与平面的法向量方向一致(在行空间中)
\end{itemize}

\section{拉格朗日乘数法}

\subsection{拉格朗日函数}

\textbf{拉格朗日函数}:
\begin{equation}
L(\mathbf{x}, \boldsymbol{\nu}) = f_0(\mathbf{x}) + \boldsymbol{\nu}^T (\mathbf{A}\mathbf{x} - \mathbf{b})
\end{equation}

其中 $\boldsymbol{\nu} \in \mathbb{R}^p$ 是拉格朗日乘数。

\subsection{最优性条件}

\textbf{KKT条件}(对于等式约束问题):
\begin{enumerate}
\item \textbf{原始可行性}:$\mathbf{A}\mathbf{x} = \mathbf{b}$

\item \textbf{对偶可行性}:$\nabla_\mathbf{x} L(\mathbf{x}, \boldsymbol{\nu}) = \mathbf{0}$

计算梯度:
\begin{align}
\nabla_\mathbf{x} L(\mathbf{x}, \boldsymbol{\nu}) &= \nabla f_0(\mathbf{x}) + \nabla_\mathbf{x} [\boldsymbol{\nu}^T (\mathbf{A}\mathbf{x} - \mathbf{b})] \\
&= \nabla f_0(\mathbf{x}) + \mathbf{A}^T \boldsymbol{\nu}
\end{align}

因此:
\begin{equation}
\nabla f_0(\mathbf{x}) + \mathbf{A}^T \boldsymbol{\nu} = \mathbf{0}
\end{equation}

这正是我们推导出的最优性条件!
\end{enumerate}

\subsection{等价性}

\textbf{结论}:最优性条件 $\nabla f_0(\mathbf{x}) + \mathbf{A}^T \boldsymbol{\nu} = \mathbf{0}$ 等价于:
\begin{itemize}
\item $\nabla f_0(\mathbf{x}) \in R(\mathbf{A}^T)$
\item $\nabla f_0(\mathbf{x}) \perp N(\mathbf{A})$
\item 拉格朗日函数的梯度为零
\end{itemize}

\section{具体例子}

\subsection{例子1:简单约束}

\textbf{问题}:$\min_{x, y} x^2 + y^2$ subject to $x + y = 1$

\textbf{矩阵表示}:
\begin{itemize}
\item $\mathbf{A} = \begin{pmatrix} 1 & 1 \end{pmatrix}$,$\mathbf{b} = 1$
\item $\mathbf{A} \in \mathbb{R}^{1 \times 2}$,$p = 1$
\end{itemize}

\textbf{梯度}:$\nabla f_0(x, y) = \begin{pmatrix} 2x \\ 2y \end{pmatrix}$

\textbf{最优性条件}:存在 $\nu \in \mathbb{R}$,使得:
\begin{equation}
\begin{pmatrix} 2x \\ 2y \end{pmatrix} + \begin{pmatrix} 1 \\ 1 \end{pmatrix} \nu = \begin{pmatrix} 0 \\ 0 \end{pmatrix}
\end{equation}

即:
\begin{align}
2x + \nu &= 0 \\
2y + \nu &= 0
\end{align}

结合约束 $x + y = 1$,得到:
\begin{align}
x = y = \frac{1}{2}, \quad \nu = -1
\end{align}

\textbf{验证}:
\begin{itemize}
\item 约束满足:$x + y = 1$ ✓
\item 最优性条件:$\nabla f_0 + \mathbf{A}^T \nu = (1, 1)^T + (1, 1)^T \cdot (-1) = (0, 0)^T$ ✓
\end{itemize}

\subsection{例子2:多个约束}

\textbf{问题}:$\min_{x, y, z} x^2 + y^2 + z^2$ subject to $x + y = 1$,$y + z = 2$

\textbf{矩阵表示}:
\begin{itemize}
\item $\mathbf{A} = \begin{pmatrix} 1 & 1 & 0 \\ 0 & 1 & 1 \end{pmatrix}$,$\mathbf{b} = \begin{pmatrix} 1 \\ 2 \end{pmatrix}$
\item $\mathbf{A} \in \mathbb{R}^{2 \times 3}$,$p = 2$
\end{itemize}

\textbf{梯度}:$\nabla f_0(x, y, z) = \begin{pmatrix} 2x \\ 2y \\ 2z \end{pmatrix}$

\textbf{最优性条件}:存在 $\boldsymbol{\nu} = \begin{pmatrix} \nu_1 \\ \nu_2 \end{pmatrix} \in \mathbb{R}^2$,使得:
\begin{equation}
\begin{pmatrix} 2x \\ 2y \\ 2z \end{pmatrix} + \begin{pmatrix} 1 & 0 \\ 1 & 1 \\ 0 & 1 \end{pmatrix} \begin{pmatrix} \nu_1 \\ \nu_2 \end{pmatrix} = \begin{pmatrix} 0 \\ 0 \\ 0 \end{pmatrix}
\end{equation}

即:
\begin{align}
2x + \nu_1 &= 0 \\
2y + \nu_1 + \nu_2 &= 0 \\
2z + \nu_2 &= 0
\end{align}

结合约束,可以求解得到最优解。

\section{总结}

\subsection{关键要点}

\begin{enumerate}
\item \textbf{最优性条件}:$\nabla f_0(\mathbf{x}) + \mathbf{A}^T \boldsymbol{\nu} = \mathbf{0}$

\item \textbf{等价表述}:$\nabla f_0(\mathbf{x}) \in R(\mathbf{A}^T)$

\item \textbf{几何意义}:梯度必须垂直于可行集的方向(零空间)

\item \textbf{拉格朗日乘数}:$\boldsymbol{\nu}$ 是拉格朗日乘数,表示约束的"价格"

\item \textbf{正交补关系}:$N(\mathbf{A})^\perp = R(\mathbf{A}^T)$ 是关键
\end{enumerate}

\subsection{推导逻辑}

\begin{itemize}
\item 从基本最优性条件出发
\item 利用可行解的表示
\item 利用线性函数的性质
\item 利用正交补关系
\item 得到列空间表示
\end{itemize}

\subsection{应用}

这个最优性条件是:
\begin{itemize}
\item 拉格朗日乘数法的基础
\item KKT条件的一部分
\item 求解等式约束优化问题的关键
\end{itemize}

理解这个推导过程,是理解凸优化理论和算法的关键!

\end{document}

