\documentclass[12pt,a4paper]{article}
\usepackage[UTF8]{ctex}
\usepackage{amsmath}
\usepackage{amssymb}
\usepackage{amsthm}
\usepackage{geometry}
\geometry{left=2.5cm,right=2.5cm,top=2.5cm,bottom=2.5cm}

\title{第4.1.3节:等价问题的简单例子详解}
\subtitle{理解缩放问题如何与标准形式等价}
\author{}
\date{\today}

\begin{document}

\maketitle

\section{问题提出}

在《Convex Optimization》第4.1.3节中,给出了一个简单的等价问题例子:

\begin{align}
\begin{array}{ll}
\text{minimize} & \tilde{f}_0(\mathbf{x}) = \alpha_0 f_0(\mathbf{x}) \\
\text{subject to} & \tilde{f}_i(\mathbf{x}) = \alpha_i f_i(\mathbf{x}) \leq 0, \quad i = 1, \ldots, m \\
& \tilde{h}_i(\mathbf{x}) = \beta_i h_i(\mathbf{x}) = 0, \quad i = 1, \ldots, p
\end{array}
\end{align}

其中 $\alpha_i > 0$($i = 0, \ldots, m$),$\beta_i \neq 0$($i = 1, \ldots, p$)。

\textbf{问题}:为什么这个问题与标准形式(4.1)等价?如何理解这种等价性?

\section{标准形式回顾}

\subsection{标准形式(4.1)}

标准优化问题形式为:

\begin{align}
\begin{array}{ll}
\text{minimize} & f_0(\mathbf{x}) \\
\text{subject to} & f_i(\mathbf{x}) \leq 0, \quad i = 1, \ldots, m \\
& h_i(\mathbf{x}) = 0, \quad i = 1, \ldots, p
\end{array}
\end{align}

\subsection{缩放形式(4.3)}

缩放后的形式为:

\begin{align}
\begin{array}{ll}
\text{minimize} & \tilde{f}_0(\mathbf{x}) = \alpha_0 f_0(\mathbf{x}) \\
\text{subject to} & \tilde{f}_i(\mathbf{x}) = \alpha_i f_i(\mathbf{x}) \leq 0, \quad i = 1, \ldots, m \\
& \tilde{h}_i(\mathbf{x}) = \beta_i h_i(\mathbf{x}) = 0, \quad i = 1, \ldots, p
\end{array}
\end{align}

其中:
\begin{itemize}
\item $\alpha_0 > 0$:目标函数的缩放因子
\item $\alpha_i > 0$($i = 1, \ldots, m$):不等式约束的缩放因子
\item $\beta_i \neq 0$($i = 1, \ldots, p$):等式约束的缩放因子
\end{itemize}

\section{为什么等价?}

\subsection{等价性的定义}

两个优化问题等价,如果:
\begin{enumerate}
\item 可行集相同
\item 最优点相同(或可以互相转换)
\end{enumerate}

\subsection{证明可行集相同}

\textbf{关键观察}:缩放不改变约束的满足情况。

\textbf{对于不等式约束}:

原始约束:$f_i(\mathbf{x}) \leq 0$

缩放后约束:$\alpha_i f_i(\mathbf{x}) \leq 0$

\textbf{为什么等价?}

由于 $\alpha_i > 0$,我们可以将缩放后的约束两边同时除以 $\alpha_i$:
\begin{equation}
\alpha_i f_i(\mathbf{x}) \leq 0 \quad \Leftrightarrow \quad f_i(\mathbf{x}) \leq 0
\end{equation}

\textbf{详细证明}:

\textbf{方向1}:如果 $f_i(\mathbf{x}) \leq 0$,则 $\alpha_i f_i(\mathbf{x}) \leq \alpha_i \cdot 0 = 0$(因为 $\alpha_i > 0$)

\textbf{方向2}:如果 $\alpha_i f_i(\mathbf{x}) \leq 0$,则 $f_i(\mathbf{x}) = \frac{1}{\alpha_i} \cdot \alpha_i f_i(\mathbf{x}) \leq \frac{1}{\alpha_i} \cdot 0 = 0$(因为 $\alpha_i > 0$)

因此,$f_i(\mathbf{x}) \leq 0$ 当且仅当 $\alpha_i f_i(\mathbf{x}) \leq 0$。

\textbf{对于等式约束}:

原始约束:$h_i(\mathbf{x}) = 0$

缩放后约束:$\beta_i h_i(\mathbf{x}) = 0$

\textbf{为什么等价?}

由于 $\beta_i \neq 0$,我们可以将缩放后的约束两边同时除以 $\beta_i$:
\begin{equation}
\beta_i h_i(\mathbf{x}) = 0 \quad \Leftrightarrow \quad h_i(\mathbf{x}) = 0
\end{equation}

\textbf{详细证明}:

\textbf{方向1}:如果 $h_i(\mathbf{x}) = 0$,则 $\beta_i h_i(\mathbf{x}) = \beta_i \cdot 0 = 0$

\textbf{方向2}:如果 $\beta_i h_i(\mathbf{x}) = 0$,则 $h_i(\mathbf{x}) = \frac{1}{\beta_i} \cdot \beta_i h_i(\mathbf{x}) = \frac{1}{\beta_i} \cdot 0 = 0$(因为 $\beta_i \neq 0$)

因此,$h_i(\mathbf{x}) = 0$ 当且仅当 $\beta_i h_i(\mathbf{x}) = 0$。

\textbf{结论}:可行集完全相同!

\subsection{证明最优点相同}

\textbf{关键观察}:目标函数的缩放不改变最优点的位置(只改变最优值)。

\textbf{证明}:

设 $\mathbf{x}^*$ 是标准形式(4.1)的最优点,即:
\begin{itemize}
\item $\mathbf{x}^*$ 可行
\item $f_0(\mathbf{x}^*) \leq f_0(\mathbf{x})$ 对所有可行 $\mathbf{x}$ 成立
\end{itemize}

由于 $\alpha_0 > 0$,我们有:
\begin{equation}
\alpha_0 f_0(\mathbf{x}^*) \leq \alpha_0 f_0(\mathbf{x}) \quad \Leftrightarrow \quad \tilde{f}_0(\mathbf{x}^*) \leq \tilde{f}_0(\mathbf{x})
\end{equation}

因此 $\mathbf{x}^*$ 也是缩放形式(4.3)的最优点。

反过来,如果 $\mathbf{x}^*$ 是缩放形式(4.3)的最优点,类似地可以证明它也是标准形式(4.1)的最优点。

\textbf{注意}:虽然最优点相同,但最优值不同:
\begin{itemize}
\item 标准形式的最优值:$p^* = f_0(\mathbf{x}^*)$
\item 缩放形式的最优值:$\tilde{p}^* = \alpha_0 f_0(\mathbf{x}^*) = \alpha_0 p^*$
\end{itemize}

\section{具体例子}

\subsection{例子1:简单的缩放}

\textbf{标准形式}:
\begin{align}
\begin{array}{ll}
\text{minimize} & f_0(x) = x^2 \\
\text{subject to} & f_1(x) = x - 1 \leq 0
\end{array}
\end{align}

\textbf{缩放形式}(取 $\alpha_0 = 2$,$\alpha_1 = 3$):
\begin{align}
\begin{array}{ll}
\text{minimize} & \tilde{f}_0(x) = 2x^2 \\
\text{subject to} & \tilde{f}_1(x) = 3(x - 1) \leq 0
\end{array}
\end{align}

\textbf{验证等价性}:

\textbf{可行集}:
\begin{itemize}
\item 标准形式:$x - 1 \leq 0$,即 $x \leq 1$
\item 缩放形式:$3(x - 1) \leq 0$,即 $x - 1 \leq 0$,即 $x \leq 1$
\item 可行集相同!✓
\end{itemize}

\textbf{最优点}:
\begin{itemize}
\item 标准形式:$x^* = 1$(在可行集 $x \leq 1$ 上,$x^2$ 在 $x = 1$ 处最小)
\item 缩放形式:$x^* = 1$(在可行集 $x \leq 1$ 上,$2x^2$ 在 $x = 1$ 处最小)
\item 最优点相同!✓
\end{itemize}

\textbf{最优值}:
\begin{itemize}
\item 标准形式:$p^* = f_0(1) = 1$
\item 缩放形式:$\tilde{p}^* = \tilde{f}_0(1) = 2 \cdot 1 = 2 = 2p^*$
\item 最优值不同,但关系明确:$\tilde{p}^* = \alpha_0 p^*$
\end{itemize}

\subsection{例子2:带等式约束}

\textbf{标准形式}:
\begin{align}
\begin{array}{ll}
\text{minimize} & f_0(x, y) = x^2 + y^2 \\
\text{subject to} & f_1(x, y) = x + y - 1 \leq 0 \\
& h_1(x, y) = x - y = 0
\end{array}
\end{align}

\textbf{缩放形式}(取 $\alpha_0 = 0.5$,$\alpha_1 = 2$,$\beta_1 = -1$):
\begin{align}
\begin{array}{ll}
\text{minimize} & \tilde{f}_0(x, y) = 0.5(x^2 + y^2) \\
\text{subject to} & \tilde{f}_1(x, y) = 2(x + y - 1) \leq 0 \\
& \tilde{h}_1(x, y) = -1(x - y) = 0
\end{array}
\end{align}

\textbf{验证等价性}:

\textbf{可行集}:
\begin{itemize}
\item 标准形式:$x + y - 1 \leq 0$ 且 $x - y = 0$
\item 缩放形式:$2(x + y - 1) \leq 0$ 且 $-(x - y) = 0$
\item 第一个约束:$2(x + y - 1) \leq 0 \Leftrightarrow x + y - 1 \leq 0$(因为 $2 > 0$)
\item 第二个约束:$-(x - y) = 0 \Leftrightarrow x - y = 0$(因为 $-1 \neq 0$)
\item 可行集相同!✓
\end{itemize}

\textbf{最优点}:
\begin{itemize}
\item 从 $x - y = 0$ 得到 $x = y$
\item 从 $x + y - 1 \leq 0$ 和 $x = y$ 得到 $2x - 1 \leq 0$,即 $x \leq 0.5$
\item 在可行集上,$x^2 + y^2 = 2x^2$ 在 $x = 0.5$ 处最小
\item 最优点:$(x^*, y^*) = (0.5, 0.5)$
\item 两个形式的最优点相同!✓
\end{itemize}

\section{为什么需要 $\alpha_i > 0$ 和 $\beta_i \neq 0$?}

\subsection{为什么 $\alpha_i > 0$?

\textbf{如果 $\alpha_i < 0$}:

不等式约束的方向会反转!

\textbf{例子}:
\begin{itemize}
\item 原始约束:$x - 1 \leq 0$(即 $x \leq 1$)
\item 如果 $\alpha_1 = -1$:$-1(x - 1) \leq 0$,即 $x - 1 \geq 0$,即 $x \geq 1$
\item 可行集完全不同!✗
\end{itemize}

\textbf{如果 $\alpha_i = 0$}:

约束会变成 $0 \leq 0$,这是恒成立的,约束失效。

\textbf{因此}:必须 $\alpha_i > 0$ 才能保持不等式的方向。

\subsection{为什么 $\beta_i \neq 0$?

\textbf{如果 $\beta_i = 0$}:

等式约束会变成 $0 = 0$,这是恒成立的,约束失效。

\textbf{例子}:
\begin{itemize}
\item 原始约束:$x - y = 0$(即 $x = y$)
\item 如果 $\beta_1 = 0$:$0(x - y) = 0$,即 $0 = 0$(恒成立)
\item 可行集完全不同!✗
\end{itemize}

\textbf{因此}:必须 $\beta_i \neq 0$ 才能保持等式的有效性。

\section{几何直观}

\subsection{不等式约束的缩放}

\begin{itemize}
\item 原始约束 $f_i(\mathbf{x}) \leq 0$ 定义了一个"半空间"
\item 缩放后 $\alpha_i f_i(\mathbf{x}) \leq 0$ 定义的是同一个半空间(因为 $\alpha_i > 0$)
\item 只是"描述方式"不同,但"几何对象"相同
\end{itemize}

\textbf{类比}:
\begin{itemize}
\item 就像用不同的单位测量距离:1米 = 100厘米
\item 数值不同,但表示的是同一个长度
\item 约束的"本质"没有改变
\end{itemize}

\subsection{等式约束的缩放}

\begin{itemize}
\item 原始约束 $h_i(\mathbf{x}) = 0$ 定义了一个"超平面"
\item 缩放后 $\beta_i h_i(\mathbf{x}) = 0$ 定义的是同一个超平面(因为 $\beta_i \neq 0$)
\item 只是方程的"系数"不同,但"几何对象"相同
\end{itemize}

\subsection{目标函数的缩放}

\begin{itemize}
\item 原始目标函数 $f_0(\mathbf{x})$ 的"等高线"形状
\item 缩放后 $\alpha_0 f_0(\mathbf{x})$ 的等高线形状相同(只是数值不同)
\item 最优点位置不变(因为只是"垂直拉伸",不改变"水平位置")
\end{itemize}

\textbf{类比}:
\begin{itemize}
\item 就像把地图上的高度图乘以常数
\item 山峰的位置不变,只是高度数值改变
\item 最低点(最优点)的位置不变
\end{itemize}

\section{实际应用}

\subsection{数值稳定性}

缩放可以改善数值稳定性:
\begin{itemize}
\item 如果约束函数的数值范围很大,可能导致数值问题
\item 通过适当的缩放,可以使数值范围更合理
\item 例如:将约束从 $10^6 x \leq 10^6$ 缩放到 $x \leq 1$
\end{itemize}

\subsection{算法实现}

某些优化算法对缩放敏感:
\begin{itemize}
\item 通过适当的缩放,可以改善算法的收敛性
\item 例如:使约束的梯度具有相似的量级
\end{itemize}

\subsection{问题标准化}

将问题转换为标准形式时,可能需要缩放:
\begin{itemize}
\item 某些约束可能不是 $\leq 0$ 的形式
\item 通过缩放可以转换为标准形式
\item 例如:$g(\mathbf{x}) \leq b$ 可以写成 $g(\mathbf{x}) - b \leq 0$
\end{itemize}

\section{总结}

\begin{enumerate}
\item \textbf{等价性的含义}:
   \begin{itemize}
   \item 可行集相同
   \item 最优点相同
   \item 最优值可能不同(但关系明确)
   \end{itemize}

\item \textbf{为什么等价?}:
   \begin{itemize}
   \item 不等式约束:$\alpha_i > 0$ 保证方向不变
   \item 等式约束:$\beta_i \neq 0$ 保证有效性不变
   \item 目标函数:$\alpha_0 > 0$ 保证最优点位置不变
   \end{itemize}

\item \textbf{关键条件}:
   \begin{itemize}
   \item $\alpha_i > 0$($i = 0, \ldots, m$):保持不等式方向和最优点
   \item $\beta_i \neq 0$($i = 1, \ldots, p$):保持等式有效性
   \end{itemize}

\item \textbf{几何直观}:
   \begin{itemize}
   \item 缩放不改变可行集的几何形状
   \item 缩放不改变最优点的位置
   \item 只是改变了"描述方式"和"数值大小"
   \end{itemize}
\end{enumerate}

理解这种等价性,有助于理解优化问题的各种变换,以及为什么某些变换不会改变问题的本质!

\end{document}

