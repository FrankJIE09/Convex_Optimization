\documentclass[12pt,a4paper]{article}
\usepackage[UTF8]{ctex}
\usepackage{amsmath}
\usepackage{amssymb}
\usepackage{amsthm}
\usepackage{geometry}
\geometry{left=2.5cm,right=2.5cm,top=2.5cm,bottom=2.5cm}

\title{第4.2.3节:最优性条件证明详解}
\author{}
\date{\today}

\begin{document}

\maketitle

\section{引言}

第4.2.3节给出了可微凸优化问题的最优性条件。这个条件不仅提供了判断最优性的方法,还有深刻的几何意义。本节详细解释这个最优性条件及其证明。

\section{最优性条件}

\subsection{定理陈述}

\textbf{定理}:对于可微的凸优化问题,点 $\mathbf{x}$ 是最优的,当且仅当:

\begin{enumerate}
\item $\mathbf{x}$ 是可行的:$\mathbf{x} \in X$

\item 对于所有可行点 $\mathbf{y} \in X$,有:
   \begin{equation}
   \nabla f_0(\mathbf{x})^T (\mathbf{y} - \mathbf{x}) \geq 0
   \end{equation}
\end{enumerate}

\textbf{符号说明}:
\begin{itemize}
\item $X = \{\mathbf{x} \mid f_i(\mathbf{x}) \leq 0, i = 1, \ldots, m, h_i(\mathbf{x}) = 0, i = 1, \ldots, p\}$:可行集
\item $\nabla f_0(\mathbf{x})$:目标函数在 $\mathbf{x}$ 处的梯度
\end{itemize}

\subsection{几何意义}

\textbf{核心思想}:如果 $\nabla f_0(\mathbf{x}) \neq \mathbf{0}$,则 $-\nabla f_0(\mathbf{x})$ 定义了可行集在点 $\mathbf{x}$ 处的支撑超平面。

\textbf{详细解释}:
\begin{itemize}
\item \textbf{梯度方向}:$\nabla f_0(\mathbf{x})$ 指向函数值增加最快的方向
\item \textbf{负梯度方向}:$-\nabla f_0(\mathbf{x})$ 指向函数值减小最快的方向
\item \textbf{最优性条件}:所有可行方向与梯度方向的内积非负
\item \textbf{几何意义}:在 $\mathbf{x}$ 处,没有可行方向能使函数值减小
\end{itemize}

\section{证明:充分性(条件 $\Rightarrow$ 最优性)}

\subsection{目标}

\textbf{需要证明}:如果 $\mathbf{x} \in X$ 且对于所有 $\mathbf{y} \in X$,有 $\nabla f_0(\mathbf{x})^T (\mathbf{y} - \mathbf{x}) \geq 0$,则 $\mathbf{x}$ 是最优的。

\subsection{证明步骤}

\textbf{步骤1}:已知条件。

\begin{itemize}
\item $\mathbf{x} \in X$(可行)
\item 对于所有 $\mathbf{y} \in X$,有 $\nabla f_0(\mathbf{x})^T (\mathbf{y} - \mathbf{x}) \geq 0$
\end{itemize}

\textbf{步骤2}:应用一阶条件。

由于 $f_0$ 是凸函数且可微,一阶条件成立:

\begin{equation}
f_0(\mathbf{y}) \geq f_0(\mathbf{x}) + \nabla f_0(\mathbf{x})^T (\mathbf{y} - \mathbf{x})
\end{equation}

\textbf{步骤3}:结合条件。

由于 $\nabla f_0(\mathbf{x})^T (\mathbf{y} - \mathbf{x}) \geq 0$,有:

\begin{align}
f_0(\mathbf{y}) &\geq f_0(\mathbf{x}) + \nabla f_0(\mathbf{x})^T (\mathbf{y} - \mathbf{x}) \\
&\geq f_0(\mathbf{x}) + 0 \\
&= f_0(\mathbf{x})
\end{equation}

\textbf{步骤4}:结论。

对于所有可行点 $\mathbf{y} \in X$,有 $f_0(\mathbf{y}) \geq f_0(\mathbf{x})$,因此 $\mathbf{x}$ 是最优的。$\square$

\subsection{证明总结}

\begin{enumerate}
\item 利用一阶条件:$f_0(\mathbf{y}) \geq f_0(\mathbf{x}) + \nabla f_0(\mathbf{x})^T (\mathbf{y} - \mathbf{x})$
\item 结合最优性条件:$\nabla f_0(\mathbf{x})^T (\mathbf{y} - \mathbf{x}) \geq 0$
\item 得到:$f_0(\mathbf{y}) \geq f_0(\mathbf{x})$
\end{enumerate}

\section{证明:必要性(最优性 $\Rightarrow$ 条件)}

\subsection{目标}

\textbf{需要证明}:如果 $\mathbf{x}$ 是最优的,则对于所有 $\mathbf{y} \in X$,有 $\nabla f_0(\mathbf{x})^T (\mathbf{y} - \mathbf{x}) \geq 0$。

\textbf{方法}:反证法

\subsection{证明步骤}

\textbf{步骤1}:假设 $\mathbf{x}$ 是最优的,但条件不成立。

即:存在可行点 $\mathbf{y} \in X$,使得:

\begin{equation}
\nabla f_0(\mathbf{x})^T (\mathbf{y} - \mathbf{x}) < 0
\end{equation}

\textbf{步骤2}:构造参数化的点。

\textbf{构造}:在连接 $\mathbf{x}$ 和 $\mathbf{y}$ 的线段上,考虑点:

\begin{equation}
\mathbf{z}(t) = t\mathbf{y} + (1-t)\mathbf{x} = \mathbf{x} + t(\mathbf{y} - \mathbf{x})
\end{equation}

其中 $t \in [0, 1]$ 是参数。

\textbf{性质}:
\begin{itemize}
\item 当 $t = 0$ 时:$\mathbf{z}(0) = \mathbf{x}$
\item 当 $t = 1$ 时:$\mathbf{z}(1) = \mathbf{y}$
\item 当 $t \in (0, 1)$ 时:$\mathbf{z}(t)$ 在连接 $\mathbf{x}$ 和 $\mathbf{y}$ 的线段上
\end{itemize}

\textbf{步骤3}:验证 $\mathbf{z}(t)$ 的可行性。

由于可行集 $X$ 是凸集,且 $\mathbf{x}, \mathbf{y} \in X$,凸组合 $\mathbf{z}(t) = t\mathbf{y} + (1-t)\mathbf{x} \in X$。

因此 $\mathbf{z}(t)$ 是可行的。✓

\textbf{步骤4}:计算 $f_0(\mathbf{z}(t))$ 在 $t = 0$ 处的导数。

\textbf{关键观察}:考虑函数 $g(t) = f_0(\mathbf{z}(t)) = f_0(\mathbf{x} + t(\mathbf{y} - \mathbf{x}))$。

\textbf{计算导数}:

使用链式法则:

\begin{align}
\frac{d}{dt} f_0(\mathbf{z}(t)) &= \frac{d}{dt} f_0(\mathbf{x} + t(\mathbf{y} - \mathbf{x})) \\
&= \nabla f_0(\mathbf{x} + t(\mathbf{y} - \mathbf{x}))^T \cdot \frac{d}{dt}[\mathbf{x} + t(\mathbf{y} - \mathbf{x})] \\
&= \nabla f_0(\mathbf{x} + t(\mathbf{y} - \mathbf{x}))^T (\mathbf{y} - \mathbf{x})
\end{equation}

\textbf{在 $t = 0$ 处}:

\begin{equation}
\frac{d}{dt} f_0(\mathbf{z}(t)) \Big|_{t=0} = \nabla f_0(\mathbf{x})^T (\mathbf{y} - \mathbf{x}) < 0
\end{equation}

\textbf{步骤5}:利用导数信息。

\textbf{关键}:由于 $\frac{d}{dt} f_0(\mathbf{z}(t)) \Big|_{t=0} < 0$,对于小的正数 $t$,有 $f_0(\mathbf{z}(t)) < f_0(\mathbf{z}(0)) = f_0(\mathbf{x})$。

\textbf{严格证明}:

根据导数的定义,存在 $\delta > 0$,使得对于 $t \in (0, \delta)$,有:

\begin{equation}
\frac{f_0(\mathbf{z}(t)) - f_0(\mathbf{z}(0))}{t} < 0
\end{equation}

因此 $f_0(\mathbf{z}(t)) < f_0(\mathbf{x})$。

\textbf{步骤6}:导出矛盾。

\begin{itemize}
\item $\mathbf{z}(t)$ 是可行的(步骤3)
\item $f_0(\mathbf{z}(t)) < f_0(\mathbf{x})$(步骤5)
\item 但 $\mathbf{x}$ 是最优的,应该有 $f_0(\mathbf{x}) \leq f_0(\mathbf{z}(t))$
\item 矛盾!
\end{itemize}

\textbf{步骤7}:结论。

假设"条件不成立"导致矛盾,因此条件必须成立。$\square$

\section{详细解释关键步骤}

\subsection{为什么考虑 $\mathbf{z}(t) = t\mathbf{y} + (1-t)\mathbf{x}$?}

\textbf{原因}:
\begin{enumerate}
\item \textbf{可行性}:由于可行集是凸集,$\mathbf{z}(t)$ 是可行的

\item \textbf{接近 $\mathbf{x}$}:当 $t$ 很小时,$\mathbf{z}(t)$ 接近 $\mathbf{x}$

\item \textbf{方向性}:$\mathbf{z}(t)$ 在从 $\mathbf{x}$ 指向 $\mathbf{y}$ 的方向上
\end{enumerate}

\textbf{几何意义}:
\begin{itemize}
\item $\mathbf{z}(t)$ 是从 $\mathbf{x}$ 出发,沿方向 $(\mathbf{y} - \mathbf{x})$ 移动的点
\item 当 $t$ 很小时,$\mathbf{z}(t)$ 在 $\mathbf{x}$ 附近
\item 如果在这个方向上的导数 $< 0$,则函数值会减小
\end{itemize}

\subsection{为什么 $\frac{d}{dt} f_0(\mathbf{z}(t)) \Big|_{t=0} < 0$ 意味着 $f_0(\mathbf{z}(t)) < f_0(\mathbf{x})$?}

\textbf{导数的定义}:

\begin{equation}
\frac{d}{dt} f_0(\mathbf{z}(t)) \Big|_{t=0} = \lim_{t \to 0^+} \frac{f_0(\mathbf{z}(t)) - f_0(\mathbf{z}(0))}{t}
\end{equation}

\textbf{如果导数 $< 0$}:

\begin{itemize}
\item 对于充分小的 $t > 0$,有 $\frac{f_0(\mathbf{z}(t)) - f_0(\mathbf{x})}{t} < 0$
\item 由于 $t > 0$,有 $f_0(\mathbf{z}(t)) - f_0(\mathbf{x}) < 0$
\item 因此 $f_0(\mathbf{z}(t)) < f_0(\mathbf{x})$
\end{itemize}

\textbf{几何直观}:
\begin{itemize}
\item 导数 $< 0$ 意味着函数在 $t = 0$ 处是递减的
\item 对于小的 $t > 0$,函数值会减小
\item 这与 $\mathbf{x}$ 是最优的矛盾
\end{itemize}

\section{几何直观}

\subsection{最优性条件的几何意义}

\textbf{条件}:$\nabla f_0(\mathbf{x})^T (\mathbf{y} - \mathbf{x}) \geq 0$ 对所有可行 $\mathbf{y}$ 成立

\textbf{几何解释}:
\begin{itemize}
\item \textbf{梯度方向}:$\nabla f_0(\mathbf{x})$ 指向函数值增加最快的方向
\item \textbf{可行方向}:$(\mathbf{y} - \mathbf{x})$ 是从 $\mathbf{x}$ 指向可行点 $\mathbf{y}$ 的方向
\item \textbf{内积非负}:所有可行方向与梯度方向的内积 $\geq 0$
\item \textbf{含义}:没有可行方向能使函数值减小
\end{itemize}

\subsection{支撑超平面}

\textbf{定义}:超平面 $H = \{\mathbf{z} \mid \mathbf{a}^T (\mathbf{z} - \mathbf{x}) = 0\}$ 是集合 $X$ 在点 $\mathbf{x}$ 处的支撑超平面,如果:
\begin{enumerate}
\item $\mathbf{x} \in H \cap X$
\item $X$ 在超平面的一侧:$\mathbf{a}^T (\mathbf{z} - \mathbf{x}) \geq 0$ 对所有 $\mathbf{z} \in X$ 成立
\end{enumerate}

\textbf{最优性条件的几何意义}:

如果 $\nabla f_0(\mathbf{x}) \neq \mathbf{0}$,则超平面:

\begin{equation}
H = \{\mathbf{z} \mid -\nabla f_0(\mathbf{x})^T (\mathbf{z} - \mathbf{x}) = 0\}
\end{equation}

是可行集 $X$ 在点 $\mathbf{x}$ 处的支撑超平面。

\textbf{验证}:
\begin{itemize}
\item $\mathbf{x} \in H \cap X$ ✓
\item 对于所有 $\mathbf{y} \in X$,有 $-\nabla f_0(\mathbf{x})^T (\mathbf{y} - \mathbf{x}) \leq 0$
\item 等价地:$\nabla f_0(\mathbf{x})^T (\mathbf{y} - \mathbf{x}) \geq 0$ ✓
\end{itemize}

\section{无约束问题的特殊情况}

\subsection{定理}

\textbf{无约束问题}:如果 $m = p = 0$(无约束),则最优性条件简化为:

\begin{equation}
\nabla f_0(\mathbf{x}) = \mathbf{0}
\end{equation}

\textbf{这是经典的必要和充分条件}。

\subsection{证明}

\textbf{步骤1}:从一般条件出发。

对于无约束问题,所有 $\mathbf{y}$ 都是可行的(只要在定义域内)。

最优性条件:$\nabla f_0(\mathbf{x})^T (\mathbf{y} - \mathbf{x}) \geq 0$ 对所有 $\mathbf{y}$ 成立。

\textbf{步骤2}:取特定的 $\mathbf{y}$。

\textbf{构造}:$\mathbf{y} = \mathbf{x} - t\nabla f_0(\mathbf{x})$,其中 $t > 0$ 是小的正数。

\textbf{为什么可行?}

由于 $f_0$ 可微,定义域是开的,对于小的 $t$,$\mathbf{y}$ 在定义域内,因此是可行的。

\textbf{步骤3}:应用最优性条件。

\begin{align}
\nabla f_0(\mathbf{x})^T (\mathbf{y} - \mathbf{x}) &= \nabla f_0(\mathbf{x})^T (\mathbf{x} - t\nabla f_0(\mathbf{x}) - \mathbf{x}) \\
&= \nabla f_0(\mathbf{x})^T (-t\nabla f_0(\mathbf{x})) \\
&= -t \|\nabla f_0(\mathbf{x})\|_2^2 \geq 0
\end{equation}

\textbf{步骤4}:得到结论。

由于 $t > 0$,有 $-t \|\nabla f_0(\mathbf{x})\|_2^2 \geq 0$,即 $\|\nabla f_0(\mathbf{x})\|_2^2 \leq 0$。

因此 $\|\nabla f_0(\mathbf{x})\|_2 = 0$,即 $\nabla f_0(\mathbf{x}) = \mathbf{0}$。$\square$

\section{具体例子}

\subsection{例子1:无约束二次优化}

\textbf{问题}:$\min_{\mathbf{x}} f_0(\mathbf{x}) = \frac{1}{2}\mathbf{x}^T \mathbf{P} \mathbf{x} + \mathbf{q}^T \mathbf{x} + r$

其中 $\mathbf{P} \succeq 0$(凸函数)。

\textbf{最优性条件}:

\begin{equation}
\nabla f_0(\mathbf{x}) = \mathbf{P}\mathbf{x} + \mathbf{q} = \mathbf{0}
\end{equation}

\textbf{情况分析}:

\begin{enumerate}
\item \textbf{如果 $\mathbf{q} \notin R(\mathbf{P})$}:
   \begin{itemize}
   \item 方程无解
   \item $f_0$ 无下界
   \end{itemize}

\item \textbf{如果 $\mathbf{P} \succ 0$(严格正定)}:
   \begin{itemize}
   \item 有唯一解:$\mathbf{x}^* = -\mathbf{P}^{-1}\mathbf{q}$
   \item 这是全局最优解
   \end{itemize}

\item \textbf{如果 $\mathbf{P} \succeq 0$(半正定)}:
   \begin{itemize}
   \item 可能有多个解
   \item 所有解都是最优解
   \end{itemize}
\end{enumerate}

\subsection{例子2:带约束的问题}

\textbf{问题}:$\min_{x, y} x^2 + y^2$ subject to $x + y = 1$

\textbf{可行集}:$X = \{(x, y) \mid x + y = 1\}$(直线)

\textbf{最优性条件}:

\begin{itemize}
\item 梯度:$\nabla f_0(x, y) = (2x, 2y)^T$
\item 条件:$\nabla f_0(x, y)^T (\mathbf{y} - \mathbf{x}) \geq 0$ 对所有可行 $\mathbf{y}$ 成立
\end{itemize}

\textbf{分析}:

由于可行集是直线 $x + y = 1$,可行方向是沿直线的方向。

设 $(x, y)$ 是最优点,则对于沿直线的任意方向 $\mathbf{d}$,有 $\nabla f_0(x, y)^T \mathbf{d} = 0$。

这等价于梯度垂直于可行集(直线)。

\section{总结}

\subsection{最优性条件}

\begin{enumerate}
\item \textbf{条件}:$\nabla f_0(\mathbf{x})^T (\mathbf{y} - \mathbf{x}) \geq 0$ 对所有可行 $\mathbf{y}$ 成立

\item \textbf{几何意义}:$-\nabla f_0(\mathbf{x})$ 定义了可行集在 $\mathbf{x}$ 处的支撑超平面

\item \textbf{无约束情况}:$\nabla f_0(\mathbf{x}) = \mathbf{0}$
\end{enumerate}

\subsection{证明方法}

\begin{enumerate}
\item \textbf{充分性}:
   \begin{itemize}
   \item 利用一阶条件
   \item 结合最优性条件
   \item 直接得到 $f_0(\mathbf{y}) \geq f_0(\mathbf{x})$
   \end{itemize}

\item \textbf{必要性}:
   \begin{itemize}
   \item 反证法
   \item 构造参数化的点 $\mathbf{z}(t)$
   \item 利用导数信息证明存在更好的点
   \item 导出矛盾
   \end{itemize}
\end{enumerate}

\subsection{关键理解}

\begin{enumerate}
\item \textbf{最优性条件}:没有可行方向能使函数值减小

\item \textbf{几何意义}:梯度定义了支撑超平面

\item \textbf{证明技巧}:参数化、导数、反证法

\item \textbf{应用}:判断最优性、设计算法
\end{enumerate}

理解这个最优性条件,是理解凸优化理论和算法的基础!

\end{document}

