\documentclass[12pt,a4paper]{article}
\usepackage[UTF8]{ctex}
\usepackage{amsmath}
\usepackage{amssymb}
\usepackage{amsthm}
\usepackage{geometry}
\geometry{left=2.5cm,right=2.5cm,top=2.5cm,bottom=2.5cm}

\title{第4.1.1节:优化问题的基本术语详解}
\subtitle{基于《Convex Optimization》第4章}
\author{}
\date{\today}

\begin{document}

\maketitle

\section{引言}

本节详细讲解优化问题的基本术语和概念。即使没有学习过前面的章节,通过本节的详细解释,您也能理解优化问题的基本结构。

\section{优化问题的标准形式}

\subsection{数学表示}

优化问题的标准形式为:

\begin{align}
\begin{array}{ll}
\text{minimize} & f_0(\mathbf{x}) \\
\text{subject to} & f_i(\mathbf{x}) \leq 0, \quad i = 1, \ldots, m \\
& h_i(\mathbf{x}) = 0, \quad i = 1, \ldots, p
\end{array}
\end{align}

其中:
\begin{itemize}
\item $\mathbf{x} \in \mathbb{R}^n$ 是\textbf{优化变量}(optimization variable)
\item $f_0: \mathbb{R}^n \to \mathbb{R}$ 是\textbf{目标函数}(objective function)或\textbf{代价函数}(cost function)
\item $f_i(\mathbf{x}) \leq 0$ 是\textbf{不等式约束}(inequality constraints),$i = 1, \ldots, m$
\item $h_i(\mathbf{x}) = 0$ 是\textbf{等式约束}(equality constraints),$i = 1, \ldots, p$
\end{itemize}

\subsection{通俗理解}

优化问题就是:\textbf{在满足某些条件的前提下,找到一个使目标函数值最小的变量值}。

\textbf{类比}:
\begin{itemize}
\item \textbf{目标函数}:就像考试分数,我们希望它越小越好(或越大越好,但这里假设最小化)
\item \textbf{约束条件}:就像考试规则,必须遵守
\item \textbf{优化变量}:就像我们的答题策略,可以调整
\end{itemize}

\section{各部分详细解释}

\subsection{优化变量(Optimization Variable)}

\textbf{定义}:$\mathbf{x} \in \mathbb{R}^n$ 是我们要寻找的变量。

\textbf{理解}:
\begin{itemize}
\item $\mathbf{x}$ 是一个 $n$ 维向量:$\mathbf{x} = (x_1, x_2, \ldots, x_n)$
\item 每个分量 $x_i$ 可以是任意实数
\item 我们的目标就是找到"最好"的 $\mathbf{x}$ 值
\end{itemize}

\textbf{例子}:
\begin{itemize}
\item 如果 $n = 2$,$\mathbf{x} = (x_1, x_2)$ 是平面上的点
\item 如果 $n = 3$,$\mathbf{x} = (x_1, x_2, x_3)$ 是空间中的点
\item 在机器学习中,$\mathbf{x}$ 可能是模型参数
\end{itemize}

\subsection{目标函数(Objective Function)}

\textbf{定义}:$f_0: \mathbb{R}^n \to \mathbb{R}$ 是一个函数,将 $n$ 维向量映射到实数。

\textbf{作用}:衡量一个解 $\mathbf{x}$ 的"好坏"。

\textbf{理解}:
\begin{itemize}
\item 对于每个 $\mathbf{x}$,$f_0(\mathbf{x})$ 给出一个数值
\item 我们希望这个数值尽可能小(最小化问题)
\item 如果问题是最大化,可以转化为最小化 $-f_0(\mathbf{x})$
\end{itemize}

\textbf{例子}:
\begin{itemize}
\item $f_0(\mathbf{x}) = x_1^2 + x_2^2$:到原点的距离的平方
\item $f_0(\mathbf{x}) = \|\mathbf{A}\mathbf{x} - \mathbf{b}\|_2^2$:最小二乘问题
\item $f_0(\mathbf{x}) = \mathbf{c}^T \mathbf{x}$:线性规划问题
\end{itemize}

\subsection{不等式约束(Inequality Constraints)}

\textbf{定义}:$f_i(\mathbf{x}) \leq 0$,$i = 1, \ldots, m$

\textbf{作用}:限制 $\mathbf{x}$ 的取值范围。

\textbf{理解}:
\begin{itemize}
\item 每个 $f_i$ 是一个函数
\item 约束 $f_i(\mathbf{x}) \leq 0$ 意味着函数值必须小于或等于0
\item 所有 $m$ 个不等式约束都必须同时满足
\end{itemize}

\textbf{例子}:
\begin{itemize}
\item $x_1 + x_2 \leq 1$:可以写成 $f_1(\mathbf{x}) = x_1 + x_2 - 1 \leq 0$
\item $x_1^2 + x_2^2 \leq 4$:可以写成 $f_1(\mathbf{x}) = x_1^2 + x_2^2 - 4 \leq 0$
\item $x_1 \geq 0$:可以写成 $f_1(\mathbf{x}) = -x_1 \leq 0$
\end{itemize}

\textbf{注意}:任何不等式都可以写成 $\leq 0$ 的形式:
\begin{itemize}
\item $g(\mathbf{x}) \leq b$ 等价于 $g(\mathbf{x}) - b \leq 0$
\item $g(\mathbf{x}) \geq b$ 等价于 $-g(\mathbf{x}) + b \leq 0$
\end{itemize}

\subsection{等式约束(Equality Constraints)}

\textbf{定义}:$h_i(\mathbf{x}) = 0$,$i = 1, \ldots, p$

\textbf{作用}:要求某些函数值必须精确等于0。

\textbf{理解}:
\begin{itemize}
\item 每个 $h_i$ 是一个函数
\item 约束 $h_i(\mathbf{x}) = 0$ 意味着函数值必须精确等于0
\item 所有 $p$ 个等式约束都必须同时满足
\end{itemize}

\textbf{例子}:
\begin{itemize}
\item $x_1 + x_2 = 1$:可以写成 $h_1(\mathbf{x}) = x_1 + x_2 - 1 = 0$
\item $\mathbf{A}\mathbf{x} = \mathbf{b}$:可以写成 $h_i(\mathbf{x}) = (\mathbf{A}\mathbf{x} - \mathbf{b})_i = 0$,$i = 1, \ldots, p$
\end{itemize}

\subsection{无约束问题(Unconstrained Problems)}

\textbf{定义}:如果 $m = 0$ 且 $p = 0$(没有约束),则问题是无约束的。

\textbf{形式}:
\begin{equation}
\text{minimize } f_0(\mathbf{x})
\end{equation}

\textbf{理解}:可以在整个 $\mathbb{R}^n$ 中自由寻找使 $f_0(\mathbf{x})$ 最小的点。

\textbf{例子}:
\begin{itemize}
\item 最小化 $f_0(x) = x^2$:最优解是 $x = 0$
\item 最小化 $f_0(\mathbf{x}) = \|\mathbf{x}\|_2^2$:最优解是 $\mathbf{x} = \mathbf{0}$
\end{itemize}

\section{可行域和可行点}

\subsection{定义域(Domain)}

\textbf{定义}:优化问题的定义域是所有函数都有定义的点集:

\begin{equation}
\mathcal{D} = \bigcap_{i=0}^m \text{dom } f_i \cap \bigcap_{i=1}^p \text{dom } h_i
\end{equation}

其中 $\text{dom } f$ 表示函数 $f$ 的定义域。

\textbf{理解}:
\begin{itemize}
\item 定义域是使所有函数都有意义的点的集合
\item 例如,如果 $f_0(x) = \log x$,则 $\text{dom } f_0 = \{x \mid x > 0\}$
\item 定义域是所有函数定义域的交集
\end{itemize}

\subsection{可行点(Feasible Point)}

\textbf{定义}:点 $\mathbf{x} \in \mathcal{D}$ 是\textbf{可行点},如果它满足所有约束:
\begin{itemize}
\item $f_i(\mathbf{x}) \leq 0$,$i = 1, \ldots, m$
\item $h_i(\mathbf{x}) = 0$,$i = 1, \ldots, p$
\end{itemize}

\textbf{理解}:
\begin{itemize}
\item 可行点是"合法"的解,满足所有约束条件
\item 我们只能在可行点中寻找最优解
\item 如果一个问题没有可行点,我们说它是不可行的
\end{itemize}

\subsection{可行集(Feasible Set)}

\textbf{定义}:所有可行点的集合称为\textbf{可行集}或\textbf{约束集}:

\begin{equation}
\mathcal{X} = \{\mathbf{x} \in \mathcal{D} \mid f_i(\mathbf{x}) \leq 0, i = 1, \ldots, m, h_i(\mathbf{x}) = 0, i = 1, \ldots, p\}
\end{equation}

\textbf{理解}:
\begin{itemize}
\item 可行集是我们可以选择的 $\mathbf{x}$ 的"候选集合"
\item 优化问题就是在可行集中找到使目标函数最小的点
\item 可行集可能是空的(不可行),也可能包含无限多个点
\end{itemize}

\textbf{例子}:
\begin{itemize}
\item 如果约束是 $x_1^2 + x_2^2 \leq 1$,可行集是单位圆盘
\item 如果约束是 $x_1 + x_2 = 1$ 且 $x_1, x_2 \geq 0$,可行集是一条线段
\end{itemize}

\subsection{可行性问题(Feasibility)}

\textbf{可行}:如果存在至少一个可行点,问题称为\textbf{可行的}(feasible)。

\textbf{不可行}:如果不存在可行点,问题称为\textbf{不可行的}(infeasible)。

\textbf{例子}:
\begin{itemize}
\item 可行:约束 $x_1 + x_2 \leq 1$,$x_1, x_2 \geq 0$(有解,如 $(0, 0)$)
\item 不可行:约束 $x_1 + x_2 \leq -1$,$x_1, x_2 \geq 0$(无解)
\end{itemize}

\section{最优值}

\subsection{定义}

\textbf{最优值}(optimal value)定义为:

\begin{equation}
p^* = \inf \{f_0(\mathbf{x}) \mid \mathbf{x} \in \mathcal{X}\}
\end{equation}

即所有可行点的目标函数值的下确界(infimum)。

\textbf{理解}:
\begin{itemize}
\item $p^*$ 是目标函数在可行集上能达到的"最好"值
\item 使用 $\inf$(下确界)而不是 $\min$(最小值),因为最小值可能不存在
\item $p^*$ 可以是 $+\infty$、$-\infty$ 或有限值
\end{itemize}

\subsection{特殊情况}

\textbf{情况1:不可行问题}

如果问题不可行(可行集为空),则:
\begin{equation}
p^* = +\infty
\end{equation}

(按照约定,空集的下确界是 $+\infty$)

\textbf{情况2:无下界问题}

如果存在可行点序列 $\{\mathbf{x}_k\}$,使得 $f_0(\mathbf{x}_k) \to -\infty$,则:
\begin{equation}
p^* = -\infty
\end{equation}

此时问题称为\textbf{无下界的}(unbounded below)。

\textbf{情况3:有界问题}

如果 $p^*$ 是有限值,问题是有界的。

\section{最优点和最优集}

\subsection{最优点(Optimal Point)}

\textbf{定义}:点 $\mathbf{x}^*$ 是\textbf{最优点}(optimal point),如果:
\begin{enumerate}
\item $\mathbf{x}^*$ 是可行的
\item $f_0(\mathbf{x}^*) = p^*$
\end{enumerate}

\textbf{理解}:
\begin{itemize}
\item 最优点是使目标函数达到最优值的可行点
\item 最优点可能不存在(即使 $p^*$ 是有限的)
\item 最优点可能不唯一
\end{itemize}

\subsection{最优集(Optimal Set)}

\textbf{定义}:所有最优点的集合:

\begin{equation}
\mathcal{X}_{\text{opt}} = \{\mathbf{x} \in \mathcal{X} \mid f_0(\mathbf{x}) = p^*\}
\end{equation}

\textbf{理解}:
\begin{itemize}
\item 如果最优集非空,我们说最优值\textbf{达到}(attained)或\textbf{实现}(achieved)
\item 如果最优集为空,我们说最优值\textbf{未达到}(not attained)
\item 无下界问题的最优集总是空的
\end{itemize}

\subsection{可解性(Solvability)}

\textbf{定义}:如果存在最优点,问题称为\textbf{可解的}(solvable)。

\textbf{注意}:
\begin{itemize}
\item 可解 $\Leftrightarrow$ 最优集非空
\item 可解 $\Leftrightarrow$ 最优值达到
\end{itemize}

\section{$\epsilon$-次优解}

\subsection{定义}

\textbf{$\epsilon$-次优点}:可行点 $\mathbf{x}$ 如果满足 $f_0(\mathbf{x}) \leq p^* + \epsilon$(其中 $\epsilon > 0$),则称为\textbf{$\epsilon$-次优}的。

\textbf{$\epsilon$-次优集}:所有 $\epsilon$-次优点的集合。

\textbf{理解}:
\begin{itemize}
\item $\epsilon$-次优解虽然不是最优的,但"足够好"
\item 在实际应用中,我们通常只能找到 $\epsilon$-次优解
\item $\epsilon$ 越小,解越接近最优
\end{itemize}

\textbf{例子}:
\begin{itemize}
\item 如果 $p^* = 10$,$\epsilon = 0.1$,则任何满足 $f_0(\mathbf{x}) \leq 10.1$ 的可行点都是 $0.1$-次优的
\item 如果 $p^* = 10$,$\epsilon = 1$,则任何满足 $f_0(\mathbf{x}) \leq 11$ 的可行点都是 $1$-次优的
\end{itemize}

\section{局部最优和全局最优}

\subsection{局部最优点(Locally Optimal Point)}

\textbf{定义}:可行点 $\mathbf{x}$ 是\textbf{局部最优}的,如果存在 $R > 0$,使得 $\mathbf{x}$ 是以下问题的解:

\begin{align}
\begin{array}{ll}
\text{minimize} & f_0(\mathbf{z}) \\
\text{subject to} & f_i(\mathbf{z}) \leq 0, \quad i = 1, \ldots, m \\
& h_i(\mathbf{z}) = 0, \quad i = 1, \ldots, p \\
& \|\mathbf{z} - \mathbf{x}\|_2 \leq R
\end{array}
\end{align}

\textbf{理解}:
\begin{itemize}
\item 局部最优点在"附近"的可行点中是最优的
\item "附近"由半径 $R$ 定义
\item 局部最优不一定是全局最优
\end{itemize}

\textbf{几何直观}:
\begin{itemize}
\item 想象在一个多峰函数上
\item 局部最优点是在某个"山谷"的底部
\item 全局最优点是在"最低的山谷"的底部
\end{itemize}

\subsection{全局最优点(Globally Optimal Point)}

\textbf{定义}:全局最优点就是在整个可行集上使目标函数最小的点。

\textbf{注意}:
\begin{itemize}
\item 本书中,\textbf{最优}(optimal)默认指全局最优
\item 局部最优和全局最优的区别很重要
\item 在凸优化中,局部最优就是全局最优(这是凸优化的优势之一)
\end{itemize}

\section{约束的活性}

\subsection{活性约束(Active Constraints)}

\textbf{定义}:对于可行点 $\mathbf{x}$:
\begin{itemize}
\item 如果 $f_i(\mathbf{x}) = 0$,我们说不等式约束 $f_i(\mathbf{x}) \leq 0$ 在 $\mathbf{x}$ 处是\textbf{活性的}(active)
\item 如果 $f_i(\mathbf{x}) < 0$,我们说不等式约束 $f_i(\mathbf{x}) \leq 0$ 在 $\mathbf{x}$ 处是\textbf{非活性的}(inactive)
\item 等式约束 $h_i(\mathbf{x}) = 0$ 在所有可行点处都是活性的
\end{itemize}

\textbf{理解}:
\begin{itemize}
\item 活性约束是"紧"的约束,正好在边界上
\item 非活性约束是"松"的约束,还有"余量"
\item 活性约束限制了进一步优化的方向
\end{itemize}

\textbf{例子}:
\begin{itemize}
\item 约束 $x_1 + x_2 \leq 1$,点 $(0.5, 0.5)$:约束是活性的($0.5 + 0.5 = 1$)
\item 约束 $x_1 + x_2 \leq 1$,点 $(0.3, 0.3)$:约束是非活性的($0.3 + 0.3 = 0.6 < 1$)
\end{itemize}

\subsection{冗余约束(Redundant Constraints)}

\textbf{定义}:如果删除某个约束后可行集不变,则该约束是\textbf{冗余的}。

\textbf{理解}:
\begin{itemize}
\item 冗余约束不影响可行集
\item 但保留它们可能有助于算法或分析
\item 识别冗余约束可以简化问题
\end{itemize}

\textbf{例子}:
\begin{itemize}
\item 约束 $x_1 \leq 1$ 和 $x_1 \leq 2$:第二个是冗余的
\item 约束 $x_1^2 + x_2^2 \leq 1$ 和 $x_1^2 + x_2^2 \leq 4$:第二个是冗余的
\end{itemize}

\section{具体例子}

\subsection{例子1:无约束问题}

考虑问题:
\begin{equation}
\text{minimize } f_0(x) = \frac{1}{x}, \quad \text{dom } f_0 = \mathbb{R}_{++}
\end{equation}

\textbf{分析}:
\begin{itemize}
\item 可行集:$\mathcal{X} = \{x \mid x > 0\}$
\item 最优值:$p^* = \inf_{x > 0} \frac{1}{x} = 0$
\item 最优值未达到:因为不存在 $x > 0$ 使得 $\frac{1}{x} = 0$
\item 最优集:$\mathcal{X}_{\text{opt}} = \emptyset$
\end{itemize}

\subsection{例子2:无下界问题}

考虑问题:
\begin{equation}
\text{minimize } f_0(x) = -\log x, \quad \text{dom } f_0 = \mathbb{R}_{++}
\end{equation}

\textbf{分析}:
\begin{itemize}
\item 可行集:$\mathcal{X} = \{x \mid x > 0\}$
\item 当 $x \to 0^+$ 时,$f_0(x) = -\log x \to +\infty$
\item 当 $x \to +\infty$ 时,$f_0(x) = -\log x \to -\infty$
\item 最优值:$p^* = -\infty$
\item 问题无下界
\end{itemize}

\subsection{例子3:有最优解的问题}

考虑问题:
\begin{equation}
\text{minimize } f_0(x) = x \log x, \quad \text{dom } f_0 = \mathbb{R}_{++}
\end{equation}

\textbf{分析}:
\begin{itemize}
\item 可行集:$\mathcal{X} = \{x \mid x > 0\}$
\item 求导:$f_0'(x) = \log x + 1 = 0$,得到 $x = e^{-1} = 1/e$
\item 最优值:$p^* = f_0(1/e) = (1/e) \log(1/e) = -1/e$
\item 最优点:$x^* = 1/e$(唯一)
\item 最优集:$\mathcal{X}_{\text{opt}} = \{1/e\}$
\end{itemize}

\section{可行性问题}

\subsection{定义}

如果目标函数恒等于零,问题变为:

\begin{align}
\begin{array}{ll}
\text{find} & \mathbf{x} \\
\text{subject to} & f_i(\mathbf{x}) \leq 0, \quad i = 1, \ldots, m \\
& h_i(\mathbf{x}) = 0, \quad i = 1, \ldots, p
\end{array}
\end{align}

这称为\textbf{可行性问题}(feasibility problem)。

\textbf{理解}:
\begin{itemize}
\item 可行性问题只关心是否存在可行点
\item 不关心目标函数的值
\item 如果可行集非空,最优值为0;如果可行集为空,最优值为 $+\infty$
\end{itemize}

\textbf{应用}:
\begin{itemize}
\item 检查约束是否相容
\item 寻找满足所有条件的解
\item 作为优化问题的子问题
\end{itemize}

\section{总结}

\begin{enumerate}
\item \textbf{优化问题的结构}:
   \begin{itemize}
   \item 优化变量:$\mathbf{x} \in \mathbb{R}^n$
   \item 目标函数:$f_0(\mathbf{x})$(要最小化)
   \item 不等式约束:$f_i(\mathbf{x}) \leq 0$
   \item 等式约束:$h_i(\mathbf{x}) = 0$
   \end{itemize}

\item \textbf{可行性和最优性}:
   \begin{itemize}
   \item 可行点:满足所有约束的点
   \item 可行集:所有可行点的集合
   \item 最优值:$p^* = \inf \{f_0(\mathbf{x}) \mid \mathbf{x} \in \mathcal{X}\}$
   \item 最优点:使 $f_0(\mathbf{x}) = p^*$ 的可行点
   \end{itemize}

\item \textbf{特殊情况}:
   \begin{itemize}
   \item 不可行:可行集为空,$p^* = +\infty$
   \item 无下界:$p^* = -\infty$
   \item 最优值未达到:最优集为空
   \end{itemize}

\item \textbf{局部 vs 全局}:
   \begin{itemize}
   \item 局部最优:在附近区域最优
   \item 全局最优:在整个可行集上最优
   \end{itemize}

\item \textbf{约束的活性}:
   \begin{itemize}
   \item 活性约束:$f_i(\mathbf{x}) = 0$(在边界上)
   \item 非活性约束:$f_i(\mathbf{x}) < 0$(有余量)
   \end{itemize}
\end{enumerate}

掌握这些基本术语是学习凸优化的基础!

\end{document}

