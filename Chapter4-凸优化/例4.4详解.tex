\documentclass[12pt,a4paper]{article}
\usepackage[UTF8]{ctex}
\usepackage{amsmath}
\usepackage{amssymb}
\usepackage{amsthm}
\usepackage{geometry}
\geometry{left=2.5cm,right=2.5cm,top=2.5cm,bottom=2.5cm}

\title{例4.4详解:最小化二次函数(部分变量无约束)}
\subtitle{理解如何通过消除无约束变量简化问题}
\author{}
\date{\today}

\begin{document}

\maketitle

\section{问题提出}

例4.4考虑以下优化问题:

\begin{align}
\begin{array}{ll}
\text{minimize} & \mathbf{x}_1^T \mathbf{P}_{11} \mathbf{x}_1 + 2\mathbf{x}_1^T \mathbf{P}_{12} \mathbf{x}_2 + \mathbf{x}_2^T \mathbf{P}_{22} \mathbf{x}_2 \\
\text{subject to} & f_i(\mathbf{x}_1) \leq 0, \quad i = 1, \ldots, m
\end{array}
\end{align}

\textbf{关键特点}:
\begin{itemize}
\item 目标函数是严格凸的二次函数
\item 变量分为两部分:$\mathbf{x}_1$(有约束)和 $\mathbf{x}_2$(无约束)
\item 约束只涉及 $\mathbf{x}_1$,不涉及 $\mathbf{x}_2$
\end{itemize}

\textbf{问题}:如何通过消除无约束变量 $\mathbf{x}_2$ 来简化这个问题?

\section{问题分析}

\subsection{变量分解}

将变量 $\mathbf{x}$ 分解为两部分:
\begin{equation}
\mathbf{x} = \begin{pmatrix} \mathbf{x}_1 \\ \mathbf{x}_2 \end{pmatrix}
\end{equation}

其中:
\begin{itemize}
\item $\mathbf{x}_1 \in \mathbb{R}^{n_1}$:有约束的变量
\item $\mathbf{x}_2 \in \mathbb{R}^{n_2}$:无约束的变量
\item $n_1 + n_2 = n$
\end{itemize}

\subsection{目标函数的结构}

目标函数可以写成:
\begin{equation}
f_0(\mathbf{x}_1, \mathbf{x}_2) = \mathbf{x}_1^T \mathbf{P}_{11} \mathbf{x}_1 + 2\mathbf{x}_1^T \mathbf{P}_{12} \mathbf{x}_2 + \mathbf{x}_2^T \mathbf{P}_{22} \mathbf{x}_2
\end{equation}

\textbf{矩阵分块形式}:

如果我们将完整的二次型写成:
\begin{equation}
f_0(\mathbf{x}) = \mathbf{x}^T \mathbf{P} \mathbf{x}
\end{equation}

其中 $\mathbf{P}$ 是分块矩阵:
\begin{equation}
\mathbf{P} = \begin{pmatrix} \mathbf{P}_{11} & \mathbf{P}_{12} \\ \mathbf{P}_{12}^T & \mathbf{P}_{22} \end{pmatrix}
\end{equation}

展开:
\begin{align}
\mathbf{x}^T \mathbf{P} \mathbf{x} &= \begin{pmatrix} \mathbf{x}_1^T & \mathbf{x}_2^T \end{pmatrix} \begin{pmatrix} \mathbf{P}_{11} & \mathbf{P}_{12} \\ \mathbf{P}_{12}^T & \mathbf{P}_{22} \end{pmatrix} \begin{pmatrix} \mathbf{x}_1 \\ \mathbf{x}_2 \end{pmatrix} \\
&= \mathbf{x}_1^T \mathbf{P}_{11} \mathbf{x}_1 + \mathbf{x}_1^T \mathbf{P}_{12} \mathbf{x}_2 + \mathbf{x}_2^T \mathbf{P}_{12}^T \mathbf{x}_1 + \mathbf{x}_2^T \mathbf{P}_{22} \mathbf{x}_2 \\
&= \mathbf{x}_1^T \mathbf{P}_{11} \mathbf{x}_1 + 2\mathbf{x}_1^T \mathbf{P}_{12} \mathbf{x}_2 + \mathbf{x}_2^T \mathbf{P}_{22} \mathbf{x}_2
\end{align}

(注意:$\mathbf{x}_1^T \mathbf{P}_{12} \mathbf{x}_2 = \mathbf{x}_2^T \mathbf{P}_{12}^T \mathbf{x}_1$,因为都是标量)

\section{关键思想:先优化无约束变量}

\subsection{核心观察}

由于 $\mathbf{x}_2$ 无约束,对于固定的 $\mathbf{x}_1$,我们可以先找到使目标函数最小的 $\mathbf{x}_2$。

\textbf{思路}:
\begin{enumerate}
\item 对于固定的 $\mathbf{x}_1$,最小化关于 $\mathbf{x}_2$ 的函数
\item 得到最优的 $\mathbf{x}_2^*(\mathbf{x}_1)$(作为 $\mathbf{x}_1$ 的函数)
\item 将 $\mathbf{x}_2^*(\mathbf{x}_1)$ 代入原目标函数,得到只关于 $\mathbf{x}_1$ 的函数
\item 然后优化关于 $\mathbf{x}_1$ 的问题
\end{enumerate}

\subsection{数学表述}

对于固定的 $\mathbf{x}_1$,考虑:
\begin{equation}
g(\mathbf{x}_1) = \inf_{\mathbf{x}_2} \left[\mathbf{x}_1^T \mathbf{P}_{11} \mathbf{x}_1 + 2\mathbf{x}_1^T \mathbf{P}_{12} \mathbf{x}_2 + \mathbf{x}_2^T \mathbf{P}_{22} \mathbf{x}_2\right]
\end{equation}

由于 $\mathbf{x}_1^T \mathbf{P}_{11} \mathbf{x}_1$ 不依赖于 $\mathbf{x}_2$,可以提出:
\begin{equation}
g(\mathbf{x}_1) = \mathbf{x}_1^T \mathbf{P}_{11} \mathbf{x}_1 + \inf_{\mathbf{x}_2} \left[2\mathbf{x}_1^T \mathbf{P}_{12} \mathbf{x}_2 + \mathbf{x}_2^T \mathbf{P}_{22} \mathbf{x}_2\right]
\end{equation}

\section{关于 $\mathbf{x}_2$ 的最小化}

\subsection{目标}

最小化关于 $\mathbf{x}_2$ 的二次函数:
\begin{equation}
h(\mathbf{x}_2) = 2\mathbf{x}_1^T \mathbf{P}_{12} \mathbf{x}_2 + \mathbf{x}_2^T \mathbf{P}_{22} \mathbf{x}_2
\end{equation}

\subsection{求导(梯度方法)}

对 $\mathbf{x}_2$ 求梯度并令其为零:

\begin{align}
\nabla_{\mathbf{x}_2} h(\mathbf{x}_2) &= 2\mathbf{P}_{12}^T \mathbf{x}_1 + 2\mathbf{P}_{22} \mathbf{x}_2 = \mathbf{0} \\
\Rightarrow \quad \mathbf{P}_{22} \mathbf{x}_2 &= -\mathbf{P}_{12}^T \mathbf{x}_1
\end{align}

\textbf{关键假设}:$\mathbf{P}_{22} \succ 0$(严格正定),因此可逆。

\textbf{最优解}:
\begin{equation}
\mathbf{x}_2^* = -\mathbf{P}_{22}^{-1} \mathbf{P}_{12}^T \mathbf{x}_1
\end{equation}

\subsection{验证二阶条件}

由于 $\mathbf{P}_{22} \succ 0$,Hessian矩阵是正定的,所以这是最小值点。

\section{代入得到简化形式}

\subsection{计算最优值}

将 $\mathbf{x}_2^* = -\mathbf{P}_{22}^{-1} \mathbf{P}_{12}^T \mathbf{x}_1$ 代入原目标函数:

\begin{align}
g(\mathbf{x}_1) &= \mathbf{x}_1^T \mathbf{P}_{11} \mathbf{x}_1 + 2\mathbf{x}_1^T \mathbf{P}_{12} (-\mathbf{P}_{22}^{-1} \mathbf{P}_{12}^T \mathbf{x}_1) + (-\mathbf{P}_{22}^{-1} \mathbf{P}_{12}^T \mathbf{x}_1)^T \mathbf{P}_{22} (-\mathbf{P}_{22}^{-1} \mathbf{P}_{12}^T \mathbf{x}_1) \\
&= \mathbf{x}_1^T \mathbf{P}_{11} \mathbf{x}_1 - 2\mathbf{x}_1^T \mathbf{P}_{12} \mathbf{P}_{22}^{-1} \mathbf{P}_{12}^T \mathbf{x}_1 + \mathbf{x}_1^T \mathbf{P}_{12} \mathbf{P}_{22}^{-1} \mathbf{P}_{22} \mathbf{P}_{22}^{-1} \mathbf{P}_{12}^T \mathbf{x}_1 \\
&= \mathbf{x}_1^T \mathbf{P}_{11} \mathbf{x}_1 - 2\mathbf{x}_1^T \mathbf{P}_{12} \mathbf{P}_{22}^{-1} \mathbf{P}_{12}^T \mathbf{x}_1 + \mathbf{x}_1^T \mathbf{P}_{12} \mathbf{P}_{22}^{-1} \mathbf{P}_{12}^T \mathbf{x}_1 \\
&= \mathbf{x}_1^T \mathbf{P}_{11} \mathbf{x}_1 - \mathbf{x}_1^T \mathbf{P}_{12} \mathbf{P}_{22}^{-1} \mathbf{P}_{12}^T \mathbf{x}_1 \\
&= \mathbf{x}_1^T \left(\mathbf{P}_{11} - \mathbf{P}_{12} \mathbf{P}_{22}^{-1} \mathbf{P}_{12}^T\right) \mathbf{x}_1
\end{align}

\subsection{Schur补}

\textbf{关键结果}:
\begin{equation}
\inf_{\mathbf{x}_2} f_0(\mathbf{x}_1, \mathbf{x}_2) = \mathbf{x}_1^T \left(\mathbf{P}_{11} - \mathbf{P}_{12} \mathbf{P}_{22}^{-1} \mathbf{P}_{12}^T\right) \mathbf{x}_1
\end{equation}

矩阵 $\mathbf{P}_{11} - \mathbf{P}_{12} \mathbf{P}_{22}^{-1} \mathbf{P}_{12}^T$ 称为 $\mathbf{P}$ 关于 $\mathbf{P}_{22}$ 的\textbf{Schur补}(Schur Complement)。

\section{等价问题}

\subsection{简化后的问题}

原问题等价于:

\begin{align}
\begin{array}{ll}
\text{minimize} & \mathbf{x}_1^T \left(\mathbf{P}_{11} - \mathbf{P}_{12} \mathbf{P}_{22}^{-1} \mathbf{P}_{12}^T\right) \mathbf{x}_1 \\
\text{subject to} & f_i(\mathbf{x}_1) \leq 0, \quad i = 1, \ldots, m
\end{array}
\end{align}

\textbf{优势}:
\begin{itemize}
\item 变量数减少:从 $n$ 个减少到 $n_1$ 个
\item 消除了无约束变量 $\mathbf{x}_2$
\item 问题维度降低
\end{itemize}

\section{详细推导步骤}

\subsection{步骤1:固定 $\mathbf{x}_1$,优化 $\mathbf{x}_2$}

对于固定的 $\mathbf{x}_1$,最小化:
\begin{equation}
\min_{\mathbf{x}_2} \left[2\mathbf{x}_1^T \mathbf{P}_{12} \mathbf{x}_2 + \mathbf{x}_2^T \mathbf{P}_{22} \mathbf{x}_2\right]
\end{equation}

\textbf{求梯度}:
\begin{equation}
\nabla_{\mathbf{x}_2} = 2\mathbf{P}_{12}^T \mathbf{x}_1 + 2\mathbf{P}_{22} \mathbf{x}_2
\end{equation}

\textbf{令梯度为零}:
\begin{equation}
2\mathbf{P}_{12}^T \mathbf{x}_1 + 2\mathbf{P}_{22} \mathbf{x}_2 = \mathbf{0}
\end{equation}

\textbf{解出 $\mathbf{x}_2$}:
\begin{equation}
\mathbf{x}_2^* = -\mathbf{P}_{22}^{-1} \mathbf{P}_{12}^T \mathbf{x}_1
\end{equation}

\subsection{步骤2:代入原目标函数}

将 $\mathbf{x}_2^*$ 代入:
\begin{align}
f_0(\mathbf{x}_1, \mathbf{x}_2^*) &= \mathbf{x}_1^T \mathbf{P}_{11} \mathbf{x}_1 + 2\mathbf{x}_1^T \mathbf{P}_{12} (-\mathbf{P}_{22}^{-1} \mathbf{P}_{12}^T \mathbf{x}_1) \\
&\quad + (-\mathbf{P}_{22}^{-1} \mathbf{P}_{12}^T \mathbf{x}_1)^T \mathbf{P}_{22} (-\mathbf{P}_{22}^{-1} \mathbf{P}_{12}^T \mathbf{x}_1)
\end{align}

\subsection{步骤3:简化表达式}

\textbf{第二项}:
\begin{equation}
2\mathbf{x}_1^T \mathbf{P}_{12} (-\mathbf{P}_{22}^{-1} \mathbf{P}_{12}^T \mathbf{x}_1) = -2\mathbf{x}_1^T \mathbf{P}_{12} \mathbf{P}_{22}^{-1} \mathbf{P}_{12}^T \mathbf{x}_1
\end{equation}

\textbf{第三项}:
\begin{align}
(-\mathbf{P}_{22}^{-1} \mathbf{P}_{12}^T \mathbf{x}_1)^T \mathbf{P}_{22} (-\mathbf{P}_{22}^{-1} \mathbf{P}_{12}^T \mathbf{x}_1) &= \mathbf{x}_1^T \mathbf{P}_{12} \mathbf{P}_{22}^{-T} \mathbf{P}_{22} \mathbf{P}_{22}^{-1} \mathbf{P}_{12}^T \mathbf{x}_1 \\
&= \mathbf{x}_1^T \mathbf{P}_{12} \mathbf{P}_{22}^{-1} \mathbf{P}_{12}^T \mathbf{x}_1
\end{align}

(因为 $\mathbf{P}_{22}$ 对称,所以 $\mathbf{P}_{22}^{-T} = \mathbf{P}_{22}^{-1}$)

\textbf{合并}:
\begin{align}
f_0(\mathbf{x}_1, \mathbf{x}_2^*) &= \mathbf{x}_1^T \mathbf{P}_{11} \mathbf{x}_1 - 2\mathbf{x}_1^T \mathbf{P}_{12} \mathbf{P}_{22}^{-1} \mathbf{P}_{12}^T \mathbf{x}_1 + \mathbf{x}_1^T \mathbf{P}_{12} \mathbf{P}_{22}^{-1} \mathbf{P}_{12}^T \mathbf{x}_1 \\
&= \mathbf{x}_1^T \mathbf{P}_{11} \mathbf{x}_1 - \mathbf{x}_1^T \mathbf{P}_{12} \mathbf{P}_{22}^{-1} \mathbf{P}_{12}^T \mathbf{x}_1 \\
&= \mathbf{x}_1^T \left(\mathbf{P}_{11} - \mathbf{P}_{12} \mathbf{P}_{22}^{-1} \mathbf{P}_{12}^T\right) \mathbf{x}_1
\end{align}

\section{Schur补}

\subsection{定义}

对于分块矩阵:
\begin{equation}
\mathbf{P} = \begin{pmatrix} \mathbf{P}_{11} & \mathbf{P}_{12} \\ \mathbf{P}_{12}^T & \mathbf{P}_{22} \end{pmatrix}
\end{equation}

其中 $\mathbf{P}_{22}$ 可逆,$\mathbf{P}$ 关于 $\mathbf{P}_{22}$ 的\textbf{Schur补}定义为:

\begin{equation}
\mathbf{P}_{11} - \mathbf{P}_{12} \mathbf{P}_{22}^{-1} \mathbf{P}_{12}^T
\end{equation}

\subsection{性质}

\begin{itemize}
\item 如果 $\mathbf{P} \succ 0$,则 Schur补也是正定的
\item Schur补在矩阵理论中非常重要
\item 在优化中,Schur补经常出现在消除变量后
\end{itemize}

\section{具体例子}

\subsection{例子1:简单的2维情况}

\textbf{问题}:
\begin{align}
\text{minimize} \quad & x_1^2 + 2x_1 x_2 + 2x_2^2 \\
\text{subject to} \quad & x_1 \geq 0
\end{align}

\textbf{分析}:
\begin{itemize}
\item $\mathbf{x}_1 = x_1$(1维,有约束)
\item $\mathbf{x}_2 = x_2$(1维,无约束)
\item $\mathbf{P}_{11} = 1$,$\mathbf{P}_{12} = 1$,$\mathbf{P}_{22} = 2$
\end{itemize}

\textbf{步骤1:优化 $x_2$}

对于固定的 $x_1$,最小化 $2x_1 x_2 + 2x_2^2$:
\begin{equation}
\frac{d}{dx_2}(2x_1 x_2 + 2x_2^2) = 2x_1 + 4x_2 = 0 \quad \Rightarrow \quad x_2^* = -\frac{x_1}{2}
\end{equation}

\textbf{步骤2:代入}

\begin{align}
f_0(x_1, x_2^*) &= x_1^2 + 2x_1 \left(-\frac{x_1}{2}\right) + 2\left(-\frac{x_1}{2}\right)^2 \\
&= x_1^2 - x_1^2 + 2 \cdot \frac{x_1^2}{4} \\
&= \frac{x_1^2}{2}
\end{align}

\textbf{使用公式验证}:

\begin{equation}
\mathbf{P}_{11} - \mathbf{P}_{12} \mathbf{P}_{22}^{-1} \mathbf{P}_{12}^T = 1 - 1 \cdot \frac{1}{2} \cdot 1 = 1 - \frac{1}{2} = \frac{1}{2}
\end{equation}

因此:$f_0(x_1, x_2^*) = \frac{1}{2} x_1^2$ ✓

\textbf{简化后的问题}:
\begin{align}
\text{minimize} \quad & \frac{1}{2} x_1^2 \\
\text{subject to} \quad & x_1 \geq 0
\end{align}

最优解:$x_1^* = 0$,$x_2^* = 0$。

\subsection{例子2:3维情况}

\textbf{问题}:
\begin{align}
\text{minimize} \quad & x_1^2 + 2x_1 x_2 + x_2^2 + 2x_2 x_3 + 2x_3^2 \\
\text{subject to} \quad & x_1 \geq 0
\end{align}

\textbf{分析}:
\begin{itemize}
\item $\mathbf{x}_1 = x_1$(1维)
\item $\mathbf{x}_2 = (x_2, x_3)^T$(2维)
\item 矩阵分块:
  \begin{equation}
  \mathbf{P} = \begin{pmatrix} 1 & 1 & 0 \\ 1 & 1 & 1 \\ 0 & 1 & 2 \end{pmatrix} = \begin{pmatrix} \mathbf{P}_{11} & \mathbf{P}_{12} \\ \mathbf{P}_{12}^T & \mathbf{P}_{22} \end{pmatrix}
  \end{equation}
  其中:
  \begin{align}
  \mathbf{P}_{11} &= 1 \\
  \mathbf{P}_{12} &= \begin{pmatrix} 1 & 0 \end{pmatrix} \\
  \mathbf{P}_{22} &= \begin{pmatrix} 1 & 1 \\ 1 & 2 \end{pmatrix}
  \end{align}
\end{itemize}

\textbf{计算 Schur补}:

\begin{align}
\mathbf{P}_{22}^{-1} &= \begin{pmatrix} 1 & 1 \\ 1 & 2 \end{pmatrix}^{-1} = \begin{pmatrix} 2 & -1 \\ -1 & 1 \end{pmatrix} \\
\mathbf{P}_{12} \mathbf{P}_{22}^{-1} \mathbf{P}_{12}^T &= \begin{pmatrix} 1 & 0 \end{pmatrix} \begin{pmatrix} 2 & -1 \\ -1 & 1 \end{pmatrix} \begin{pmatrix} 1 \\ 0 \end{pmatrix} = 2
\end{align}

因此:
\begin{equation}
\mathbf{P}_{11} - \mathbf{P}_{12} \mathbf{P}_{22}^{-1} \mathbf{P}_{12}^T = 1 - 2 = -1
\end{equation}

\textbf{简化后的问题}:
\begin{align}
\text{minimize} \quad & -x_1^2 \\
\text{subject to} \quad & x_1 \geq 0
\end{align}

注意:这里目标函数是 $-x_1^2$,在 $x_1 \geq 0$ 的约束下,最优值趋向于 $-\infty$(无下界)。

\section{为什么需要 $\mathbf{P}_{22} \succ 0$?}

\subsection{严格正定的要求}

\textbf{原因1:可逆性}

如果 $\mathbf{P}_{22}$ 不是正定的,可能不可逆,无法计算 $\mathbf{P}_{22}^{-1}$。

\textbf{原因2:唯一最优解}

如果 $\mathbf{P}_{22} \succ 0$,则关于 $\mathbf{x}_2$ 的优化问题有唯一最优解。

如果 $\mathbf{P}_{22}$ 只是半正定,可能有无穷多个最优解。

\textbf{原因3:严格凸性}

如果 $\mathbf{P}_{22} \succ 0$,则关于 $\mathbf{x}_2$ 的二次函数是严格凸的,保证有唯一最小值。

\section{几何直观}

\subsection{等高线理解}

\begin{itemize}
\item 原目标函数 $f_0(\mathbf{x}_1, \mathbf{x}_2)$ 的等高线是椭圆(在 $(\mathbf{x}_1, \mathbf{x}_2)$ 空间中)
\item 对于固定的 $\mathbf{x}_1$,在 $\mathbf{x}_2$ 方向上找到最低点
\item 这些最低点构成新的目标函数 $g(\mathbf{x}_1)$
\item $g(\mathbf{x}_1)$ 的等高线是 $f_0$ 的"投影"
\end{itemize}

\subsection{降维理解}

\begin{itemize}
\item 原问题在 $n$ 维空间中
\item 通过消除 $\mathbf{x}_2$,问题降维到 $n_1$ 维空间
\item 新问题的可行域是原问题可行域在 $\mathbf{x}_1$ 子空间上的投影
\end{itemize}

\section{应用}

\subsection{在优化算法中}

\begin{itemize}
\item 某些算法可以利用这种结构
\item 先优化无约束变量,再优化有约束变量
\item 减少问题维度,提高计算效率
\end{itemize}

\subsection{在问题建模中}

\begin{itemize}
\item 如果某些变量自然无约束,可以利用这个技巧
\item 例如:某些中间变量、辅助变量等
\end{itemize}

\section{总结}

\begin{enumerate}
\item \textbf{核心思想}:
   \begin{itemize}
   \item 对于无约束变量,可以先优化
   \item 得到最优值作为有约束变量的函数
   \item 然后优化关于有约束变量的问题
   \end{itemize}

\item \textbf{关键公式}:
   \begin{equation}
   \inf_{\mathbf{x}_2} f_0(\mathbf{x}_1, \mathbf{x}_2) = \mathbf{x}_1^T \left(\mathbf{P}_{11} - \mathbf{P}_{12} \mathbf{P}_{22}^{-1} \mathbf{P}_{12}^T\right) \mathbf{x}_1
   \end{equation}

\item \textbf{Schur补}:
   \begin{itemize}
   \item $\mathbf{P}_{11} - \mathbf{P}_{12} \mathbf{P}_{22}^{-1} \mathbf{P}_{12}^T$ 是 Schur补
   \item 出现在消除变量后的简化形式中
   \end{itemize}

\item \textbf{效果}:
   \begin{itemize}
   \item 变量数减少:从 $n$ 到 $n_1$
   \item 问题维度降低
   \item 保持等价性
   \end{itemize}

\item \textbf{条件}:
   \begin{itemize}
   \item 需要 $\mathbf{P}_{22} \succ 0$(严格正定)
   \item 保证可逆性和唯一最优解
   \end{itemize}
\end{enumerate}

理解这个例子,有助于掌握通过消除变量简化优化问题的方法!

\end{document}


