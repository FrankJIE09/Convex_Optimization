\documentclass[12pt,a4paper]{article}
\usepackage[UTF8]{ctex}
\usepackage{amsmath}
\usepackage{amssymb}
\usepackage{amsthm}
\usepackage{geometry}
\usepackage{tikz}
\usetikzlibrary{shapes.geometric}
\geometry{left=2.5cm,right=2.5cm,top=2.5cm,bottom=2.5cm}

\title{仿射集合与凸集合}
\author{}
\date{\today}

\begin{document}

\maketitle

\section{引言}

仿射集合(Affine Set)和凸集合(Convex Set)是凸优化理论中的两个基本概念。理解它们的定义、性质和关系,对于深入学习凸优化至关重要。本文将详细介绍这两个概念,并通过具体例子说明它们的区别和联系。

\section{仿射集合}

\subsection{定义}

集合 $C \subseteq \mathbb{R}^n$ 称为\textbf{仿射集合}(Affine Set),如果对于任意 $\mathbf{x}_1, \mathbf{x}_2 \in C$ 和任意 $\theta \in \mathbb{R}$,都有:

\begin{equation}
\theta \mathbf{x}_1 + (1-\theta) \mathbf{x}_2 \in C
\end{equation}

换句话说,仿射集合包含通过其中任意两点的\textbf{整条直线}。

\subsection{几何意义}

仿射集合在几何上可以理解为"平移后的线性子空间"。具体来说:
\begin{itemize}
\item 如果仿射集合包含原点,那么它就是一个线性子空间
\item 如果仿射集合不包含原点,那么它是某个线性子空间的平移
\item 仿射集合是"无界"的,向两个方向无限延伸
\end{itemize}

\subsection{例子}

\textbf{例子1}:在 $\mathbb{R}^2$ 中,任意一条直线都是仿射集合。例如,直线 $y = 2x + 1$ 是一个仿射集合。

\textbf{例子2}:在 $\mathbb{R}^3$ 中,任意一个平面都是仿射集合。例如,平面 $x + y + z = 1$ 是一个仿射集合。

\textbf{例子3}:整个空间 $\mathbb{R}^n$ 本身是一个仿射集合。

\textbf{例子4}:空集 $\emptyset$ 和单点集 $\{\mathbf{x}_0\}$ 也是仿射集合(平凡情况)。

\textbf{反例}:在 $\mathbb{R}^2$ 中,去掉一个点的直线\textbf{不是}仿射集合,因为通过剩余两点的直线会经过被去掉的点。

\subsection{仿射组合}

更一般地,如果集合 $C$ 是仿射集合,那么对于任意 $k$ 个点 $\mathbf{x}_1, \mathbf{x}_2, \ldots, \mathbf{x}_k \in C$ 和满足 $\sum_{i=1}^k \theta_i = 1$ 的系数 $\theta_1, \theta_2, \ldots, \theta_k \in \mathbb{R}$,都有:

\begin{equation}
\theta_1 \mathbf{x}_1 + \theta_2 \mathbf{x}_2 + \cdots + \theta_k \mathbf{x}_k \in C
\end{equation}

这称为\textbf{仿射组合}(Affine Combination)。注意这里系数之和必须等于1,但系数可以是任意实数(正数、负数或零)。

\section{凸集合}

\subsection{定义}

集合 $C \subseteq \mathbb{R}^n$ 称为\textbf{凸集合}(Convex Set),如果对于任意 $\mathbf{x}_1, \mathbf{x}_2 \in C$ 和任意 $\theta \in [0, 1]$,都有:

\begin{equation}
\theta \mathbf{x}_1 + (1-\theta) \mathbf{x}_2 \in C
\end{equation}

换句话说,凸集合包含连接其中任意两点的\textbf{线段}。

\subsection{几何意义}

凸集合的几何特征是:连接集合中任意两点的线段完全包含在集合内。直观上,凸集合"没有凹陷",形状是"向外凸出"的。

\subsection{例子}

\textbf{例子1}:在 $\mathbb{R}^2$ 中,圆盘(包括边界)是凸集合。

\textbf{例子2}:在 $\mathbb{R}^2$ 中,三角形、矩形、椭圆等都是凸集合。

\textbf{例子3}:在 $\mathbb{R}^n$ 中,任意超平面(hyperplane)$\{\mathbf{x} \mid \mathbf{a}^T \mathbf{x} = b\}$ 是凸集合。

\textbf{例子4}:在 $\mathbb{R}^n$ 中,任意半空间(halfspace)$\{\mathbf{x} \mid \mathbf{a}^T \mathbf{x} \leq b\}$ 是凸集合。

\textbf{例子5}:在 $\mathbb{R}^n$ 中,球 $\{\mathbf{x} \mid \|\mathbf{x} - \mathbf{x}_c\|_2 \leq r\}$ 是凸集合。

\textbf{反例}:在 $\mathbb{R}^2$ 中,月牙形、星形、字母"C"的形状等都不是凸集合,因为存在两点,连接它们的线段不完全在集合内。

\subsection{凸组合}

更一般地,如果集合 $C$ 是凸集合,那么对于任意 $k$ 个点 $\mathbf{x}_1, \mathbf{x}_2, \ldots, \mathbf{x}_k \in C$ 和满足 $\sum_{i=1}^k \theta_i = 1$ 且 $\theta_i \geq 0$ 的系数 $\theta_1, \theta_2, \ldots, \theta_k$,都有:

\begin{equation}
\theta_1 \mathbf{x}_1 + \theta_2 \mathbf{x}_2 + \cdots + \theta_k \mathbf{x}_k \in C
\end{equation}

这称为\textbf{凸组合}(Convex Combination)。注意这里系数之和必须等于1,且所有系数必须非负。

\section{仿射集合与凸集合的关系}

\subsection{包含关系}

\textbf{重要结论}:\textbf{所有仿射集合都是凸集合},但反之不成立。

\begin{proof}
设 $C$ 是仿射集合。对于任意 $\mathbf{x}_1, \mathbf{x}_2 \in C$ 和 $\theta \in [0, 1]$,由于 $C$ 是仿射集合,对于任意 $\theta \in \mathbb{R}$(包括 $[0,1]$),都有:
\begin{equation}
\theta \mathbf{x}_1 + (1-\theta) \mathbf{x}_2 \in C
\end{equation}
因此 $C$ 也是凸集合。
\end{proof}

\textbf{反例}:在 $\mathbb{R}^2$ 中,单位圆盘 $\{\mathbf{x} \mid \|\mathbf{x}\|_2 \leq 1\}$ 是凸集合,但不是仿射集合,因为它是有界的,而仿射集合必须是无界的(除非是单点集或空集)。

\subsection{关键区别}

\begin{table}[h]
\centering
\begin{tabular}{|l|l|l|}
\hline
\textbf{性质} & \textbf{仿射集合} & \textbf{凸集合} \\
\hline
系数范围 & $\theta \in \mathbb{R}$(任意实数) & $\theta \in [0, 1]$(0到1之间) \\
\hline
包含内容 & 通过两点的整条直线 & 连接两点的线段 \\
\hline
有界性 & 通常无界(除单点集和空集) & 可以有界 \\
\hline
例子 & 直线、平面、整个空间 & 圆盘、三角形、球、半空间 \\
\hline
\end{tabular}
\caption{仿射集合与凸集合的区别}
\end{table}

\subsection{具体对比例子}

\textbf{例子1}:考虑 $\mathbb{R}^2$ 中的直线 $L = \{(x, y) \mid y = x\}$

\begin{itemize}
\item $L$ 是仿射集合:对于任意两点 $(x_1, x_1), (x_2, x_2) \in L$ 和任意 $\theta \in \mathbb{R}$,点 $\theta(x_1, x_1) + (1-\theta)(x_2, x_2) = (\theta x_1 + (1-\theta)x_2, \theta x_1 + (1-\theta)x_2)$ 仍在直线 $y = x$ 上。
\item $L$ 也是凸集合:因为所有仿射集合都是凸集合。
\end{itemize}

\textbf{例子2}:考虑 $\mathbb{R}^2$ 中的单位圆盘 $D = \{(x, y) \mid x^2 + y^2 \leq 1\}$

\begin{itemize}
\item $D$ 是凸集合:对于任意两点 $\mathbf{x}_1, \mathbf{x}_2 \in D$ 和 $\theta \in [0, 1]$,点 $\theta \mathbf{x}_1 + (1-\theta) \mathbf{x}_2$ 仍在圆盘内(这可以通过三角不等式证明)。
\item $D$ 不是仿射集合:例如,取两点 $(1, 0)$ 和 $(-1, 0)$,当 $\theta = 2$ 时,点 $2(1, 0) + (1-2)(-1, 0) = (3, 0)$ 不在圆盘内。
\end{itemize}

\textbf{例子3}:考虑 $\mathbb{R}^2$ 中的半平面 $H = \{(x, y) \mid x \geq 0\}$

\begin{itemize}
\item $H$ 是凸集合:对于任意两点 $(x_1, y_1), (x_2, y_2) \in H$(即 $x_1 \geq 0, x_2 \geq 0$)和 $\theta \in [0, 1]$,有:
\begin{equation}
\theta x_1 + (1-\theta) x_2 \geq 0
\end{equation}
因此 $H$ 是凸集合。
\item $H$ 不是仿射集合:例如,取两点 $(1, 0)$ 和 $(2, 0)$,当 $\theta = -1$ 时,点 $-1(1, 0) + (1-(-1))(2, 0) = (-1, 0)$ 不在半平面内。
\end{itemize}

\section{仿射包与凸包}

\subsection{仿射包}

对于任意集合 $S \subseteq \mathbb{R}^n$,包含 $S$ 的最小仿射集合称为 $S$ 的\textbf{仿射包}(Affine Hull),记为 $\text{aff } S$。

\textbf{例子}:在 $\mathbb{R}^2$ 中,如果 $S = \{(0, 0), (1, 0)\}$(两个点),则 $\text{aff } S$ 是通过这两点的直线。

\textbf{例子}:在 $\mathbb{R}^3$ 中,如果 $S = \{(0, 0, 0), (1, 0, 0), (0, 1, 0)\}$(三个不共线的点),则 $\text{aff } S$ 是通过这三个点的平面。

\subsection{凸包}

对于任意集合 $S \subseteq \mathbb{R}^n$,包含 $S$ 的最小凸集合称为 $S$ 的\textbf{凸包}(Convex Hull),记为 $\text{conv } S$。

\textbf{例子}:在 $\mathbb{R}^2$ 中,如果 $S = \{(0, 0), (1, 0), (0, 1)\}$(三个点),则 $\text{conv } S$ 是以这三个点为顶点的三角形(包括内部和边界)。

\textbf{例子}:在 $\mathbb{R}^2$ 中,如果 $S$ 是一个圆,则 $\text{conv } S$ 是包含该圆的最小圆盘。

\subsection{关系}

对于任意集合 $S$,有:
\begin{equation}
S \subseteq \text{conv } S \subseteq \text{aff } S
\end{equation}

这是因为:
\begin{itemize}
\item 凸包包含原集合
\item 仿射包包含凸包(因为仿射集合是凸集合的超集)
\end{itemize}

\section{应用}

\subsection{仿射集合的应用}

\begin{itemize}
\item \textbf{线性方程组}:线性方程组 $\mathbf{A}\mathbf{x} = \mathbf{b}$ 的解集是一个仿射集合
\item \textbf{仿射函数}:仿射函数 $f(\mathbf{x}) = \mathbf{A}\mathbf{x} + \mathbf{b}$ 将仿射集合映射为仿射集合
\item \textbf{优化问题}:等式约束优化问题中的可行域通常是仿射集合
\end{itemize}

\subsection{凸集合的应用}

\begin{itemize}
\item \textbf{凸优化}:凸优化问题的可行域必须是凸集合
\item \textbf{分离定理}:两个不相交的凸集合可以用超平面分离
\item \textbf{支撑超平面}:凸集合的边界点有支撑超平面
\item \textbf{机器学习}:支持向量机、逻辑回归等算法的可行域都是凸集合
\end{itemize}

\section{总结}

\begin{enumerate}
\item \textbf{定义区别}:
   \begin{itemize}
   \item 仿射集合:包含通过任意两点的整条直线($\theta \in \mathbb{R}$)
   \item 凸集合:包含连接任意两点的线段($\theta \in [0, 1]$)
   \end{itemize}

\item \textbf{包含关系}:所有仿射集合都是凸集合,但反之不成立。

\item \textbf{有界性}:仿射集合通常无界(除单点集和空集),而凸集合可以有界。

\item \textbf{几何特征}:
   \begin{itemize}
   \item 仿射集合:直线、平面、整个空间
   \item 凸集合:圆盘、三角形、球、半空间等
   \end{itemize}

\item \textbf{组合区别}:
   \begin{itemize}
   \item 仿射组合:系数之和为1,系数可以是任意实数
   \item 凸组合:系数之和为1,系数必须非负
   \end{itemize}
\end{enumerate}

理解仿射集合和凸集合的区别与联系,是深入学习凸优化理论的基础。

\end{document}

