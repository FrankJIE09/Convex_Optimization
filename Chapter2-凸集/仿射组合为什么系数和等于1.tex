\documentclass[12pt,a4paper]{article}
\usepackage[UTF8]{ctex}
\usepackage{amsmath}
\usepackage{amssymb}
\usepackage{amsthm}
\usepackage{geometry}
\geometry{left=2.5cm,right=2.5cm,top=2.5cm,bottom=2.5cm}

\title{仿射组合为什么系数和等于1?}
\author{}
\date{\today}

\begin{document}

\maketitle

\section{问题提出}

在学习仿射集合时,我们遇到这样的定义:

\textbf{仿射组合}:对于 $k$ 个点 $\mathbf{x}_1, \mathbf{x}_2, \ldots, \mathbf{x}_k$,如果系数 $\theta_1, \theta_2, \ldots, \theta_k \in \mathbb{R}$ 满足:
\begin{equation}
\theta_1 + \theta_2 + \cdots + \theta_k = 1
\end{equation}
那么点 $\theta_1 \mathbf{x}_1 + \theta_2 \mathbf{x}_2 + \cdots + \theta_k \mathbf{x}_k$ 称为这 $k$ 个点的仿射组合。

\textbf{疑问}:为什么要求"所有系数的和等于1",而不是"两两的 $\theta$ 之和等于1"?

\section{为什么是"所有系数和等于1"?}

\subsection{从两个点的情况开始}

首先,让我们回顾两个点的仿射组合:

对于两个点 $\mathbf{x}_1$ 和 $\mathbf{x}_2$,仿射组合定义为:
\begin{equation}
\theta \mathbf{x}_1 + (1-\theta) \mathbf{x}_2, \quad \theta \in \mathbb{R}
\end{equation}

注意:这里 $\theta + (1-\theta) = 1$,所以两个系数之和等于1。

\subsection{推广到多个点}

当我们推广到 $k$ 个点时,自然的推广方式是:
\begin{equation}
\theta_1 \mathbf{x}_1 + \theta_2 \mathbf{x}_2 + \cdots + \theta_k \mathbf{x}_k
\end{equation}

为了保持与两个点情况的一致性,我们要求:
\begin{equation}
\theta_1 + \theta_2 + \cdots + \theta_k = 1
\end{equation}

\subsection{几何意义}

系数和等于1的几何意义是:\textbf{仿射组合的结果点位于由这些点确定的仿射空间中}。

\textbf{关键理解}:
\begin{itemize}
\item 如果系数和等于1,那么结果点一定在通过这 $k$ 个点的仿射空间中
\item 如果系数和不等于1,结果点可能不在这个仿射空间中
\end{itemize}

\section{具体例子说明}

\subsection{例子1:三个点的情况}

在 $\mathbb{R}^2$ 中,考虑三个点:
\begin{align}
\mathbf{x}_1 &= (0, 0) \\
\mathbf{x}_2 &= (1, 0) \\
\mathbf{x}_3 &= (0, 1)
\end{align}

\textbf{情况1:系数和等于1}

设 $\theta_1 = 0.2$,$\theta_2 = 0.3$,$\theta_3 = 0.5$,则 $\theta_1 + \theta_2 + \theta_3 = 1$。

计算仿射组合:
\begin{align}
\mathbf{y} &= 0.2(0, 0) + 0.3(1, 0) + 0.5(0, 1) \\
&= (0, 0) + (0.3, 0) + (0, 0.5) \\
&= (0.3, 0.5)
\end{align}

点 $(0.3, 0.5)$ 位于由这三个点确定的平面(整个 $\mathbb{R}^2$)中。

\textbf{情况2:系数和不等于1}

设 $\theta_1 = 0.5$,$\theta_2 = 0.5$,$\theta_3 = 0.5$,则 $\theta_1 + \theta_2 + \theta_3 = 1.5 \neq 1$。

计算:
\begin{align}
\mathbf{y} &= 0.5(0, 0) + 0.5(1, 0) + 0.5(0, 1) \\
&= (0, 0) + (0.5, 0) + (0, 0.5) \\
&= (0.5, 0.5)
\end{align}

虽然这个点也在 $\mathbb{R}^2$ 中,但如果我们考虑更一般的情况(比如在 $\mathbb{R}^3$ 中),系数和不等于1时,结果点可能不在通过原点的仿射空间中。

\subsection{例子2:为什么不是"两两之和等于1"?}

假设我们要求"两两的 $\theta$ 之和等于1",即:
\begin{align}
\theta_1 + \theta_2 &= 1 \\
\theta_1 + \theta_3 &= 1 \\
\theta_2 + \theta_3 &= 1
\end{align}

从第一个和第二个方程,我们得到 $\theta_2 = \theta_3$。

从第一个和第三个方程,我们得到 $\theta_1 = \theta_3$。

因此 $\theta_1 = \theta_2 = \theta_3$。

从 $\theta_1 + \theta_2 = 1$,我们得到 $2\theta_1 = 1$,所以 $\theta_1 = \theta_2 = \theta_3 = 0.5$。

但这样 $\theta_1 + \theta_2 + \theta_3 = 1.5 \neq 1$,这与仿射组合的定义矛盾。

\textbf{问题}:如果要求"两两之和等于1",那么对于 $k > 2$ 个点,这个条件要么无解,要么过于严格,无法灵活地表示仿射空间中的所有点。

\section{数学原理}

\subsection{仿射组合的线性不变性}

\textbf{重要性质}:如果 $f: \mathbb{R}^n \to \mathbb{R}^m$ 是仿射函数(即 $f(\mathbf{x}) = \mathbf{A}\mathbf{x} + \mathbf{b}$),那么:

\begin{equation}
f\left(\sum_{i=1}^k \theta_i \mathbf{x}_i\right) = \sum_{i=1}^k \theta_i f(\mathbf{x}_i)
\end{equation}

\textbf{当且仅当} $\sum_{i=1}^k \theta_i = 1$。

\textbf{证明}:
\begin{align}
f\left(\sum_{i=1}^k \theta_i \mathbf{x}_i\right) &= \mathbf{A}\left(\sum_{i=1}^k \theta_i \mathbf{x}_i\right) + \mathbf{b} \\
&= \sum_{i=1}^k \theta_i \mathbf{A}\mathbf{x}_i + \mathbf{b} \\
&= \sum_{i=1}^k \theta_i \mathbf{A}\mathbf{x}_i + \left(\sum_{i=1}^k \theta_i\right) \mathbf{b} \\
&= \sum_{i=1}^k \theta_i (\mathbf{A}\mathbf{x}_i + \mathbf{b}) \\
&= \sum_{i=1}^k \theta_i f(\mathbf{x}_i)
\end{align}

关键步骤是:$\mathbf{b} = \left(\sum_{i=1}^k \theta_i\right) \mathbf{b}$,这要求 $\sum_{i=1}^k \theta_i = 1$。

\subsection{坐标无关性}

仿射组合的定义是\textbf{坐标无关}的。如果我们对空间进行仿射变换(平移、旋转、缩放等),仿射组合仍然是仿射组合。

这个性质依赖于系数和等于1。

\section{与凸组合的关系}

\subsection{凸组合的定义}

\textbf{凸组合}:对于 $k$ 个点,如果系数满足:
\begin{align}
\theta_1 + \theta_2 + \cdots + \theta_k &= 1 \\
\theta_i &\geq 0, \quad i = 1, 2, \ldots, k
\end{align}

那么 $\sum_{i=1}^k \theta_i \mathbf{x}_i$ 是凸组合。

\subsection{关系}

\begin{itemize}
\item \textbf{所有凸组合都是仿射组合}(因为凸组合也满足系数和等于1)
\item \textbf{但仿射组合不一定是凸组合}(因为仿射组合允许负系数)
\end{itemize}

\textbf{例子}:
\begin{itemize}
\item 对于点 $(0, 0)$ 和 $(1, 0)$:
  \begin{itemize}
  \item $\frac{1}{2}(0, 0) + \frac{1}{2}(1, 0) = (0.5, 0)$ 是凸组合(也是仿射组合)
  \item $2(0, 0) + (-1)(1, 0) = (-1, 0)$ 是仿射组合(但不是凸组合,因为系数有负数)
  \end{itemize}
\end{itemize}

\section{为什么"两两之和等于1"不合理?}

\subsection{数学上的问题}

如果我们要求"对于任意 $i \neq j$,都有 $\theta_i + \theta_j = 1$",那么:

对于 $k = 3$ 个点,我们需要:
\begin{align}
\theta_1 + \theta_2 &= 1 \label{eq:12} \\
\theta_1 + \theta_3 &= 1 \label{eq:13} \\
\theta_2 + \theta_3 &= 1 \label{eq:23}
\end{align}

从 \eqref{eq:12} 和 \eqref{eq:13},我们得到 $\theta_2 = \theta_3$。

从 \eqref{eq:12} 和 \eqref{eq:23},我们得到 $\theta_1 = \theta_3$。

因此 $\theta_1 = \theta_2 = \theta_3 = \frac{1}{2}$。

但这样 $\theta_1 + \theta_2 + \theta_3 = \frac{3}{2} \neq 1$。

\textbf{问题}:这个条件过于严格,只能表示一个特定的点,无法表示仿射空间中的所有点。

\subsection{几何上的问题}

仿射集合应该能够表示通过给定点的整个仿射空间。如果要求"两两之和等于1",我们只能得到非常有限的点,无法覆盖整个仿射空间。

\section{正确的理解方式}

\subsection{两个点的仿射组合}

对于两个点 $\mathbf{x}_1$ 和 $\mathbf{x}_2$:
\begin{equation}
\theta \mathbf{x}_1 + (1-\theta) \mathbf{x}_2 = \theta \mathbf{x}_1 + (1-\theta) \mathbf{x}_2
\end{equation}

这里两个系数是 $\theta$ 和 $(1-\theta)$,它们的和是1。

\subsection{多个点的仿射组合}

对于 $k$ 个点,我们直接推广:
\begin{equation}
\theta_1 \mathbf{x}_1 + \theta_2 \mathbf{x}_2 + \cdots + \theta_k \mathbf{x}_k
\end{equation}

要求所有系数的和等于1:
\begin{equation}
\theta_1 + \theta_2 + \cdots + \theta_k = 1
\end{equation}

\textbf{注意}:这里我们\textbf{不要求}每两个系数的和等于1,只要求所有系数的总和等于1。

\subsection{为什么这样定义?}

\begin{enumerate}
\item \textbf{一致性}:与两个点的情况保持一致
\item \textbf{完备性}:能够表示仿射空间中的所有点
\item \textbf{数学性质}:保持仿射变换下的不变性
\item \textbf{灵活性}:允许负系数,从而能够表示仿射空间中任意位置的点
\end{enumerate}

\section{具体数值例子}

\subsection{例子:三个点的仿射组合}

设 $\mathbf{x}_1 = (0, 0)$,$\mathbf{x}_2 = (1, 0)$,$\mathbf{x}_3 = (0, 1)$。

\textbf{情况1}:$\theta_1 = 1$,$\theta_2 = 0$,$\theta_3 = 0$(和等于1)
\begin{equation}
\mathbf{y} = 1 \cdot (0, 0) + 0 \cdot (1, 0) + 0 \cdot (0, 1) = (0, 0) = \mathbf{x}_1
\end{equation}

\textbf{情况2}:$\theta_1 = 0.5$,$\theta_2 = 0.3$,$\theta_3 = 0.2$(和等于1)
\begin{equation}
\mathbf{y} = 0.5(0, 0) + 0.3(1, 0) + 0.2(0, 1) = (0.3, 0.2)
\end{equation}

\textbf{情况3}:$\theta_1 = 2$,$\theta_2 = -1$,$\theta_3 = 0$(和等于1,允许负系数)
\begin{equation}
\mathbf{y} = 2(0, 0) + (-1)(1, 0) + 0(0, 1) = (-1, 0)
\end{equation}

注意:情况3中,虽然 $\theta_2 = -1 < 0$,但 $\theta_1 + \theta_2 + \theta_3 = 1$,所以这是有效的仿射组合。

\textbf{如果要求"两两之和等于1"}:
\begin{itemize}
\item $\theta_1 + \theta_2 = 1$ 且 $\theta_1 + \theta_3 = 1$ 意味着 $\theta_2 = \theta_3$
\item $\theta_1 + \theta_2 = 1$ 且 $\theta_2 + \theta_3 = 1$ 意味着 $\theta_1 = \theta_3$
\item 因此 $\theta_1 = \theta_2 = \theta_3 = 0.5$
\item 只能得到 $\mathbf{y} = 0.5(0, 0) + 0.5(1, 0) + 0.5(0, 1) = (0.5, 0.5)$
\item 无法表示其他点,比如 $(0, 0)$、$(1, 0)$、$(-1, 0)$ 等
\end{itemize}

\section{总结}

\begin{enumerate}
\item \textbf{仿射组合要求所有系数的和等于1},而不是"两两之和等于1"。

\item \textbf{原因}:
   \begin{itemize}
   \item 与两个点的情况保持一致
   \item 能够表示仿射空间中的所有点
   \item 保持仿射变换下的不变性
   \item 允许负系数,提供更大的灵活性
   \end{itemize}

\item \textbf{"两两之和等于1"的问题}:
   \begin{itemize}
   \item 对于 $k > 2$ 个点,条件过于严格
   \item 只能表示非常有限的点
   \item 无法覆盖整个仿射空间
   \end{itemize}

\item \textbf{正确理解}:
   \begin{itemize}
   \item 对于两个点:$\theta + (1-\theta) = 1$
   \item 对于 $k$ 个点:$\theta_1 + \theta_2 + \cdots + \theta_k = 1$
   \item 这是自然的推广,保持了数学的一致性和完备性
   \end{itemize}
\end{enumerate}

理解这一点对于掌握仿射集合、凸集合等概念非常重要!

\end{document}

