\documentclass[12pt,a4paper]{article}
\usepackage[UTF8]{ctex}
\usepackage{amsmath}
\usepackage{amssymb}
\usepackage{amsthm}
\usepackage{geometry}
\geometry{left=2.5cm,right=2.5cm,top=2.5cm,bottom=2.5cm}

\title{为什么相对内部定义中范数可以是任意的?}
\subtitle{基于《Convex Optimization》第2.1.3节}
\author{}
\date{\today}

\begin{document}

\maketitle

\section{问题提出}

在《Convex Optimization》第2.1.3节中,相对内部(relative interior)的定义为:

\begin{equation}
\text{relint } C = \{\mathbf{x} \in C \mid B(\mathbf{x}, r) \cap \text{aff } C \subseteq C \text{ 对某个 } r > 0\}
\end{equation}

其中 $B(\mathbf{x}, r) = \{\mathbf{y} \mid \|\mathbf{y} - \mathbf{x}\| \leq r\}$ 是以 $\mathbf{x}$ 为中心、半径为 $r$ 的球。

书中提到:\textbf{"Here $\|\cdot\|$ is any norm; all norms define the same relative interior."}

\textbf{问题}:为什么说"可以是任意范数"?为什么所有范数定义相同的相对内部?

\section{范数等价性定理}

\subsection{核心定理}

\textbf{定理}(范数等价性):在有限维向量空间 $\mathbb{R}^n$ 中,所有范数都是等价的。

具体地,对于任意两个范数 $\|\cdot\|_a$ 和 $\|\cdot\|_b$,存在正常数 $c_1, c_2 > 0$,使得对任意 $\mathbf{x} \in \mathbb{R}^n$,有:

\begin{equation}
c_1 \|\mathbf{x}\|_a \leq \|\mathbf{x}\|_b \leq c_2 \|\mathbf{x}\|_a
\end{equation}

\subsection{几何意义}

这个定理意味着:
\begin{itemize}
\item 不同范数定义的"球"形状可能不同(例如,L2范数是圆,L1范数是菱形,L∞范数是正方形)
\item 但它们定义的"开集"和"闭集"是相同的
\item 因此,它们定义的"内部"和"边界"也是相同的
\end{itemize}

\section{为什么相对内部与范数选择无关?}

\subsection{相对内部的本质}

相对内部的定义依赖于:
\begin{equation}
B(\mathbf{x}, r) \cap \text{aff } C \subseteq C
\end{equation}

这里的关键是:是否存在某个 $r > 0$,使得以 $\mathbf{x}$ 为中心的球与仿射包的交集完全包含在 $C$ 中。

\subsection{范数等价性的作用}

\textbf{关键观察}:如果 $\mathbf{x}$ 是 $C$ 的相对内点(相对于某个范数),那么它也是相对内点(相对于任何其他范数)。

\textbf{证明思路}:

设 $\|\cdot\|_1$ 和 $\|\cdot\|_2$ 是两个范数,根据范数等价性,存在 $c_1, c_2 > 0$,使得:
\begin{equation}
c_1 \|\mathbf{y}\|_1 \leq \|\mathbf{y}\|_2 \leq c_2 \|\mathbf{y}\|_1
\end{equation}

如果 $\mathbf{x}$ 是相对于 $\|\cdot\|_1$ 的相对内点,则存在 $r_1 > 0$,使得:
\begin{equation}
B_1(\mathbf{x}, r_1) \cap \text{aff } C \subseteq C
\end{equation}
其中 $B_1(\mathbf{x}, r_1) = \{\mathbf{y} \mid \|\mathbf{y} - \mathbf{x}\|_1 \leq r_1\}$。

现在考虑相对于 $\|\cdot\|_2$ 的球 $B_2(\mathbf{x}, r_2)$,其中 $r_2 = c_1 r_1$。

对于任意 $\mathbf{y} \in B_2(\mathbf{x}, r_2) \cap \text{aff } C$,有:
\begin{equation}
\|\mathbf{y} - \mathbf{x}\|_2 \leq r_2 = c_1 r_1
\end{equation}

根据范数等价性:
\begin{equation}
c_1 \|\mathbf{y} - \mathbf{x}\|_1 \leq \|\mathbf{y} - \mathbf{x}\|_2 \leq c_1 r_1
\end{equation}

因此:
\begin{equation}
\|\mathbf{y} - \mathbf{x}\|_1 \leq r_1
\end{equation}

所以 $\mathbf{y} \in B_1(\mathbf{x}, r_1) \cap \text{aff } C \subseteq C$。

因此 $B_2(\mathbf{x}, r_2) \cap \text{aff } C \subseteq C$,所以 $\mathbf{x}$ 也是相对于 $\|\cdot\|_2$ 的相对内点。

类似地,可以证明如果 $\mathbf{x}$ 不是相对于 $\|\cdot\|_1$ 的相对内点,那么它也不是相对于 $\|\cdot\|_2$ 的相对内点。

因此,相对内部的定义与范数的选择无关!$\square$

\section{具体例子}

\subsection{例子1:二维正方形}

在 $\mathbb{R}^2$ 中,考虑集合:
\begin{equation}
C = \{(x, y) \mid -1 \leq x \leq 1, -1 \leq y \leq 1\}
\end{equation}

\textbf{使用L2范数}:

点 $(0, 0)$ 是相对内点,因为存在 $r = 0.5$,使得:
\begin{equation}
B_2((0, 0), 0.5) = \{(x, y) \mid x^2 + y^2 \leq 0.25\} \subseteq C
\end{equation}

\textbf{使用L1范数}:

点 $(0, 0)$ 也是相对内点,因为存在 $r = 0.5$,使得:
\begin{equation}
B_1((0, 0), 0.5) = \{(x, y) \mid |x| + |y| \leq 0.5\} \subseteq C
\end{equation}

\textbf{使用L∞范数}:

点 $(0, 0)$ 也是相对内点,因为存在 $r = 0.5$,使得:
\begin{equation}
B_\infty((0, 0), 0.5) = \{(x, y) \mid \max\{|x|, |y|\} \leq 0.5\} \subseteq C
\end{equation}

虽然三个"球"的形状不同(圆、菱形、正方形),但都确认 $(0, 0)$ 是相对内点。

\subsection{例子2:边界点}

考虑点 $(1, 0)$(在边界上)。

\textbf{使用L2范数}:

对于任意 $r > 0$,球 $B_2((1, 0), r)$ 都会包含 $x > 1$ 的点,这些点不在 $C$ 中,所以 $(1, 0)$ 不是相对内点。

\textbf{使用L1范数}:

同样,对于任意 $r > 0$,球 $B_1((1, 0), r)$ 都会包含 $x > 1$ 的点,所以 $(1, 0)$ 不是相对内点。

\textbf{使用L∞范数}:

同样,$(1, 0)$ 不是相对内点。

所有范数都给出相同的结论:$(1, 0)$ 不是相对内点。

\section{为什么在有限维中成立?}

\subsection{关键:有限维性}

\textbf{重要事实}:范数等价性定理只在\textbf{有限维}向量空间中成立!

在无限维空间中,不同的范数可能定义不同的拓扑,因此可能定义不同的开集和闭集。

\subsection{为什么有限维中成立?}

在 $\mathbb{R}^n$ 中,我们可以证明:

\begin{enumerate}
\item 所有范数都是连续的(在标准拓扑下)
\item 单位球面(在任意范数下)是紧致的
\item 紧致集合上的连续函数达到最大值和最小值
\item 因此可以找到范数之间的等价常数
\end{enumerate}

\subsection{具体构造}

对于 $\mathbb{R}^n$ 中的任意范数 $\|\cdot\|$,考虑单位球面:
\begin{equation}
S = \{\mathbf{x} \in \mathbb{R}^n \mid \|\mathbf{x}\|_2 = 1\}
\end{equation}

由于 $S$ 是紧致的,且 $\|\cdot\|$ 在 $S$ 上连续,所以存在:
\begin{align}
m &= \min_{\mathbf{x} \in S} \|\mathbf{x}\| > 0 \\
M &= \max_{\mathbf{x} \in S} \|\mathbf{x}\| < \infty
\end{align}

对于任意 $\mathbf{x} \neq \mathbf{0}$,设 $\mathbf{y} = \frac{\mathbf{x}}{\|\mathbf{x}\|_2} \in S$,则:
\begin{equation}
m \leq \|\mathbf{y}\| \leq M
\end{equation}

因此:
\begin{equation}
m \|\mathbf{x}\|_2 \leq \|\mathbf{x}\| \leq M \|\mathbf{x}\|_2
\end{equation}

这证明了任意范数与L2范数等价,因此所有范数都等价。

\section{相对内部定义的合理性}

\subsection{为什么这样定义是合理的?}

相对内部的定义使用"球"的概念,而:
\begin{itemize}
\item 在有限维中,所有范数定义的拓扑相同
\item 因此,相对内部的概念是"内在的",不依赖于范数的具体选择
\item 这使定义更加"几何化",而不是依赖于具体的度量
\end{itemize}

\subsection{实际应用}

在实际计算中:
\begin{itemize}
\item 我们可以选择最方便的范数(通常是L2范数)
\item 不用担心选择不同范数会得到不同的结果
\item 这简化了理论分析和实际计算
\end{itemize}

\section{与普通内部的对比}

\subsection{普通内部}

普通内部(interior)的定义也使用球:
\begin{equation}
\text{int } C = \{\mathbf{x} \in C \mid B(\mathbf{x}, r) \subseteq C \text{ 对某个 } r > 0\}
\end{equation}

同样,由于范数等价性,普通内部也与范数选择无关。

\subsection{区别}

\begin{itemize}
\item \textbf{普通内部}:球 $B(\mathbf{x}, r)$ 完全在 $C$ 中
\item \textbf{相对内部}:球 $B(\mathbf{x}, r)$ 与仿射包的交集在 $C$ 中
\end{itemize}

两者都因为范数等价性而与范数选择无关。

\section{总结}

\begin{enumerate}
\item \textbf{范数等价性}:在有限维 $\mathbb{R}^n$ 中,所有范数都是等价的。

\item \textbf{拓扑相同}:不同范数定义相同的开集、闭集、内部、边界等概念。

\item \textbf{相对内部与范数无关}:
   \begin{itemize}
   \item 如果 $\mathbf{x}$ 是相对于某个范数的相对内点
   \item 那么它也是相对于任何其他范数的相对内点
   \end{itemize}

\item \textbf{实际意义}:
   \begin{itemize}
   \item 我们可以选择最方便的范数(通常是L2范数)
   \item 不用担心范数选择会影响结果
   \item 使定义更加"几何化"和"内在化"
   \end{itemize}

\item \textbf{关键限制}:这个结论只在\textbf{有限维}空间中成立。在无限维空间中,不同范数可能定义不同的拓扑。
\end{enumerate}

理解这一点对于正确使用相对内部的概念非常重要!

\end{document}

