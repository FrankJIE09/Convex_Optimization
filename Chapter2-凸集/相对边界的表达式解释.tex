\documentclass[12pt,a4paper]{article}
\usepackage[UTF8]{ctex}
\usepackage{amsmath}
\usepackage{amssymb}
\usepackage{amsthm}
\usepackage{geometry}
\geometry{left=2.5cm,right=2.5cm,top=2.5cm,bottom=2.5cm}

\title{相对边界表达式 $\max\{|x_1|, |x_2|\} = 1$ 的解释}
\subtitle{基于《Convex Optimization》Example 2.2}
\author{}
\date{\today}

\begin{document}

\maketitle

\section{问题提出}

在《Convex Optimization》第2.1.3节的Example 2.2中,考虑 $\mathbb{R}^3$ 中的一个正方形:

\begin{equation}
C = \{\mathbf{x} \in \mathbb{R}^3 \mid -1 \leq x_1 \leq 1, -1 \leq x_2 \leq 1, x_3 = 0\}
\end{equation}

其相对边界(relative boundary)为:
\begin{equation}
\{\mathbf{x} \in \mathbb{R}^3 \mid \max\{|x_1|, |x_2|\} = 1, x_3 = 0\}
\end{equation}

\textbf{问题}:为什么相对边界是 $\max\{|x_1|, |x_2|\} = 1$?这个表达式是怎么来的?

\section{正方形的定义}

\subsection{集合 $C$ 的描述}

集合 $C$ 是 $\mathbb{R}^3$ 中位于 $x_1$-$x_2$ 平面($x_3 = 0$)上的一个正方形:

\begin{itemize}
\item $-1 \leq x_1 \leq 1$:$x_1$ 坐标在 $[-1, 1]$ 之间
\item $-1 \leq x_2 \leq 1$:$x_2$ 坐标在 $[-1, 1]$ 之间
\item $x_3 = 0$:位于 $x_1$-$x_2$ 平面上
\end{itemize}

这个正方形包括:
\begin{itemize}
\item 内部:$-1 < x_1 < 1$ 且 $-1 < x_2 < 1$
\item 边界:$x_1 = \pm 1$ 或 $x_2 = \pm 1$(在 $[-1, 1] \times [-1, 1]$ 范围内)
\end{itemize}

\section{相对内部和相对边界}

\subsection{相对内部}

根据定义,相对内部为:
\begin{equation}
\text{relint } C = \{\mathbf{x} \in \mathbb{R}^3 \mid -1 < x_1 < 1, -1 < x_2 < 1, x_3 = 0\}
\end{equation}

这是正方形的\textbf{开正方形}(不包括边界)。

\subsection{相对边界}

相对边界定义为:
\begin{equation}
\text{相对边界} = \text{闭包}(C) \setminus \text{relint } C
\end{equation}

由于 $C$ 是闭集(包含边界),所以:
\begin{equation}
\text{相对边界} = C \setminus \text{relint } C
\end{equation}

即:相对边界 = 整个正方形 - 相对内部 = 正方形的边界

\section{为什么是 $\max\{|x_1|, |x_2|\} = 1$?}

\subsection{正方形的边界}

正方形的边界由四条边组成:
\begin{enumerate}
\item 右边:$x_1 = 1$,$-1 \leq x_2 \leq 1$
\item 左边:$x_1 = -1$,$-1 \leq x_2 \leq 1$
\item 上边:$x_2 = 1$,$-1 \leq x_1 \leq 1$
\item 下边:$x_2 = -1$,$-1 \leq x_1 \leq 1$
\end{enumerate}

\subsection{用 $\max\{|x_1|, |x_2|\} = 1$ 表示}

\textbf{关键观察}:点 $(x_1, x_2)$ 在正方形边界上当且仅当:
\begin{itemize}
\item $|x_1| = 1$ 且 $|x_2| \leq 1$,或
\item $|x_2| = 1$ 且 $|x_1| \leq 1$
\end{itemize}

这等价于:
\begin{equation}
\max\{|x_1|, |x_2|\} = 1 \quad \text{且} \quad |x_1| \leq 1, |x_2| \leq 1
\end{equation}

但由于 $\max\{|x_1|, |x_2|\} = 1$ 已经隐含了 $|x_1| \leq 1$ 和 $|x_2| \leq 1$(因为最大值是1),所以可以简化为:
\begin{equation}
\max\{|x_1|, |x_2|\} = 1
\end{equation}

\subsection{详细分析}

\textbf{情况1}:$|x_1| = 1$ 且 $|x_2| \leq 1$

此时 $\max\{|x_1|, |x_2|\} = \max\{1, |x_2|\} = 1$(因为 $|x_2| \leq 1$)。

这对应正方形的左右两条边。

\textbf{情况2}:$|x_2| = 1$ 且 $|x_1| \leq 1$

此时 $\max\{|x_1|, |x_2|\} = \max\{|x_1|, 1\} = 1$(因为 $|x_1| \leq 1$)。

这对应正方形的上下两条边。

\textbf{情况3}:$|x_1| = 1$ 且 $|x_2| = 1$(四个顶点)

此时 $\max\{|x_1|, |x_2|\} = \max\{1, 1\} = 1$。

这对应正方形的四个角。

\textbf{结论}:所有边界点都满足 $\max\{|x_1|, |x_2|\} = 1$。

\section{几何直观}

\subsection{$\max\{|x_1|, |x_2|\}$ 的几何意义}

在 $\mathbb{R}^2$ 中,集合 $\{\mathbf{x} \mid \max\{|x_1|, |x_2|\} \leq r\}$ 表示:
\begin{itemize}
\item 以原点为中心的正方形
\item 边长为 $2r$
\item 边平行于坐标轴
\end{itemize}

当 $r = 1$ 时,这是边长为2的正方形。

\textbf{关键}:$\max\{|x_1|, |x_2|\} = 1$ 表示这个正方形的\textbf{边界}。

\subsection{与L∞范数的关系}

注意:$\max\{|x_1|, |x_2|\}$ 正是L∞范数在二维空间中的定义:
\begin{equation}
\|\mathbf{x}\|_\infty = \max\{|x_1|, |x_2|\}
\end{equation}

因此,$\max\{|x_1|, |x_2|\} = 1$ 等价于 $\|\mathbf{x}\|_\infty = 1$,即L∞范数等于1的点的集合。

\section{具体例子}

\subsection{边界上的点}

\textbf{例子1}:点 $(1, 0.5)$
\begin{itemize}
\item $|x_1| = 1$,$|x_2| = 0.5$
\item $\max\{|x_1|, |x_2|\} = \max\{1, 0.5\} = 1$ ✓
\item 这个点在正方形的右边上
\end{itemize}

\textbf{例子2}:点 $(0.3, 1)$
\begin{itemize}
\item $|x_1| = 0.3$,$|x_2| = 1$
\item $\max\{|x_1|, |x_2|\} = \max\{0.3, 1\} = 1$ ✓
\item 这个点在正方形的上边上
\end{itemize}

\textbf{例子3}:点 $(1, 1)$(角点)
\begin{itemize}
\item $|x_1| = 1$,$|x_2| = 1$
\item $\max\{|x_1|, |x_2|\} = \max\{1, 1\} = 1$ ✓
\item 这个点在正方形的右上角
\end{itemize}

\subsection{内部的点}

\textbf{例子4}:点 $(0.5, 0.3)$
\begin{itemize}
\item $|x_1| = 0.5$,$|x_2| = 0.3$
\item $\max\{|x_1|, |x_2|\} = \max\{0.5, 0.3\} = 0.5 < 1$ ✗
\item 这个点在正方形内部,不在边界上
\end{itemize}

\subsection{外部的点}

\textbf{例子5}:点 $(1.5, 0.5)$
\begin{itemize}
\item $|x_1| = 1.5$,$|x_2| = 0.5$
\item $\max\{|x_1|, |x_2|\} = \max\{1.5, 0.5\} = 1.5 > 1$ ✗
\item 这个点在正方形外部
\end{itemize}

\section{一般情况:$n$ 维情况}

\subsection{推广}

对于 $\mathbb{R}^n$ 中的"超立方体":
\begin{equation}
C = \{\mathbf{x} \in \mathbb{R}^n \mid -1 \leq x_i \leq 1, i = 1, \ldots, n\}
\end{equation}

其相对边界为:
\begin{equation}
\{\mathbf{x} \in \mathbb{R}^n \mid \max\{|x_1|, |x_2|, \ldots, |x_n|\} = 1\}
\end{equation}

\subsection{证明思路}

类似地,点 $\mathbf{x}$ 在边界上当且仅当至少有一个坐标的绝对值等于1,且其他坐标的绝对值都不超过1,这等价于 $\max\{|x_1|, \ldots, |x_n|\} = 1$。

\section{与相对内部的关系}

\subsection{完整的描述}

对于正方形 $C$:
\begin{itemize}
\item \textbf{相对内部}:$\max\{|x_1|, |x_2|\} < 1$(严格小于)
\item \textbf{相对边界}:$\max\{|x_1|, |x_2|\} = 1$(等于)
\item \textbf{外部}:$\max\{|x_1|, |x_2|\} > 1$(严格大于)
\end{itemize}

\subsection{为什么叫"wire-frame outline"?}

书中将相对边界称为"wire-frame outline"(线框轮廓),因为:
\begin{itemize}
\item 相对边界只包含正方形的四条边
\item 不包含内部
\item 就像用线框勾勒出的轮廓
\end{itemize}

\section{总结}

\begin{enumerate}
\item \textbf{正方形的边界}:由满足 $x_1 = \pm 1$ 或 $x_2 = \pm 1$(在范围内)的点组成。

\item \textbf{等价表示}:边界可以等价地表示为 $\max\{|x_1|, |x_2|\} = 1$。

\item \textbf{几何意义}:
   \begin{itemize}
   \item $\max\{|x_1|, |x_2|\} = 1$ 表示L∞范数等于1的点的集合
   \item 这正好是边长为2的正方形的边界
   \end{itemize}

\item \textbf{验证方法}:
   \begin{itemize}
   \item 边界上的点:$\max\{|x_1|, |x_2|\} = 1$
   \item 内部的点:$\max\{|x_1|, |x_2|\} < 1$
   \item 外部的点:$\max\{|x_1|, |x_2|\} > 1$
   \end{itemize}
\end{enumerate}

理解这个表达式对于掌握相对边界的概念非常重要!

\end{document}

