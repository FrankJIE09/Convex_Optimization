\documentclass[12pt,a4paper]{article}
\usepackage[UTF8]{ctex}
\usepackage{amsmath}
\usepackage{amssymb}
\usepackage{amsthm}
\usepackage{geometry}
\geometry{left=2.5cm,right=2.5cm,top=2.5cm,bottom=2.5cm}

\title{单纯形与凸包的关系}
\subtitle{基于《Convex Optimization》第2.2节}
\author{}
\date{\today}

\begin{document}

\maketitle

\section{引言}

单纯形(Simplex)和凸包(Convex Hull)是凸优化中两个密切相关的重要概念。理解它们的关系对于掌握凸集理论至关重要。本文将详细解释单纯形的定义、性质,以及它与凸包的关系。

\section{凸包(Convex Hull)回顾}

\subsection{定义}

对于集合 $C \subseteq \mathbb{R}^n$,$C$ 的\textbf{凸包}定义为:

\begin{equation}
\text{conv } C = \left\{\sum_{i=1}^k \theta_i \mathbf{x}_i \middle| \mathbf{x}_i \in C, \theta_i \geq 0, \sum_{i=1}^k \theta_i = 1\right\}
\end{equation}

即 $C$ 中所有点的所有凸组合的集合。

\subsection{性质}

\begin{itemize}
\item 凸包是包含 $C$ 的最小凸集
\item 凸包本身是凸集
\item 如果 $C$ 是有限点集,凸包是这些点的所有凸组合
\end{itemize}

\section{单纯形(Simplex)}

\subsection{定义}

\textbf{定义}:设 $\mathbf{v}_0, \mathbf{v}_1, \ldots, \mathbf{v}_k \in \mathbb{R}^n$ 是 $k+1$ 个\textbf{仿射无关}(affinely independent)的点。

\textbf{仿射无关}意味着:$\mathbf{v}_1 - \mathbf{v}_0, \mathbf{v}_2 - \mathbf{v}_0, \ldots, \mathbf{v}_k - \mathbf{v}_0$ 是线性无关的。

由这 $k+1$ 个点确定的\textbf{单纯形}定义为:

\begin{equation}
C = \text{conv}\{\mathbf{v}_0, \mathbf{v}_1, \ldots, \mathbf{v}_k\} = \left\{\sum_{i=0}^k \theta_i \mathbf{v}_i \middle| \theta_i \geq 0, \sum_{i=0}^k \theta_i = 1\right\}
\end{equation}

\subsection{关键要点}

\begin{enumerate}
\item \textbf{仿射无关性}:这是单纯形的关键要求
\item \textbf{凸包形式}:单纯形就是这 $k+1$ 个点的凸包
\item \textbf{维度}:$k$ 维单纯形(在 $\mathbb{R}^n$ 中,$k \leq n$)
\end{enumerate}

\subsection{仿射无关性的含义}

\textbf{为什么要求仿射无关?}

如果点不是仿射无关的,那么:
\begin{itemize}
\item 某些点可能共线、共面等
\item 凸包可能"退化"(维度降低)
\item 不再是"标准"的单纯形
\end{itemize}

\textbf{例子}:
\begin{itemize}
\item 三个点如果共线,它们的凸包是线段(1维),不是三角形(2维)
\item 四个点如果共面,它们的凸包可能是三角形或四边形,不是四面体(3维)
\end{itemize}

\section{单纯形与凸包的关系}

\subsection{核心关系}

\textbf{关键结论}:\textbf{单纯形就是特定点集的凸包}。

具体地:
\begin{equation}
\text{单纯形} = \text{conv}\{\mathbf{v}_0, \mathbf{v}_1, \ldots, \mathbf{v}_k\}
\end{equation}

其中 $\mathbf{v}_0, \mathbf{v}_1, \ldots, \mathbf{v}_k$ 是仿射无关的点。

\subsection{关系总结}

\begin{enumerate}
\item \textbf{单纯形是凸包的特殊情况}:
   \begin{itemize}
   \item 单纯形 = 仿射无关点集的凸包
   \item 不是所有凸包都是单纯形
   \item 只有当点集满足仿射无关性时,凸包才是单纯形
   \end{itemize}

\item \textbf{凸包是更一般的概念}:
   \begin{itemize}
   \item 凸包可以应用于任意点集
   \item 单纯形只适用于仿射无关的点集
   \item 单纯形是凸包的"非退化"情况
   \end{itemize}

\item \textbf{表示形式相同}:
   \begin{itemize}
   \item 两者都表示为凸组合的集合
   \item 系数和等于1,系数非负
   \item 区别在于点的选择(是否仿射无关)
   \end{itemize}
\end{enumerate}

\section{具体例子}

\subsection{例子1:一维单纯形(线段)}

在 $\mathbb{R}^2$ 中,设 $\mathbf{v}_0 = (0, 0)$,$\mathbf{v}_1 = (1, 0)$。

这两个点是仿射无关的(因为 $\mathbf{v}_1 - \mathbf{v}_0 = (1, 0)$ 是线性无关的)。

单纯形:
\begin{equation}
C = \text{conv}\{(0, 0), (1, 0)\} = \{\theta_0(0, 0) + \theta_1(1, 0) \mid \theta_i \geq 0, \theta_0 + \theta_1 = 1\}
\end{equation}

这正好是连接 $(0, 0)$ 和 $(1, 0)$ 的线段。

\textbf{关系}:单纯形 = 这两个点的凸包

\subsection{例子2:二维单纯形(三角形)}

在 $\mathbb{R}^2$ 中,设 $\mathbf{v}_0 = (0, 0)$,$\mathbf{v}_1 = (1, 0)$,$\mathbf{v}_2 = (0, 1)$。

验证仿射无关性:
\begin{itemize}
\item $\mathbf{v}_1 - \mathbf{v}_0 = (1, 0)$
\item $\mathbf{v}_2 - \mathbf{v}_0 = (0, 1)$
\item 这两个向量线性无关 ✓
\end{itemize}

单纯形:
\begin{equation}
C = \text{conv}\{(0, 0), (1, 0), (0, 1)\} = \{\theta_0(0, 0) + \theta_1(1, 0) + \theta_2(0, 1) \mid \theta_i \geq 0, \sum \theta_i = 1\}
\end{equation}

这正好是以 $(0, 0)$、$(1, 0)$、$(0, 1)$ 为顶点的三角形(包括内部和边界)。

\textbf{关系}:单纯形 = 这三个点的凸包

\subsection{例子3:三维单纯形(四面体)}

在 $\mathbb{R}^3$ 中,设 $\mathbf{v}_0 = (0, 0, 0)$,$\mathbf{v}_1 = (1, 0, 0)$,$\mathbf{v}_2 = (0, 1, 0)$,$\mathbf{v}_3 = (0, 0, 1)$。

验证仿射无关性:
\begin{itemize}
\item $\mathbf{v}_1 - \mathbf{v}_0 = (1, 0, 0)$
\item $\mathbf{v}_2 - \mathbf{v}_0 = (0, 1, 0)$
\item $\mathbf{v}_3 - \mathbf{v}_0 = (0, 0, 1)$
\item 这三个向量线性无关 ✓
\end{itemize}

单纯形:
\begin{equation}
C = \text{conv}\{\mathbf{v}_0, \mathbf{v}_1, \mathbf{v}_2, \mathbf{v}_3\}
\end{equation}

这正好是以这四个点为顶点的四面体。

\textbf{关系}:单纯形 = 这四个点的凸包

\subsection{例子4:非单纯形的凸包}

在 $\mathbb{R}^2$ 中,设 $C = \{(0, 0), (1, 0), (1, 1), (0, 1)\}$(正方形的四个顶点)。

这四个点不是仿射无关的(因为它们在同一个平面上,且多于3个点)。

凸包:
\begin{equation}
\text{conv } C = \text{正方形(包括内部和边界)}
\end{equation}

这不是单纯形,因为:
\begin{itemize}
\item 单纯形要求 $k+1$ 个仿射无关的点
\item 在 $\mathbb{R}^2$ 中,最多只能有3个仿射无关的点
\item 4个点的凸包不是单纯形
\end{itemize}

\section{常见单纯形}

\subsection{单位单纯形(Unit Simplex)}

\textbf{定义}:$n$ 维单位单纯形由零向量和单位向量确定:

\begin{equation}
\text{conv}\{\mathbf{0}, \mathbf{e}_1, \mathbf{e}_2, \ldots, \mathbf{e}_n\}
\end{equation}

其中 $\mathbf{e}_i$ 是第 $i$ 个单位向量。

\textbf{等价表示}:
\begin{equation}
\{\mathbf{x} \in \mathbb{R}^n \mid \mathbf{x} \succeq \mathbf{0}, \mathbf{1}^T \mathbf{x} \leq 1\}
\end{equation}

\textbf{关系}:单位单纯形 = $\{\mathbf{0}, \mathbf{e}_1, \ldots, \mathbf{e}_n\}$ 的凸包

\subsection{概率单纯形(Probability Simplex)}

\textbf{定义}:$(n-1)$ 维概率单纯形由单位向量确定:

\begin{equation}
\text{conv}\{\mathbf{e}_1, \mathbf{e}_2, \ldots, \mathbf{e}_n\}
\end{equation}

\textbf{等价表示}:
\begin{equation}
\{\mathbf{x} \in \mathbb{R}^n \mid \mathbf{x} \succeq \mathbf{0}, \mathbf{1}^T \mathbf{x} = 1\}
\end{equation}

\textbf{关系}:概率单纯形 = $\{\mathbf{e}_1, \ldots, \mathbf{e}_n\}$ 的凸包

\textbf{应用}:概率分布,其中 $x_i$ 表示第 $i$ 个元素的概率。

\section{单纯形的性质}

\subsection{凸性}

单纯形是凸集,因为它是凸包。

\subsection{维度}

$k$ 维单纯形(由 $k+1$ 个仿射无关的点确定)的仿射维度是 $k$。

\subsection{顶点}

单纯形的顶点就是确定它的那 $k+1$ 个点。

\subsection{唯一表示}

对于单纯形内的任意点,存在唯一的凸组合表示(因为点仿射无关)。

\section{单纯形 vs 一般凸包}

\subsection{相同点}

\begin{itemize}
\item 都是凸集
\item 都表示为凸组合的集合
\item 都包含确定它们的点
\end{itemize}

\subsection{不同点}

\begin{table}[h]
\centering
\begin{tabular}{|l|l|l|}
\hline
\textbf{性质} & \textbf{单纯形} & \textbf{一般凸包} \\
\hline
点的要求 & 必须仿射无关 & 任意点集 \\
\hline
点的数量 & $k+1$ 个($k$ 维) & 任意数量 \\
\hline
维度 & 固定为 $k$ & 可能小于 $k$(退化) \\
\hline
唯一表示 & 是 & 不一定 \\
\hline
例子 & 三角形、四面体 & 多边形、多面体 \\
\hline
\end{tabular}
\caption{单纯形与一般凸包的区别}
\end{table}

\section{几何直观}

\subsection{单纯形}

\begin{itemize}
\item \textbf{一维}:线段(2个点)
\item \textbf{二维}:三角形(3个点)
\item \textbf{三维}:四面体(4个点)
\item \textbf{$k$维}:$k+1$ 个顶点的"广义三角形"
\end{itemize}

\subsection{一般凸包}

\begin{itemize}
\item 可以是任意形状的凸集
\item 顶点数量可以任意多
\item 可能"退化"(维度降低)
\item 例子:正方形、五边形、任意凸多边形
\end{itemize}

\section{应用}

\subsection{在优化中}

\begin{itemize}
\item \textbf{单纯形法}:线性规划中的经典算法
\item \textbf{概率约束}:概率单纯形用于概率优化
\item \textbf{离散优化}:单纯形用于离散问题的连续松弛
\end{itemize}

\subsection{在计算几何中}

\begin{itemize}
\item \textbf{三角剖分}:将区域分解为单纯形
\item \textbf{有限元方法}:使用单纯形作为基本单元
\item \textbf{体积计算}:单纯形的体积容易计算
\end{itemize}

\section{总结}

\begin{enumerate}
\item \textbf{核心关系}:
   \begin{itemize}
   \item 单纯形 = 仿射无关点集的凸包
   \item 单纯形是凸包的特殊情况
   \item 凸包是更一般的概念
   \end{itemize}

\item \textbf{关键区别}:
   \begin{itemize}
   \item 单纯形要求点仿射无关
   \item 一般凸包对点没有特殊要求
   \item 单纯形有固定的维度
   \end{itemize}

\item \textbf{表示形式}:
   \begin{itemize}
   \item 两者都表示为凸组合
   \item 系数和等于1,系数非负
   \item 单纯形有唯一的表示
   \end{itemize}

\item \textbf{几何意义}:
   \begin{itemize}
   \item 单纯形:三角形、四面体的推广
   \item 凸包:包含点集的最小凸集
   \end{itemize}
\end{enumerate}

理解单纯形与凸包的关系,对于掌握凸优化理论和应用都非常重要!

\end{document}

