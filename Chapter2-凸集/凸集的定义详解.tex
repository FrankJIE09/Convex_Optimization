\documentclass[12pt,a4paper]{article}
\usepackage[UTF8]{ctex}
\usepackage{amsmath}
\usepackage{amssymb}
\usepackage{amsthm}
\usepackage{geometry}
\geometry{left=2.5cm,right=2.5cm,top=2.5cm,bottom=2.5cm}

\title{凸集(Convex Set)的定义详解}
\author{}
\date{\today}

\begin{document}

\maketitle

\section{凸集的基本定义}

\subsection{标准定义}

\textbf{凸集}:集合 $C \subseteq \mathbb{R}^n$ 是凸集,如果对于任意 $\mathbf{x}_1, \mathbf{x}_2 \in C$ 和任意 $\theta \in [0, 1]$,有:

\begin{equation}
\theta \mathbf{x}_1 + (1-\theta) \mathbf{x}_2 \in C
\end{equation}

\textbf{符号说明}:
\begin{itemize}
\item $C \subseteq \mathbb{R}^n$:$n$ 维欧几里得空间中的集合
\item $\mathbf{x}_1, \mathbf{x}_2 \in C$:集合 $C$ 中的任意两点
\item $\theta \in [0, 1]$:权重参数,在 $0$ 和 $1$ 之间
\item $\theta \mathbf{x}_1 + (1-\theta) \mathbf{x}_2$:两点 $\mathbf{x}_1$ 和 $\mathbf{x}_2$ 的凸组合
\end{itemize}

\subsection{凸组合}

\textbf{凸组合}:对于点 $\mathbf{x}_1, \mathbf{x}_2$ 和 $\theta \in [0, 1]$,点 $\theta \mathbf{x}_1 + (1-\theta) \mathbf{x}_2$ 称为 $\mathbf{x}_1$ 和 $\mathbf{x}_2$ 的凸组合。

\textbf{几何意义}:
\begin{itemize}
\item 当 $\theta = 0$ 时:$\mathbf{x} = \mathbf{x}_2$
\item 当 $\theta = 1$ 时:$\mathbf{x} = \mathbf{x}_1$
\item 当 $\theta = 0.5$ 时:$\mathbf{x}$ 是 $\mathbf{x}_1$ 和 $\mathbf{x}_2$ 的中点
\item 当 $\theta \in (0, 1)$ 时:$\mathbf{x}$ 在连接 $\mathbf{x}_1$ 和 $\mathbf{x}_2$ 的线段上
\end{itemize}

\textbf{一般凸组合}:对于 $k$ 个点 $\mathbf{x}_1, \ldots, \mathbf{x}_k$ 和权重 $\theta_1, \ldots, \theta_k \geq 0$ 满足 $\sum_{i=1}^k \theta_i = 1$,点 $\sum_{i=1}^k \theta_i \mathbf{x}_i$ 称为这些点的凸组合。

\section{几何意义}

\subsection{核心思想}

\textbf{凸集的几何特征}:集合中任意两点的连线完全在集合内部(或边界上)。

\textbf{几何描述}:
\begin{enumerate}
\item 在集合中取任意两点 $\mathbf{x}_1$ 和 $\mathbf{x}_2$
\item 画一条连接这两点的线段
\item 如果线段上的所有点都在集合中,则集合是凸集
\item 如果线段上有部分点不在集合中,则集合不是凸集
\end{enumerate}

\subsection{直观理解}

\textbf{凸集的形象}:
\begin{itemize}
\item 集合"凸出来",没有凹陷
\item 像圆盘、三角形、矩形这样的形状
\item 没有"洞"或"缺口"
\end{itemize}

\textbf{非凸集的形象}:
\begin{itemize}
\item 集合有凹陷或缺口
\item 像月牙形、星形、圆环这样的形状
\item 某些两点连线会穿过凹陷部分
\end{itemize}

\section{一维情况的具体例子}

\subsection{凸集的例子}

\textbf{例子1:区间}

集合 $C = [a, b] = \{x \in \mathbb{R} \mid a \leq x \leq b\}$ 是凸集。

\textbf{证明}:
\begin{itemize}
\item 取任意 $x_1, x_2 \in [a, b]$
\item 对于 $\theta \in [0, 1]$,$x = \theta x_1 + (1-\theta) x_2$
\item 由于 $a \leq x_1, x_2 \leq b$,有 $a \leq x \leq b$
\item 因此 $x \in [a, b]$,集合是凸集
\end{itemize}

\textbf{例子2:半直线}

集合 $C = [a, +\infty) = \{x \in \mathbb{R} \mid x \geq a\}$ 是凸集。

\textbf{例子3:整个实数轴}

集合 $C = \mathbb{R}$ 是凸集。

\subsection{非凸集的例子}

\textbf{例子1:两个不相交的区间}

集合 $C = [0, 1] \cup [3, 4]$ 不是凸集。

\textbf{证明}:
\begin{itemize}
\item 取 $x_1 = 1 \in C$,$x_2 = 3 \in C$
\item 中点 $x = 0.5 \times 1 + 0.5 \times 3 = 2$
\item 但 $2 \notin C$(因为 $2$ 不在 $[0, 1]$ 或 $[3, 4]$ 中)
\item 因此集合不是凸集
\end{itemize}

\textbf{例子2:去掉一个点的区间}

集合 $C = [0, 1] \setminus \{0.5\}$ 不是凸集。

\textbf{证明}:
\begin{itemize}
\item 取 $x_1 = 0.4 \in C$,$x_2 = 0.6 \in C$
\item 中点 $x = 0.5 \times 0.4 + 0.5 \times 0.6 = 0.5$
\item 但 $0.5 \notin C$(因为 $0.5$ 被去掉了)
\item 因此集合不是凸集
\end{itemize}

\section{二维情况的具体例子}

\subsection{凸集的例子}

\textbf{例子1:圆盘}

集合 $C = \{\mathbf{x} \in \mathbb{R}^2 \mid \|\mathbf{x}\|_2 \leq r\}$ 是凸集。

\textbf{几何意义}:以原点为中心、半径为 $r$ 的圆及其内部。

\textbf{证明思路}:
\begin{itemize}
\item 取圆盘内任意两点 $\mathbf{x}_1, \mathbf{x}_2$
\item 连接这两点的线段
\item 由于圆盘是"凸出来"的,线段完全在圆盘内
\item 可以使用三角不等式严格证明
\end{itemize}

\textbf{例子2:三角形}

三角形及其内部是凸集。

\textbf{几何意义}:三角形是三个顶点的凸包。

\textbf{例子3:矩形}

矩形及其内部是凸集。

\textbf{例子4:半平面}

集合 $C = \{\mathbf{x} \in \mathbb{R}^2 \mid \mathbf{a}^T \mathbf{x} \leq b\}$ 是凸集。

\textbf{几何意义}:直线 $\mathbf{a}^T \mathbf{x} = b$ 一侧的所有点。

\textbf{例子5:多面体}

有限个半空间的交集(多面体)是凸集。

\subsection{非凸集的例子}

\textbf{例子1:月牙形}

月牙形集合不是凸集。

\textbf{证明}:
\begin{itemize}
\item 取月牙形"两端"的两点
\item 连接这两点的线段会穿过凹陷部分
\item 线段上的某些点不在月牙形内
\item 因此不是凸集
\end{itemize}

\textbf{例子2:星形}

星形集合不是凸集(有尖角凹陷)。

\textbf{例子3:圆环}

集合 $C = \{\mathbf{x} \in \mathbb{R}^2 \mid r_1 \leq \|\mathbf{x}\|_2 \leq r_2\}$ 不是凸集。

\textbf{证明}:
\begin{itemize}
\item 取内圆上一点和外圆上一点
\item 连接这两点的线段会穿过中间的"洞"
\item 线段上的某些点不在圆环内
\item 因此不是凸集
\end{itemize}

\section{凸集的判断方法}

\subsection{方法1:通过定义}

\textbf{步骤}:
\begin{enumerate}
\item 取集合中任意两点 $\mathbf{x}_1, \mathbf{x}_2$
\item 对于任意 $\theta \in [0, 1]$,计算 $\mathbf{x} = \theta \mathbf{x}_1 + (1-\theta) \mathbf{x}_2$
\item 检查 $\mathbf{x}$ 是否在集合中
\item 如果对所有点都成立,则是凸集
\item 如果存在反例,则不是凸集
\end{enumerate}

\subsection{方法2:通过几何直观}

\textbf{步骤}:
\begin{enumerate}
\item 画出集合的几何形状
\item 检查是否有凹陷或缺口
\item 取"最可能违反凸性"的两点(如凹陷两端)
\item 检查连接这两点的线段是否完全在集合中
\end{enumerate}

\subsection{方法3:通过运算}

\textbf{定理}:某些运算保持凸性:
\begin{itemize}
\item 凸集的交集是凸集
\item 凸集的仿射变换是凸集
\item 凸集的投影是凸集
\end{itemize}

如果集合可以通过这些运算从已知凸集得到,则它是凸集。

\section{凸集的性质}

\subsection{基本性质}

\begin{enumerate}
\item \textbf{空集是凸集}:空集 $\emptyset$ 是凸集(空真条件)

\item \textbf{单点集是凸集}:单点集 $\{\mathbf{x}\}$ 是凸集

\item \textbf{整个空间是凸集}:$\mathbb{R}^n$ 是凸集

\item \textbf{凸集的交集是凸集}:
   \begin{itemize}
   \item 如果 $C_1, C_2$ 是凸集,则 $C_1 \cap C_2$ 是凸集
   \item 更一般地,任意多个凸集的交集是凸集
   \end{itemize}

\item \textbf{凸集的仿射变换是凸集}:
   \begin{itemize}
   \item 如果 $C$ 是凸集,$f(\mathbf{x}) = \mathbf{A}\mathbf{x} + \mathbf{b}$ 是仿射函数
   \item 则 $f(C) = \{\mathbf{A}\mathbf{x} + \mathbf{b} \mid \mathbf{x} \in C\}$ 是凸集
   \end{itemize}
\end{enumerate}

\subsection{凸集的运算}

\begin{enumerate}
\item \textbf{凸组合}:凸集中任意有限个点的凸组合仍在凸集中

\item \textbf{凸包}:集合 $S$ 的凸包 $\text{conv}(S)$ 是包含 $S$ 的最小凸集

\item \textbf{和集}:如果 $C_1, C_2$ 是凸集,则 $C_1 + C_2 = \{\mathbf{x}_1 + \mathbf{x}_2 \mid \mathbf{x}_1 \in C_1, \mathbf{x}_2 \in C_2\}$ 是凸集

\item \textbf{标量乘法}:如果 $C$ 是凸集,$\alpha \in \mathbb{R}$,则 $\alpha C = \{\alpha \mathbf{x} \mid \mathbf{x} \in C\}$ 是凸集
\end{enumerate}

\section{在优化中的应用}

\subsection{可行域}

\textbf{凸优化问题}:
\begin{align}
\begin{array}{ll}
\text{minimize} & f_0(\mathbf{x}) \\
\text{subject to} & f_i(\mathbf{x}) \leq 0, \quad i = 1, \ldots, m \\
& \mathbf{A}\mathbf{x} = \mathbf{b}
\end{array}
\end{align}

\textbf{可行域}:
\begin{equation}
\mathcal{F} = \{\mathbf{x} \mid f_i(\mathbf{x}) \leq 0, i = 1, \ldots, m, \mathbf{A}\mathbf{x} = \mathbf{b}\}
\end{equation}

\textbf{关键性质}:
\begin{itemize}
\item 如果所有 $f_i$ 是凸函数,则可行域是凸集
\item 可行域是多个凸集的交集
\item 凸集的交集是凸集
\end{itemize}

\subsection{最优性}

\textbf{凸优化的优势}:
\begin{itemize}
\item 如果可行域是凸集,目标函数是凸函数
\item 则局部最优 = 全局最优
\item 这是凸优化的关键优势
\end{itemize}

\section{常见凸集}

\subsection{基本凸集}

\begin{enumerate}
\item \textbf{空集}:$\emptyset$

\item \textbf{单点集}:$\{\mathbf{x}\}$

\item \textbf{整个空间}:$\mathbb{R}^n$

\item \textbf{仿射集合}:$\{\mathbf{x} \mid \mathbf{A}\mathbf{x} = \mathbf{b}\}$

\item \textbf{半空间}:$\{\mathbf{x} \mid \mathbf{a}^T \mathbf{x} \leq b\}$

\item \textbf{多面体}:有限个半空间的交集
\end{enumerate}

\subsection{特殊凸集}

\begin{enumerate}
\item \textbf{欧几里得球}:$\{\mathbf{x} \mid \|\mathbf{x} - \mathbf{x}_c\|_2 \leq r\}$

\item \textbf{椭球}:$\{\mathbf{x} \mid (\mathbf{x} - \mathbf{x}_c)^T \mathbf{P}^{-1} (\mathbf{x} - \mathbf{x}_c) \leq 1\}$,其中 $\mathbf{P} \succ 0$

\item \textbf{范数球}:$\{\mathbf{x} \mid \|\mathbf{x}\| \leq r\}$(对于任意范数)

\item \textbf{范数锥}:$\{(\mathbf{x}, t) \mid \|\mathbf{x}\| \leq t\}$

\item \textbf{半正定锥}:$\{\mathbf{X} \in \mathbb{S}^n \mid \mathbf{X} \succeq 0\}$
\end{enumerate}

\section{总结}

\subsection{凸集的定义}

\textbf{数学定义}:集合 $C \subseteq \mathbb{R}^n$ 是凸集,如果对于任意 $\mathbf{x}_1, \mathbf{x}_2 \in C$ 和 $\theta \in [0, 1]$,有:

\begin{equation}
\theta \mathbf{x}_1 + (1-\theta) \mathbf{x}_2 \in C
\end{equation}

\textbf{几何意义}:集合中任意两点的连线完全在集合内部(或边界上)。

\subsection{关键特征}

\begin{enumerate}
\item \textbf{没有凹陷}:集合"凸出来"
\item \textbf{没有洞}:集合是"实心"的
\item \textbf{任意两点连线在集合内}
\end{enumerate}

\subsection{判断方法}

\begin{enumerate}
\item \textbf{通过定义}:验证凸组合是否在集合中
\item \textbf{通过几何直观}:检查是否有凹陷
\item \textbf{通过运算}:利用凸集的运算性质
\end{enumerate}

\subsection{在优化中的重要性}

\begin{enumerate}
\item \textbf{可行域必须是凸集}:凸优化问题的可行域是凸集
\item \textbf{局部最优 = 全局最优}:在凸集上优化凸函数
\item \textbf{理论保证}:凸集的性质保证了优化的良好性质
\end{enumerate}

理解凸集的定义,是理解凸优化理论的基础!

\end{document}

