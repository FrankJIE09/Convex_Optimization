\documentclass[12pt,a4paper]{article}
\usepackage[UTF8]{ctex}
\usepackage{amsmath}
\usepackage{amssymb}
\usepackage{amsthm}
\usepackage{geometry}
\geometry{left=2.5cm,right=2.5cm,top=2.5cm,bottom=2.5cm}

\title{支撑超平面(Supporting Hyperplane)详解}
\author{}
\date{\today}

\begin{document}

\maketitle

\section{引言}

支撑超平面是凸分析和优化理论中的重要概念,与最优性条件、分离定理等密切相关。理解支撑超平面有助于理解凸优化的几何本质。

\section{超平面的定义}

\subsection{基本定义}

\textbf{超平面}:在 $\mathbb{R}^n$ 中,超平面是 $n-1$ 维的仿射集合。

\textbf{数学表述}:

\begin{equation}
H = \{\mathbf{x} \in \mathbb{R}^n \mid \mathbf{a}^T \mathbf{x} = b\}
\end{equation}

其中:
\begin{itemize}
\item $\mathbf{a} \in \mathbb{R}^n$,$\mathbf{a} \neq \mathbf{0}$:超平面的法向量
\item $b \in \mathbb{R}$:常数项
\end{itemize}

\subsection{几何意义}

\textbf{几何描述}:
\begin{itemize}
\item 超平面将空间分成两个半空间
\item 法向量 $\mathbf{a}$ 垂直于超平面
\item 超平面是"平的",没有曲率
\end{itemize}

\textbf{半空间}:

超平面 $H = \{\mathbf{x} \mid \mathbf{a}^T \mathbf{x} = b\}$ 将空间分成两个半空间:

\begin{align}
H_+ &= \{\mathbf{x} \in \mathbb{R}^n \mid \mathbf{a}^T \mathbf{x} \geq b\} \quad \text{(正半空间)} \\
H_- &= \{\mathbf{x} \in \mathbb{R}^n \mid \mathbf{a}^T \mathbf{x} \leq b\} \quad \text{(负半空间)}
\end{align}

\section{支撑超平面的定义}

\subsection{基本定义}

\textbf{支撑超平面}:超平面 $H = \{\mathbf{x} \mid \mathbf{a}^T \mathbf{x} = b\}$ 是集合 $C \subseteq \mathbb{R}^n$ 在点 $\mathbf{x}_0 \in \text{bd } C$ 处的支撑超平面,如果:

\begin{enumerate}
\item $\mathbf{x}_0 \in H \cap C$(超平面经过点 $\mathbf{x}_0$,且 $\mathbf{x}_0$ 在集合的边界上)

\item $C \subseteq H_+$ 或 $C \subseteq H_-$(集合完全在超平面的一侧)
\end{enumerate}

\textbf{等价表述}:

对于所有 $\mathbf{x} \in C$,有 $\mathbf{a}^T \mathbf{x} \geq b$(或 $\mathbf{a}^T \mathbf{x} \leq b$),且 $\mathbf{a}^T \mathbf{x}_0 = b$。

\subsection{数学表述}

\textbf{支撑超平面}:超平面 $H = \{\mathbf{x} \mid \mathbf{a}^T \mathbf{x} = b\}$ 是集合 $C$ 在点 $\mathbf{x}_0$ 处的支撑超平面,如果:

\begin{enumerate}
\item $\mathbf{x}_0 \in \text{bd } C$($\mathbf{x}_0$ 在集合的边界上)

\item $\mathbf{a}^T \mathbf{x}_0 = b$(超平面经过点 $\mathbf{x}_0$)

\item 对于所有 $\mathbf{x} \in C$,有 $\mathbf{a}^T \mathbf{x} \geq b$(集合在超平面的一侧)
\end{enumerate}

\section{几何直观}

\subsection{二维情况}

\textbf{例子}:圆盘 $C = \{\mathbf{x} \in \mathbb{R}^2 \mid \|\mathbf{x}\|_2 \leq 1\}$

\textbf{支撑超平面}:
\begin{itemize}
\item 在边界点 $(1, 0)$ 处:超平面 $x = 1$ 是支撑超平面
\item 在边界点 $(0, 1)$ 处:超平面 $y = 1$ 是支撑超平面
\item 在任意边界点处:存在唯一的支撑超平面(切线)
\end{itemize}

\textbf{几何意义}:
\begin{itemize}
\item 支撑超平面"支撑"集合,使其不越过超平面
\item 集合完全在超平面的一侧
\item 超平面与集合在边界点处相切
\end{itemize}

\subsection{三维情况}

\textbf{例子}:球 $C = \{\mathbf{x} \in \mathbb{R}^3 \mid \|\mathbf{x}\|_2 \leq 1\}$

\textbf{支撑超平面}:
\begin{itemize}
\item 在边界点处:支撑超平面是切平面
\item 法向量指向球心
\item 球完全在超平面的一侧
\end{itemize}

\section{支撑超平面的性质}

\subsection{存在性}

\textbf{定理}(支撑超平面定理):如果 $C$ 是凸集,且 $\mathbf{x}_0 \in \text{bd } C$,则存在支撑超平面在点 $\mathbf{x}_0$ 处。

\textbf{注意}:
\begin{itemize}
\item 对于凸集,边界点处总是存在支撑超平面
\item 对于非凸集,可能不存在支撑超平面
\end{itemize}

\subsection{唯一性}

\textbf{唯一性}:
\begin{itemize}
\item 对于光滑的凸集(如球、椭球),每个边界点处有唯一的支撑超平面
\item 对于有"角"的凸集(如多面体),在角点处可能有多个支撑超平面
\end{itemize}

\textbf{例子}:
\begin{itemize}
\item 圆盘:每个边界点处有唯一的支撑超平面(切线)
\item 正方形:在顶点处有多个支撑超平面
\item 球:每个边界点处有唯一的支撑超平面(切平面)
\end{itemize}

\section{在优化中的应用}

\subsection{最优性条件}

\textbf{定理}:对于可微的凸优化问题,点 $\mathbf{x}$ 是最优的,当且仅当 $-\nabla f_0(\mathbf{x})$ 定义了可行集在点 $\mathbf{x}$ 处的支撑超平面。

\textbf{数学表述}:

如果 $\mathbf{x}$ 是最优的,则超平面:

\begin{equation}
H = \{\mathbf{z} \mid -\nabla f_0(\mathbf{x})^T (\mathbf{z} - \mathbf{x}) = 0\}
\end{equation}

是可行集 $X$ 在点 $\mathbf{x}$ 处的支撑超平面。

\textbf{验证}:
\begin{itemize}
\item $\mathbf{x} \in H \cap X$ ✓
\item 对于所有 $\mathbf{y} \in X$,有 $-\nabla f_0(\mathbf{x})^T (\mathbf{y} - \mathbf{x}) \leq 0$
\item 等价地:$\nabla f_0(\mathbf{x})^T (\mathbf{y} - \mathbf{x}) \geq 0$ ✓
\end{itemize}

\subsection{几何解释}

\textbf{最优性条件的几何意义}:
\begin{itemize}
\item \textbf{梯度方向}:$\nabla f_0(\mathbf{x})$ 指向函数值增加最快的方向
\item \textbf{负梯度方向}:$-\nabla f_0(\mathbf{x})$ 指向函数值减小最快的方向
\item \textbf{支撑超平面}:$-\nabla f_0(\mathbf{x})$ 定义了支撑超平面
\item \textbf{含义}:在最优解处,没有可行方向能使函数值减小
\end{itemize}

\section{具体例子}

\subsection{例子1:二维优化问题}

\textbf{问题}:$\min_{x, y} x^2 + y^2$ subject to $x + y \geq 1$

\textbf{可行集}:$X = \{(x, y) \mid x + y \geq 1\}$(半平面)

\textbf{最优解}:$\mathbf{x}^* = (0.5, 0.5)$(在约束边界上)

\textbf{梯度}:$\nabla f_0(0.5, 0.5) = (1, 1)^T$

\textbf{支撑超平面}:

\begin{equation}
H = \{(x, y) \mid -(1, 1)^T ((x, y) - (0.5, 0.5)) = 0\} = \{(x, y) \mid x + y = 1\}
\end{equation}

\textbf{验证}:
\begin{itemize}
\item $(0.5, 0.5) \in H \cap X$ ✓
\item 对于所有 $(x, y) \in X$($x + y \geq 1$),有 $x + y \geq 1$,即 $-(1, 1)^T ((x, y) - (0.5, 0.5)) \leq 0$ ✓
\end{itemize}

\subsection{例子2:圆盘上的优化}

\textbf{问题}:$\min_{x, y} x + y$ subject to $x^2 + y^2 \leq 1$

\textbf{可行集}:$X = \{(x, y) \mid x^2 + y^2 \leq 1\}$(单位圆盘)

\textbf{最优解}:$\mathbf{x}^* = \left(-\frac{1}{\sqrt{2}}, -\frac{1}{\sqrt{2}}\right)$(在边界上)

\textbf{梯度}:$\nabla f_0 = (1, 1)^T$

\textbf{支撑超平面}:

在点 $\mathbf{x}^*$ 处,支撑超平面是:

\begin{equation}
H = \left\{(x, y) \mid -(1, 1)^T \left((x, y) - \left(-\frac{1}{\sqrt{2}}, -\frac{1}{\sqrt{2}}\right)\right) = 0\right\}
\end{equation}

即:$x + y = -\sqrt{2}$

\textbf{几何意义}:
\begin{itemize}
\item 支撑超平面是圆盘在点 $\mathbf{x}^*$ 处的切线
\item 法向量是 $-\nabla f_0 = -(1, 1)^T$
\item 圆盘完全在超平面的一侧
\end{itemize}

\section{支撑超平面与分离定理}

\subsection{分离定理}

\textbf{定理}(分离超平面定理):如果 $C$ 和 $D$ 是两个不相交的凸集,则存在超平面分离它们。

\textbf{支撑超平面的作用}:
\begin{itemize}
\item 支撑超平面是分离定理的特殊情况
\item 当两个集合"相切"时,支撑超平面就是分离超平面
\end{itemize}

\subsection{关系}

\textbf{支撑超平面 vs 分离超平面}:
\begin{itemize}
\item \textbf{分离超平面}:分离两个不相交的集合
\item \textbf{支撑超平面}:支撑一个集合(可以看作集合与"外部"的分离)
\end{itemize}

\section{支撑超平面的计算}

\subsection{对于凸集}

\textbf{方法1}:通过边界点的法向量

如果 $\mathbf{x}_0 \in \text{bd } C$,且 $C$ 在 $\mathbf{x}_0$ 处可微,则:

\begin{equation}
H = \{\mathbf{x} \mid \mathbf{n}^T (\mathbf{x} - \mathbf{x}_0) = 0\}
\end{equation}

其中 $\mathbf{n}$ 是边界在 $\mathbf{x}_0$ 处的法向量。

\textbf{方法2}:通过优化问题

对于优化问题,支撑超平面由梯度决定:

\begin{equation}
H = \{\mathbf{z} \mid -\nabla f_0(\mathbf{x})^T (\mathbf{z} - \mathbf{x}) = 0\}
\end{equation}

\section{总结}

\subsection{支撑超平面的定义}

\begin{enumerate}
\item \textbf{超平面经过边界点}:$\mathbf{x}_0 \in H \cap \text{bd } C$

\item \textbf{集合在一侧}:$C \subseteq H_+$ 或 $C \subseteq H_-$

\item \textbf{支撑性质}:超平面"支撑"集合,使其不越过超平面
\end{enumerate}

\subsection{在优化中的应用}

\begin{enumerate}
\item \textbf{最优性条件}:$-\nabla f_0(\mathbf{x})$ 定义支撑超平面

\item \textbf{几何意义}:在最优解处,没有可行方向能使函数值减小

\item \textbf{判断最优性}:通过检查支撑超平面是否存在
\end{enumerate}

\subsection{关键理解}

\begin{enumerate}
\item \textbf{支撑超平面}:超平面"支撑"集合,集合在超平面的一侧

\item \textbf{存在性}:对于凸集,边界点处总是存在支撑超平面

\item \textbf{唯一性}:对于光滑的凸集,每个边界点处有唯一的支撑超平面

\item \textbf{在优化中}:最优性条件可以用支撑超平面来表述
\end{enumerate}

理解支撑超平面,是理解凸优化几何本质的关键!

\end{document}

