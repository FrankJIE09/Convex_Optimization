\documentclass[12pt,a4paper]{article}
\usepackage[UTF8]{ctex}
\usepackage{amsmath}
\usepackage{amssymb}
\usepackage{amsthm}
\usepackage{geometry}
\geometry{left=2.5cm,right=2.5cm,top=2.5cm,bottom=2.5cm}

\title{如何理解仿射包的最小性?}
\subtitle{为什么 $\text{aff } C$ 是包含 $C$ 的最小仿射集合?}
\author{}
\date{\today}

\begin{document}

\maketitle

\section{问题提出}

\textbf{定义}:对于集合 $C \subseteq \mathbb{R}^n$,$C$ 的\textbf{仿射包}(Affine Hull)定义为:
\begin{equation}
\text{aff } C = \{\theta_1 \mathbf{x}_1 + \cdots + \theta_k \mathbf{x}_k \mid \mathbf{x}_1, \ldots, \mathbf{x}_k \in C, \theta_1 + \cdots + \theta_k = 1\}
\end{equation}

\textbf{性质}:$\text{aff } C$ 是包含 $C$ 的\textbf{最小}仿射集合。

\textbf{问题}:如何理解"最小"的含义?为什么说它是"最小"的?

\section{"最小"的含义}

\subsection{集合论中的"最小"}

在集合论中,当我们说集合 $A$ 是满足某个性质的"最小"集合时,通常有两种含义:

\begin{enumerate}
\item \textbf{包含关系下的最小}:如果 $B$ 是任何满足该性质的集合,且 $A \subseteq B$,那么 $A = B$。
\item \textbf{等价表述}:$A$ 包含在任何满足该性质的集合中,即:如果 $S$ 满足该性质且 $A \subseteq S$,那么 $A$ 是"最小"的意味着不存在更小的满足该性质的集合 $A'$ 使得 $A' \subsetneq A$。
\end{enumerate}

\subsection{在仿射包中的含义}

对于仿射包,\textbf{"最小"的含义}是:

如果 $S$ 是任何包含 $C$ 的仿射集合,那么:
\begin{equation}
\text{aff } C \subseteq S
\end{equation}

换句话说:
\begin{itemize}
\item $\text{aff } C$ 包含在\textbf{所有}包含 $C$ 的仿射集合中
\item 不存在更小的仿射集合同时包含 $C$
\item $\text{aff } C$ 是包含 $C$ 的所有仿射集合的\textbf{交集}
\end{itemize}

\section{为什么仿射包是最小的?}

\subsection{证明思路}

要证明 $\text{aff } C$ 是最小的,需要证明两点:

\begin{enumerate}
\item $\text{aff } C$ 是仿射集合
\item 对于任何包含 $C$ 的仿射集合 $S$,都有 $\text{aff } C \subseteq S$
\end{enumerate}

\subsection{证明:$\text{aff } C$ 是仿射集合}

\textbf{证明}:设 $\mathbf{y}_1, \mathbf{y}_2 \in \text{aff } C$,$\theta \in \mathbb{R}$。

根据定义,存在 $\mathbf{x}_1^{(1)}, \ldots, \mathbf{x}_{k_1}^{(1)} \in C$ 和系数 $\alpha_1, \ldots, \alpha_{k_1}$(和为1),使得:
\begin{equation}
\mathbf{y}_1 = \sum_{i=1}^{k_1} \alpha_i \mathbf{x}_i^{(1)}
\end{equation}

类似地,存在 $\mathbf{x}_1^{(2)}, \ldots, \mathbf{x}_{k_2}^{(2)} \in C$ 和系数 $\beta_1, \ldots, \beta_{k_2}$(和为1),使得:
\begin{equation}
\mathbf{y}_2 = \sum_{j=1}^{k_2} \beta_j \mathbf{x}_j^{(2)}
\end{equation}

现在考虑 $\theta \mathbf{y}_1 + (1-\theta) \mathbf{y}_2$:
\begin{align}
\theta \mathbf{y}_1 + (1-\theta) \mathbf{y}_2 &= \theta \sum_{i=1}^{k_1} \alpha_i \mathbf{x}_i^{(1)} + (1-\theta) \sum_{j=1}^{k_2} \beta_j \mathbf{x}_j^{(2)} \\
&= \sum_{i=1}^{k_1} \theta \alpha_i \mathbf{x}_i^{(1)} + \sum_{j=1}^{k_2} (1-\theta) \beta_j \mathbf{x}_j^{(2)}
\end{align}

这是 $C$ 中点的仿射组合,因为:
\begin{align}
\sum_{i=1}^{k_1} \theta \alpha_i + \sum_{j=1}^{k_2} (1-\theta) \beta_j &= \theta \sum_{i=1}^{k_1} \alpha_i + (1-\theta) \sum_{j=1}^{k_2} \beta_j \\
&= \theta \cdot 1 + (1-\theta) \cdot 1 \\
&= 1
\end{align}

因此 $\theta \mathbf{y}_1 + (1-\theta) \mathbf{y}_2 \in \text{aff } C$,所以 $\text{aff } C$ 是仿射集合。$\square$

\subsection{证明:$\text{aff } C$ 包含在任何包含 $C$ 的仿射集合中}

\textbf{证明}:设 $S$ 是任何包含 $C$ 的仿射集合,即 $C \subseteq S$。

我们需要证明:$\text{aff } C \subseteq S$。

对于任意 $\mathbf{y} \in \text{aff } C$,根据定义,存在 $\mathbf{x}_1, \ldots, \mathbf{x}_k \in C$ 和系数 $\theta_1, \ldots, \theta_k$(和为1),使得:
\begin{equation}
\mathbf{y} = \sum_{i=1}^k \theta_i \mathbf{x}_i
\end{equation}

由于 $C \subseteq S$,所以 $\mathbf{x}_1, \ldots, \mathbf{x}_k \in S$。

由于 $S$ 是仿射集合,它包含其中任意点的所有仿射组合。因此:
\begin{equation}
\mathbf{y} = \sum_{i=1}^k \theta_i \mathbf{x}_i \in S
\end{equation}

所以 $\text{aff } C \subseteq S$。$\square$

\subsection{结论}

\begin{enumerate}
\item $\text{aff } C$ 是仿射集合
\item $\text{aff } C$ 包含在任何包含 $C$ 的仿射集合中
\item 因此,$\text{aff } C$ 是包含 $C$ 的\textbf{最小}仿射集合
\end{enumerate}

\section{另一种理解:交集观点}

\subsection{交集定义}

我们可以将仿射包定义为所有包含 $C$ 的仿射集合的交集:

\begin{equation}
\text{aff } C = \bigcap \{S \mid S \text{ 是仿射集合且 } C \subseteq S\}
\end{equation}

\subsection{为什么这两种定义等价?}

\textbf{证明等价性}:

设 $A = \{\theta_1 \mathbf{x}_1 + \cdots + \theta_k \mathbf{x}_k \mid \mathbf{x}_i \in C, \sum \theta_i = 1\}$(定义1)

设 $B = \bigcap \{S \mid S \text{ 是仿射集合且 } C \subseteq S\}$(定义2)

\textbf{证明 $A \subseteq B$}:

对于任意 $\mathbf{y} \in A$,存在 $\mathbf{x}_1, \ldots, \mathbf{x}_k \in C$ 和系数(和为1),使得 $\mathbf{y} = \sum \theta_i \mathbf{x}_i$。

对于任何包含 $C$ 的仿射集合 $S$,由于 $\mathbf{x}_i \in C \subseteq S$,且 $S$ 是仿射集合,所以 $\mathbf{y} \in S$。

因此 $\mathbf{y} \in B$,所以 $A \subseteq B$。

\textbf{证明 $B \subseteq A$}:

我们已经证明 $A$ 是仿射集合且包含 $C$,所以 $A$ 是交集中的一个集合。

因此 $B \subseteq A$。

所以 $A = B$。$\square$

\subsection{交集观点的优势}

交集观点更直观地说明了"最小性":
\begin{itemize}
\item 交集是包含在所有集合中的最大集合
\item 因此是"最小"的满足条件的集合
\end{itemize}

\section{具体例子}

\subsection{例子1:两个点}

在 $\mathbb{R}^2$ 中,设 $C = \{(0, 0), (1, 0)\}$(两个点)。

\textbf{仿射包}:
\begin{equation}
\text{aff } C = \{\theta(0, 0) + (1-\theta)(1, 0) \mid \theta \in \mathbb{R}\} = \{(1-\theta, 0) \mid \theta \in \mathbb{R}\} = \{(x, 0) \mid x \in \mathbb{R}\}
\end{equation}

这是 $x$ 轴(一条直线)。

\textbf{验证最小性}:
\begin{itemize}
\item 任何包含这两个点的仿射集合必须包含通过这两点的整条直线
\item 因此必须包含 $x$ 轴
\item 所以 $\text{aff } C$($x$ 轴)是最小的
\end{itemize}

\subsection{例子2:三个不共线的点}

在 $\mathbb{R}^2$ 中,设 $C = \{(0, 0), (1, 0), (0, 1)\}$(三个点)。

\textbf{仿射包}:
\begin{equation}
\text{aff } C = \{\theta_1(0, 0) + \theta_2(1, 0) + \theta_3(0, 1) \mid \theta_1 + \theta_2 + \theta_3 = 1\}
\end{equation}

这实际上是整个 $\mathbb{R}^2$(因为三个不共线的点确定整个平面)。

\textbf{验证最小性}:
\begin{itemize}
\item 任何包含这三个点的仿射集合必须包含通过它们的仿射组合
\item 由于三点不共线,这些组合覆盖整个平面
\item 因此 $\text{aff } C = \mathbb{R}^2$ 是最小的
\end{itemize}

\subsection{例子3:圆上的点}

在 $\mathbb{R}^2$ 中,设 $C = \{(x, y) \mid x^2 + y^2 = 1\}$(单位圆)。

\textbf{仿射包}:
\begin{equation}
\text{aff } C = \mathbb{R}^2
\end{equation}

因为单位圆上的点不都在同一条直线上,它们的仿射组合可以覆盖整个平面。

\textbf{验证最小性}:
\begin{itemize}
\item 任何包含单位圆的仿射集合必须包含整个平面
\item 因此 $\text{aff } C = \mathbb{R}^2$ 是最小的
\end{itemize}

\subsection{例子4:共线的点}

在 $\mathbb{R}^2$ 中,设 $C = \{(0, 0), (1, 0), (2, 0)\}$(三个共线的点)。

\textbf{仿射包}:
\begin{equation}
\text{aff } C = \{(x, 0) \mid x \in \mathbb{R}\}
\end{equation}

这是 $x$ 轴。

\textbf{验证最小性}:
\begin{itemize}
\item 虽然有三个点,但它们都在同一条直线上
\item 任何包含这三个点的仿射集合必须包含这条直线
\item 因此 $\text{aff } C$($x$ 轴)是最小的
\end{itemize}

\section{最小性的直观理解}

\subsection{类比:凸包}

类似于凸包(convex hull):
\begin{itemize}
\item 凸包是包含集合的最小凸集合
\item 仿射包是包含集合的最小仿射集合
\end{itemize}

\subsection{几何直观}

\begin{itemize}
\item \textbf{最小}意味着"刚好够用":
  \begin{itemize}
  \item 包含 $C$ 所需的最小仿射空间
  \item 不多不少,正好是 $C$ 中点的所有仿射组合
  \end{itemize}

\item \textbf{唯一性}:
  \begin{itemize}
  \item 虽然可能有多个包含 $C$ 的仿射集合
  \item 但最小的那个是唯一的,就是 $\text{aff } C$
  \end{itemize}

\item \textbf{构造性}:
  \begin{itemize}
  \item $\text{aff } C$ 由 $C$ 中点的所有仿射组合构成
  \item 这是"最小"的,因为任何更小的集合都无法包含所有这些组合
  \end{itemize}
\end{itemize}

\section{最小性与维度的关系}

\subsection{仿射维度}

\textbf{定义}:集合 $C$ 的\textbf{仿射维度}(Affine Dimension)定义为 $\text{aff } C$ 的维度。

\textbf{性质}:
\begin{itemize}
\item 如果 $C$ 中的点处于"一般位置"(不都在低维仿射空间中),则 $\dim(\text{aff } C)$ 等于 $C$ 中点的"有效"数量减1
\item 例如:两个点确定一条直线(1维),三个不共线的点确定一个平面(2维)
\end{itemize}

\subsection{最小性与维度}

\begin{itemize}
\item $\text{aff } C$ 的维度是包含 $C$ 的仿射集合的最小可能维度
\item 任何包含 $C$ 的仿射集合的维度至少等于 $\dim(\text{aff } C)$
\item 如果另一个仿射集合 $S$ 的维度等于 $\dim(\text{aff } C)$ 且 $C \subseteq S$,那么 $S = \text{aff } C$
\end{itemize}

\section{应用:为什么最小性重要?}

\subsection{在优化问题中}

\begin{itemize}
\item 如果优化问题的可行域是仿射集合,我们通常关心它的维度
\item 仿射包的维度告诉我们问题的"自由度"
\item 最小性确保我们考虑的是最紧凑的表示
\end{itemize}

\subsection{在几何问题中}

\begin{itemize}
\item 确定一组点的"有效维度"
\item 判断点是否共线、共面等
\item 计算相对内部(relative interior)时需要知道仿射包
\end{itemize}

\section{总结}

\begin{enumerate}
\item \textbf{"最小"的含义}:
   \begin{itemize}
   \item $\text{aff } C$ 包含在任何包含 $C$ 的仿射集合中
   \item 不存在更小的仿射集合同时包含 $C$
   \end{itemize}

\item \textbf{为什么是最小的}:
   \begin{itemize}
   \item $\text{aff } C$ 由 $C$ 中点的所有仿射组合构成
   \item 任何包含 $C$ 的仿射集合必须包含这些组合
   \item 因此必须包含 $\text{aff } C$
   \end{itemize}

\item \textbf{等价表述}:
   \begin{itemize}
   \item $\text{aff } C$ 是所有包含 $C$ 的仿射集合的交集
   \item 这更直观地说明了"最小性"
   \end{itemize}

\item \textbf{几何意义}:
   \begin{itemize}
   \item 最小意味着"刚好够用"
   \item 是包含 $C$ 所需的最小仿射空间
   \item 维度是最小的可能维度
   \end{itemize}
\end{enumerate}

理解最小性对于掌握仿射包、相对内部等概念非常重要!

\end{document}

