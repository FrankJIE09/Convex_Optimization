\documentclass[12pt,a4paper]{article}
\usepackage[UTF8]{ctex}
\usepackage{amsmath}
\usepackage{amssymb}
\usepackage{amsthm}
\usepackage{geometry}
\geometry{left=2.5cm,right=2.5cm,top=2.5cm,bottom=2.5cm}

\title{线性子空间与子空间}
\author{}
\date{\today}

\begin{document}

\maketitle

\section{引言}

线性子空间(Linear Subspace)和子空间(Subspace)是线性代数中的基础概念,在凸优化理论中也经常出现。理解这些概念对于学习仿射集合、凸集合等概念至关重要。本文将详细解释这两个概念的定义、性质和区别。

\section{线性子空间(Linear Subspace)}

\subsection{定义}

设 $V$ 是向量空间 $\mathbb{R}^n$ 的一个子集。如果 $V$ 满足以下三个条件,则称 $V$ 为 $\mathbb{R}^n$ 的\textbf{线性子空间}(Linear Subspace)或\textbf{向量子空间}(Vector Subspace):

\begin{enumerate}
\item \textbf{包含零向量}:$\mathbf{0} \in V$
\item \textbf{对加法封闭}:对于任意 $\mathbf{v}_1, \mathbf{v}_2 \in V$,都有 $\mathbf{v}_1 + \mathbf{v}_2 \in V$
\item \textbf{对数乘封闭}:对于任意 $\mathbf{v} \in V$ 和任意标量 $\alpha \in \mathbb{R}$,都有 $\alpha \mathbf{v} \in V$
\end{enumerate}

\subsection{等价定义}

线性子空间也可以等价地定义为:对于任意 $\mathbf{v}_1, \mathbf{v}_2 \in V$ 和任意 $\alpha, \beta \in \mathbb{R}$,都有:

\begin{equation}
\alpha \mathbf{v}_1 + \beta \mathbf{v}_2 \in V
\end{equation}

这个定义包含了上述三个条件:
\begin{itemize}
\item 当 $\alpha = \beta = 0$ 时,得到 $\mathbf{0} \in V$
\item 当 $\alpha = \beta = 1$ 时,得到对加法封闭
\item 当 $\beta = 0$ 时,得到对数乘封闭
\end{itemize}

\subsection{几何直观}

线性子空间在几何上可以理解为:
\begin{itemize}
\item \textbf{通过原点的直线}(一维子空间)
\item \textbf{通过原点的平面}(二维子空间)
\item \textbf{通过原点的超平面}($n-1$ 维子空间)
\item \textbf{整个空间}($n$ 维子空间)
\item \textbf{只包含原点}(零维子空间,$\{\mathbf{0}\}$)
\end{itemize}

\textbf{关键特征}:所有线性子空间都必须经过原点!

\subsection{例子}

\textbf{例子1:一维子空间(直线)}

在 $\mathbb{R}^2$ 中,集合 $V = \{(t, 2t) \mid t \in \mathbb{R}\}$ 是一个一维线性子空间。

验证:
\begin{itemize}
\item $\mathbf{0} = (0, 0) \in V$(当 $t = 0$ 时)
\item 对加法封闭:$(t_1, 2t_1) + (t_2, 2t_2) = (t_1 + t_2, 2(t_1 + t_2)) \in V$
\item 对数乘封闭:$\alpha(t, 2t) = (\alpha t, 2\alpha t) \in V$
\end{itemize}

几何上,这是通过原点且方向向量为 $(1, 2)$ 的直线。

\textbf{例子2:二维子空间(平面)}

在 $\mathbb{R}^3$ 中,集合 $V = \{(x, y, 0) \mid x, y \in \mathbb{R}\}$ 是一个二维线性子空间。

验证:
\begin{itemize}
\item $\mathbf{0} = (0, 0, 0) \in V$
\item 对加法和数乘封闭(容易验证)
\end{itemize}

几何上,这是 $xy$ 平面($z = 0$ 的平面)。

\textbf{例子3:零空间(Null Space)}

对于矩阵 $\mathbf{A} \in \mathbb{R}^{m \times n}$,集合
\begin{equation}
\text{null}(\mathbf{A}) = \{\mathbf{x} \in \mathbb{R}^n \mid \mathbf{A}\mathbf{x} = \mathbf{0}\}
\end{equation}
是 $\mathbb{R}^n$ 的一个线性子空间,称为 $\mathbf{A}$ 的\textbf{零空间}或\textbf{核}(Kernel)。

验证:
\begin{itemize}
\item $\mathbf{A}\mathbf{0} = \mathbf{0}$,所以 $\mathbf{0} \in \text{null}(\mathbf{A})$
\item 如果 $\mathbf{x}_1, \mathbf{x}_2 \in \text{null}(\mathbf{A})$,则 $\mathbf{A}(\mathbf{x}_1 + \mathbf{x}_2) = \mathbf{A}\mathbf{x}_1 + \mathbf{A}\mathbf{x}_2 = \mathbf{0} + \mathbf{0} = \mathbf{0}$,所以对加法封闭
\item 如果 $\mathbf{x} \in \text{null}(\mathbf{A})$,$\alpha \in \mathbb{R}$,则 $\mathbf{A}(\alpha \mathbf{x}) = \alpha \mathbf{A}\mathbf{x} = \alpha \mathbf{0} = \mathbf{0}$,所以对数乘封闭
\end{itemize}

\textbf{例子4:列空间(Column Space)}

对于矩阵 $\mathbf{A} \in \mathbb{R}^{m \times n}$,集合
\begin{equation}
\text{col}(\mathbf{A}) = \{\mathbf{A}\mathbf{x} \mid \mathbf{x} \in \mathbb{R}^n\}
\end{equation}
是 $\mathbb{R}^m$ 的一个线性子空间,称为 $\mathbf{A}$ 的\textbf{列空间}(Column Space)或\textbf{值域}(Range)。

\subsection{线性子空间的性质}

\begin{enumerate}
\item \textbf{必须包含原点}:这是线性子空间与仿射集合的关键区别。

\item \textbf{维度的限制}:$\mathbb{R}^n$ 的线性子空间的维度可以是 $0, 1, 2, \ldots, n$。

\item \textbf{生成子空间}:对于向量集合 $\{\mathbf{v}_1, \mathbf{v}_2, \ldots, \mathbf{v}_k\}$,所有线性组合的集合
  \begin{equation}
  \text{span}\{\mathbf{v}_1, \mathbf{v}_2, \ldots, \mathbf{v}_k\} = \{\alpha_1 \mathbf{v}_1 + \cdots + \alpha_k \mathbf{v}_k \mid \alpha_i \in \mathbb{R}\}
  \end{equation}
  是一个线性子空间,称为由这些向量\textbf{生成}(span)的子空间。

\item \textbf{交集是子空间}:如果 $V_1$ 和 $V_2$ 都是线性子空间,那么 $V_1 \cap V_2$ 也是线性子空间。

\item \textbf{和空间}:如果 $V_1$ 和 $V_2$ 都是线性子空间,那么
  \begin{equation}
  V_1 + V_2 = \{\mathbf{v}_1 + \mathbf{v}_2 \mid \mathbf{v}_1 \in V_1, \mathbf{v}_2 \in V_2\}
  \end{equation}
  也是线性子空间。
\end{enumerate}

\section{子空间(Subspace)}

\subsection{一般定义}

在更广泛的数学语境中,\textbf{子空间}(Subspace)通常指某个更大空间的子集,该子集本身也具有相同的结构。

\subsection{在不同语境中的含义}

\begin{enumerate}
\item \textbf{在线性代数中}:
   \begin{itemize}
   \item 子空间通常指\textbf{线性子空间}(Linear Subspace)
   \item 即我们上面定义的概念
   \end{itemize}

\item \textbf{在拓扑学中}:
   \begin{itemize}
   \item 子空间指拓扑空间的子集,继承原空间的拓扑结构
   \item 与线性子空间概念不同
   \end{itemize}

\item \textbf{在度量空间中}:
   \begin{itemize}
   \item 子空间指度量空间的子集,继承原空间的度量
   \end{itemize}
\end{enumerate}

\subsection{在凸优化中的用法}

在凸优化和线性代数中,当我们说"子空间"时,通常指的是\textbf{线性子空间},除非特别说明。

\section{线性子空间与仿射集合的关系}

\subsection{重要关系}

\textbf{定理}:仿射集合可以表示为线性子空间的平移。

具体地,如果 $C$ 是仿射集合,$\mathbf{x}_0 \in C$,那么:
\begin{equation}
V = C - \mathbf{x}_0 = \{\mathbf{x} - \mathbf{x}_0 \mid \mathbf{x} \in C\}
\end{equation}
是一个线性子空间,且:
\begin{equation}
C = V + \mathbf{x}_0 = \{\mathbf{v} + \mathbf{x}_0 \mid \mathbf{v} \in V\}
\end{equation}

\subsection{区别总结}

\begin{table}[h]
\centering
\begin{tabular}{|l|l|l|}
\hline
\textbf{性质} & \textbf{线性子空间} & \textbf{仿射集合} \\
\hline
必须包含原点 & 是 & 不一定 \\
\hline
结构 & 通过原点的直线/平面 & 直线/平面的平移 \\
\hline
表示 & $V$(子空间本身) & $V + \mathbf{x}_0$(子空间+偏移) \\
\hline
例子 & $\{(x, y, 0) \mid x, y \in \mathbb{R}\}$ & $\{(x, y, 1) \mid x, y \in \mathbb{R}\}$ \\
\hline
\end{tabular}
\caption{线性子空间与仿射集合的区别}
\end{table}

\subsection{具体例子}

\textbf{例子1}:

在 $\mathbb{R}^2$ 中:
\begin{itemize}
\item $V = \{(x, 2x) \mid x \in \mathbb{R}\}$ 是线性子空间(通过原点的直线)
\item $C = \{(x, 2x + 1) \mid x \in \mathbb{R}\}$ 是仿射集合(不通过原点的直线)
\item 关系:$C = V + (0, 1)$
\end{itemize}

\textbf{例子2}:

在 $\mathbb{R}^3$ 中:
\begin{itemize}
\item $V = \{(x, y, 0) \mid x, y \in \mathbb{R}\}$ 是线性子空间($xy$ 平面)
\item $C = \{(x, y, 1) \mid x, y \in \mathbb{R}\}$ 是仿射集合(平行于 $xy$ 平面,但 $z = 1$)
\item 关系:$C = V + (0, 0, 1)$
\end{itemize}

\section{生成子空间(Span)}

\subsection{定义}

对于向量集合 $S = \{\mathbf{v}_1, \mathbf{v}_2, \ldots, \mathbf{v}_k\} \subseteq \mathbb{R}^n$,由 $S$ \textbf{生成}(span)的线性子空间定义为:

\begin{equation}
\text{span}(S) = \{\alpha_1 \mathbf{v}_1 + \alpha_2 \mathbf{v}_2 + \cdots + \alpha_k \mathbf{v}_k \mid \alpha_1, \alpha_2, \ldots, \alpha_k \in \mathbb{R}\}
\end{equation}

即所有可能的线性组合的集合。

\subsection{性质}

\begin{enumerate}
\item $\text{span}(S)$ 是包含 $S$ 的最小线性子空间
\item 如果 $S$ 中的向量线性无关,则 $\dim(\text{span}(S)) = k$
\item 如果 $S$ 中的向量线性相关,则 $\dim(\text{span}(S)) < k$
\end{enumerate}

\subsection{例子}

\textbf{例子}:在 $\mathbb{R}^3$ 中,设 $\mathbf{v}_1 = (1, 0, 0)$,$\mathbf{v}_2 = (0, 1, 0)$,则:
\begin{equation}
\text{span}\{\mathbf{v}_1, \mathbf{v}_2\} = \{(x, y, 0) \mid x, y \in \mathbb{R}\}
\end{equation}
这是 $xy$ 平面,是一个二维线性子空间。

\section{常见线性子空间}

\subsection{零空间(Null Space)}

对于矩阵 $\mathbf{A} \in \mathbb{R}^{m \times n}$:
\begin{equation}
\text{null}(\mathbf{A}) = \{\mathbf{x} \in \mathbb{R}^n \mid \mathbf{A}\mathbf{x} = \mathbf{0}\}
\end{equation}

\textbf{性质}:
\begin{itemize}
\item 维度:$\dim(\text{null}(\mathbf{A})) = n - \text{rank}(\mathbf{A})$
\item 这是齐次线性方程组 $\mathbf{A}\mathbf{x} = \mathbf{0}$ 的解空间
\end{itemize}

\subsection{列空间(Column Space)}

对于矩阵 $\mathbf{A} \in \mathbb{R}^{m \times n}$:
\begin{equation}
\text{col}(\mathbf{A}) = \text{span}\{\mathbf{a}_1, \mathbf{a}_2, \ldots, \mathbf{a}_n\}
\end{equation}
其中 $\mathbf{a}_i$ 是 $\mathbf{A}$ 的第 $i$ 列。

\textbf{性质}:
\begin{itemize}
\item 维度:$\dim(\text{col}(\mathbf{A})) = \text{rank}(\mathbf{A})$
\item 这是非齐次线性方程组 $\mathbf{A}\mathbf{x} = \mathbf{b}$ 有解时,$\mathbf{b}$ 必须属于的空间
\end{itemize}

\subsection{行空间(Row Space)}

对于矩阵 $\mathbf{A} \in \mathbb{R}^{m \times n}$,行空间是 $\mathbf{A}^T$ 的列空间:
\begin{equation}
\text{row}(\mathbf{A}) = \text{col}(\mathbf{A}^T)
\end{equation}

\section{维度和基}

\subsection{基(Basis)}

线性子空间 $V$ 的\textbf{基}(Basis)是一组线性无关的向量 $\{\mathbf{v}_1, \mathbf{v}_2, \ldots, \mathbf{v}_k\}$,使得:
\begin{equation}
V = \text{span}\{\mathbf{v}_1, \mathbf{v}_2, \ldots, \mathbf{v}_k\}
\end{equation}

\textbf{性质}:
\begin{itemize}
\item 基中向量的个数等于子空间的维度
\item 基的选择不唯一,但基中向量的个数唯一
\item 子空间中的任意向量都可以唯一地表示为基向量的线性组合
\end{itemize}

\subsection{例子}

\textbf{例子}:在 $\mathbb{R}^3$ 中,$xy$ 平面的一个基是:
\begin{itemize}
\item $\{(1, 0, 0), (0, 1, 0)\}$
\item 或 $\{(1, 1, 0), (1, -1, 0)\}$(另一个基)
\end{itemize}
两个基都包含2个向量,因为 $xy$ 平面是二维的。

\section{总结}

\subsection{线性子空间的关键特征}

\begin{enumerate}
\item \textbf{必须包含原点}:$\mathbf{0} \in V$
\item \textbf{对线性运算封闭}:对加法和数乘封闭
\item \textbf{几何上}:通过原点的直线、平面或超平面
\item \textbf{代数上}:可以表示为某些向量的线性组合的集合
\end{enumerate}

\subsection{与仿射集合的关系}

\begin{itemize}
\item 线性子空间是特殊的仿射集合(通过原点的仿射集合)
\item 仿射集合是线性子空间的平移
\item 每个仿射集合都可以表示为:线性子空间 + 偏移向量
\end{itemize}

\subsection{在凸优化中的重要性}

\begin{itemize}
\item 理解线性子空间有助于理解仿射集合
\item 线性方程组的解集是仿射集合,其关联子空间是系数矩阵的零空间
\item 在优化问题中,可行域的几何结构往往涉及子空间和仿射集合
\end{itemize}

理解线性子空间的概念是学习线性代数、凸优化等高级数学主题的基础。

\end{document}

