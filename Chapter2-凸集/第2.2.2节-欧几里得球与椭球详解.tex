\documentclass[12pt,a4paper]{article}
\usepackage[UTF8]{ctex}
\usepackage{amsmath}
\usepackage{amssymb}
\usepackage{amsthm}
\usepackage{geometry}
\geometry{left=2.5cm,right=2.5cm,top=2.5cm,bottom=2.5cm}

\title{第2.2.2节:欧几里得球与椭球详解}
\subtitle{基于《Convex Optimization》Stephen Boyd}
\author{}
\date{\today}

\begin{document}

\maketitle

\section{引言}

欧几里得球(Euclidean Ball)和椭球(Ellipsoid)是凸优化中最重要和最常见的凸集之一。本节将详细讲解它们的定义、性质、表示方法以及凸性证明。

\section{欧几里得球(Euclidean Ball)}

\subsection{定义}

\textbf{定义}:在 $\mathbb{R}^n$ 中,以 $\mathbf{x}_c$ 为中心、半径为 $r > 0$ 的\textbf{欧几里得球}定义为:

\begin{equation}
B(\mathbf{x}_c, r) = \{\mathbf{x} \in \mathbb{R}^n \mid \|\mathbf{x} - \mathbf{x}_c\|_2 \leq r\}
\end{equation}

其中 $\|\cdot\|_2$ 是欧几里得范数(L2范数),定义为:
\begin{equation}
\|\mathbf{u}\|_2 = \sqrt{\mathbf{u}^T \mathbf{u}} = \left(\sum_{i=1}^n u_i^2\right)^{1/2}
\end{equation}

\subsection{等价表示}

欧几里得球也可以用内积形式表示:

\begin{equation}
B(\mathbf{x}_c, r) = \{\mathbf{x} \in \mathbb{R}^n \mid (\mathbf{x} - \mathbf{x}_c)^T (\mathbf{x} - \mathbf{x}_c) \leq r^2\}
\end{equation}

这是因为:
\begin{equation}
\|\mathbf{x} - \mathbf{x}_c\|_2^2 = (\mathbf{x} - \mathbf{x}_c)^T (\mathbf{x} - \mathbf{x}_c)
\end{equation}

\subsection{参数化表示}

另一种常见的表示方法是:

\begin{equation}
B(\mathbf{x}_c, r) = \{\mathbf{x}_c + r \mathbf{u} \mid \|\mathbf{u}\|_2 \leq 1\}
\end{equation}

\textbf{几何意义}:
\begin{itemize}
\item $\mathbf{u}$ 是单位球内的点($\|\mathbf{u}\|_2 \leq 1$)
\item $r \mathbf{u}$ 将单位向量缩放 $r$ 倍
\item $\mathbf{x}_c + r \mathbf{u}$ 从中心 $\mathbf{x}_c$ 出发,沿方向 $\mathbf{u}$ 移动距离 $r$
\item 所有这样的点构成半径为 $r$ 的球
\end{itemize}

\subsection{几何直观}

\begin{itemize}
\item \textbf{二维}($\mathbb{R}^2$):圆盘(包括边界)
\item \textbf{三维}($\mathbb{R}^3$):球体(包括表面)
\item \textbf{$n$维}($\mathbb{R}^n$):超球(hypersphere)
\end{itemize}

\subsection{凸性证明}

\textbf{定理}:欧几里得球是凸集。

\textbf{证明}:设 $\mathbf{x}_1, \mathbf{x}_2 \in B(\mathbf{x}_c, r)$,即:
\begin{align}
\|\mathbf{x}_1 - \mathbf{x}_c\|_2 &\leq r \\
\|\mathbf{x}_2 - \mathbf{x}_c\|_2 &\leq r
\end{align}

对于任意 $\theta \in [0, 1]$,考虑凸组合 $\theta \mathbf{x}_1 + (1-\theta) \mathbf{x}_2$:

\begin{align}
\|\theta \mathbf{x}_1 + (1-\theta) \mathbf{x}_2 - \mathbf{x}_c\|_2 &= \|\theta(\mathbf{x}_1 - \mathbf{x}_c) + (1-\theta)(\mathbf{x}_2 - \mathbf{x}_c)\|_2 \\
&\leq \theta \|\mathbf{x}_1 - \mathbf{x}_c\|_2 + (1-\theta) \|\mathbf{x}_2 - \mathbf{x}_c\|_2 \quad \text{(三角不等式)} \\
&\leq \theta r + (1-\theta) r = r
\end{align}

因此 $\theta \mathbf{x}_1 + (1-\theta) \mathbf{x}_2 \in B(\mathbf{x}_c, r)$,所以球是凸集。$\square$

\textbf{关键步骤}:
\begin{enumerate}
\item \textbf{代数恒等式}:将 $\mathbf{x}_c$ 写成 $\theta \mathbf{x}_c + (1-\theta) \mathbf{x}_c$
\item \textbf{三角不等式}:$\|\mathbf{u} + \mathbf{v}\|_2 \leq \|\mathbf{u}\|_2 + \|\mathbf{v}\|_2$
\item \textbf{齐次性}:$\|\alpha \mathbf{w}\|_2 = |\alpha| \|\mathbf{w}\|_2$
\item \textbf{利用已知条件}:两个距离都 $\leq r$
\end{enumerate}

\section{椭球(Ellipsoid)}

\subsection{定义}

\textbf{定义}:在 $\mathbb{R}^n$ 中,\textbf{椭球}定义为:

\begin{equation}
\mathcal{E} = \{\mathbf{x} \in \mathbb{R}^n \mid (\mathbf{x} - \mathbf{x}_c)^T \mathbf{P}^{-1} (\mathbf{x} - \mathbf{x}_c) \leq 1\}
\end{equation}

其中:
\begin{itemize}
\item $\mathbf{x}_c \in \mathbb{R}^n$ 是椭球的\textbf{中心}
\item $\mathbf{P} = \mathbf{P}^T \succ 0$ 是\textbf{对称正定矩阵}(symmetric positive definite matrix)
\item $\mathbf{P}^{-1}$ 是 $\mathbf{P}$ 的逆矩阵
\end{itemize}

\subsection{正定矩阵的含义}

\textbf{正定矩阵} $\mathbf{P} \succ 0$ 意味着:
\begin{itemize}
\item $\mathbf{P}$ 是对称的:$\mathbf{P} = \mathbf{P}^T$
\item 对于任意非零向量 $\mathbf{v}$,有 $\mathbf{v}^T \mathbf{P} \mathbf{v} > 0$
\item $\mathbf{P}$ 的所有特征值都是正数
\item $\mathbf{P}$ 可逆,且 $\mathbf{P}^{-1}$ 也是正定的
\end{itemize}

\subsection{为什么使用 $\mathbf{P}^{-1}$?}

使用 $\mathbf{P}^{-1}$ 而不是 $\mathbf{P}$ 的原因:
\begin{itemize}
\item 使得椭球的半轴长度与 $\mathbf{P}$ 的特征值有简单关系
\item 半轴长度 = $\sqrt{\lambda_i}$,其中 $\lambda_i$ 是 $\mathbf{P}$ 的特征值
\item 如果使用 $\mathbf{P}$,半轴长度 = $1/\sqrt{\lambda_i}$,关系更复杂
\end{itemize}

\subsection{椭球的几何性质}

\textbf{半轴长度}:椭球在各个方向的"延伸程度"由 $\mathbf{P}$ 的特征值决定。

设 $\mathbf{P}$ 的特征值分解为:
\begin{equation}
\mathbf{P} = \mathbf{Q} \boldsymbol{\Lambda} \mathbf{Q}^T
\end{equation}
其中 $\mathbf{Q}$ 是正交矩阵,$\boldsymbol{\Lambda} = \text{diag}(\lambda_1, \lambda_2, \ldots, \lambda_n)$,$\lambda_i > 0$。

则椭球在特征向量方向上的半轴长度为 $\sqrt{\lambda_i}$。

\textbf{证明思路}:
\begin{itemize}
\item 在特征向量 $\mathbf{q}_i$ 方向上,椭球的"半径"为 $\sqrt{\lambda_i}$
\item 这是因为 $(\mathbf{q}_i)^T \mathbf{P}^{-1} \mathbf{q}_i = 1/\lambda_i$
\item 所以 $(\sqrt{\lambda_i} \mathbf{q}_i)^T \mathbf{P}^{-1} (\sqrt{\lambda_i} \mathbf{q}_i) = \lambda_i / \lambda_i = 1$
\end{itemize}

\subsection{椭球是球的推广}

\textbf{特殊情况}:当 $\mathbf{P} = r^2 \mathbf{I}$ 时(其中 $\mathbf{I}$ 是单位矩阵),椭球退化为球:

\begin{align}
\mathcal{E} &= \{\mathbf{x} \mid (\mathbf{x} - \mathbf{x}_c)^T (r^2 \mathbf{I})^{-1} (\mathbf{x} - \mathbf{x}_c) \leq 1\} \\
&= \{\mathbf{x} \mid (\mathbf{x} - \mathbf{x}_c)^T (r^{-2} \mathbf{I}) (\mathbf{x} - \mathbf{x}_c) \leq 1\} \\
&= \{\mathbf{x} \mid r^{-2} (\mathbf{x} - \mathbf{x}_c)^T (\mathbf{x} - \mathbf{x}_c) \leq 1\} \\
&= \{\mathbf{x} \mid \|\mathbf{x} - \mathbf{x}_c\|_2^2 \leq r^2\} \\
&= \{\mathbf{x} \mid \|\mathbf{x} - \mathbf{x}_c\|_2 \leq r\} = B(\mathbf{x}_c, r)
\end{align}

因此,球是椭球的特殊情况。

\section{椭球的另一种表示}

\subsection{参数化表示}

椭球也可以用另一种形式表示:

\begin{equation}
\mathcal{E} = \{\mathbf{x}_c + \mathbf{A} \mathbf{u} \mid \|\mathbf{u}\|_2 \leq 1\}
\end{equation}

其中 $\mathbf{A}$ 是方阵且非奇异(nonsingular)。

\subsection{两种表示的关系}

\textbf{定理}:两种表示是等价的,且可以假设 $\mathbf{A}$ 是对称正定矩阵。

\textbf{证明}:

\textbf{从表示1到表示2}:

设 $\mathbf{P} \succ 0$,取 $\mathbf{A} = \mathbf{P}^{1/2}$($\mathbf{P}$ 的平方根矩阵)。

由于 $\mathbf{P}$ 正定,存在唯一的对称正定矩阵 $\mathbf{P}^{1/2}$,使得 $(\mathbf{P}^{1/2})^2 = \mathbf{P}$。

设 $\mathbf{x} = \mathbf{x}_c + \mathbf{A} \mathbf{u}$,其中 $\|\mathbf{u}\|_2 \leq 1$,则:
\begin{align}
(\mathbf{x} - \mathbf{x}_c)^T \mathbf{P}^{-1} (\mathbf{x} - \mathbf{x}_c) &= (\mathbf{A} \mathbf{u})^T \mathbf{P}^{-1} (\mathbf{A} \mathbf{u}) \\
&= \mathbf{u}^T \mathbf{A}^T \mathbf{P}^{-1} \mathbf{A} \mathbf{u}
\end{align}

由于 $\mathbf{A} = \mathbf{P}^{1/2}$ 且 $\mathbf{A}$ 对称,$\mathbf{A}^T = \mathbf{A}$,所以:
\begin{align}
\mathbf{A}^T \mathbf{P}^{-1} \mathbf{A} &= \mathbf{P}^{1/2} \mathbf{P}^{-1} \mathbf{P}^{1/2} \\
&= \mathbf{P}^{1/2} (\mathbf{P}^{1/2})^{-2} \mathbf{P}^{1/2} \\
&= \mathbf{I}
\end{align}

因此:
\begin{equation}
(\mathbf{x} - \mathbf{x}_c)^T \mathbf{P}^{-1} (\mathbf{x} - \mathbf{x}_c) = \mathbf{u}^T \mathbf{u} = \|\mathbf{u}\|_2^2 \leq 1
\end{equation}

\textbf{从表示2到表示1}:

设 $\mathbf{x} = \mathbf{x}_c + \mathbf{A} \mathbf{u}$,其中 $\|\mathbf{u}\|_2 \leq 1$。

设 $\mathbf{P} = \mathbf{A}^2$(如果 $\mathbf{A}$ 对称正定),则 $\mathbf{P}^{-1} = \mathbf{A}^{-2}$。

类似地可以证明等价性。

\subsection{几何意义}

\begin{itemize}
\item $\mathbf{u}$ 是单位球内的点
\item $\mathbf{A} \mathbf{u}$ 对单位向量进行线性变换(拉伸、旋转)
\item $\mathbf{x}_c + \mathbf{A} \mathbf{u}$ 从中心出发,经过变换后的向量
\item 椭球是单位球经过仿射变换(线性变换+平移)得到的
\end{itemize}

\section{退化椭球(Degenerate Ellipsoid)}

\subsection{定义}

如果表示 $\mathcal{E} = \{\mathbf{x}_c + \mathbf{A} \mathbf{u} \mid \|\mathbf{u}\|_2 \leq 1\}$ 中的矩阵 $\mathbf{A}$ 是\textbf{对称半正定但奇异}的(即 $\mathbf{A} \succeq 0$ 但 $\det(\mathbf{A}) = 0$),则称为\textbf{退化椭球}。

\subsection{性质}

\begin{itemize}
\item \textbf{仿射维度}:退化椭球的仿射维度等于 $\mathbf{A}$ 的秩(rank)
\item \textbf{凸性}:退化椭球仍然是凸集
\item \textbf{几何意义}:退化椭球是"扁平"的,在某些方向上没有厚度
\end{itemize}

\subsection{例子}

在 $\mathbb{R}^3$ 中,如果 $\mathbf{A}$ 的秩为2,则退化椭球是一个"椭圆盘"(ellipsoidal disk),即三维空间中的二维椭圆。

\section{具体例子}

\subsection{例子1:二维单位圆}

在 $\mathbb{R}^2$ 中,单位圆可以表示为:

\textbf{作为球}:
\begin{equation}
B(\mathbf{0}, 1) = \{(x, y) \mid x^2 + y^2 \leq 1\}
\end{equation}

\textbf{作为椭球}:
\begin{equation}
\mathcal{E} = \{(x, y) \mid (x, y)^T \mathbf{I}^{-1} (x, y) \leq 1\} = \{(x, y) \mid x^2 + y^2 \leq 1\}
\end{equation}
其中 $\mathbf{P} = \mathbf{I}$(单位矩阵)。

\subsection{例子2:二维椭圆}

在 $\mathbb{R}^2$ 中,考虑椭圆:
\begin{equation}
\frac{x^2}{a^2} + \frac{y^2}{b^2} \leq 1
\end{equation}

可以写成椭球形式:
\begin{equation}
\mathcal{E} = \{(x, y) \mid (x, y)^T \mathbf{P}^{-1} (x, y) \leq 1\}
\end{equation}
其中:
\begin{equation}
\mathbf{P} = \begin{pmatrix} a^2 & 0 \\ 0 & b^2 \end{pmatrix}, \quad \mathbf{P}^{-1} = \begin{pmatrix} 1/a^2 & 0 \\ 0 & 1/b^2 \end{pmatrix}
\end{equation}

验证:
\begin{align}
(x, y)^T \mathbf{P}^{-1} (x, y) &= (x, y)^T \begin{pmatrix} 1/a^2 & 0 \\ 0 & 1/b^2 \end{pmatrix} \begin{pmatrix} x \\ y \end{pmatrix} \\
&= \frac{x^2}{a^2} + \frac{y^2}{b^2} \leq 1
\end{align}

\textbf{半轴长度}:
\begin{itemize}
\item $\mathbf{P}$ 的特征值为 $\lambda_1 = a^2$,$\lambda_2 = b^2$
\item 半轴长度为 $\sqrt{a^2} = a$ 和 $\sqrt{b^2} = b$
\end{itemize}

\subsection{例子3:旋转椭圆}

在 $\mathbb{R}^2$ 中,考虑旋转后的椭圆。

设旋转角度为 $\theta$,旋转矩阵为:
\begin{equation}
\mathbf{R} = \begin{pmatrix} \cos\theta & -\sin\theta \\ \sin\theta & \cos\theta \end{pmatrix}
\end{equation}

椭球可以表示为:
\begin{equation}
\mathcal{E} = \{\mathbf{x}_c + \mathbf{R} \mathbf{D} \mathbf{u} \mid \|\mathbf{u}\|_2 \leq 1\}
\end{equation}
其中 $\mathbf{D} = \text{diag}(a, b)$ 是对角矩阵。

对应的 $\mathbf{P}$ 矩阵为:
\begin{equation}
\mathbf{P} = (\mathbf{R} \mathbf{D})^2 = \mathbf{R} \mathbf{D}^2 \mathbf{R}^T
\end{equation}

\section{椭球的凸性}

\subsection{椭球是凸集}

\textbf{定理}:椭球是凸集。

\textbf{证明思路}:

椭球可以表示为单位球经过仿射变换得到:
\begin{equation}
\mathcal{E} = \{\mathbf{x}_c + \mathbf{A} \mathbf{u} \mid \|\mathbf{u}\|_2 \leq 1\}
\end{equation}

由于:
\begin{itemize}
\item 单位球是凸集
\item 仿射变换保持凸性
\item 因此椭球是凸集
\end{itemize}

\textbf{直接证明}:

设 $\mathbf{x}_1, \mathbf{x}_2 \in \mathcal{E}$,即:
\begin{align}
(\mathbf{x}_1 - \mathbf{x}_c)^T \mathbf{P}^{-1} (\mathbf{x}_1 - \mathbf{x}_c) &\leq 1 \\
(\mathbf{x}_2 - \mathbf{x}_c)^T \mathbf{P}^{-1} (\mathbf{x}_2 - \mathbf{x}_c) &\leq 1
\end{align}

对于 $\theta \in [0, 1]$,考虑凸组合 $\theta \mathbf{x}_1 + (1-\theta) \mathbf{x}_2$。

定义 $\mathbf{y}_i = \mathbf{P}^{-1/2} (\mathbf{x}_i - \mathbf{x}_c)$,则 $\|\mathbf{y}_i\|_2^2 \leq 1$。

\begin{align}
&(\theta \mathbf{x}_1 + (1-\theta) \mathbf{x}_2 - \mathbf{x}_c)^T \mathbf{P}^{-1} (\theta \mathbf{x}_1 + (1-\theta) \mathbf{x}_2 - \mathbf{x}_c) \\
&= \|\mathbf{P}^{-1/2} [\theta(\mathbf{x}_1 - \mathbf{x}_c) + (1-\theta)(\mathbf{x}_2 - \mathbf{x}_c)]\|_2^2 \\
&= \|\theta \mathbf{y}_1 + (1-\theta) \mathbf{y}_2\|_2^2 \\
&\leq [\theta \|\mathbf{y}_1\|_2 + (1-\theta) \|\mathbf{y}_2\|_2]^2 \quad \text{(三角不等式)} \\
&\leq [\theta \cdot 1 + (1-\theta) \cdot 1]^2 = 1
\end{align}

因此椭球是凸集。$\square$

\section{应用}

\subsection{在优化问题中}

\begin{itemize}
\item \textbf{约束集合}:椭球常用作优化问题的约束集合
\item \textbf{不确定性集合}:在鲁棒优化中,椭球用于表示不确定参数的可能范围
\item \textbf{置信区域}:在统计学中,椭球用于表示参数的置信区域
\end{itemize}

\subsection{在机器学习中}

\begin{itemize}
\item \textbf{正则化}:椭球约束用于正则化
\item \textbf{支持向量机}:某些变体使用椭球约束
\item \textbf{主成分分析}:数据的主成分可以用椭球表示
\end{itemize}

\section{总结}

\begin{enumerate}
\item \textbf{欧几里得球}:
   \begin{itemize}
   \item 定义:$B(\mathbf{x}_c, r) = \{\mathbf{x} \mid \|\mathbf{x} - \mathbf{x}_c\|_2 \leq r\}$
   \item 性质:凸集
   \item 表示:可以用参数化形式表示
   \end{itemize}

\item \textbf{椭球}:
   \begin{itemize}
   \item 定义:$\mathcal{E} = \{\mathbf{x} \mid (\mathbf{x} - \mathbf{x}_c)^T \mathbf{P}^{-1} (\mathbf{x} - \mathbf{x}_c) \leq 1\}$
   \item 性质:凸集,是球的推广
   \item 半轴长度:由 $\mathbf{P}$ 的特征值决定
   \item 表示:两种等价表示方法
   \end{itemize}

\item \textbf{关系}:
   \begin{itemize}
   \item 球是椭球的特殊情况($\mathbf{P} = r^2 \mathbf{I}$)
   \item 椭球是单位球经过仿射变换得到的
   \end{itemize}

\item \textbf{凸性}:
   \begin{itemize}
   \item 球和椭球都是凸集
   \item 证明依赖于三角不等式和范数的性质
   \end{itemize}
\end{enumerate}

理解欧几里得球和椭球对于深入学习凸优化至关重要!

\end{document}


