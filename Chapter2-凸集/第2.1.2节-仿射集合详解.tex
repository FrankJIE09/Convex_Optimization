\documentclass[12pt,a4paper]{article}
\usepackage[UTF8]{ctex}
\usepackage{amsmath}
\usepackage{amssymb}
\usepackage{amsthm}
\usepackage{geometry}
\geometry{left=2.5cm,right=2.5cm,top=2.5cm,bottom=2.5cm}

\title{第2.1.2节:仿射集合详解}
\subtitle{基于《Convex Optimization》Stephen Boyd}
\author{}
\date{\today}

\begin{document}

\maketitle

\section{引言}

本节详细讲解《Convex Optimization》第2.1.2节关于仿射集合(Affine Sets)的内容。仿射集合是凸优化理论中的基础概念,理解它对于后续学习至关重要。

\section{仿射集合的定义}

\subsection{基本定义}

\textbf{定义}:集合 $C \subseteq \mathbb{R}^n$ 称为\textbf{仿射集合}(Affine Set),如果通过其中任意两个不同点的\textbf{整条直线}都在 $C$ 中。

用数学语言表述:对于任意 $\mathbf{x}_1, \mathbf{x}_2 \in C$($\mathbf{x}_1 \neq \mathbf{x}_2$)和任意 $\theta \in \mathbb{R}$,都有:

\begin{equation}
\theta \mathbf{x}_1 + (1-\theta) \mathbf{x}_2 \in C
\end{equation}

\subsection{关键理解点}

\begin{enumerate}
\item \textbf{系数和为1}:注意 $\theta + (1-\theta) = 1$,这是仿射组合的关键特征。

\item \textbf{$\theta$ 可以是任意实数}:
   \begin{itemize}
   \item 当 $\theta = 0$ 时,得到点 $\mathbf{x}_2$
   \item 当 $\theta = 1$ 时,得到点 $\mathbf{x}_1$
   \item 当 $0 < \theta < 1$ 时,得到 $\mathbf{x}_1$ 和 $\mathbf{x}_2$ 之间的点(线段)
   \item 当 $\theta > 1$ 时,得到 $\mathbf{x}_1$ 延长线上的点
   \item 当 $\theta < 0$ 时,得到 $\mathbf{x}_2$ 延长线上的点
   \end{itemize}

\item \textbf{包含整条直线}:与凸集合只包含线段不同,仿射集合必须包含通过两点的整条直线(向两个方向无限延伸)。
\end{enumerate}

\subsection{几何直观}

想象在二维空间中:
\begin{itemize}
\item 如果 $C$ 是一条直线,那么通过直线上任意两点的直线就是这条直线本身,所以直线是仿射集合。
\item 如果 $C$ 是一个圆盘,那么通过圆盘内两点的直线会延伸到圆盘外,所以圆盘\textbf{不是}仿射集合。
\item 如果 $C$ 是整个平面 $\mathbb{R}^2$,那么它当然是仿射集合。
\end{itemize}

\section{仿射组合}

\subsection{两个点的仿射组合}

对于两个点 $\mathbf{x}_1, \mathbf{x}_2$,表达式 $\theta \mathbf{x}_1 + (1-\theta) \mathbf{x}_2$(其中 $\theta \in \mathbb{R}$)称为这两个点的\textbf{仿射组合}(Affine Combination)。

\subsection{多个点的仿射组合}

\textbf{定义}:对于 $k$ 个点 $\mathbf{x}_1, \mathbf{x}_2, \ldots, \mathbf{x}_k$,如果存在系数 $\theta_1, \theta_2, \ldots, \theta_k \in \mathbb{R}$ 满足:

\begin{equation}
\theta_1 + \theta_2 + \cdots + \theta_k = 1
\end{equation}

那么点 $\theta_1 \mathbf{x}_1 + \theta_2 \mathbf{x}_2 + \cdots + \theta_k \mathbf{x}_k$ 称为这 $k$ 个点的\textbf{仿射组合}。

\textbf{重要性质}:如果 $C$ 是仿射集合,那么它包含其中任意有限个点的所有仿射组合。

\textbf{证明思路}:使用数学归纳法。已知 $C$ 包含任意两个点的仿射组合(定义),可以证明它包含任意 $k$ 个点的仿射组合。

\subsection{仿射组合与凸组合的区别}

\begin{table}[h]
\centering
\begin{tabular}{|l|l|l|}
\hline
\textbf{类型} & \textbf{系数条件} & \textbf{几何意义} \\
\hline
仿射组合 & $\sum \theta_i = 1$,$\theta_i \in \mathbb{R}$ & 可以是任意实数(正、负、零) \\
\hline
凸组合 & $\sum \theta_i = 1$,$\theta_i \geq 0$ & 系数必须非负 \\
\hline
\end{tabular}
\caption{仿射组合与凸组合的区别}
\end{table}

\textbf{例子}:对于点 $\mathbf{x}_1 = (1, 0)$ 和 $\mathbf{x}_2 = (0, 1)$
\begin{itemize}
\item 仿射组合:$(2, -1) = 2(1,0) + (-1)(0,1)$(系数和为1,但系数可以是负数)
\item 凸组合:$(0.3, 0.7) = 0.3(1,0) + 0.7(0,1)$(系数和为1,且都非负)
\end{itemize}

\section{仿射集合与子空间的关系}

\subsection{核心定理}

\textbf{重要结论}:如果 $C$ 是仿射集合,且 $\mathbf{x}_0 \in C$,那么集合

\begin{equation}
V = C - \mathbf{x}_0 = \{\mathbf{x} - \mathbf{x}_0 \mid \mathbf{x} \in C\}
\end{equation}

是一个\textbf{线性子空间}(Subspace)。

\subsection{为什么是子空间?}

要证明 $V$ 是子空间,需要证明它对加法和数乘封闭:

\textbf{证明}:设 $\mathbf{v}_1, \mathbf{v}_2 \in V$,$\alpha, \beta \in \mathbb{R}$。

由于 $\mathbf{v}_1, \mathbf{v}_2 \in V$,存在 $\mathbf{x}_1, \mathbf{x}_2 \in C$ 使得:
\begin{align}
\mathbf{v}_1 &= \mathbf{x}_1 - \mathbf{x}_0 \\
\mathbf{v}_2 &= \mathbf{x}_2 - \mathbf{x}_0
\end{align}

现在考虑 $\alpha \mathbf{v}_1 + \beta \mathbf{v}_2$:
\begin{align}
\alpha \mathbf{v}_1 + \beta \mathbf{v}_2 &= \alpha(\mathbf{x}_1 - \mathbf{x}_0) + \beta(\mathbf{x}_2 - \mathbf{x}_0) \\
&= \alpha \mathbf{x}_1 + \beta \mathbf{x}_2 - (\alpha + \beta) \mathbf{x}_0
\end{align}

重新整理:
\begin{align}
\alpha \mathbf{v}_1 + \beta \mathbf{v}_2 + \mathbf{x}_0 &= \alpha \mathbf{x}_1 + \beta \mathbf{x}_2 + (1 - \alpha - \beta) \mathbf{x}_0
\end{align}

注意 $\alpha + \beta + (1 - \alpha - \beta) = 1$,所以这是 $\mathbf{x}_1, \mathbf{x}_2, \mathbf{x}_0$ 的仿射组合。由于 $C$ 是仿射集合,这个仿射组合仍在 $C$ 中,即:

\begin{equation}
\alpha \mathbf{v}_1 + \beta \mathbf{v}_2 + \mathbf{x}_0 \in C
\end{equation}

因此:
\begin{equation}
\alpha \mathbf{v}_1 + \beta \mathbf{v}_2 = (\alpha \mathbf{v}_1 + \beta \mathbf{v}_2 + \mathbf{x}_0) - \mathbf{x}_0 \in V
\end{equation}

所以 $V$ 对加法和数乘封闭,是线性子空间。$\square$

\subsection{几何意义}

这个定理告诉我们:\textbf{仿射集合可以表示为线性子空间的平移}。

具体来说:
\begin{equation}
C = V + \mathbf{x}_0 = \{\mathbf{v} + \mathbf{x}_0 \mid \mathbf{v} \in V\}
\end{equation}

其中 $V$ 是线性子空间,$\mathbf{x}_0$ 是 $C$ 中的任意一点(称为"偏移"或"基点")。

\textbf{直观理解}:
\begin{itemize}
\item 如果仿射集合 $C$ 经过原点(即 $\mathbf{0} \in C$),那么 $C$ 本身就是一个线性子空间。
\item 如果仿射集合 $C$ 不经过原点,那么它是某个线性子空间的平移。
\end{itemize}

\textbf{例子}:
\begin{itemize}
\item 在 $\mathbb{R}^2$ 中,直线 $y = 2x + 1$ 是仿射集合。它可以表示为:子空间 $\{(t, 2t) \mid t \in \mathbb{R}\}$(通过原点的直线)加上偏移 $(0, 1)$。
\item 在 $\mathbb{R}^3$ 中,平面 $x + y + z = 1$ 是仿射集合。它可以表示为:子空间 $\{(x, y, -x-y) \mid x, y \in \mathbb{R}\}$(通过原点的平面)加上偏移 $(0, 0, 1)$。
\end{itemize}

\subsection{维度的定义}

\textbf{定义}:仿射集合 $C$ 的\textbf{维度}(Dimension)定义为关联子空间 $V = C - \mathbf{x}_0$ 的维度,其中 $\mathbf{x}_0$ 是 $C$ 中的任意点。

注意:这个定义不依赖于 $\mathbf{x}_0$ 的选择,因为对于不同的基点,得到的子空间是平行的(具有相同的维度)。

\section{线性方程组的解集}

\subsection{重要例子}

\textbf{定理}:线性方程组 $\mathbf{A}\mathbf{x} = \mathbf{b}$ 的解集是一个仿射集合,其中 $\mathbf{A} \in \mathbb{R}^{m \times n}$,$\mathbf{b} \in \mathbb{R}^m$。

\textbf{证明}:设 $C = \{\mathbf{x} \mid \mathbf{A}\mathbf{x} = \mathbf{b}\}$。对于任意 $\mathbf{x}_1, \mathbf{x}_2 \in C$ 和 $\theta \in \mathbb{R}$,有:

\begin{align}
\mathbf{A}(\theta \mathbf{x}_1 + (1-\theta) \mathbf{x}_2) &= \theta \mathbf{A}\mathbf{x}_1 + (1-\theta) \mathbf{A}\mathbf{x}_2 \\
&= \theta \mathbf{b} + (1-\theta) \mathbf{b} \\
&= \mathbf{b}
\end{align}

因此 $\theta \mathbf{x}_1 + (1-\theta) \mathbf{x}_2 \in C$,所以 $C$ 是仿射集合。$\square$

\subsection{几何解释}

\begin{itemize}
\item 如果 $\mathbf{b} = \mathbf{0}$(齐次方程组),解集是线性子空间($\mathbf{A}$ 的零空间)。
\item 如果 $\mathbf{b} \neq \mathbf{0}$(非齐次方程组),解集是零空间的平移。
\item 具体地,如果 $\mathbf{x}_0$ 是 $\mathbf{A}\mathbf{x} = \mathbf{b}$ 的一个特解,那么解集为:
  \begin{equation}
  C = \{\mathbf{x}_0 + \mathbf{v} \mid \mathbf{A}\mathbf{v} = \mathbf{0}\} = \mathbf{x}_0 + \text{null}(\mathbf{A})
  \end{equation}
  其中 $\text{null}(\mathbf{A})$ 是 $\mathbf{A}$ 的零空间。
\end{itemize}

\subsection{逆定理}

\textbf{逆定理}:每个仿射集合都可以表示为某个线性方程组的解集。

\textbf{证明思路}:如果 $C$ 是 $k$ 维仿射集合,那么存在 $n-k$ 个线性无关的方程,使得 $C$ 是它们的解集。

\section{仿射包(Affine Hull)}

\subsection{定义}

对于任意集合 $C \subseteq \mathbb{R}^n$,$C$ 的\textbf{仿射包}(Affine Hull)定义为 $C$ 中所有点的所有仿射组合的集合:

\begin{equation}
\text{aff } C = \{\theta_1 \mathbf{x}_1 + \cdots + \theta_k \mathbf{x}_k \mid \mathbf{x}_1, \ldots, \mathbf{x}_k \in C, \theta_1 + \cdots + \theta_k = 1\}
\end{equation}

\subsection{性质}

\begin{enumerate}
\item \textbf{仿射包是仿射集合}:$\text{aff } C$ 本身是一个仿射集合。

\item \textbf{最小性}:$\text{aff } C$ 是包含 $C$ 的最小仿射集合。也就是说,如果 $S$ 是任何包含 $C$ 的仿射集合,那么 $\text{aff } C \subseteq S$。

\item \textbf{包含关系}:$C \subseteq \text{aff } C$
\end{enumerate}

\subsection{例子}

\textbf{例子1}:在 $\mathbb{R}^2$ 中,如果 $C = \{(0, 0), (1, 0)\}$(两个点),则 $\text{aff } C$ 是通过这两点的直线。

\textbf{例子2}:在 $\mathbb{R}^3$ 中,如果 $C = \{(0, 0, 0), (1, 0, 0), (0, 1, 0)\}$(三个不共线的点),则 $\text{aff } C$ 是通过这三个点的平面。

\textbf{例子3}:在 $\mathbb{R}^2$ 中,如果 $C$ 是一个圆,则 $\text{aff } C = \mathbb{R}^2$(整个平面)。

\section{总结与关键点}

\subsection{核心概念回顾}

\begin{enumerate}
\item \textbf{仿射集合的定义}:包含通过任意两点的整条直线的集合。

\item \textbf{仿射组合}:系数和为1的线性组合(系数可以是任意实数)。

\item \textbf{仿射集合的结构}:仿射集合 = 线性子空间 + 偏移。

\item \textbf{线性方程组的解集}:是仿射集合的典型例子。

\item \textbf{仿射包}:包含集合的最小仿射集合。
\end{enumerate}

\subsection{与凸集合的关系}

\begin{itemize}
\item \textbf{所有仿射集合都是凸集合}(因为包含整条直线,必然包含线段)。
\item \textbf{但凸集合不一定是仿射集合}(例如,圆盘是凸的但不是仿射的)。
\item \textbf{区别}:仿射集合要求 $\theta \in \mathbb{R}$,凸集合只要求 $\theta \in [0, 1]$。
\end{itemize}

\subsection{实际应用}

\begin{itemize}
\item \textbf{等式约束优化}:等式约束 $\mathbf{A}\mathbf{x} = \mathbf{b}$ 定义的可行域是仿射集合。
\item \textbf{线性规划}:线性规划问题的可行域通常是多个半空间的交集,但如果有等式约束,则包含仿射集合。
\item \textbf{几何问题}:仿射集合在计算几何、计算机图形学中有广泛应用。
\end{itemize}

\section{练习题思考}

\begin{enumerate}
\item 证明:如果 $C_1$ 和 $C_2$ 都是仿射集合,那么 $C_1 \cap C_2$ 也是仿射集合。

\item 证明:如果 $C$ 是仿射集合,$\mathbf{A}$ 是矩阵,$\mathbf{b}$ 是向量,那么 $\{\mathbf{A}\mathbf{x} + \mathbf{b} \mid \mathbf{x} \in C\}$ 也是仿射集合。

\item 在 $\mathbb{R}^2$ 中,找出包含点 $(0, 0)$、$(1, 0)$、$(0, 1)$ 的仿射包,并给出其参数表示。
\end{enumerate}

\end{document}

