\documentclass[12pt,a4paper]{article}
\usepackage[UTF8]{ctex}
\usepackage{amsmath}
\usepackage{amssymb}
\usepackage{amsthm}
\usepackage{geometry}
\geometry{left=2.5cm,right=2.5cm,top=2.5cm,bottom=2.5cm}

\title{集合边界(Boundary)详解}
\subtitle{理解 $\text{bd } C$ 的含义}
\author{}
\date{\today}

\begin{document}

\maketitle

\section{引言}

在凸分析和优化理论中,$\text{bd } C$ 表示集合 $C$ 的边界(boundary)。理解边界的概念对于理解支撑超平面、相对内部等概念非常重要。

\section{边界的定义}

\subsection{基本定义}

\textbf{边界}:集合 $C \subseteq \mathbb{R}^n$ 的边界,记作 $\text{bd } C$ 或 $\partial C$,定义为:

\begin{equation}
\text{bd } C = \text{cl } C \setminus \text{int } C
\end{equation}

其中:
\begin{itemize}
\item $\text{cl } C$:集合 $C$ 的闭包(closure)
\item $\text{int } C$:集合 $C$ 的内部(interior)
\item $\setminus$:集合差运算
\end{itemize}

\textbf{含义}:边界是闭包中不属于内部的所有点。

\subsection{等价定义}

\textbf{定义2}:点 $\mathbf{x}$ 属于 $\text{bd } C$,当且仅当:

对于任意 $\epsilon > 0$,球 $B(\mathbf{x}, \epsilon) = \{\mathbf{y} \mid \|\mathbf{y} - \mathbf{x}\|_2 < \epsilon\}$ 既包含 $C$ 中的点,也包含不在 $C$ 中的点。

\textbf{含义}:
\begin{itemize}
\item 边界上的点:无论多小的邻域,都同时包含集合内外的点
\item 边界是集合"边缘"的点
\end{itemize}

\section{闭包和内部}

\subsection{闭包(Closure)}

\textbf{闭包}:集合 $C$ 的闭包,记作 $\text{cl } C$ 或 $\bar{C}$,定义为:

\begin{equation}
\text{cl } C = \{\mathbf{x} \in \mathbb{R}^n \mid \text{对于任意 } \epsilon > 0, B(\mathbf{x}, \epsilon) \cap C \neq \emptyset\}
\end{equation}

\textbf{含义}:
\begin{itemize}
\item 闭包包含 $C$ 的所有点
\item 还包含 $C$ 的"极限点"(边界点)
\item 闭包是包含 $C$ 的最小闭集
\end{itemize}

\subsection{内部(Interior)}

\textbf{内部}:集合 $C$ 的内部,记作 $\text{int } C$,定义为:

\begin{equation}
\text{int } C = \{\mathbf{x} \in C \mid \text{存在 } \epsilon > 0, B(\mathbf{x}, \epsilon) \subseteq C\}
\end{equation}

\textbf{含义}:
\begin{itemize}
\item 内部的点:存在一个邻域完全在集合内
\item 内部是集合"内部"的点(不包括边界)
\end{itemize}

\section{边界的性质}

\subsection{基本性质}

\begin{enumerate}
\item \textbf{边界是闭集}:$\text{bd } C$ 是闭集

\item \textbf{边界与内部不相交}:$\text{int } C \cap \text{bd } C = \emptyset$

\item \textbf{闭包 = 内部 + 边界}:$\text{cl } C = \text{int } C \cup \text{bd } C$

\item \textbf{集合 = 内部 + 边界的一部分}:$C = \text{int } C \cup (\text{bd } C \cap C)$
\end{enumerate}

\subsection{对于凸集}

\textbf{性质}:如果 $C$ 是凸集,则:
\begin{itemize}
\item 边界上的点可以定义支撑超平面
\item 边界是"光滑"的(对于严格凸集)
\end{itemize}

\section{具体例子}

\subsection{例子1:闭区间}

\textbf{集合}:$C = [a, b] = \{x \in \mathbb{R} \mid a \leq x \leq b\}$

\textbf{分析}:
\begin{itemize}
\item 内部:$\text{int } C = (a, b)$(开区间)
\item 闭包:$\text{cl } C = [a, b]$(闭区间)
\item 边界:$\text{bd } C = \{a, b\}$(两个端点)
\end{itemize}

\textbf{验证}:
\begin{itemize}
\item 点 $a$:任意邻域既包含 $[a, b]$ 中的点,也包含 $< a$ 的点
\item 点 $b$:任意邻域既包含 $[a, b]$ 中的点,也包含 $> b$ 的点
\end{itemize}

\subsection{例子2:开区间}

\textbf{集合}:$C = (a, b) = \{x \in \mathbb{R} \mid a < x < b\}$

\textbf{分析}:
\begin{itemize}
\item 内部:$\text{int } C = (a, b)$
\item 闭包:$\text{cl } C = [a, b]$
\item 边界:$\text{bd } C = \{a, b\}$(仍然是两个端点)
\end{itemize}

\textbf{注意}:即使 $C$ 是开集,边界点 $a, b$ 不在 $C$ 中,但仍然是 $C$ 的边界点。

\subsection{例子3:圆盘}

\textbf{集合}:$C = \{\mathbf{x} \in \mathbb{R}^2 \mid \|\mathbf{x}\|_2 \leq 1\}$(闭圆盘)

\textbf{分析}:
\begin{itemize}
\item 内部:$\text{int } C = \{\mathbf{x} \in \mathbb{R}^2 \mid \|\mathbf{x}\|_2 < 1\}$(开圆盘)
\item 闭包:$\text{cl } C = C$(闭圆盘)
\item 边界:$\text{bd } C = \{\mathbf{x} \in \mathbb{R}^2 \mid \|\mathbf{x}\|_2 = 1\}$(单位圆周)
\end{itemize}

\textbf{几何意义}:
\begin{itemize}
\item 边界是圆周上的所有点
\item 每个边界点都可以定义支撑超平面(切线)
\end{itemize}

\subsection{例子4:开圆盘}

\textbf{集合}:$C = \{\mathbf{x} \in \mathbb{R}^2 \mid \|\mathbf{x}\|_2 < 1\}$(开圆盘)

\textbf{分析}:
\begin{itemize}
\item 内部:$\text{int } C = C$(开圆盘本身)
\item 闭包:$\text{cl } C = \{\mathbf{x} \in \mathbb{R}^2 \mid \|\mathbf{x}\|_2 \leq 1\}$(闭圆盘)
\item 边界:$\text{bd } C = \{\mathbf{x} \in \mathbb{R}^2 \mid \|\mathbf{x}\|_2 = 1\}$(单位圆周)
\end{itemize}

\textbf{注意}:即使 $C$ 是开集,边界仍然是圆周(虽然边界点不在 $C$ 中)。

\subsection{例子5:单点集}

\textbf{集合}:$C = \{\mathbf{x}_0\}$(单点集)

\textbf{分析}:
\begin{itemize}
\item 内部:$\text{int } C = \emptyset$(空集,因为没有邻域完全在单点内)
\item 闭包:$\text{cl } C = \{\mathbf{x}_0\}$
\item 边界:$\text{bd } C = \{\mathbf{x}_0\}$(点本身)
\end{itemize}

\subsection{例子6:整个空间}

\textbf{集合}:$C = \mathbb{R}^n$

\textbf{分析}:
\begin{itemize}
\item 内部:$\text{int } C = \mathbb{R}^n$
\item 闭包:$\text{cl } C = \mathbb{R}^n$
\item 边界:$\text{bd } C = \emptyset$(空集,因为没有"外部")
\end{itemize}

\section{边界与支撑超平面}

\subsection{关系}

\textbf{定理}:如果 $C$ 是凸集,且 $\mathbf{x}_0 \in \text{bd } C$,则存在支撑超平面在点 $\mathbf{x}_0$ 处。

\textbf{含义}:
\begin{itemize}
\item 边界点处可以定义支撑超平面
\item 内部点处不能定义支撑超平面(因为集合在超平面两侧都有点)
\end{itemize}

\subsection{为什么需要边界点?}

\textbf{原因}:
\begin{itemize}
\item 支撑超平面要求集合完全在超平面的一侧
\item 如果点在内部,则存在邻域完全在集合内,无法定义支撑超平面
\item 只有在边界上,才能"支撑"集合
\end{itemize}

\section{相对边界}

\subsection{定义}

\textbf{相对边界}:对于集合 $C \subseteq \mathbb{R}^n$,相对边界定义为:

\begin{equation}
\text{relbd } C = \text{cl } C \setminus \text{relint } C
\end{equation}

其中 $\text{relint } C$ 是相对内部。

\textbf{含义}:
\begin{itemize}
\item 相对边界是相对于仿射包(affine hull)的边界
\item 对于低维集合(如线段、三角形),相对边界更有意义
\end{itemize}

\subsection{例子}

\textbf{集合}:$C = \{(x, y) \in \mathbb{R}^2 \mid x \in [0, 1], y = 0\}$(线段)

\textbf{分析}:
\begin{itemize}
\item 内部:$\text{int } C = \emptyset$(在 $\mathbb{R}^2$ 中没有内部)
\item 相对内部:$\text{relint } C = \{(x, 0) \mid x \in (0, 1)\}$(开线段)
\item 相对边界:$\text{relbd } C = \{(0, 0), (1, 0)\}$(两个端点)
\end{itemize}

\section{在优化中的应用}

\subsection{最优解在边界上}

\textbf{性质}:对于约束优化问题,最优解通常在可行域的边界上。

\textbf{原因}:
\begin{itemize}
\item 如果最优解在内部,则可以通过移动减小目标函数值
\item 只有在边界上,约束才"起作用"
\end{itemize}

\subsection{支撑超平面}

\textbf{应用}:在最优解处,$-\nabla f_0(\mathbf{x})$ 定义了可行域在边界点处的支撑超平面。

\textbf{条件}:$\mathbf{x} \in \text{bd } X$(最优解在可行域的边界上)

\section{总结}

\subsection{边界的定义}

\begin{enumerate}
\item \textbf{基本定义}:$\text{bd } C = \text{cl } C \setminus \text{int } C$

\item \textbf{等价定义}:边界上的点,任意邻域既包含集合内的点,也包含集合外的点

\item \textbf{性质}:边界是闭集,与内部不相交
\end{enumerate}

\subsection{关键理解}

\begin{enumerate}
\item \textbf{边界}:集合"边缘"的点

\item \textbf{与内部的区别}:
   \begin{itemize}
   \item 内部:存在邻域完全在集合内
   \item 边界:任意邻域都包含集合内外的点
   \end{itemize}

\item \textbf{在优化中}:
   \begin{itemize}
   \item 最优解通常在边界上
   \item 边界点处可以定义支撑超平面
   \end{itemize}
</enumerate}

\subsection{符号说明}

\begin{itemize}
\item $\text{bd } C$ 或 $\partial C$:集合 $C$ 的边界
\item $\text{cl } C$ 或 $\bar{C}$:集合 $C$ 的闭包
\item $\text{int } C$:集合 $C$ 的内部
\item $\text{relbd } C$:集合 $C$ 的相对边界
\item $\text{relint } C$:集合 $C$ 的相对内部
\end{itemize}

理解边界的概念,是理解支撑超平面和优化理论的基础!

\end{document}

