\documentclass[12pt,a4paper]{article}
\usepackage[UTF8]{ctex}
\usepackage{amsmath}
\usepackage{amssymb}
\usepackage{amsthm}
\usepackage{geometry}
\geometry{left=2.5cm,right=2.5cm,top=2.5cm,bottom=2.5cm}

\title{凸集、凸组合、凸包的定义}
\subtitle{基于《Convex Optimization》第2.1.4节}
\author{}
\date{\today}

\begin{document}

\maketitle

\section{引言}

凸集(Convex Set)、凸组合(Convex Combination)和凸包(Convex Hull)是凸优化理论中的三个核心概念。它们密切相关但又有区别。本文将详细解释这三个概念的定义、性质和关系。

\section{凸集(Convex Set)}

\subsection{定义}

\textbf{定义}:集合 $C \subseteq \mathbb{R}^n$ 称为\textbf{凸集}(Convex Set),如果连接其中任意两点的\textbf{线段}都在 $C$ 中。

用数学语言表述:对于任意 $\mathbf{x}_1, \mathbf{x}_2 \in C$ 和任意 $\theta \in [0, 1]$,都有:

\begin{equation}
\theta \mathbf{x}_1 + (1-\theta) \mathbf{x}_2 \in C
\end{equation}

\subsection{关键要点}

\begin{enumerate}
\item \textbf{系数范围}:$\theta \in [0, 1]$(与仿射组合不同,仿射组合允许 $\theta \in \mathbb{R}$)

\item \textbf{包含内容}:只要求包含\textbf{线段},不要求包含整条直线

\item \textbf{几何直观}:集合"没有凹陷",形状是"向外凸出"的
\end{enumerate}

\subsection{几何直观}

一个集合是凸的,如果:
\begin{itemize}
\item 集合中任意两点之间的直线路径完全在集合内
\item 没有"凹陷"或"缺口"
\item 从集合内任意一点"看"到集合内任意另一点,视线不被阻挡
\end{itemize}

\subsection{例子}

\textbf{凸集的例子}:
\begin{itemize}
\item 圆盘:$\{\mathbf{x} \in \mathbb{R}^2 \mid \|\mathbf{x}\|_2 \leq r\}$
\item 三角形、矩形、椭圆(包括边界)
\item 半空间:$\{\mathbf{x} \in \mathbb{R}^n \mid \mathbf{a}^T \mathbf{x} \leq b\}$
\item 超平面:$\{\mathbf{x} \in \mathbb{R}^n \mid \mathbf{a}^T \mathbf{x} = b\}$
\item 整个空间 $\mathbb{R}^n$
\item 空集 $\emptyset$ 和单点集 $\{\mathbf{x}_0\}$
\end{itemize}

\textbf{非凸集的例子}:
\begin{itemize}
\item 月牙形、星形
\item 字母"C"的形状
\item 去掉一个点的圆盘
\item 两个不相交的圆盘
\end{itemize}

\subsection{性质}

\begin{enumerate}
\item \textbf{所有仿射集合都是凸集}(因为包含整条直线,必然包含线段)

\item \textbf{凸集的交集是凸集}:如果 $C_1$ 和 $C_2$ 都是凸集,那么 $C_1 \cap C_2$ 也是凸集

\item \textbf{凸集的并集不一定是凸集}
\end{enumerate}

\section{凸组合(Convex Combination)}

\subsection{定义}

\textbf{定义}:对于 $k$ 个点 $\mathbf{x}_1, \mathbf{x}_2, \ldots, \mathbf{x}_k \in \mathbb{R}^n$,如果存在系数 $\theta_1, \theta_2, \ldots, \theta_k$ 满足:

\begin{align}
\theta_1 + \theta_2 + \cdots + \theta_k &= 1 \label{eq:sum1} \\
\theta_i &\geq 0, \quad i = 1, 2, \ldots, k \label{eq:nonneg}
\end{align}

那么点 $\theta_1 \mathbf{x}_1 + \theta_2 \mathbf{x}_2 + \cdots + \theta_k \mathbf{x}_k$ 称为这 $k$ 个点的\textbf{凸组合}(Convex Combination)。

\subsection{关键要点}

\begin{enumerate}
\item \textbf{系数和等于1}:$\sum_{i=1}^k \theta_i = 1$

\item \textbf{系数非负}:$\theta_i \geq 0$(这是与仿射组合的关键区别)

\item \textbf{几何意义}:凸组合表示这些点的"加权平均"或"混合"
\end{enumerate}

\subsection{与仿射组合的区别}

\begin{table}[h]
\centering
\begin{tabular}{|l|l|l|}
\hline
\textbf{性质} & \textbf{凸组合} & \textbf{仿射组合} \\
\hline
系数和 & $\sum \theta_i = 1$ & $\sum \theta_i = 1$ \\
\hline
系数符号 & $\theta_i \geq 0$(必须非负) & $\theta_i \in \mathbb{R}$(可以是负数) \\
\hline
几何意义 & 加权平均(在点之间) & 可以延伸到点之外 \\
\hline
\end{tabular}
\caption{凸组合与仿射组合的区别}
\end{table}

\subsection{两个点的凸组合}

对于两个点 $\mathbf{x}_1$ 和 $\mathbf{x}_2$,凸组合为:
\begin{equation}
\theta \mathbf{x}_1 + (1-\theta) \mathbf{x}_2, \quad \theta \in [0, 1]
\end{equation}

这正好是连接 $\mathbf{x}_1$ 和 $\mathbf{x}_2$ 的线段上的点。

\subsection{三个点的凸组合}

对于三个点 $\mathbf{x}_1, \mathbf{x}_2, \mathbf{x}_3$,凸组合为:
\begin{equation}
\theta_1 \mathbf{x}_1 + \theta_2 \mathbf{x}_2 + \theta_3 \mathbf{x}_3
\end{equation}

其中 $\theta_1 + \theta_2 + \theta_3 = 1$ 且 $\theta_i \geq 0$。

\textbf{几何意义}:如果三点不共线,所有凸组合的集合是以这三点为顶点的三角形(包括内部和边界)。

\subsection{例子}

\textbf{例子1}:在 $\mathbb{R}^2$ 中,设 $\mathbf{x}_1 = (0, 0)$,$\mathbf{x}_2 = (1, 0)$,$\mathbf{x}_3 = (0, 1)$。

凸组合:$\theta_1(0, 0) + \theta_2(1, 0) + \theta_3(0, 1) = (\theta_2, \theta_3)$,其中 $\theta_1 + \theta_2 + \theta_3 = 1$,$\theta_i \geq 0$。

这正好是以 $(0, 0)$、$(1, 0)$、$(0, 1)$ 为顶点的三角形。

\textbf{例子2}:设 $\theta_1 = 0.3$,$\theta_2 = 0.5$,$\theta_3 = 0.2$,则:
\begin{equation}
0.3(0, 0) + 0.5(1, 0) + 0.2(0, 1) = (0.5, 0.2)
\end{equation}

这个点在三角形内部。

\subsection{重要性质}

\textbf{定理}:集合 $C$ 是凸集当且仅当它包含其中任意有限个点的所有凸组合。

\textbf{证明思路}:
\begin{itemize}
\item 如果 $C$ 是凸集,通过数学归纳法可以证明它包含任意有限个点的凸组合
\item 反过来,如果 $C$ 包含任意两个点的凸组合,那么它是凸集
\end{itemize}

\section{凸包(Convex Hull)}

\subsection{定义}

\textbf{定义}:对于集合 $C \subseteq \mathbb{R}^n$,$C$ 的\textbf{凸包}(Convex Hull),记为 $\text{conv } C$,定义为 $C$ 中所有点的所有凸组合的集合:

\begin{equation}
\text{conv } C = \left\{\sum_{i=1}^k \theta_i \mathbf{x}_i \middle| \mathbf{x}_i \in C, \theta_i \geq 0, i = 1, \ldots, k, \sum_{i=1}^k \theta_i = 1\right\}
\end{equation}

\subsection{关键要点}

\begin{enumerate}
\item \textbf{构造性定义}:凸包由 $C$ 中所有点的所有凸组合构成

\item \textbf{凸性}:$\text{conv } C$ 本身是凸集

\item \textbf{最小性}:$\text{conv } C$ 是包含 $C$ 的\textbf{最小}凸集
\end{enumerate}

\subsection{最小性}

\textbf{定理}:$\text{conv } C$ 是包含 $C$ 的最小凸集。

具体地:
\begin{itemize}
\item $\text{conv } C$ 是凸集
\item $C \subseteq \text{conv } C$
\item 对于任何包含 $C$ 的凸集 $B$,都有 $\text{conv } C \subseteq B$
\end{itemize}

\textbf{证明思路}:
\begin{enumerate}
\item $\text{conv } C$ 是凸集(因为凸组合的凸组合仍是凸组合)
\item $C \subseteq \text{conv } C$(因为每个点都是自己的凸组合)
\item 如果 $B$ 是包含 $C$ 的凸集,那么 $B$ 包含 $C$ 中所有点的所有凸组合,因此 $B \supseteq \text{conv } C$
\end{enumerate}

\subsection{几何直观}

凸包可以理解为:
\begin{itemize}
\item 用"橡皮筋"围绕集合 $C$,凸包就是橡皮筋收缩后的形状
\item 是包含 $C$ 的"最紧"的凸集
\item 如果 $C$ 是有限个点,凸包是一个凸多面体(或多边形)
\end{itemize}

\subsection{例子}

\textbf{例子1}:在 $\mathbb{R}^2$ 中,设 $C = \{(0, 0), (1, 0), (0, 1)\}$(三个点)。

$\text{conv } C$ 是以这三个点为顶点的三角形(包括内部和边界)。

\textbf{例子2}:在 $\mathbb{R}^2$ 中,设 $C$ 是一个圆。

$\text{conv } C$ 是包含该圆的最小圆盘(即圆本身,因为圆是凸集)。

\textbf{例子3}:在 $\mathbb{R}^2$ 中,设 $C$ 是字母"C"的形状(非凸)。

$\text{conv } C$ 是包含"C"的最小凸集,形状类似于一个"D"。

\textbf{例子4}:在 $\mathbb{R}^2$ 中,设 $C = \{(x, y) \mid x^2 + y^2 = 1\}$(单位圆周)。

$\text{conv } C = \{(x, y) \mid x^2 + y^2 \leq 1\}$(单位圆盘)。

\section{三个概念的关系}

\subsection{层次关系}

\begin{enumerate}
\item \textbf{凸组合}:点的组合方式(操作)

\item \textbf{凸集}:满足某种性质的集合(对象)

\item \textbf{凸包}:由凸组合构造的集合(构造)
\end{enumerate}

\subsection{逻辑关系}

\begin{itemize}
\item 如果集合 $C$ 是凸集,那么 $\text{conv } C = C$

\item 对于任意集合 $C$,$\text{conv } C$ 是包含 $C$ 的最小凸集

\item 集合 $C$ 是凸集当且仅当它包含其中所有点的所有凸组合
\end{itemize}

\subsection{与仿射集合的类比}

\begin{table}[h]
\centering
\begin{tabular}{|l|l|l|}
\hline
\textbf{概念} & \textbf{仿射} & \textbf{凸} \\
\hline
组合 & 仿射组合(系数任意) & 凸组合(系数非负) \\
\hline
集合 & 仿射集合 & 凸集 \\
\hline
包 & 仿射包(aff $C$) & 凸包(conv $C$) \\
\hline
包含关系 & 所有仿射集合都是凸集 & 凸集不一定是仿射集合 \\
\hline
\end{tabular}
\caption{仿射与凸的对应关系}
\end{table}

\section{推广:无限凸组合}

\subsection{可数无限凸组合}

对于可数无限个点 $\mathbf{x}_1, \mathbf{x}_2, \ldots$,如果系数 $\theta_1, \theta_2, \ldots$ 满足:
\begin{align}
\theta_i &\geq 0, \quad i = 1, 2, \ldots \\
\sum_{i=1}^{\infty} \theta_i &= 1
\end{align}

且级数 $\sum_{i=1}^{\infty} \theta_i \mathbf{x}_i$ 收敛,那么它也是凸组合。

\subsection{积分形式}

对于连续情况,如果 $p: \mathbb{R}^n \to \mathbb{R}$ 满足:
\begin{align}
p(\mathbf{x}) &\geq 0, \quad \forall \mathbf{x} \in C \\
\int_C p(\mathbf{x}) \, d\mathbf{x} &= 1
\end{align}

那么 $\int_C p(\mathbf{x}) \mathbf{x} \, d\mathbf{x}$(如果积分存在)也是凸组合。

\subsection{概率分布形式}

最一般的形式:如果 $\mathbf{x}$ 是随机向量,且 $\mathbf{x} \in C$ 的概率为1,那么 $\mathbb{E}[\mathbf{x}] \in \text{conv } C$。

这包含了所有其他形式作为特例。

\section{应用}

\subsection{在优化问题中}

\begin{itemize}
\item 凸优化问题的可行域必须是凸集
\item 如果可行域不是凸集,可以考虑其凸包
\item 凸包可以帮助找到问题的下界
\end{itemize}

\subsection{在几何问题中}

\begin{itemize}
\item 计算几何:计算点集的凸包
\item 分离定理:两个不相交的凸集可以用超平面分离
\item 支撑超平面:凸集的边界点有支撑超平面
\end{itemize}

\subsection{在机器学习中}

\begin{itemize}
\item 支持向量机:可行域是凸集
\item 逻辑回归:损失函数的水平集是凸集
\item 神经网络:某些激活函数保证凸性
\end{itemize}

\section{总结}

\begin{enumerate}
\item \textbf{凸集}:
   \begin{itemize}
   \item 定义:包含连接任意两点的线段
   \item 条件:$\theta \mathbf{x}_1 + (1-\theta) \mathbf{x}_2 \in C$,$\theta \in [0, 1]$
   \item 几何:没有凹陷,形状向外凸出
   \end{itemize}

\item \textbf{凸组合}:
   \begin{itemize}
   \item 定义:系数和等于1且系数非负的线性组合
   \item 条件:$\sum \theta_i = 1$,$\theta_i \geq 0$
   \item 几何:加权平均,在点之间
   \end{itemize}

\item \textbf{凸包}:
   \begin{itemize}
   \item 定义:所有凸组合的集合
   \item 性质:是包含原集合的最小凸集
   \item 几何:用橡皮筋围绕集合后的形状
   \end{itemize}

\item \textbf{关系}:
   \begin{itemize}
   \item 凸集包含其中所有点的所有凸组合
   \item 凸包是包含集合的最小凸集
   \item 如果集合是凸集,则其凸包等于自身
   \end{itemize}
\end{enumerate}

理解这三个概念及其关系对于深入学习凸优化理论至关重要!

\end{document}


