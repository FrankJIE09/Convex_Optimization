\documentclass[12pt,a4paper]{article}
\usepackage[UTF8]{ctex}
\usepackage{amsmath}
\usepackage{amssymb}
\usepackage{amsthm}
\usepackage{geometry}
\geometry{left=2.5cm,right=2.5cm,top=2.5cm,bottom=2.5cm}

\title{仿射变换与椭球表示详解}
\author{}
\date{\today}

\begin{document}

\maketitle

\section{仿射变换的定义}

\subsection{基本定义}

\textbf{仿射变换}:函数 $f: \mathbb{R}^n \to \mathbb{R}^m$ 是仿射的,如果它可以表示为:

\begin{equation}
f(\mathbf{x}) = \mathbf{A}\mathbf{x} + \mathbf{b}
\end{equation}

其中:
\begin{itemize}
\item $\mathbf{A} \in \mathbb{R}^{m \times n}$:线性变换矩阵
\item $\mathbf{b} \in \mathbb{R}^m$:平移向量
\end{itemize}

\subsection{组成部分}

\textbf{仿射变换 = 线性变换 + 平移}

\begin{itemize}
\item \textbf{线性变换部分}:$\mathbf{A}\mathbf{x}$(旋转、缩放、剪切等)
\item \textbf{平移部分}:$\mathbf{b}$(位移)
\end{itemize}

\subsection{特殊情况}

\begin{enumerate}
\item \textbf{如果 $\mathbf{b} = \mathbf{0}$}:仿射变换退化为线性变换 $f(\mathbf{x}) = \mathbf{A}\mathbf{x}$

\item \textbf{如果 $\mathbf{A} = \mathbf{I}$}:仿射变换退化为平移 $f(\mathbf{x}) = \mathbf{x} + \mathbf{b}$

\item \textbf{如果 $\mathbf{A} = \mathbf{I}$ 且 $\mathbf{b} = \mathbf{0}$}:恒等变换 $f(\mathbf{x}) = \mathbf{x}$
\end{enumerate}

\section{仿射变换的性质}

\subsection{保持凸性}

\textbf{重要性质}:仿射变换保持凸性。

\textbf{定理}:如果 $C \subseteq \mathbb{R}^n$ 是凸集,$f(\mathbf{x}) = \mathbf{A}\mathbf{x} + \mathbf{b}$ 是仿射变换,则 $f(C) = \{\mathbf{A}\mathbf{x} + \mathbf{b} \mid \mathbf{x} \in C\}$ 是凸集。

\textbf{证明思路}:
\begin{itemize}
\item 取 $f(C)$ 中任意两点 $\mathbf{y}_1, \mathbf{y}_2$
\item 它们来自 $C$ 中的点 $\mathbf{x}_1, \mathbf{x}_2$:$\mathbf{y}_i = \mathbf{A}\mathbf{x}_i + \mathbf{b}$
\item 凸组合:$\theta \mathbf{y}_1 + (1-\theta) \mathbf{y}_2 = \mathbf{A}(\theta \mathbf{x}_1 + (1-\theta) \mathbf{x}_2) + \mathbf{b}$
\item 由于 $C$ 是凸集,$\theta \mathbf{x}_1 + (1-\theta) \mathbf{x}_2 \in C$
\item 因此凸组合在 $f(C)$ 中
\end{itemize}

\subsection{保持仿射性}

\textbf{性质}:仿射变换的复合仍然是仿射变换。

如果 $f(\mathbf{x}) = \mathbf{A}\mathbf{x} + \mathbf{b}$ 和 $g(\mathbf{x}) = \mathbf{C}\mathbf{x} + \mathbf{d}$ 都是仿射变换,则:

\begin{equation}
(g \circ f)(\mathbf{x}) = \mathbf{C}(\mathbf{A}\mathbf{x} + \mathbf{b}) + \mathbf{d} = \mathbf{C}\mathbf{A}\mathbf{x} + (\mathbf{C}\mathbf{b} + \mathbf{d})
\end{equation}

也是仿射变换。

\section{椭球的定义}

\subsection{标准椭球}

\textbf{标准椭球}:在 $\mathbb{R}^n$ 中,以原点为中心的标准椭球定义为:

\begin{equation}
\mathcal{E}_0 = \left\{\mathbf{u} \in \mathbb{R}^n \mid \sum_{i=1}^n \frac{u_i^2}{a_i^2} \leq 1\right\}
\end{equation}

其中 $a_i > 0$ 是椭球在各坐标轴方向的半轴长度。

\textbf{矩阵形式}:

设 $\mathbf{D} = \text{diag}(1/a_1^2, \ldots, 1/a_n^2)$,则:

\begin{equation}
\mathcal{E}_0 = \{\mathbf{u} \in \mathbb{R}^n \mid \mathbf{u}^T \mathbf{D} \mathbf{u} \leq 1\}
\end{equation}

\subsection{一般椭球}

\textbf{一般椭球}:通过仿射变换从标准椭球得到。

\textbf{定义}:

\begin{equation}
\mathcal{E} = \{\mathbf{x} \in \mathbb{R}^n \mid (\mathbf{x} - \mathbf{x}_c)^T \mathbf{P}^{-1} (\mathbf{x} - \mathbf{x}_c) \leq 1\}
\end{equation}

其中:
\begin{itemize}
\item $\mathbf{x}_c \in \mathbb{R}^n$:椭球的中心
\item $\mathbf{P} \in \mathbb{S}^n_{++}$:正定矩阵($\mathbf{P} \succ 0$),决定椭球的形状、大小和方向
\end{itemize}

\section{为什么椭球可以这样表示?}

\subsection{从标准椭球到一般椭球}

\textbf{步骤1}:从标准单位球开始

\textbf{单位球}:$\mathcal{B} = \{\mathbf{u} \in \mathbb{R}^n \mid \|\mathbf{u}\|_2^2 \leq 1\} = \{\mathbf{u} \in \mathbb{R}^n \mid \mathbf{u}^T \mathbf{u} \leq 1\}$

\textbf{步骤2}:通过线性变换改变形状

\textbf{线性变换}:$\mathbf{v} = \mathbf{P}^{1/2} \mathbf{u}$,其中 $\mathbf{P}^{1/2}$ 是 $\mathbf{P}$ 的平方根矩阵($\mathbf{P}^{1/2} \mathbf{P}^{1/2} = \mathbf{P}$)。

\textbf{结果}:$\mathcal{E}_1 = \{\mathbf{v} \in \mathbb{R}^n \mid \mathbf{v}^T \mathbf{P}^{-1} \mathbf{v} \leq 1\}$

\textbf{验证}:
\begin{itemize}
\item 从 $\mathbf{u}^T \mathbf{u} \leq 1$ 开始
\item 由于 $\mathbf{v} = \mathbf{P}^{1/2} \mathbf{u}$,有 $\mathbf{u} = \mathbf{P}^{-1/2} \mathbf{v}$
\item 代入:$(\mathbf{P}^{-1/2} \mathbf{v})^T (\mathbf{P}^{-1/2} \mathbf{v}) = \mathbf{v}^T \mathbf{P}^{-1} \mathbf{v} \leq 1$
\end{itemize}

\textbf{步骤3}:通过平移改变中心

\textbf{平移}:$\mathbf{x} = \mathbf{v} + \mathbf{x}_c$

\textbf{结果}:$\mathcal{E} = \{\mathbf{x} \in \mathbb{R}^n \mid (\mathbf{x} - \mathbf{x}_c)^T \mathbf{P}^{-1} (\mathbf{x} - \mathbf{x}_c) \leq 1\}$

\textbf{完整变换}:

\begin{equation}
\mathbf{x} = \mathbf{P}^{1/2} \mathbf{u} + \mathbf{x}_c
\end{equation}

这是仿射变换:$\mathbf{x} = \mathbf{A}\mathbf{u} + \mathbf{b}$,其中 $\mathbf{A} = \mathbf{P}^{1/2}$,$\mathbf{b} = \mathbf{x}_c$。

\subsection{详细推导}

\textbf{从单位球开始}:

\begin{equation}
\mathcal{B} = \{\mathbf{u} \in \mathbb{R}^n \mid \mathbf{u}^T \mathbf{u} \leq 1\}
\end{equation}

\textbf{步骤1:线性变换(改变形状)}

\textbf{变换}:$\mathbf{v} = \mathbf{P}^{1/2} \mathbf{u}$

\textbf{逆变换}:$\mathbf{u} = \mathbf{P}^{-1/2} \mathbf{v}$

\textbf{代入单位球定义}:

\begin{align}
\mathbf{u}^T \mathbf{u} &\leq 1 \\
(\mathbf{P}^{-1/2} \mathbf{v})^T (\mathbf{P}^{-1/2} \mathbf{v}) &\leq 1 \\
\mathbf{v}^T \mathbf{P}^{-1/2} \mathbf{P}^{-1/2} \mathbf{v} &\leq 1 \\
\mathbf{v}^T \mathbf{P}^{-1} \mathbf{v} &\leq 1
\end{equation}

\textbf{结果}:$\mathcal{E}_1 = \{\mathbf{v} \in \mathbb{R}^n \mid \mathbf{v}^T \mathbf{P}^{-1} \mathbf{v} \leq 1\}$

\textbf{步骤2:平移(改变中心)}

\textbf{变换}:$\mathbf{x} = \mathbf{v} + \mathbf{x}_c$

\textbf{逆变换}:$\mathbf{v} = \mathbf{x} - \mathbf{x}_c$

\textbf{代入}:

\begin{align}
\mathbf{v}^T \mathbf{P}^{-1} \mathbf{v} &\leq 1 \\
(\mathbf{x} - \mathbf{x}_c)^T \mathbf{P}^{-1} (\mathbf{x} - \mathbf{x}_c) &\leq 1
\end{equation}

\textbf{结果}:$\mathcal{E} = \{\mathbf{x} \in \mathbb{R}^n \mid (\mathbf{x} - \mathbf{x}_c)^T \mathbf{P}^{-1} (\mathbf{x} - \mathbf{x}_c) \leq 1\}$

\section{矩阵 $\mathbf{P}$ 的几何意义}

\subsection{特征值分解}

\textbf{设}:$\mathbf{P}$ 的特征值分解为 $\mathbf{P} = \mathbf{Q} \boldsymbol{\Lambda} \mathbf{Q}^T$,其中:
\begin{itemize}
\item $\mathbf{Q}$ 是正交矩阵(列向量是单位正交向量)
\item $\boldsymbol{\Lambda} = \text{diag}(\lambda_1, \ldots, \lambda_n)$,$\lambda_i > 0$(因为 $\mathbf{P} \succ 0$)
\end{itemize}

\textbf{几何意义}:
\begin{itemize}
\item \textbf{特征向量}:$\mathbf{Q}$ 的列向量是椭球的主轴方向
\item \textbf{特征值}:$\lambda_i$ 决定各主轴方向的"拉伸"程度
\item \textbf{半轴长度}:$\sqrt{\lambda_i}$ 是第 $i$ 个主轴方向的半轴长度
\end{itemize}

\subsection{为什么是 $\mathbf{P}^{-1}$?}

\textbf{关键观察}:在椭球定义中使用 $\mathbf{P}^{-1}$ 而不是 $\mathbf{P}$。

\textbf{原因}:

\begin{align}
\mathbf{P} &= \mathbf{Q} \boldsymbol{\Lambda} \mathbf{Q}^T \\
\mathbf{P}^{-1} &= \mathbf{Q} \boldsymbol{\Lambda}^{-1} \mathbf{Q}^T = \mathbf{Q} \text{diag}(1/\lambda_1, \ldots, 1/\lambda_n) \mathbf{Q}^T
\end{equation}

\textbf{几何意义}:
\begin{itemize}
\item 如果 $\lambda_i$ 大,则 $1/\lambda_i$ 小,该方向"压缩"
\item 如果 $\lambda_i$ 小,则 $1/\lambda_i$ 大,该方向"拉伸"
\item $\mathbf{P}^{-1}$ 的特征值是 $\mathbf{P}$ 的特征值的倒数
\end{itemize}

\textbf{为什么这样?}

从单位球 $\mathbf{u}^T \mathbf{u} \leq 1$ 开始,通过 $\mathbf{v} = \mathbf{P}^{1/2} \mathbf{u}$ 变换:

\begin{align}
\mathbf{u}^T \mathbf{u} &\leq 1 \\
(\mathbf{P}^{-1/2} \mathbf{v})^T (\mathbf{P}^{-1/2} \mathbf{v}) &\leq 1 \\
\mathbf{v}^T \mathbf{P}^{-1} \mathbf{v} &\leq 1
\end{equation}

因此自然出现 $\mathbf{P}^{-1}$。

\section{具体例子}

\subsection{例子1:二维椭圆}

\textbf{标准椭圆}:$\mathcal{E}_0 = \{(u, v) \mid u^2 + v^2 \leq 1\}$(单位圆)

\textbf{矩阵}:$\mathbf{P} = \begin{pmatrix} 4 & 0 \\ 0 & 1 \end{pmatrix}$($\mathbf{P} \succ 0$)

\textbf{特征值分解}:
\begin{itemize}
\item 特征值:$\lambda_1 = 4$,$\lambda_2 = 1$
\item 特征向量:$(1, 0)^T$ 和 $(0, 1)^T$(坐标轴方向)
\end{itemize}

\textbf{椭球表示}:

\begin{equation}
\mathcal{E} = \left\{(x, y) \mid \frac{x^2}{4} + y^2 \leq 1\right\}
\end{equation}

\textbf{验证}:

\begin{align}
(x, y)^T \mathbf{P}^{-1} (x, y) &= (x, y) \begin{pmatrix} 1/4 & 0 \\ 0 & 1 \end{pmatrix} \begin{pmatrix} x \\ y \end{pmatrix} \\
&= \frac{x^2}{4} + y^2
\end{equation}

因此 $\frac{x^2}{4} + y^2 \leq 1$ 等价于 $(x, y)^T \mathbf{P}^{-1} (x, y) \leq 1$。✓

\textbf{几何意义}:
\begin{itemize}
\item 在 $x$ 方向:半轴长度为 $2$(因为 $\lambda_1 = 4$,$\sqrt{\lambda_1} = 2$)
\item 在 $y$ 方向:半轴长度为 $1$(因为 $\lambda_2 = 1$,$\sqrt{\lambda_2} = 1$)
\item 这是一个水平方向拉伸的椭圆
\end{itemize}

\subsection{例子2:带中心的椭圆}

\textbf{椭球}:$\mathcal{E} = \{(x, y) \mid (x-1)^2/4 + (y-2)^2 \leq 1\}$

\textbf{标准形式}:

\begin{equation}
\mathcal{E} = \left\{(x, y) \mid \begin{pmatrix} x-1 \\ y-2 \end{pmatrix}^T \begin{pmatrix} 1/4 & 0 \\ 0 & 1 \end{pmatrix} \begin{pmatrix} x-1 \\ y-2 \end{pmatrix} \leq 1\right\}
\end{equation}

\textbf{参数}:
\begin{itemize}
\item 中心:$\mathbf{x}_c = (1, 2)^T$
\item 矩阵:$\mathbf{P} = \begin{pmatrix} 4 & 0 \\ 0 & 1 \end{pmatrix}$,$\mathbf{P}^{-1} = \begin{pmatrix} 1/4 & 0 \\ 0 & 1 \end{pmatrix}$
\end{itemize}

\subsection{例子3:旋转的椭圆}

\textbf{矩阵}:$\mathbf{P} = \begin{pmatrix} 2 & 1 \\ 1 & 2 \end{pmatrix}$($\mathbf{P} \succ 0$)

\textbf{特征值分解}:
\begin{itemize}
\item 特征值:$\lambda_1 = 3$,$\lambda_2 = 1$
\item 特征向量:$(1, 1)^T/\sqrt{2}$ 和 $(1, -1)^T/\sqrt{2}$(旋转 $45^\circ$)
\end{itemize}

\textbf{椭球表示}:

\begin{equation}
\mathcal{E} = \{\mathbf{x} \mid \mathbf{x}^T \mathbf{P}^{-1} \mathbf{x} \leq 1\}
\end{equation}

\textbf{几何意义}:
\begin{itemize}
\item 主轴方向:沿 $(1, 1)^T$ 和 $(1, -1)^T$ 方向
\item 半轴长度:$\sqrt{3}$ 和 $1$
\item 这是一个旋转的椭圆
\end{itemize}

\section{从单位球到椭球的完整变换}

\subsection{变换过程}

\textbf{步骤1}:单位球

\begin{equation}
\mathcal{B} = \{\mathbf{u} \in \mathbb{R}^n \mid \mathbf{u}^T \mathbf{u} \leq 1\}
\end{equation}

\textbf{步骤2}:线性变换(改变形状和方向)

\begin{equation}
\mathbf{v} = \mathbf{P}^{1/2} \mathbf{u}
\end{equation}

\textbf{结果}:

\begin{equation}
\mathcal{E}_1 = \{\mathbf{v} \in \mathbb{R}^n \mid \mathbf{v}^T \mathbf{P}^{-1} \mathbf{v} \leq 1\}
\end{equation}

\textbf{步骤3}:平移(改变中心)

\begin{equation}
\mathbf{x} = \mathbf{v} + \mathbf{x}_c
\end{equation}

\textbf{最终结果}:

\begin{equation}
\mathcal{E} = \{\mathbf{x} \in \mathbb{R}^n \mid (\mathbf{x} - \mathbf{x}_c)^T \mathbf{P}^{-1} (\mathbf{x} - \mathbf{x}_c) \leq 1\}
\end{equation}

\subsection{完整仿射变换}

\textbf{仿射变换}:$\mathbf{x} = f(\mathbf{u}) = \mathbf{A}\mathbf{u} + \mathbf{b}$

其中:
\begin{itemize}
\item $\mathbf{A} = \mathbf{P}^{1/2}$:线性变换部分(旋转、缩放)
\item $\mathbf{b} = \mathbf{x}_c$:平移部分
\end{itemize}

\textbf{逆变换}:$\mathbf{u} = \mathbf{A}^{-1}(\mathbf{x} - \mathbf{b}) = \mathbf{P}^{-1/2}(\mathbf{x} - \mathbf{x}_c)$

\section{为什么使用 $\mathbf{P}^{-1}$ 而不是 $\mathbf{P}$?}

\subsection{从变换过程理解}

\textbf{关键}:我们从单位球 $\mathbf{u}^T \mathbf{u} \leq 1$ 开始。

\textbf{变换}:$\mathbf{v} = \mathbf{P}^{1/2} \mathbf{u}$,因此 $\mathbf{u} = \mathbf{P}^{-1/2} \mathbf{v}$。

\textbf{代入}:

\begin{align}
\mathbf{u}^T \mathbf{u} &\leq 1 \\
(\mathbf{P}^{-1/2} \mathbf{v})^T (\mathbf{P}^{-1/2} \mathbf{v}) &\leq 1 \\
\mathbf{v}^T \mathbf{P}^{-1/2} \mathbf{P}^{-1/2} \mathbf{v} &\leq 1 \\
\mathbf{v}^T \mathbf{P}^{-1} \mathbf{v} &\leq 1
\end{equation}

\textbf{因此自然出现 $\mathbf{P}^{-1}$}。

\subsection{几何理解}

\textbf{如果使用 $\mathbf{P}$}:

\begin{equation}
\mathcal{E} = \{\mathbf{x} \mid (\mathbf{x} - \mathbf{x}_c)^T \mathbf{P} (\mathbf{x} - \mathbf{x}_c) \leq 1\}
\end{equation}

\textbf{问题}:
\begin{itemize}
\item 这定义了不同的椭球
\item 如果 $\mathbf{P}$ 的特征值是 $\lambda_i$,则半轴长度是 $1/\sqrt{\lambda_i}$
\item 与标准定义不一致
\end{itemize}

\textbf{标准定义使用 $\mathbf{P}^{-1}$}:
\begin{itemize}
\item 如果 $\mathbf{P}$ 的特征值是 $\lambda_i$,则 $\mathbf{P}^{-1}$ 的特征值是 $1/\lambda_i$
\item 半轴长度是 $\sqrt{\lambda_i}$(更直观)
\item 与从单位球变换的过程一致
\end{itemize}

\section{总结}

\subsection{仿射变换}

\begin{enumerate}
\item \textbf{定义}:$f(\mathbf{x}) = \mathbf{A}\mathbf{x} + \mathbf{b}$

\item \textbf{组成}:线性变换 + 平移

\item \textbf{性质}:保持凸性
\end{enumerate}

\subsection{椭球的表示}

\begin{enumerate}
\item \textbf{标准形式}:$\mathcal{E} = \{\mathbf{x} \mid (\mathbf{x} - \mathbf{x}_c)^T \mathbf{P}^{-1} (\mathbf{x} - \mathbf{x}_c) \leq 1\}$

\item \textbf{从单位球得到}:
   \begin{itemize}
   \item 单位球:$\mathbf{u}^T \mathbf{u} \leq 1$
   \item 线性变换:$\mathbf{v} = \mathbf{P}^{1/2} \mathbf{u}$
   \item 平移:$\mathbf{x} = \mathbf{v} + \mathbf{x}_c$
   \end{itemize}

\item \textbf{为什么是 $\mathbf{P}^{-1}$?}
   \begin{itemize}
   \item 从单位球变换过程中自然出现
   \item 与特征值的几何意义一致
   \end{itemize}
\end{enumerate}

\subsection{关键理解}

\begin{enumerate}
\item \textbf{仿射变换}:将单位球变换为椭球

\item \textbf{$\mathbf{P}$ 的几何意义}:
   \begin{itemize}
   \item 特征向量:主轴方向
   \item 特征值:决定半轴长度
   \end{itemize}

\item \textbf{$\mathbf{P}^{-1}$ 的作用}:
   \begin{itemize}
   \item 从单位球变换过程中自然出现
   \item 保证几何意义的一致性
   \end{itemize}
\end{enumerate}

理解仿射变换和椭球表示,是理解凸优化中几何问题的基础!

\end{document}

