\documentclass[12pt,a4paper]{article}
\usepackage[UTF8]{ctex}
\usepackage{amsmath}
\usepackage{amssymb}
\usepackage{amsthm}
\usepackage{geometry}
\geometry{left=2.5cm,right=2.5cm,top=2.5cm,bottom=2.5cm}

\title{直线、子空间与凸锥的关系}
\subtitle{理解"Any line is affine. If it passes through zero, it is a subspace, hence also a convex cone"}
\author{}
\date{\today}

\begin{document}

\maketitle

\section{问题提出}

在《Convex Optimization》第2.2节中,有这样一句话:

\textbf{"Any line is affine. If it passes through zero, it is a subspace, hence also a convex cone."}

\textbf{问题}:如何理解这句话?为什么直线是仿射集合?为什么通过原点的直线是子空间?为什么子空间是凸锥?

\section{第一部分:任何直线都是仿射集合}

\subsection{直线的定义}

在 $\mathbb{R}^n$ 中,一条直线可以表示为:
\begin{equation}
L = \{\mathbf{x}_0 + t \mathbf{v} \mid t \in \mathbb{R}\}
\end{equation}
其中 $\mathbf{x}_0$ 是直线上的一点,$\mathbf{v} \neq \mathbf{0}$ 是方向向量。

\subsection{为什么直线是仿射集合?}

\textbf{证明}:设 $\mathbf{x}_1, \mathbf{x}_2 \in L$,$\theta \in \mathbb{R}$。

由于 $\mathbf{x}_1, \mathbf{x}_2 \in L$,存在 $t_1, t_2 \in \mathbb{R}$,使得:
\begin{align}
\mathbf{x}_1 &= \mathbf{x}_0 + t_1 \mathbf{v} \\
\mathbf{x}_2 &= \mathbf{x}_0 + t_2 \mathbf{v}
\end{align}

考虑仿射组合:
\begin{align}
\theta \mathbf{x}_1 + (1-\theta) \mathbf{x}_2 &= \theta(\mathbf{x}_0 + t_1 \mathbf{v}) + (1-\theta)(\mathbf{x}_0 + t_2 \mathbf{v}) \\
&= \theta \mathbf{x}_0 + \theta t_1 \mathbf{v} + (1-\theta) \mathbf{x}_0 + (1-\theta) t_2 \mathbf{v} \\
&= \mathbf{x}_0 + [\theta t_1 + (1-\theta) t_2] \mathbf{v}
\end{align}

设 $t = \theta t_1 + (1-\theta) t_2$,则 $t \in \mathbb{R}$,所以:
\begin{equation}
\theta \mathbf{x}_1 + (1-\theta) \mathbf{x}_2 = \mathbf{x}_0 + t \mathbf{v} \in L
\end{equation}

因此 $L$ 是仿射集合。$\square$

\subsection{几何直观}

\begin{itemize}
\item 直线包含通过其上任意两点的整条直线(就是它自己)
\item 因此满足仿射集合的定义
\item 这是最直观的仿射集合例子
\end{itemize}

\section{第二部分:通过原点的直线是子空间}

\subsection{子空间的定义回顾}

集合 $V \subseteq \mathbb{R}^n$ 是线性子空间,如果:
\begin{enumerate}
\item $\mathbf{0} \in V$
\item 对加法封闭:$\mathbf{v}_1 + \mathbf{v}_2 \in V$,$\forall \mathbf{v}_1, \mathbf{v}_2 \in V$
\item 对数乘封闭:$\alpha \mathbf{v} \in V$,$\forall \mathbf{v} \in V$,$\alpha \in \mathbb{R}$
\end{enumerate}

\subsection{为什么通过原点的直线是子空间?}

\textbf{证明}:设直线 $L = \{t \mathbf{v} \mid t \in \mathbb{R}\}$,其中 $\mathbf{v} \neq \mathbf{0}$(通过原点)。

\textbf{验证条件1}:$\mathbf{0} \in L$(当 $t = 0$ 时)

\textbf{验证条件2}(对加法封闭):
设 $\mathbf{x}_1 = t_1 \mathbf{v}$,$\mathbf{x}_2 = t_2 \mathbf{v} \in L$,则:
\begin{equation}
\mathbf{x}_1 + \mathbf{x}_2 = t_1 \mathbf{v} + t_2 \mathbf{v} = (t_1 + t_2) \mathbf{v} \in L
\end{equation}

\textbf{验证条件3}(对数乘封闭):
设 $\mathbf{x} = t \mathbf{v} \in L$,$\alpha \in \mathbb{R}$,则:
\begin{equation}
\alpha \mathbf{x} = \alpha(t \mathbf{v}) = (\alpha t) \mathbf{v} \in L
\end{equation}

因此 $L$ 是线性子空间。$\square$

\subsection{关键区别}

\begin{itemize}
\item \textbf{不通过原点的直线}:是仿射集合,但不是子空间(因为不包含原点)
\item \textbf{通过原点的直线}:既是仿射集合,也是子空间
\end{itemize}

\subsection{例子}

\textbf{例子1}:在 $\mathbb{R}^2$ 中,直线 $y = 2x$(通过原点)
\begin{equation}
L = \{(x, 2x) \mid x \in \mathbb{R}\} = \{t(1, 2) \mid t \in \mathbb{R}\}
\end{equation}
这是子空间(一维子空间)。

\textbf{例子2}:在 $\mathbb{R}^2$ 中,直线 $y = 2x + 1$(不通过原点)
\begin{equation}
L = \{(x, 2x + 1) \mid x \in \mathbb{R}\} = \{(0, 1) + t(1, 2) \mid t \in \mathbb{R}\}
\end{equation}
这是仿射集合,但不是子空间。

\section{第三部分:子空间是凸锥}

\subsection{锥(Cone)的定义}

集合 $C$ 称为\textbf{锥}(Cone),如果对于任意 $\mathbf{x} \in C$ 和 $\theta \geq 0$,都有:
\begin{equation}
\theta \mathbf{x} \in C
\end{equation}

换句话说,锥对非负数乘封闭。

\subsection{凸锥(Convex Cone)的定义}

集合 $C$ 称为\textbf{凸锥}(Convex Cone),如果它既是凸集又是锥。

等价地,对于任意 $\mathbf{x}_1, \mathbf{x}_2 \in C$ 和 $\theta_1, \theta_2 \geq 0$,都有:
\begin{equation}
\theta_1 \mathbf{x}_1 + \theta_2 \mathbf{x}_2 \in C
\end{equation}

\subsection{为什么子空间是凸锥?}

\textbf{方法1:使用凸锥的等价定义}

凸锥的等价定义:对于任意 $\mathbf{x}_1, \mathbf{x}_2 \in C$ 和 $\theta_1, \theta_2 \geq 0$(\textbf{不要求和为1}),都有:
\begin{equation}
\theta_1 \mathbf{x}_1 + \theta_2 \mathbf{x}_2 \in C
\end{equation}

\textbf{证明}:设 $V$ 是线性子空间。

对于任意 $\mathbf{x}_1, \mathbf{x}_2 \in V$ 和 $\theta_1, \theta_2 \geq 0$,由于 $V$ 对加法和数乘封闭,所以:
\begin{equation}
\theta_1 \mathbf{x}_1 + \theta_2 \mathbf{x}_2 \in V
\end{equation}

因此 $V$ 是凸锥。$\square$

\textbf{方法2:分别证明是锥和凸集}

\textbf{步骤1:证明 $V$ 是锥}

对于任意 $\mathbf{x} \in V$ 和 $\theta \geq 0$,由于 $V$ 对数乘封闭(包括非负数乘),所以 $\theta \mathbf{x} \in V$。

因此 $V$ 是锥。

\textbf{步骤2:证明 $V$ 是凸集}

对于任意 $\mathbf{x}_1, \mathbf{x}_2 \in V$ 和 $\theta \in [0, 1]$,由于 $V$ 对加法和数乘封闭,所以:
\begin{equation}
\theta \mathbf{x}_1 + (1-\theta) \mathbf{x}_2 \in V
\end{equation}

因此 $V$ 是凸集。

\textbf{结论}:$V$ 既是锥又是凸集,所以是凸锥。$\square$

\textbf{注意}:方法2中,凸组合($\theta \in [0, 1]$)只证明了是凸集,但结合"是锥"的证明,可以得出是凸锥。方法1直接使用凸锥的定义,更加直接。

\subsection{几何直观}

\begin{itemize}
\item \textbf{锥}:从原点出发的"扇形"区域
\item \textbf{凸锥}:从原点出发的凸扇形区域
\item \textbf{子空间}:从原点出发的直线、平面等,既是锥(可以向两个方向延伸),也是凸集
\end{itemize}

\subsection{注意:子空间是特殊的凸锥}

子空间比一般的凸锥更特殊:
\begin{itemize}
\item 子空间允许负系数:$\theta \in \mathbb{R}$(可以是负数)
\item 一般凸锥只允许非负系数:$\theta \geq 0$
\item 但子空间也满足凸锥的条件(因为包含非负数乘的情况)
\end{itemize}

\section{完整逻辑链条}

\subsection{推理过程}

\begin{enumerate}
\item \textbf{任何直线都是仿射集合}
   \begin{itemize}
   \item 因为直线包含通过其上任意两点的整条直线
   \item 满足仿射集合的定义
   \end{itemize}

\item \textbf{如果直线通过原点,它是子空间}
   \begin{itemize}
   \item 通过原点的直线可以表示为 $\{t \mathbf{v} \mid t \in \mathbb{R}\}$
   \item 满足子空间的三个条件(包含原点、对加法封闭、对数乘封闭)
   \end{itemize}

\item \textbf{子空间是凸锥}
   \begin{itemize}
   \item 子空间对非负数乘封闭(是锥)
   \item 子空间是凸集
   \item 因此是凸锥
   \end{itemize}

\item \textbf{结论}:通过原点的直线是仿射集合、子空间和凸锥
\end{enumerate}

\section{具体例子}

\subsection{例子1:通过原点的直线}

在 $\mathbb{R}^2$ 中,考虑直线 $L = \{(x, 2x) \mid x \in \mathbb{R}\}$。

\textbf{验证是仿射集合}:
对于 $(x_1, 2x_1), (x_2, 2x_2) \in L$ 和 $\theta \in \mathbb{R}$:
\begin{equation}
\theta(x_1, 2x_1) + (1-\theta)(x_2, 2x_2) = (\theta x_1 + (1-\theta)x_2, 2(\theta x_1 + (1-\theta)x_2)) \in L
\end{equation}
✓ 是仿射集合

\textbf{验证是子空间}:
\begin{itemize}
\item $(0, 0) \in L$ ✓
\item 对加法封闭 ✓
\item 对数乘封闭 ✓
\end{itemize}
✓ 是子空间

\textbf{验证是凸锥}:

\textbf{方法1(直接使用凸锥定义)}:
对于 $(x_1, 2x_1), (x_2, 2x_2) \in L$ 和 $\theta_1, \theta_2 \geq 0$(\textbf{不要求和为1}):
\begin{align}
\theta_1(x_1, 2x_1) + \theta_2(x_2, 2x_2) &= (\theta_1 x_1 + \theta_2 x_2, 2(\theta_1 x_1 + \theta_2 x_2)) \\
&= (t, 2t) \in L
\end{align}
其中 $t = \theta_1 x_1 + \theta_2 x_2 \in \mathbb{R}$。✓

\textbf{方法2(分别证明)}:
\begin{itemize}
\item 是锥:对于 $(x, 2x) \in L$ 和 $\theta \geq 0$,$\theta(x, 2x) = (\theta x, 2\theta x) \in L$ ✓
\item 是凸集:对于 $(x_1, 2x_1), (x_2, 2x_2) \in L$ 和 $\theta \in [0, 1]$,凸组合仍在 $L$ 中 ✓
\end{itemize}
✓ 是凸锥

\subsection{例子2:不通过原点的直线}

在 $\mathbb{R}^2$ 中,考虑直线 $L = \{(x, 2x + 1) \mid x \in \mathbb{R}\}$。

\textbf{验证是仿射集合}:✓(任何直线都是仿射集合)

\textbf{验证是子空间}:
\begin{itemize}
\item $(0, 1) \in L$,但 $(0, 0) \notin L$ ✗
\item 不包含原点,所以不是子空间
\end{itemize}

\textbf{验证是凸锥}:
\begin{itemize}
\item 不是锥:例如 $(0, 1) \in L$,但 $2(0, 1) = (0, 2) \notin L$ ✗
\item 因此不是凸锥
\end{itemize}

\section{概念层次关系}

\subsection{包含关系}

\begin{align}
\text{子空间} &\subseteq \text{仿射集合} \\
\text{子空间} &\subseteq \text{凸锥} \\
\text{凸锥} &\subseteq \text{凸集} \\
\text{仿射集合} &\subseteq \text{凸集}
\end{align}

\textbf{注意}:子空间同时是仿射集合和凸锥,因此也是凸集。

\subsection{维恩图理解}

\begin{itemize}
\item \textbf{仿射集合}:包含通过任意两点的整条直线
\item \textbf{凸集}:包含连接任意两点的线段
\item \textbf{锥}:对非负数乘封闭
\item \textbf{子空间}:包含原点,对加法和数乘封闭
\end{itemize}

通过原点的直线同时满足所有这些条件。

\section{推广}

\subsection{一般子空间}

\textbf{定理}:任何子空间都是仿射集合和凸锥。

\textbf{证明}:
\begin{itemize}
\item 子空间是仿射集合:因为子空间包含通过任意两点的整条直线
\item 子空间是凸锥:因为子空间对非负数乘封闭(是锥),且是凸集
\end{itemize}

\subsection{通过原点的仿射集合}

\textbf{定理}:如果仿射集合包含原点,那么它是子空间(因此也是凸锥)。

这是我们在第2.1.2节中学到的:仿射集合 = 子空间 + 偏移。如果偏移为0,则仿射集合就是子空间。

\section{重要澄清:凸锥与凸组合的区别}

\subsection{用户的疑问}

\textbf{疑问}:在证明凸锥时,如果只使用凸组合($\theta \in [0, 1]$),这不是只证明了是凸集(线段)吗?如何证明是凸锥?

\subsection{回答}

\textbf{关键区别}:

\begin{enumerate}
\item \textbf{凸集的定义}:对于 $\mathbf{x}_1, \mathbf{x}_2 \in C$ 和 $\theta \in [0, 1]$,有:
   \begin{equation}
   \theta \mathbf{x}_1 + (1-\theta) \mathbf{x}_2 \in C
   \end{equation}
   这里系数和必须等于1,且都在 $[0, 1]$ 之间。

\item \textbf{凸锥的定义}:对于 $\mathbf{x}_1, \mathbf{x}_2 \in C$ 和 $\theta_1, \theta_2 \geq 0$(\textbf{不要求和为1}),有:
   \begin{equation}
   \theta_1 \mathbf{x}_1 + \theta_2 \mathbf{x}_2 \in C
   \end{equation}
   这里系数只需要非负,不要求和为1。
\end{enumerate}

\subsection{为什么子空间是凸锥?}

对于子空间 $V$,对于任意 $\mathbf{x}_1, \mathbf{x}_2 \in V$ 和 $\theta_1, \theta_2 \geq 0$(不要求和为1),由于 $V$ 对加法和数乘封闭,所以:
\begin{equation}
\theta_1 \mathbf{x}_1 + \theta_2 \mathbf{x}_2 \in V
\end{equation}

\textbf{例子}:在直线 $L = \{(x, 2x) \mid x \in \mathbb{R}\}$ 中
\begin{itemize}
\item 取 $(1, 2), (2, 4) \in L$
\item 取 $\theta_1 = 3$,$\theta_2 = 2$(和等于5,不是1)
\item $3(1, 2) + 2(2, 4) = (3, 6) + (4, 8) = (7, 14) = (7, 2 \cdot 7) \in L$ ✓
\end{itemize}

这证明了是凸锥,而不仅仅是凸集。

\subsection{几何直观}

\begin{itemize}
\item \textbf{凸组合}($\theta \in [0, 1]$):只能表示两点之间的线段
\item \textbf{凸锥组合}($\theta_1, \theta_2 \geq 0$):可以表示从原点出发的"扇形"区域中的所有点
\item \textbf{子空间}:由于对数乘封闭(包括任意实数),所以凸锥组合的结果仍在子空间中
\end{itemize}

\section{总结}

\begin{enumerate}
\item \textbf{任何直线都是仿射集合}:
   \begin{itemize}
   \item 因为直线包含通过其上任意两点的整条直线
   \item 满足仿射集合的定义
   \end{itemize}

\item \textbf{通过原点的直线是子空间}:
   \begin{itemize}
   \item 可以表示为 $\{t \mathbf{v} \mid t \in \mathbb{R}\}$
   \item 满足子空间的三个条件
   \end{itemize}

\item \textbf{子空间是凸锥}:
   \begin{itemize}
   \item 子空间对非负数乘封闭(是锥)
   \item 子空间是凸集
   \item 因此是凸锥
   \end{itemize}

\item \textbf{逻辑链条}:
   \begin{itemize}
   \item 直线 → 仿射集合
   \item 通过原点的直线 → 子空间
   \item 子空间 → 凸锥
   \item 因此:通过原点的直线是仿射集合、子空间和凸锥
   \end{itemize}
\end{enumerate}

理解这些概念之间的关系对于掌握凸优化理论非常重要!

\end{document}

