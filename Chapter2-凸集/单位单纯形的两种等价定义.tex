\documentclass[12pt,a4paper]{article}
\usepackage[UTF8]{ctex}
\usepackage{amsmath}
\usepackage{amssymb}
\usepackage{amsthm}
\usepackage{geometry}
\geometry{left=2.5cm,right=2.5cm,top=2.5cm,bottom=2.5cm}

\title{单位单纯形的两种等价定义}
\subtitle{为什么 $\text{conv}\{\mathbf{0}, \mathbf{e}_1, \ldots, \mathbf{e}_n\}$ 等价于 $\{\mathbf{x} \mid \mathbf{x} \succeq \mathbf{0}, \mathbf{1}^T \mathbf{x} \leq 1\}$?}
\author{}
\date{\today}

\begin{document}

\maketitle

\section{问题提出}

单位单纯形有两种定义:

\textbf{定义1(凸包形式)}:
\begin{equation}
\text{conv}\{\mathbf{0}, \mathbf{e}_1, \mathbf{e}_2, \ldots, \mathbf{e}_n\}
\end{equation}

\textbf{定义2(不等式形式)}:
\begin{equation}
\{\mathbf{x} \in \mathbb{R}^n \mid \mathbf{x} \succeq \mathbf{0}, \mathbf{1}^T \mathbf{x} \leq 1\}
\end{equation}

\textbf{问题}:为什么这两种定义是等价的?如何从一种形式推导出另一种形式?

\section{从凸包形式到不等式形式}

\subsection{凸包形式的展开}

根据凸包的定义:
\begin{equation}
\text{conv}\{\mathbf{0}, \mathbf{e}_1, \ldots, \mathbf{e}_n\} = \left\{\sum_{i=0}^n \theta_i \mathbf{v}_i \middle| \theta_i \geq 0, \sum_{i=0}^n \theta_i = 1\right\}
\end{equation}

其中 $\mathbf{v}_0 = \mathbf{0}$,$\mathbf{v}_i = \mathbf{e}_i$($i = 1, \ldots, n$)。

展开:
\begin{align}
\sum_{i=0}^n \theta_i \mathbf{v}_i &= \theta_0 \mathbf{0} + \theta_1 \mathbf{e}_1 + \theta_2 \mathbf{e}_2 + \cdots + \theta_n \mathbf{e}_n \\
&= (0, 0, \ldots, 0) + (\theta_1, 0, \ldots, 0) + (0, \theta_2, \ldots, 0) + \cdots + (0, 0, \ldots, \theta_n) \\
&= (\theta_1, \theta_2, \ldots, \theta_n)
\end{align}

设 $\mathbf{x} = (\theta_1, \theta_2, \ldots, \theta_n)$,则:
\begin{align}
\mathbf{x} &= (\theta_1, \theta_2, \ldots, \theta_n) \\
&= \sum_{i=1}^n \theta_i \mathbf{e}_i
\end{align}

\subsection{约束条件的转换}

从凸包定义,我们有:
\begin{align}
\theta_i &\geq 0, \quad i = 0, 1, \ldots, n \\
\sum_{i=0}^n \theta_i &= 1
\end{align}

\textbf{关键观察}:
\begin{itemize}
\item $\theta_0 \geq 0$ 且 $\sum_{i=0}^n \theta_i = 1$ 意味着 $\sum_{i=1}^n \theta_i = 1 - \theta_0 \leq 1$
\item 由于 $\theta_0 \geq 0$,所以 $\sum_{i=1}^n \theta_i \leq 1$
\item 由于 $\theta_i \geq 0$($i = 1, \ldots, n$),所以 $\mathbf{x} = (\theta_1, \ldots, \theta_n) \succeq \mathbf{0}$
\end{itemize}

因此:
\begin{align}
\mathbf{x} &\succeq \mathbf{0} \quad \text{(因为 $\theta_i \geq 0$,$i = 1, \ldots, n$)} \\
\mathbf{1}^T \mathbf{x} &= \sum_{i=1}^n \theta_i = 1 - \theta_0 \leq 1 \quad \text{(因为 $\theta_0 \geq 0$)}
\end{align}

\subsection{完整推导}

对于任意 $\mathbf{x} \in \text{conv}\{\mathbf{0}, \mathbf{e}_1, \ldots, \mathbf{e}_n\}$,存在 $\theta_0, \theta_1, \ldots, \theta_n \geq 0$,$\sum_{i=0}^n \theta_i = 1$,使得:
\begin{equation}
\mathbf{x} = \theta_0 \mathbf{0} + \sum_{i=1}^n \theta_i \mathbf{e}_i = \sum_{i=1}^n \theta_i \mathbf{e}_i = (\theta_1, \theta_2, \ldots, \theta_n)
\end{equation}

因此:
\begin{align}
\mathbf{x} &= (\theta_1, \theta_2, \ldots, \theta_n) \succeq \mathbf{0} \quad \text{(因为 $\theta_i \geq 0$)} \\
\mathbf{1}^T \mathbf{x} &= \sum_{i=1}^n \theta_i = 1 - \theta_0 \leq 1 \quad \text{(因为 $\theta_0 \geq 0$)}
\end{align}

所以 $\mathbf{x} \in \{\mathbf{x} \mid \mathbf{x} \succeq \mathbf{0}, \mathbf{1}^T \mathbf{x} \leq 1\}$。

\section{从不等式形式到凸包形式}

\subsection{逆推导}

现在证明:如果 $\mathbf{x} \succeq \mathbf{0}$ 且 $\mathbf{1}^T \mathbf{x} \leq 1$,那么 $\mathbf{x} \in \text{conv}\{\mathbf{0}, \mathbf{e}_1, \ldots, \mathbf{e}_n\}$。

设 $\mathbf{x} = (x_1, x_2, \ldots, x_n)$,满足:
\begin{align}
x_i &\geq 0, \quad i = 1, \ldots, n \\
\sum_{i=1}^n x_i &\leq 1
\end{align}

\subsection{构造凸组合}

定义:
\begin{align}
\theta_0 &= 1 - \sum_{i=1}^n x_i \geq 0 \quad \text{(因为 $\sum_{i=1}^n x_i \leq 1$)} \\
\theta_i &= x_i \geq 0, \quad i = 1, \ldots, n
\end{align}

验证:
\begin{align}
\sum_{i=0}^n \theta_i &= \theta_0 + \sum_{i=1}^n \theta_i \\
&= \left(1 - \sum_{i=1}^n x_i\right) + \sum_{i=1}^n x_i \\
&= 1
\end{align}

现在构造凸组合:
\begin{align}
\sum_{i=0}^n \theta_i \mathbf{v}_i &= \theta_0 \mathbf{0} + \sum_{i=1}^n \theta_i \mathbf{e}_i \\
&= \mathbf{0} + \sum_{i=1}^n x_i \mathbf{e}_i \\
&= (x_1, x_2, \ldots, x_n) \\
&= \mathbf{x}
\end{align}

因此 $\mathbf{x} \in \text{conv}\{\mathbf{0}, \mathbf{e}_1, \ldots, \mathbf{e}_n\}$。

\section{完整等价性证明}

\subsection{定理}

单位单纯形的两种定义等价:
\begin{align}
\text{conv}\{\mathbf{0}, \mathbf{e}_1, \ldots, \mathbf{e}_n\} = \{\mathbf{x} \in \mathbb{R}^n \mid \mathbf{x} \succeq \mathbf{0}, \mathbf{1}^T \mathbf{x} \leq 1\}
\end{align}

\subsection{证明}

\textbf{方向1}:$\text{conv}\{\mathbf{0}, \mathbf{e}_1, \ldots, \mathbf{e}_n\} \subseteq \{\mathbf{x} \mid \mathbf{x} \succeq \mathbf{0}, \mathbf{1}^T \mathbf{x} \leq 1\}$

对于任意 $\mathbf{x} \in \text{conv}\{\mathbf{0}, \mathbf{e}_1, \ldots, \mathbf{e}_n\}$,存在 $\theta_0, \theta_1, \ldots, \theta_n \geq 0$,$\sum_{i=0}^n \theta_i = 1$,使得:
\begin{equation}
\mathbf{x} = \sum_{i=0}^n \theta_i \mathbf{v}_i = \sum_{i=1}^n \theta_i \mathbf{e}_i = (\theta_1, \ldots, \theta_n)
\end{equation}

因此:
\begin{align}
\mathbf{x} &= (\theta_1, \ldots, \theta_n) \succeq \mathbf{0} \\
\mathbf{1}^T \mathbf{x} &= \sum_{i=1}^n \theta_i = 1 - \theta_0 \leq 1
\end{align}

所以 $\mathbf{x} \in \{\mathbf{x} \mid \mathbf{x} \succeq \mathbf{0}, \mathbf{1}^T \mathbf{x} \leq 1\}$。

\textbf{方向2}:$\{\mathbf{x} \mid \mathbf{x} \succeq \mathbf{0}, \mathbf{1}^T \mathbf{x} \leq 1\} \subseteq \text{conv}\{\mathbf{0}, \mathbf{e}_1, \ldots, \mathbf{e}_n\}$

对于任意 $\mathbf{x} = (x_1, \ldots, x_n)$ 满足 $\mathbf{x} \succeq \mathbf{0}$ 且 $\mathbf{1}^T \mathbf{x} \leq 1$,定义:
\begin{align}
\theta_0 &= 1 - \sum_{i=1}^n x_i \geq 0 \\
\theta_i &= x_i \geq 0, \quad i = 1, \ldots, n
\end{align}

则 $\sum_{i=0}^n \theta_i = 1$,且:
\begin{equation}
\sum_{i=0}^n \theta_i \mathbf{v}_i = \theta_0 \mathbf{0} + \sum_{i=1}^n \theta_i \mathbf{e}_i = \sum_{i=1}^n x_i \mathbf{e}_i = \mathbf{x}
\end{equation}

所以 $\mathbf{x} \in \text{conv}\{\mathbf{0}, \mathbf{e}_1, \ldots, \mathbf{e}_n\}$。

因此两种定义等价。$\square$

\section{几何直观}

\subsection{二维情况}

在 $\mathbb{R}^2$ 中,单位单纯形是:
\begin{equation}
\text{conv}\{(0, 0), (1, 0), (0, 1)\}
\end{equation}

这是以 $(0, 0)$、$(1, 0)$、$(0, 1)$ 为顶点的三角形。

\textbf{不等式形式}:
\begin{align}
x_1 &\geq 0 \\
x_2 &\geq 0 \\
x_1 + x_2 &\leq 1
\end{align}

这正好定义了同一个三角形!

\subsection{三维情况}

在 $\mathbb{R}^3$ 中,单位单纯形是:
\begin{equation}
\text{conv}\{(0, 0, 0), (1, 0, 0), (0, 1, 0), (0, 0, 1)\}
\end{equation}

这是以原点 $(0, 0, 0)$ 和三个单位向量为顶点的四面体。

\textbf{不等式形式}:
\begin{align}
x_1 &\geq 0 \\
x_2 &\geq 0 \\
x_3 &\geq 0 \\
x_1 + x_2 + x_3 &\leq 1
\end{align}

这正好定义了同一个四面体!

\section{关键理解}

\subsection{为什么是 $\leq 1$ 而不是 $= 1$?}

\textbf{关键区别}:

\begin{itemize}
\item \textbf{单位单纯形}:$\mathbf{1}^T \mathbf{x} \leq 1$(允许小于1)
\item \textbf{概率单纯形}:$\mathbf{1}^T \mathbf{x} = 1$(必须等于1)
\end{itemize}

\textbf{原因}:

在单位单纯形中,我们允许 $\theta_0 > 0$(即 $\sum_{i=1}^n \theta_i < 1$),这意味着:
\begin{equation}
\mathbf{x} = \sum_{i=1}^n \theta_i \mathbf{e}_i = (\theta_1, \ldots, \theta_n)
\end{equation}

其中 $\sum_{i=1}^n \theta_i = 1 - \theta_0 \leq 1$。

\textbf{几何意义}:
\begin{itemize}
\item 当 $\mathbf{1}^T \mathbf{x} = 1$ 时,点在"面"上(不包含原点)
\item 当 $\mathbf{1}^T \mathbf{x} < 1$ 时,点在"内部"(更靠近原点)
\item 当 $\mathbf{1}^T \mathbf{x} = 0$ 时,点就是原点
\end{itemize}

\subsection{与概率单纯形的对比}

\begin{table}[h]
\centering
\begin{tabular}{|l|l|l|}
\hline
\textbf{性质} & \textbf{单位单纯形} & \textbf{概率单纯形} \\
\hline
顶点 & $\mathbf{0}, \mathbf{e}_1, \ldots, \mathbf{e}_n$ & $\mathbf{e}_1, \ldots, \mathbf{e}_n$ \\
\hline
约束 & $\mathbf{x} \succeq \mathbf{0}, \mathbf{1}^T \mathbf{x} \leq 1$ & $\mathbf{x} \succeq \mathbf{0}, \mathbf{1}^T \mathbf{x} = 1$ \\
\hline
维度 & $n$ 维 & $n-1$ 维 \\
\hline
包含原点 & 是 & 否 \\
\hline
\end{tabular}
\caption{单位单纯形与概率单纯形的对比}
\end{table}

\section{具体例子}

\subsection{例子1:二维单位单纯形}

在 $\mathbb{R}^2$ 中:

\textbf{凸包形式}:
\begin{equation}
\text{conv}\{(0, 0), (1, 0), (0, 1)\}
\end{equation}

\textbf{不等式形式}:
\begin{align}
x_1 &\geq 0 \\
x_2 &\geq 0 \\
x_1 + x_2 &\leq 1
\end{align}

\textbf{验证}:

点 $(0.3, 0.4)$:
\begin{itemize}
\item 凸包形式:$(0.3, 0.4) = 0.3(1, 0) + 0.4(0, 1) + 0.3(0, 0)$,系数和 = 1 ✓
\item 不等式形式:$0.3 \geq 0$,$0.4 \geq 0$,$0.3 + 0.4 = 0.7 \leq 1$ ✓
\end{itemize}

点 $(0.6, 0.5)$:
\begin{itemize}
\item 凸包形式:$(0.6, 0.5) = 0.6(1, 0) + 0.5(0, 1) + (-0.1)(0, 0)$,但 $\theta_0 = -0.1 < 0$ ✗
\item 不等式形式:$0.6 + 0.5 = 1.1 > 1$ ✗
\end{itemize}

所以 $(0.6, 0.5)$ 不在单位单纯形中。

\subsection{例子2:三维单位单纯形}

在 $\mathbb{R}^3$ 中:

\textbf{凸包形式}:
\begin{equation}
\text{conv}\{(0, 0, 0), (1, 0, 0), (0, 1, 0), (0, 0, 1)\}
\end{equation}

\textbf{不等式形式}:
\begin{align}
x_1 &\geq 0 \\
x_2 &\geq 0 \\
x_3 &\geq 0 \\
x_1 + x_2 + x_3 &\leq 1
\end{align}

\textbf{验证}:

点 $(0.2, 0.3, 0.4)$:
\begin{itemize}
\item 凸包形式:$(0.2, 0.3, 0.4) = 0.2(1, 0, 0) + 0.3(0, 1, 0) + 0.4(0, 0, 1) + 0.1(0, 0, 0)$,系数和 = 1 ✓
\item 不等式形式:$0.2 + 0.3 + 0.4 = 0.9 \leq 1$ ✓
\end{itemize}

\section{总结}

\begin{enumerate}
\item \textbf{等价性}:
   \begin{itemize}
   \item 单位单纯形的两种定义完全等价
   \item 凸包形式:$\text{conv}\{\mathbf{0}, \mathbf{e}_1, \ldots, \mathbf{e}_n\}$
   \item 不等式形式:$\{\mathbf{x} \mid \mathbf{x} \succeq \mathbf{0}, \mathbf{1}^T \mathbf{x} \leq 1\}$
   \end{itemize}

\item \textbf{关键转换}:
   \begin{itemize}
   \item 从凸包到不等式:$\theta_0 = 1 - \sum_{i=1}^n \theta_i \geq 0$ 意味着 $\sum_{i=1}^n \theta_i \leq 1$
   \item 从不等式到凸包:设 $\theta_0 = 1 - \sum_{i=1}^n x_i$,$\theta_i = x_i$
   \end{itemize}

\item \textbf{为什么是 $\leq 1$}:
   \begin{itemize}
   \item 因为允许 $\theta_0 > 0$(即包含原点)
   \item $\sum_{i=1}^n \theta_i = 1 - \theta_0 \leq 1$
   \item 与概率单纯形($= 1$)的区别在于是否包含原点
   \end{itemize}

\item \textbf{几何意义}:
   \begin{itemize}
   \item 单位单纯形是包含原点的"完整"单纯形
   \item 概率单纯形是不包含原点的"面"
   \end{itemize}
\end{enumerate}

理解这两种定义的等价性,对于掌握单纯形和凸包的概念非常重要!

\end{document}


