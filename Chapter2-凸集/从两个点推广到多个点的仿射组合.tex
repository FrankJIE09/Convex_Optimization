\documentclass[12pt,a4paper]{article}
\usepackage[UTF8]{ctex}
\usepackage{amsmath}
\usepackage{amssymb}
\usepackage{amsthm}
\usepackage{geometry}
\geometry{left=2.5cm,right=2.5cm,top=2.5cm,bottom=2.5cm}

\title{为什么可以从两个点推广到多个点的仿射组合?}
\author{}
\date{\today}

\begin{document}

\maketitle

\section{问题提出}

我们知道:
\begin{itemize}
\item \textbf{两个点}的仿射组合:$\theta \mathbf{x}_1 + (1-\theta) \mathbf{x}_2$,其中 $\theta \in \mathbb{R}$,且 $\theta + (1-\theta) = 1$
\item \textbf{多个点}的仿射组合:$\theta_1 \mathbf{x}_1 + \theta_2 \mathbf{x}_2 + \cdots + \theta_k \mathbf{x}_k$,其中 $\theta_1 + \theta_2 + \cdots + \theta_k = 1$
\end{itemize}

\textbf{问题}:为什么可以从两个点的情况推广到多个点?这种推广的数学依据是什么?

\section{从两个点到三个点}

\subsection{直观思路}

假设我们已经知道两个点的仿射组合:$\theta \mathbf{x}_1 + (1-\theta) \mathbf{x}_2$。

现在考虑三个点 $\mathbf{x}_1, \mathbf{x}_2, \mathbf{x}_3$。我们可以这样思考:

\begin{enumerate}
\item 先取 $\mathbf{x}_1$ 和 $\mathbf{x}_2$ 的仿射组合:$\alpha \mathbf{x}_1 + (1-\alpha) \mathbf{x}_2$
\item 再取这个结果与 $\mathbf{x}_3$ 的仿射组合
\end{enumerate}

具体地:
\begin{align}
\mathbf{y} &= \beta [\alpha \mathbf{x}_1 + (1-\alpha) \mathbf{x}_2] + (1-\beta) \mathbf{x}_3 \\
&= \beta\alpha \mathbf{x}_1 + \beta(1-\alpha) \mathbf{x}_2 + (1-\beta) \mathbf{x}_3 \\
&= \theta_1 \mathbf{x}_1 + \theta_2 \mathbf{x}_2 + \theta_3 \mathbf{x}_3
\end{align}

其中:
\begin{align}
\theta_1 &= \beta\alpha \\
\theta_2 &= \beta(1-\alpha) \\
\theta_3 &= 1-\beta
\end{align}

\textbf{关键观察}:
\begin{align}
\theta_1 + \theta_2 + \theta_3 &= \beta\alpha + \beta(1-\alpha) + (1-\beta) \\
&= \beta[\alpha + (1-\alpha)] + (1-\beta) \\
&= \beta \cdot 1 + (1-\beta) \\
&= 1
\end{align}

所以,通过两次使用两个点的仿射组合,我们得到了三个点的组合,且系数和等于1!

\subsection{逆过程:任意三个点的仿射组合}

反过来,对于任意满足 $\theta_1 + \theta_2 + \theta_3 = 1$ 的系数,我们能否将其表示为上述形式?

\textbf{情况1}:如果 $\theta_1 + \theta_2 \neq 0$,设 $\beta = \theta_1 + \theta_2$,则:
\begin{align}
\alpha &= \frac{\theta_1}{\theta_1 + \theta_2} \\
1-\alpha &= \frac{\theta_2}{\theta_1 + \theta_2}
\end{align}

验证:
\begin{align}
\beta\alpha &= (\theta_1 + \theta_2) \cdot \frac{\theta_1}{\theta_1 + \theta_2} = \theta_1 \\
\beta(1-\alpha) &= (\theta_1 + \theta_2) \cdot \frac{\theta_2}{\theta_1 + \theta_2} = \theta_2 \\
1-\beta &= 1 - (\theta_1 + \theta_2) = \theta_3
\end{align}

\textbf{情况2}:如果 $\theta_1 + \theta_2 = 0$,则 $\theta_3 = 1$,此时:
\begin{equation}
\theta_1 \mathbf{x}_1 + \theta_2 \mathbf{x}_2 + \theta_3 \mathbf{x}_3 = \mathbf{x}_3
\end{equation}
这可以看作 $\mathbf{x}_3$ 与任意其他点的仿射组合(系数为1和0)。

\section{数学归纳法证明}

\subsection{归纳基础}

对于 $k = 2$,定义已经给出:$\theta_1 \mathbf{x}_1 + \theta_2 \mathbf{x}_2$,其中 $\theta_1 + \theta_2 = 1$。

\subsection{归纳假设}

假设对于 $k$ 个点,我们已经定义了仿射组合:$\sum_{i=1}^k \theta_i \mathbf{x}_i$,其中 $\sum_{i=1}^k \theta_i = 1$。

\subsection{归纳步骤}

现在考虑 $k+1$ 个点 $\mathbf{x}_1, \ldots, \mathbf{x}_{k+1}$。

对于任意满足 $\sum_{i=1}^{k+1} \theta_i = 1$ 的系数,我们有两种情况:

\textbf{情况1}:$\sum_{i=1}^k \theta_i \neq 0$

设 $\beta = \sum_{i=1}^k \theta_i$,则 $\theta_{k+1} = 1 - \beta$。

定义:
\begin{equation}
\alpha_i = \frac{\theta_i}{\beta}, \quad i = 1, 2, \ldots, k
\end{equation}

注意:$\sum_{i=1}^k \alpha_i = \frac{1}{\beta} \sum_{i=1}^k \theta_i = \frac{\beta}{\beta} = 1$。

根据归纳假设,$\sum_{i=1}^k \alpha_i \mathbf{x}_i$ 是前 $k$ 个点的仿射组合。

现在取这个结果与 $\mathbf{x}_{k+1}$ 的仿射组合:
\begin{align}
\mathbf{y} &= \beta \left(\sum_{i=1}^k \alpha_i \mathbf{x}_i\right) + (1-\beta) \mathbf{x}_{k+1} \\
&= \sum_{i=1}^k \beta \alpha_i \mathbf{x}_i + (1-\beta) \mathbf{x}_{k+1} \\
&= \sum_{i=1}^k \theta_i \mathbf{x}_i + \theta_{k+1} \mathbf{x}_{k+1} \\
&= \sum_{i=1}^{k+1} \theta_i \mathbf{x}_i
\end{align}

\textbf{情况2}:$\sum_{i=1}^k \theta_i = 0$

则 $\theta_{k+1} = 1$,所以:
\begin{equation}
\sum_{i=1}^{k+1} \theta_i \mathbf{x}_i = \mathbf{x}_{k+1}
\end{equation}
这是平凡的仿射组合。

\subsection{结论}

通过数学归纳法,我们证明了:对于任意 $k \geq 2$ 个点,所有满足 $\sum_{i=1}^k \theta_i = 1$ 的组合都可以通过反复使用两个点的仿射组合得到。

\section{仿射集合的定义要求}

\subsection{仿射集合的基本性质}

仿射集合 $C$ 的定义要求:对于任意 $\mathbf{x}_1, \mathbf{x}_2 \in C$ 和 $\theta \in \mathbb{R}$,有:
\begin{equation}
\theta \mathbf{x}_1 + (1-\theta) \mathbf{x}_2 \in C
\end{equation}

\subsection{为什么需要多个点的仿射组合?}

\textbf{关键问题}:如果仿射集合只要求两个点的仿射组合在集合内,为什么我们还要考虑多个点的仿射组合?

\textbf{答案}:通过数学归纳法可以证明,如果集合包含任意两个点的仿射组合,那么它自动包含任意多个点的仿射组合!

\subsection{证明:仿射集合包含多个点的仿射组合}

\textbf{定理}:如果 $C$ 是仿射集合,那么对于任意 $k$ 个点 $\mathbf{x}_1, \ldots, \mathbf{x}_k \in C$ 和满足 $\sum_{i=1}^k \theta_i = 1$ 的系数,都有:
\begin{equation}
\sum_{i=1}^k \theta_i \mathbf{x}_i \in C
\end{equation}

\textbf{证明}(数学归纳法):

\textbf{基础步骤}($k = 2$):由定义直接得到。

\textbf{归纳假设}:假设对于 $k$ 个点成立。

\textbf{归纳步骤}:考虑 $k+1$ 个点。

对于满足 $\sum_{i=1}^{k+1} \theta_i = 1$ 的系数,如果 $\sum_{i=1}^k \theta_i \neq 0$,设 $\beta = \sum_{i=1}^k \theta_i$,则:

根据归纳假设,$\sum_{i=1}^k \frac{\theta_i}{\beta} \mathbf{x}_i \in C$(因为 $\sum_{i=1}^k \frac{\theta_i}{\beta} = 1$)。

由于 $C$ 是仿射集合,对于点 $\sum_{i=1}^k \frac{\theta_i}{\beta} \mathbf{x}_i \in C$ 和 $\mathbf{x}_{k+1} \in C$,它们的仿射组合也在 $C$ 中:
\begin{align}
\beta \left(\sum_{i=1}^k \frac{\theta_i}{\beta} \mathbf{x}_i\right) + (1-\beta) \mathbf{x}_{k+1} &= \sum_{i=1}^k \theta_i \mathbf{x}_i + \theta_{k+1} \mathbf{x}_{k+1} \\
&= \sum_{i=1}^{k+1} \theta_i \mathbf{x}_i \in C
\end{align}

如果 $\sum_{i=1}^k \theta_i = 0$,则 $\theta_{k+1} = 1$,所以 $\sum_{i=1}^{k+1} \theta_i \mathbf{x}_i = \mathbf{x}_{k+1} \in C$。

因此,归纳步骤成立。$\square$

\section{几何直观}

\subsection{两个点的情况}

对于两个点 $\mathbf{x}_1$ 和 $\mathbf{x}_2$,仿射组合 $\theta \mathbf{x}_1 + (1-\theta) \mathbf{x}_2$ 表示通过这两点的整条直线。

\subsection{三个点的情况}

对于三个点 $\mathbf{x}_1, \mathbf{x}_2, \mathbf{x}_3$,我们可以这样理解:

\begin{enumerate}
\item 先取 $\mathbf{x}_1$ 和 $\mathbf{x}_2$ 的仿射组合,得到通过这两点的直线上的所有点
\item 再取这条直线上的点与 $\mathbf{x}_3$ 的仿射组合,得到通过 $\mathbf{x}_3$ 和直线上任意点的所有直线
\item 所有这些直线的并集,就是通过这三个点的平面
\end{enumerate}

\textbf{关键观察}:通过三个不共线点的所有仿射组合,恰好覆盖了整个平面!

\subsection{多个点的情况}

类似地,对于 $k$ 个点:
\begin{itemize}
\item 如果这些点不在同一个低维仿射空间中,它们的仿射组合覆盖整个 $k-1$ 维仿射空间
\item 如果这些点在同一个低维仿射空间中,它们的仿射组合覆盖该低维仿射空间
\end{itemize}

\section{为什么这种推广是自然的?}

\subsection{一致性}

两个点的仿射组合要求系数和等于1,多个点的仿射组合也要求系数和等于1,这保持了定义的一致性。

\subsection{完备性}

如果我们只允许两个点的仿射组合,那么:
\begin{itemize}
\item 对于三个点,我们只能得到通过每对点的直线
\item 但无法直接表示平面上的所有点
\end{itemize}

通过推广到多个点,我们可以表示整个仿射空间。

\subsection{数学优雅性}

多个点的仿射组合可以表示为两个点的仿射组合的复合,这显示了定义的内部一致性。

\section{具体例子}

\subsection{例子1:从两个点到三个点}

在 $\mathbb{R}^2$ 中,考虑三个点:
\begin{align}
\mathbf{x}_1 &= (0, 0) \\
\mathbf{x}_2 &= (1, 0) \\
\mathbf{x}_3 &= (0, 1)
\end{align}

\textbf{方法1}:直接使用三个点的仿射组合

对于 $\theta_1 = 0.3$,$\theta_2 = 0.5$,$\theta_3 = 0.2$(和等于1):
\begin{equation}
\mathbf{y} = 0.3(0, 0) + 0.5(1, 0) + 0.2(0, 1) = (0.5, 0.2)
\end{equation}

\textbf{方法2}:分两步使用两个点的仿射组合

第一步:取 $\mathbf{x}_1$ 和 $\mathbf{x}_2$ 的仿射组合
\begin{equation}
\mathbf{z} = \alpha \mathbf{x}_1 + (1-\alpha) \mathbf{x}_2 = (1-\alpha, 0)
\end{equation}

设 $\alpha = 0.4$,则 $\mathbf{z} = (0.6, 0)$。

第二步:取 $\mathbf{z}$ 和 $\mathbf{x}_3$ 的仿射组合
\begin{equation}
\mathbf{y} = \beta \mathbf{z} + (1-\beta) \mathbf{x}_3 = \beta(0.6, 0) + (1-\beta)(0, 1)
\end{equation}

要得到 $\mathbf{y} = (0.5, 0.2)$,我们需要:
\begin{align}
0.6\beta &= 0.5 \quad \Rightarrow \quad \beta = \frac{5}{6} \\
1-\beta &= 0.2 \quad \Rightarrow \quad \beta = 0.8
\end{align}

这看起来矛盾,但实际上我们可以调整 $\alpha$。

\textbf{正确的方法}:

要表示 $\theta_1 = 0.3$,$\theta_2 = 0.5$,$\theta_3 = 0.2$,设 $\beta = \theta_1 + \theta_2 = 0.8$,则:
\begin{align}
\alpha &= \frac{\theta_1}{\theta_1 + \theta_2} = \frac{0.3}{0.8} = 0.375 \\
\mathbf{z} &= 0.375(0, 0) + 0.625(1, 0) = (0.625, 0) \\
\mathbf{y} &= 0.8(0.625, 0) + 0.2(0, 1) = (0.5, 0.2)
\end{align}

验证:$\beta\alpha = 0.8 \times 0.375 = 0.3 = \theta_1$,$\beta(1-\alpha) = 0.8 \times 0.625 = 0.5 = \theta_2$,$1-\beta = 0.2 = \theta_3$。$\checkmark$

\section{总结}

\begin{enumerate}
\item \textbf{推广的数学依据}:
   \begin{itemize}
   \item 多个点的仿射组合可以表示为两个点的仿射组合的复合
   \item 通过数学归纳法可以严格证明
   \end{itemize}

\item \textbf{推广的自然性}:
   \begin{itemize}
   \item 保持系数和等于1的一致性
   \item 能够表示整个仿射空间
   \item 定义内部一致且优雅
   \end{itemize}

\item \textbf{推广的必要性}:
   \begin{itemize}
   \item 如果仿射集合包含两个点的仿射组合,它自动包含多个点的仿射组合
   \item 多个点的仿射组合是仿射集合定义的必然结果
   \end{itemize}

\item \textbf{几何意义}:
   \begin{itemize}
   \item 两个点的仿射组合:直线
   \item 三个点的仿射组合:平面(如果三点不共线)
   \item $k$ 个点的仿射组合:$k-1$ 维仿射空间(如果点处于一般位置)
   \end{itemize}
\end{enumerate}

因此,从两个点推广到多个点不仅是自然的,而且是数学上严格和必要的!

\end{document}

