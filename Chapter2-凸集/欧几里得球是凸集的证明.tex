\documentclass[12pt,a4paper]{article}
\usepackage[UTF8]{ctex}
\usepackage{amsmath}
\usepackage{amssymb}
\usepackage{amsthm}
\usepackage{geometry}
\geometry{left=2.5cm,right=2.5cm,top=2.5cm,bottom=2.5cm}

\title{欧几里得球是凸集的证明详解}
\subtitle{理解每一步的推导过程}
\author{}
\date{\today}

\begin{document}

\maketitle

\section{问题提出}

要证明欧几里得球是凸集,需要证明:如果 $\mathbf{x}_1$ 和 $\mathbf{x}_2$ 在球内,那么它们的凸组合也在球内。

\textbf{证明过程}:
\begin{align}
\|\theta \mathbf{x}_1 + (1-\theta) \mathbf{x}_2 - \mathbf{x}_c\|_2 &= \|\theta(\mathbf{x}_1 - \mathbf{x}_c) + (1-\theta)(\mathbf{x}_2 - \mathbf{x}_c)\|_2 \\
&\leq \theta\|\mathbf{x}_1 - \mathbf{x}_c\|_2 + (1-\theta)\|\mathbf{x}_2 - \mathbf{x}_c\|_2 \\
&\leq r
\end{align}

\textbf{问题}:每一步是怎么来的?为什么可以这样推导?

\section{第一步:代数恒等式}

\subsection{目标}

要证明:$\|\theta \mathbf{x}_1 + (1-\theta) \mathbf{x}_2 - \mathbf{x}_c\|_2 = \|\theta(\mathbf{x}_1 - \mathbf{x}_c) + (1-\theta)(\mathbf{x}_2 - \mathbf{x}_c)\|_2$

\subsection{推导过程}

\begin{align}
\theta \mathbf{x}_1 + (1-\theta) \mathbf{x}_2 - \mathbf{x}_c &= \theta \mathbf{x}_1 + (1-\theta) \mathbf{x}_2 - \mathbf{x}_c
\end{align}

关键技巧:将 $\mathbf{x}_c$ 写成 $\theta \mathbf{x}_c + (1-\theta) \mathbf{x}_c$(因为 $\theta + (1-\theta) = 1$):

\begin{align}
\theta \mathbf{x}_1 + (1-\theta) \mathbf{x}_2 - \mathbf{x}_c &= \theta \mathbf{x}_1 + (1-\theta) \mathbf{x}_2 - [\theta \mathbf{x}_c + (1-\theta) \mathbf{x}_c] \\
&= \theta \mathbf{x}_1 + (1-\theta) \mathbf{x}_2 - \theta \mathbf{x}_c - (1-\theta) \mathbf{x}_c \\
&= \theta(\mathbf{x}_1 - \mathbf{x}_c) + (1-\theta)(\mathbf{x}_2 - \mathbf{x}_c)
\end{align}

\subsection{为什么这样做?}

\begin{itemize}
\item 将问题转化为关于"相对位置"的问题
\item $\mathbf{x}_1 - \mathbf{x}_c$ 和 $\mathbf{x}_2 - \mathbf{x}_c$ 是从中心 $\mathbf{x}_c$ 到两个点的向量
\item 这样可以利用已知条件:$\|\mathbf{x}_1 - \mathbf{x}_c\|_2 \leq r$ 和 $\|\mathbf{x}_2 - \mathbf{x}_c\|_2 \leq r$
\end{itemize}

\subsection{几何直观}

\begin{itemize}
\item 原始问题:证明从中心到凸组合点的距离 $\leq r$
\item 转化后:证明两个"相对向量"的凸组合的范数 $\leq r$
\item 这更容易处理,因为我们可以利用范数的性质
\end{itemize}

\section{第二步:三角不等式}

\subsection{目标}

要证明:$\|\theta(\mathbf{x}_1 - \mathbf{x}_c) + (1-\theta)(\mathbf{x}_2 - \mathbf{x}_c)\|_2 \leq \theta\|\mathbf{x}_1 - \mathbf{x}_c\|_2 + (1-\theta)\|\mathbf{x}_2 - \mathbf{x}_c\|_2$

\subsection{使用的性质:范数的三角不等式}

\textbf{三角不等式}:对于任意向量 $\mathbf{u}, \mathbf{v}$ 和任意范数,有:
\begin{equation}
\|\mathbf{u} + \mathbf{v}\| \leq \|\mathbf{u}\| + \|\mathbf{v}\|
\end{equation}

\subsection{应用到我们的情况}

设 $\mathbf{u} = \theta(\mathbf{x}_1 - \mathbf{x}_c)$,$\mathbf{v} = (1-\theta)(\mathbf{x}_2 - \mathbf{x}_c)$。

直接应用三角不等式:
\begin{equation}
\|\theta(\mathbf{x}_1 - \mathbf{x}_c) + (1-\theta)(\mathbf{x}_2 - \mathbf{x}_c)\|_2 \leq \|\theta(\mathbf{x}_1 - \mathbf{x}_c)\|_2 + \|(1-\theta)(\mathbf{x}_2 - \mathbf{x}_c)\|_2
\end{equation}

\subsection{使用范数的齐次性}

\textbf{齐次性}:对于任意标量 $\alpha$ 和向量 $\mathbf{w}$,有:
\begin{equation}
\|\alpha \mathbf{w}\| = |\alpha| \|\mathbf{w}\|
\end{equation}

应用到我们的情况:
\begin{align}
\|\theta(\mathbf{x}_1 - \mathbf{x}_c)\|_2 &= |\theta| \|\mathbf{x}_1 - \mathbf{x}_c\|_2 = \theta \|\mathbf{x}_1 - \mathbf{x}_c\|_2 \\
\|(1-\theta)(\mathbf{x}_2 - \mathbf{x}_c)\|_2 &= |1-\theta| \|\mathbf{x}_2 - \mathbf{x}_c\|_2 = (1-\theta) \|\mathbf{x}_2 - \mathbf{x}_c\|_2
\end{align}

注意:由于 $\theta \in [0, 1]$,所以 $|\theta| = \theta$ 和 $|1-\theta| = 1-\theta$。

\subsection{完整推导}

\begin{align}
\|\theta(\mathbf{x}_1 - \mathbf{x}_c) + (1-\theta)(\mathbf{x}_2 - \mathbf{x}_c)\|_2 &\leq \|\theta(\mathbf{x}_1 - \mathbf{x}_c)\|_2 + \|(1-\theta)(\mathbf{x}_2 - \mathbf{x}_c)\|_2 \quad \text{(三角不等式)} \\
&= |\theta| \|\mathbf{x}_1 - \mathbf{x}_c\|_2 + |1-\theta| \|\mathbf{x}_2 - \mathbf{x}_c\|_2 \quad \text{(齐次性)} \\
&= \theta \|\mathbf{x}_1 - \mathbf{x}_c\|_2 + (1-\theta) \|\mathbf{x}_2 - \mathbf{x}_c\|_2 \quad \text{(因为 $\theta \geq 0$,$1-\theta \geq 0$)}
\end{align}

\section{第三步:利用已知条件}

\subsection{已知条件}

\begin{align}
\|\mathbf{x}_1 - \mathbf{x}_c\|_2 &\leq r \\
\|\mathbf{x}_2 - \mathbf{x}_c\|_2 &\leq r
\end{align}

\subsection{应用条件}

由于 $\theta \geq 0$ 和 $1-\theta \geq 0$,我们可以将不等式两边分别乘以非负系数:

\begin{align}
\theta \|\mathbf{x}_1 - \mathbf{x}_c\|_2 &\leq \theta r \\
(1-\theta) \|\mathbf{x}_2 - \mathbf{x}_c\|_2 &\leq (1-\theta) r
\end{align}

因此:
\begin{align}
\theta \|\mathbf{x}_1 - \mathbf{x}_c\|_2 + (1-\theta) \|\mathbf{x}_2 - \mathbf{x}_c\|_2 &\leq \theta r + (1-\theta) r \\
&= r(\theta + 1 - \theta) \\
&= r
\end{align}

\subsection{完整链条}

\begin{align}
\|\theta \mathbf{x}_1 + (1-\theta) \mathbf{x}_2 - \mathbf{x}_c\|_2 &= \|\theta(\mathbf{x}_1 - \mathbf{x}_c) + (1-\theta)(\mathbf{x}_2 - \mathbf{x}_c)\|_2 \quad \text{(第一步:代数恒等式)} \\
&\leq \theta \|\mathbf{x}_1 - \mathbf{x}_c\|_2 + (1-\theta) \|\mathbf{x}_2 - \mathbf{x}_c\|_2 \quad \text{(第二步:三角不等式)} \\
&\leq r \quad \text{(第三步:利用已知条件)}
\end{align}

因此,$\theta \mathbf{x}_1 + (1-\theta) \mathbf{x}_2$ 在球内,所以球是凸集。$\square$

\section{详细步骤总结}

\subsection{步骤1:代数恒等式}

\textbf{技巧}:将 $\mathbf{x}_c$ 写成 $\theta \mathbf{x}_c + (1-\theta) \mathbf{x}_c$

\textbf{目的}:转化为相对向量的形式,便于利用已知条件

\subsection{步骤2:三角不等式}

\textbf{使用}:$\|\mathbf{u} + \mathbf{v}\| \leq \|\mathbf{u}\| + \|\mathbf{v}\|$

\textbf{结合}:范数的齐次性 $\|\alpha \mathbf{w}\| = |\alpha| \|\mathbf{w}\|$

\textbf{结果}:将两个向量的和的范数,转化为两个范数的加权和

\subsection{步骤3:利用已知条件}

\textbf{使用}:$\|\mathbf{x}_1 - \mathbf{x}_c\|_2 \leq r$ 和 $\|\mathbf{x}_2 - \mathbf{x}_c\|_2 \leq r$

\textbf{结合}:$\theta + (1-\theta) = 1$

\textbf{结果}:得到最终的上界 $r$

\section{几何直观}

\subsection{二维情况}

在 $\mathbb{R}^2$ 中,考虑以原点为中心、半径为 $r$ 的圆。

\begin{itemize}
\item $\mathbf{x}_1$ 和 $\mathbf{x}_2$ 在圆内
\item 它们的凸组合 $\theta \mathbf{x}_1 + (1-\theta) \mathbf{x}_2$ 也在圆内
\item 这是因为从原点到凸组合点的距离,不超过两个距离的加权平均
\end{itemize}

\subsection{为什么三角不等式成立?}

三角不等式告诉我们:两点之间直线最短。

\begin{itemize}
\item 路径1:从 $\mathbf{x}_c$ 到 $\theta \mathbf{x}_1 + (1-\theta) \mathbf{x}_2$(直接路径)
\item 路径2:从 $\mathbf{x}_c$ 到 $\mathbf{x}_1$(距离 $\leq r$),然后到 $\mathbf{x}_2$(距离 $\leq r$),再取凸组合
\item 直接路径的长度不超过"绕路"的长度
\end{itemize}

\section{关键数学工具}

\subsection{范数的性质}

\begin{enumerate}
\item \textbf{三角不等式}:$\|\mathbf{u} + \mathbf{v}\| \leq \|\mathbf{u}\| + \|\mathbf{v}\|$
\item \textbf{齐次性}:$\|\alpha \mathbf{w}\| = |\alpha| \|\mathbf{w}\|$
\item \textbf{非负性}:$\|\mathbf{w}\| \geq 0$
\end{enumerate}

\subsection{凸组合的性质}

\begin{enumerate}
\item \textbf{系数和等于1}:$\theta + (1-\theta) = 1$
\item \textbf{系数非负}:$\theta \geq 0$,$1-\theta \geq 0$
\item \textbf{可以"分配"}:$\theta \mathbf{x}_c + (1-\theta) \mathbf{x}_c = \mathbf{x}_c$
\end{enumerate}

\section{具体数值例子}

\subsection{例子}

在 $\mathbb{R}^2$ 中,设 $\mathbf{x}_c = (0, 0)$,$r = 1$(单位圆)。

设 $\mathbf{x}_1 = (0.8, 0)$,$\mathbf{x}_2 = (0, 0.8)$,$\theta = 0.5$。

\textbf{验证}:
\begin{itemize}
\item $\|\mathbf{x}_1 - \mathbf{x}_c\|_2 = \|(0.8, 0)\|_2 = 0.8 \leq 1$ ✓
\item $\|\mathbf{x}_2 - \mathbf{x}_c\|_2 = \|(0, 0.8)\|_2 = 0.8 \leq 1$ ✓
\end{itemize}

\textbf{凸组合}:
\begin{align}
\theta \mathbf{x}_1 + (1-\theta) \mathbf{x}_2 &= 0.5(0.8, 0) + 0.5(0, 0.8) \\
&= (0.4, 0.4)
\end{align}

\textbf{验证在圆内}:
\begin{align}
\|\theta \mathbf{x}_1 + (1-\theta) \mathbf{x}_2 - \mathbf{x}_c\|_2 &= \|(0.4, 0.4)\|_2 \\
&= \sqrt{0.4^2 + 0.4^2} = \sqrt{0.32} \approx 0.566 \leq 1 \checkmark
\end{align}

\textbf{使用证明中的方法}:
\begin{align}
\|\theta(\mathbf{x}_1 - \mathbf{x}_c) + (1-\theta)(\mathbf{x}_2 - \mathbf{x}_c)\|_2 &= \|0.5(0.8, 0) + 0.5(0, 0.8)\|_2 \\
&= \|(0.4, 0.4)\|_2 = 0.566
\end{align}

上界:
\begin{align}
\theta \|\mathbf{x}_1 - \mathbf{x}_c\|_2 + (1-\theta) \|\mathbf{x}_2 - \mathbf{x}_c\|_2 &= 0.5 \times 0.8 + 0.5 \times 0.8 \\
&= 0.8 \leq 1 \checkmark
\end{align}

\section{推广到其他范数}

\subsection{关键要求}

这个证明依赖于:
\begin{itemize}
\item 三角不等式
\item 范数的齐次性
\end{itemize}

所有范数都满足这两个性质,所以:

\textbf{定理}:对于任意范数 $\|\cdot\|$,球 $\{\mathbf{x} \mid \|\mathbf{x} - \mathbf{x}_c\| \leq r\}$ 是凸集。

\subsection{例子:L1范数}

对于L1范数球:$\{\mathbf{x} \mid \|\mathbf{x} - \mathbf{x}_c\|_1 \leq r\}$

证明完全相同:
\begin{align}
\|\theta \mathbf{x}_1 + (1-\theta) \mathbf{x}_2 - \mathbf{x}_c\|_1 &= \|\theta(\mathbf{x}_1 - \mathbf{x}_c) + (1-\theta)(\mathbf{x}_2 - \mathbf{x}_c)\|_1 \\
&\leq \theta \|\mathbf{x}_1 - \mathbf{x}_c\|_1 + (1-\theta) \|\mathbf{x}_2 - \mathbf{x}_c\|_1 \\
&\leq r
\end{align}

\section{总结}

\begin{enumerate}
\item \textbf{第一步(代数恒等式)}:
   \begin{itemize}
   \item 技巧:将 $\mathbf{x}_c$ 写成 $\theta \mathbf{x}_c + (1-\theta) \mathbf{x}_c$
   \item 目的:转化为相对向量的形式
   \end{itemize}

\item \textbf{第二步(三角不等式)}:
   \begin{itemize}
   \item 使用:$\|\mathbf{u} + \mathbf{v}\| \leq \|\mathbf{u}\| + \|\mathbf{v}\|$
   \item 结合:范数的齐次性
   \item 结果:得到范数的加权和
   \end{itemize}

\item \textbf{第三步(利用已知条件)}:
   \begin{itemize}
   \item 使用:已知两个距离都 $\leq r$
   \item 结合:$\theta + (1-\theta) = 1$
   \item 结果:得到最终上界 $r$
   \end{itemize}

\item \textbf{关键工具}:
   \begin{itemize}
   \item 三角不等式(所有范数都满足)
   \item 范数的齐次性
   \item 凸组合的性质
   \end{itemize}
\end{enumerate}

这个证明方法可以推广到所有范数定义的球,证明了所有范数球都是凸集!

\end{document}


